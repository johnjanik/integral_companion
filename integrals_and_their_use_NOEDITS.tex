\documentclass{article}
\usepackage{amsmath}
\usepackage{amssymb}
\usepackage{bm}
\usepackage{hyperref}

% Suppress LaTeX automatic section numbering so only the
% explicit Gradshteyn-Ryzhik numbers appear in headings.
\setcounter{secnumdepth}{-1}

% Allow page breaks between consecutive headings (no body text).
% Without this, LaTeX suppresses breaks after headings, causing
% massive vbox overflow when headings appear back-to-back.
\makeatletter
\renewcommand{\@afterheading}{}
\makeatother
\raggedbottom

\begin{document}

%% ============================================================
%% 0  Introduction
%% ============================================================
\section{0\quad Introduction}

\subsection{0.1\quad Finite Sums}
\subsubsection{0.11\quad Progressions}
\subsubsection{0.12\quad Sums of powers of natural numbers}
\subsubsection{0.13\quad Sums of reciprocals of natural numbers}
\subsubsection{0.14\quad Sums of products of reciprocals of natural numbers}
\subsubsection{0.15\quad Sums of the binomial coefficients}

\subsection{0.2\quad Numerical Series and Infinite Products}
\subsubsection{0.21\quad The convergence of numerical series}
\subsubsection{0.22\quad Convergence tests}
\subsubsection{0.23--0.24\quad Examples of numerical series}
\subsubsection{0.25\quad Infinite products}
\subsubsection{0.26\quad Examples of infinite products}

\subsection{0.3\quad Functional Series}
\subsubsection{0.30\quad Definitions and theorems}
\subsubsection{0.31\quad Power series}
\subsubsection{0.32\quad Fourier series}
\subsubsection{0.33\quad Asymptotic series}

\subsection{0.4\quad Certain Formulas from Differential Calculus}
\subsubsection{0.41\quad Differentiation of a definite integral with respect to a parameter}
\subsubsection{0.42\quad The nth derivative of a product (Leibniz's rule)}
\subsubsection{0.43\quad The nth derivative of a composite function}
\subsubsection{0.44\quad Integration by substitution}

%% ============================================================
%% 1  Elementary Functions
%% ============================================================
\section{1\quad Elementary Functions}

\subsection{1.1\quad Power of Binomials}
\subsubsection{1.11\quad Power series}
\subsubsection{1.12\quad Series of rational fractions}

\subsection{1.2\quad The Exponential Function}
\subsubsection{1.21\quad Series representation}
\subsubsection{1.22\quad Functional relations}
\subsubsection{1.23\quad Series of exponentials}

\subsection{1.3--1.4\quad Trigonometric and Hyperbolic Functions}
\subsubsection{1.30\quad Introduction}
\subsubsection{1.31\quad The basic functional relations}
\subsubsection{1.32\quad The representation of powers of trigonometric and hyperbolic functions in terms of functions of multiples of the argument (angle)}
\subsubsection{1.33\quad The representation of trigonometric and hyperbolic functions of multiples of the argument (angle) in terms of powers of these functions}
\subsubsection{1.34\quad Certain sums of trigonometric and hyperbolic functions}
\subsubsection{1.35\quad Sums of powers of trigonometric functions of multiple angles}
\subsubsection{1.36\quad Sums of products of trigonometric functions of multiple angles}
\subsubsection{1.37\quad Sums of tangents of multiple angles}
\subsubsection{1.38\quad Sums leading to hyperbolic tangents and cotangents}
\subsubsection{1.39\quad The representation of cosines and sines of multiples of the angle as finite products}
\subsubsection{1.41\quad The expansion of trigonometric and hyperbolic functions in power series}
\subsubsection{1.42\quad Expansion in series of simple fractions}
\subsubsection{1.43\quad Representation in the form of an infinite product}
\subsubsection{1.44--1.45\quad Trigonometric (Fourier) series}
\subsubsection{1.46\quad Series of products of exponential and trigonometric functions}
\subsubsection{1.47\quad Series of hyperbolic functions}
\subsubsection{1.48\quad Lobachevskiy's ``Angle of Parallelism'' $\Pi(x)$}
\subsubsection{1.49\quad The hyperbolic amplitude (the Gudermannian) $\operatorname{gd} x$}

\subsection{1.5\quad The Logarithm}
\subsubsection{1.51\quad Series representation}
\subsubsection{1.52\quad Series of logarithms (cf.\ 1.431)}

\subsection{1.6\quad The Inverse Trigonometric and Hyperbolic Functions}
\subsubsection{1.61\quad The domain of definition}
\subsubsection{1.62--1.63\quad Functional relations}
\subsubsection{1.64\quad Series representations}

%% ============================================================
%% 2  Indefinite Integrals of Elementary Functions
%% ============================================================
\section{2\quad Indefinite Integrals of Elementary Functions}

\subsection{2.0\quad Introduction}
\subsubsection{2.00\quad General remarks}
\subsubsection{2.01\quad The basic integrals}
\subsubsection{2.02\quad General formulas}

\subsection{2.1\quad Rational Functions}
\subsubsection{2.10\quad General integration rules}
\subsubsection{2.11--2.13\quad Forms containing the binomial $a + bx^k$}
\subsubsection{2.14\quad Forms containing the binomial $1\pm x^n$}
\subsubsection{2.15\quad Forms containing pairs of binomials: $a + bx$ and $\alpha +\beta x$}
\subsubsection{2.16\quad Forms containing the trinomial $a + bx^k + cx^{2k}$}
\subsubsection{2.17\quad Forms containing the quadratic trinomial $a + bx + cx^2$ and powers of $x$}
\subsubsection{2.18\quad Forms containing the quadratic trinomial $a + bx + cx^2$ and the binomial $\alpha +\beta x$}

\subsection{2.2\quad Algebraic Functions}
\subsubsection{2.20\quad Introduction}
\subsubsection{2.21\quad Forms containing the binomial $a + bx^k$ and $\sqrt{x}$}
\subsubsection{2.22--2.23\quad Forms containing $\sqrt[n]{(a + bx)^k}$}
\subsubsection{2.24\quad Forms containing $\sqrt{a + bx}$ and the binomial $\alpha +\beta x$}
\subsubsection{2.25\quad Forms containing $\sqrt{a + bx + cx^2}$}
\subsubsection{2.26\quad Forms containing $\sqrt{a + bx + cx^2}$ and integral powers of $x$}
\subsubsection{2.27\quad Forms containing $\sqrt{a + cx^2}$ and integral powers of $x$}
\subsubsection{2.28\quad Forms containing $\sqrt{a + bx + cx^2}$ and first- and second-degree polynomials}
\subsubsection{2.29\quad Integrals that can be reduced to elliptic or pseudo-elliptic integrals}

\subsection{2.3\quad The Exponential Function}
\subsubsection{2.31\quad Forms containing $e^{ax}$}
\subsubsection{2.32\quad The exponential combined with rational functions of $x$}

\subsection{2.4\quad Hyperbolic Functions}
\subsubsection{2.41--2.43\quad Powers of $\sinh x$, $\cosh x$, $\tanh x$, and $\coth x$}
\subsubsection{2.44--2.45\quad Rational functions of hyperbolic functions}
\subsubsection{2.46\quad Algebraic functions of hyperbolic functions}
\subsubsection{2.47\quad Combinations of hyperbolic functions and powers}
\subsubsection{2.48\quad Combinations of hyperbolic functions, exponentials, and powers}

\subsection{2.5--2.6\quad Trigonometric Functions}
\subsubsection{2.50\quad Introduction}
\subsubsection{2.51--2.52\quad Powers of trigonometric functions}
\subsubsection{2.53--2.54\quad Sines and cosines of multiple angles and of linear and more complicated functions of the argument}
\subsubsection{2.55--2.56\quad Rational functions of the sine and cosine}
\subsubsection{2.57\quad Integrals containing $\sqrt{a\pm b\sin x}$ or $\sqrt{a\pm b\cos x}$}
\subsubsection{2.58--2.62\quad Integrals reducible to elliptic and pseudo-elliptic integrals}
\subsubsection{2.63--2.65\quad Products of trigonometric functions and powers}
\subsubsection{2.66\quad Combinations of trigonometric functions and exponentials}
\subsubsection{2.67\quad Combinations of trigonometric and hyperbolic functions}

\subsection{2.7\quad Logarithms and Inverse-Hyperbolic Functions}
\subsubsection{2.71\quad The logarithm}
\subsubsection{2.72--2.73\quad Combinations of logarithms and algebraic functions}
\subsubsection{2.74\quad Inverse hyperbolic functions}

\subsection{2.8\quad Inverse Trigonometric Functions}
\subsubsection{2.81\quad Arcsines and arccosines}
\subsubsection{2.82\quad The arcsecant, the arccosecant, the arctangent, and the arccotangent}
\subsubsection{2.83\quad Combinations of arcsine or arccosine and algebraic functions}
\subsubsection{2.84\quad Combinations of the arcsecant and arccosecant with powers of $x$}
\subsubsection{2.85\quad Combinations of the arctangent and arccotangent with algebraic functions}

%% ============================================================
%% 3--4  Definite Integrals of Elementary Functions
%% ============================================================
\section{3--4\quad Definite Integrals of Elementary Functions}

\subsection{3.0\quad Introduction}
\subsubsection{3.01\quad Theorems of a general nature}
\subsubsection{3.02\quad Change of variable in a definite integral}
\subsubsection{3.03\quad General formulas}
\subsubsection{3.04\quad Improper integrals}
\subsubsection{3.05\quad The principal values of improper integrals}

\subsection{3.1--3.2\quad Power and Algebraic Functions}
\subsubsection{3.11\quad Rational functions}
\subsubsection{3.12\quad Products of rational functions and expressions that can be reduced to square roots of first- and second-degree polynomials}
\subsubsection{3.13--3.17\quad Expressions that can be reduced to square roots of third- and fourth-degree polynomials and their products with rational functions}
\subsubsection{3.18\quad Expressions that can be reduced to fourth roots of second-degree polynomials and their products with rational functions}
\subsubsection{3.19--3.23\quad Combinations of powers of $x$ and powers of binomials of the form $(\alpha +\beta x)$}
\subsubsection{3.24--3.27\quad Powers of $x$, of binomials of the form $\alpha +\beta x^{p}$ and of polynomials in $x$}

\subsection{3.3--3.4\quad Exponential Functions}
\subsubsection{3.31\quad Exponential functions}
\subsubsection{3.32--3.34\quad Exponentials of more complicated arguments}
\subsubsection{3.35\quad Combinations of exponentials and rational functions}
\subsubsection{3.36--3.37\quad Combinations of exponentials and algebraic functions}
\subsubsection{3.38--3.39\quad Combinations of exponentials and arbitrary powers}
\subsubsection{3.41--3.44\quad Combinations of rational functions of powers and exponentials}
\subsubsection{3.45\quad Combinations of powers and algebraic functions of exponentials}
\subsubsection{3.46--3.48\quad Combinations of exponentials of more complicated arguments and powers}

\subsection{3.5\quad Hyperbolic Functions}
\subsubsection{3.51\quad Hyperbolic functions}
\subsubsection{3.52--3.53\quad Combinations of hyperbolic functions and algebraic functions}
\subsubsection{3.54\quad Combinations of hyperbolic functions and exponentials}
\subsubsection{3.55--3.56\quad Combinations of hyperbolic functions, exponentials, and powers}

\subsection{3.6--4.1\quad Trigonometric Functions}
\subsubsection{3.61\quad Rational functions of sines and cosines and trigonometric functions of multiple angles}
\subsubsection{3.62\quad Powers of trigonometric functions}
\subsubsection{3.63\quad Powers of trigonometric functions and trigonometric functions of linear functions}
\subsubsection{3.64--3.65\quad Powers and rational functions of trigonometric functions}
\subsubsection{3.66\quad Forms containing powers of linear functions of trigonometric functions}
\subsubsection{3.67\quad Square roots of expressions containing trigonometric functions}
\subsubsection{3.68\quad Various forms of powers of trigonometric functions}
\subsubsection{3.69--3.71\quad Trigonometric functions of more complicated arguments}
\subsubsection{3.72--3.74\quad Combinations of trigonometric and rational functions}
\subsubsection{3.75\quad Combinations of trigonometric and algebraic functions}
\subsubsection{3.76--3.77\quad Combinations of trigonometric functions and powers}
\subsubsection{3.78--3.81\quad Rational functions of $x$ and of trigonometric functions}
\subsubsection{3.82--3.83\quad Powers of trigonometric functions combined with other powers}
\subsubsection{3.84\quad Integrals containing $\sqrt{1 - k^{2}\sin^{2}x}$, $\sqrt{1 - k^{2}\cos^{2}x}$, and similar expressions}
\subsubsection{3.85--3.88\quad Trigonometric functions of more complicated arguments combined with powers}
\subsubsection{3.89--3.91\quad Trigonometric functions and exponentials}
\subsubsection{3.92\quad Trigonometric functions of more complicated arguments combined with exponentials}
\subsubsection{3.93\quad Trigonometric and exponential functions of trigonometric functions}
\subsubsection{3.94--3.97\quad Combinations involving trigonometric functions, exponentials, and powers}
\subsubsection{3.98--3.99\quad Combinations of trigonometric and hyperbolic functions}
\subsubsection{4.11--4.12\quad Combinations involving trigonometric and hyperbolic functions and powers}
\subsubsection{4.13\quad Combinations of trigonometric and hyperbolic functions and exponentials}
\subsubsection{4.14\quad Combinations of trigonometric and hyperbolic functions, exponentials, and powers}

\subsection{4.2--4.4\quad Logarithmic Functions}
\subsubsection{4.21\quad Logarithmic functions}
\subsubsection{4.22\quad Logarithms of more complicated arguments}
\subsubsection{4.23\quad Combinations of logarithms and rational functions}
\subsubsection{4.24\quad Combinations of logarithms and algebraic functions}
\subsubsection{4.25\quad Combinations of logarithms and powers}
\subsubsection{4.26--4.27\quad Combinations involving powers of the logarithm and other powers}
\subsubsection{4.28\quad Combinations of rational functions of $\ln x$ and powers}
\subsubsection{4.29--4.32\quad Combinations of logarithmic functions of more complicated arguments and powers}
\subsubsection{4.33--4.34\quad Combinations of logarithms and exponentials}
\subsubsection{4.35--4.36\quad Combinations of logarithms, exponentials, and powers}
\subsubsection{4.37\quad Combinations of logarithms and hyperbolic functions}
\subsubsection{4.38--4.41\quad Logarithms and trigonometric functions}
\subsubsection{4.42--4.43\quad Combinations of logarithms, trigonometric functions, and powers}
\subsubsection{4.44\quad Combinations of logarithms, trigonometric functions, and exponentials}

\subsection{4.5\quad Inverse Trigonometric Functions}
\subsubsection{4.51\quad Inverse trigonometric functions}
\subsubsection{4.52\quad Combinations of arcsines, arccosines, and powers}
\subsubsection{4.53--4.54\quad Combinations of arctangents, arccotangents, and powers}
\subsubsection{4.55\quad Combinations of inverse trigonometric functions and exponentials}
\subsubsection{4.56\quad A combination of the arctangent and a hyperbolic function}
\subsubsection{4.57\quad Combinations of inverse and direct trigonometric functions}
\subsubsection{4.58\quad A combination involving an inverse and a direct trigonometric function and a power}
\subsubsection{4.59\quad Combinations of inverse trigonometric functions and logarithms}

\subsection{4.6\quad Multiple Integrals}
\subsubsection{4.60\quad Change of variables in multiple integrals}
\subsubsection{4.61\quad Change of the order of integration and change of variables}
\subsubsection{4.62\quad Double and triple integrals with constant limits}
\subsubsection{4.63--4.64\quad Multiple integrals}

%% ============================================================
%% 5  Indefinite Integrals of Special Functions
%% ============================================================
\section{5\quad Indefinite Integrals of Special Functions}

\subsection{5.1\quad Elliptic Integrals and Functions}
\subsubsection{5.11\quad Complete elliptic integrals}
\subsubsection{5.12\quad Elliptic integrals}
\subsubsection{5.13\quad Jacobian elliptic functions}
\subsubsection{5.14\quad Weierstrass elliptic functions}

\subsection{5.2\quad The Exponential Integral Function}
\subsubsection{5.21\quad The exponential integral function}
\subsubsection{5.22\quad Combinations of the exponential integral function and powers}
\subsubsection{5.23\quad Combinations of the exponential integral and the exponential}

\subsection{5.3\quad The Sine Integral and the Cosine Integral}

\subsection{5.4\quad The Probability Integral and Fresnel Integrals}

\subsection{5.5\quad Bessel Functions}

%% ============================================================
%% 6--7  Definite Integrals of Special Functions
%% ============================================================
\section{6--7\quad Definite Integrals of Special Functions}

\subsection{6.1\quad Elliptic Integrals and Functions}
\subsubsection{6.11\quad Forms containing $F(x,k)$}
\subsubsection{6.12\quad Forms containing $E(x,k)$}
\subsubsection{6.13\quad Integration of elliptic integrals with respect to the modulus}
\subsubsection{6.14--6.15\quad Complete elliptic integrals}
\subsubsection{6.16\quad The theta function}
\subsubsection{6.17\quad Generalized elliptic integrals}

\subsection{6.2--6.3\quad The Exponential Integral Function and Functions Generated by It}
\subsubsection{6.21\quad The logarithm integral}
\subsubsection{6.22--6.23\quad The exponential integral function}
\subsubsection{6.24--6.26\quad The sine integral and cosine integral functions}
\subsubsection{6.27\quad The hyperbolic sine integral and hyperbolic cosine integral functions}
\subsubsection{6.28--6.31\quad The probability integral}
\subsubsection{6.32\quad Fresnel integrals}

\subsection{6.4\quad The Gamma Function and Functions Generated by It}
\subsubsection{6.41\quad The gamma function}
\subsubsection{6.42\quad Combinations of the gamma function, the exponential, and powers}
\subsubsection{6.43\quad Combinations of the gamma function and trigonometric functions}
\subsubsection{6.44\quad The logarithm of the gamma function\textsuperscript{*}}
\subsubsection{6.45\quad The incomplete gamma function}
\subsubsection{6.46--6.47\quad The function $\psi(x)$}

\subsection{6.5--6.7\quad Bessel Functions}
\subsubsection{6.51\quad Bessel functions}
\subsubsection{6.52\quad Bessel functions combined with $x$ and $x^{2}$}
\subsubsection{6.53--6.54\quad Combinations of Bessel functions and rational functions}
\subsubsection{6.55\quad Combinations of Bessel functions and algebraic functions}
\subsubsection{6.56--6.58\quad Combinations of Bessel functions and powers}
\subsubsection{6.59\quad Combinations of powers and Bessel functions of more complicated arguments}
\subsubsection{6.61\quad Combinations of Bessel functions and exponentials}
\subsubsection{6.62--6.63\quad Combinations of Bessel functions, exponentials, and powers}
\subsubsection{6.64\quad Combinations of Bessel functions of more complicated arguments, exponentials, and powers}
\subsubsection{6.65\quad Combinations of Bessel and exponential functions of more complicated arguments and powers}
\subsubsection{6.66\quad Combinations of Bessel, hyperbolic, and exponential functions}
\subsubsection{6.67--6.68\quad Combinations of Bessel and trigonometric functions}
\subsubsection{6.69--6.74\quad Combinations of Bessel and trigonometric functions and powers}
\subsubsection{6.75\quad Combinations of Bessel, trigonometric, and exponential functions and powers}
\subsubsection{6.76\quad Combinations of Bessel, trigonometric, and hyperbolic functions}
\subsubsection{6.77\quad Combinations of Bessel functions and the logarithm, or arctangent}
\subsubsection{6.78\quad Combinations of Bessel and other special functions}
\subsubsection{6.79\quad Integration of Bessel functions with respect to the order}

\subsection{6.8\quad Functions Generated by Bessel Functions}
\subsubsection{6.81\quad Struve functions}
\subsubsection{6.82\quad Combinations of Struve functions, exponentials, and powers}
\subsubsection{6.83\quad Combinations of Struve and trigonometric functions}
\subsubsection{6.84--6.85\quad Combinations of Struve and Bessel functions}
\subsubsection{6.86\quad Lommel functions}
\subsubsection{6.87\quad Thomson functions}

\subsection{6.9\quad Mathieu Functions}
\subsubsection{6.91\quad Mathieu functions}
\subsubsection{6.92\quad Combinations of Mathieu, hyperbolic, and trigonometric functions}
\subsubsection{6.93\quad Combinations of Mathieu and Bessel functions}
\subsubsection{6.94\quad Relationships between eigenfunctions of the Helmholtz equation in different coordinate systems}

\subsection{7.1--7.2\quad Associated Legendre Functions}
\subsubsection{7.11\quad Associated Legendre functions}
\subsubsection{7.12--7.13\quad Combinations of associated Legendre functions and powers}
\subsubsection{7.14\quad Combinations of associated Legendre functions, exponentials, and powers}
\subsubsection{7.15\quad Combinations of associated Legendre and hyperbolic functions}
\subsubsection{7.16\quad Combinations of associated Legendre functions, powers, and trigonometric functions}
\subsubsection{7.17\quad A combination of an associated Legendre function and the probability integral}
\subsubsection{7.18\quad Combinations of associated Legendre and Bessel functions}
\subsubsection{7.19\quad Combinations of associated Legendre functions and functions generated by Bessel functions}
\subsubsection{7.21\quad Integration of associated Legendre functions with respect to the order}
\subsubsection{7.22\quad Combinations of Legendre polynomials, rational functions, and algebraic functions}
\subsubsection{7.23\quad Combinations of Legendre polynomials and powers}
\subsubsection{7.24\quad Combinations of Legendre polynomials and other elementary functions}
\subsubsection{7.25\quad Combinations of Legendre polynomials and Bessel functions}

\subsection{7.3--7.4\quad Orthogonal Polynomials}
\subsubsection{7.31\quad Combinations of Gegenbauer polynomials $C_{n}^{\nu}(x)$ and powers}
\subsubsection{7.32\quad Combinations of Gegenbauer polynomials $C_{n}^{\nu}(x)$ and elementary functions}
\subsubsection{7.325\textsuperscript{*}\quad Complete System of Orthogonal Step Functions}
\subsubsection{7.33\quad Combinations of the polynomials $C_{n}^{\nu}(x)$ and Bessel functions; Integration of Gegenbauer functions with respect to the index}
\subsubsection{7.34\quad Combinations of Chebyshev polynomials and powers}
\subsubsection{7.35\quad Combinations of Chebyshev polynomials and elementary functions}
\subsubsection{7.36\quad Combinations of Chebyshev polynomials and Bessel functions}
\subsubsection{7.37--7.38\quad Hermite polynomials}
\subsubsection{7.39\quad Jacobi polynomials}
\subsubsection{7.41--7.42\quad Laguerre polynomials}

\subsection{7.5\quad Hypergeometric Functions}
\subsubsection{7.51\quad Combinations of hypergeometric functions and powers}
\subsubsection{7.52\quad Combinations of hypergeometric functions and exponentials}
\subsubsection{7.53\quad Hypergeometric and trigonometric functions}
\subsubsection{7.54\quad Combinations of hypergeometric and Bessel functions}

\subsection{7.6\quad Confluent Hypergeometric Functions}
\subsubsection{7.61\quad Combinations of confluent hypergeometric functions and powers}
\subsubsection{7.62--7.63\quad Combinations of confluent hypergeometric functions and exponentials}
\subsubsection{7.64\quad Combinations of confluent hypergeometric and trigonometric functions}
\subsubsection{7.65\quad Combinations of confluent hypergeometric functions and Bessel functions}
\subsubsection{7.66\quad Combinations of confluent hypergeometric functions, Bessel functions, and powers}
\subsubsection{7.67\quad Combinations of confluent hypergeometric functions, Bessel functions, exponentials, and powers}
\subsubsection{7.68\quad Combinations of confluent hypergeometric functions and other special functions}
\subsubsection{7.69\quad Integration of confluent hypergeometric functions with respect to the index}

\subsection{7.7\quad Parabolic Cylinder Functions}
\subsubsection{7.71\quad Parabolic cylinder functions}
\subsubsection{7.72\quad Combinations of parabolic cylinder functions, powers, and exponentials}
\subsubsection{7.73\quad Combinations of parabolic cylinder and hyperbolic functions}
\subsubsection{7.74\quad Combinations of parabolic cylinder and trigonometric functions}
\subsubsection{7.75\quad Combinations of parabolic cylinder and Bessel functions}
\subsubsection{7.76\quad Combinations of parabolic cylinder functions and confluent hypergeometric functions}
\subsubsection{7.77\quad Integration of a parabolic cylinder function with respect to the index}

\subsection{7.8\quad Meijer's and MacRobert's Functions (G and E)}
\subsubsection{7.81\quad Combinations of the functions G and E and the elementary functions}
\subsubsection{7.82\quad Combinations of the functions G and E and Bessel functions}
\subsubsection{7.83\quad Combinations of the functions G and E and other special functions}

%% ============================================================
%% 8--9  Special Functions
%% ============================================================
\section{8--9\quad Special Functions}

\subsection{8.1\quad Elliptic Integrals and Functions}
\subsubsection{8.11\quad Elliptic integrals}
\subsubsection{8.12\quad Functional relations between elliptic integrals}
\subsubsection{8.13\quad Elliptic functions}
\subsubsection{8.14\quad Jacobian elliptic functions}
\subsubsection{8.15\quad Properties of Jacobian elliptic functions and functional relationships between them}
\subsubsection{8.16\quad The Weierstrass function $\wp(u)$}
\subsubsection{8.17\quad The functions $\zeta(u)$ and $\sigma(u)$}
\subsubsection{8.18--8.19\quad Theta functions}

\subsection{8.2\quad The Exponential Integral Function and Functions Generated by It}
\subsubsection{8.21\quad The exponential integral function $\operatorname{Ei}(x)$}
\subsubsection{8.22\quad The hyperbolic sine integral $\operatorname{shi} x$ and the hyperbolic cosine integral $\operatorname{chi} x$}
\subsubsection{8.23\quad The sine integral and the cosine integral: $\operatorname{si} x$ and $\operatorname{ci} x$}
\subsubsection{8.24\quad The logarithm integral $\operatorname{li}(x)$}
\subsubsection{8.25\quad The probability integral $\Phi(x)$, the Fresnel integrals $S(x)$ and $C(x)$, the error function $\operatorname{erf}(x)$, and the complementary error function $\operatorname{erfc}(x)$}
\subsubsection{8.26\quad Lobachevskiy's function $L(x)$}

\subsection{8.3\quad Euler's Integrals of the First and Second Kinds}
\subsubsection{8.31\quad The gamma function (Euler's integral of the second kind): $\Gamma(z)$}
\subsubsection{8.32\quad Representation of the gamma function as series and products}
\subsubsection{8.33\quad Functional relations involving the gamma function}
\subsubsection{8.34\quad The logarithm of the gamma function}
\subsubsection{8.35\quad The incomplete gamma function}
\subsubsection{8.36\quad The psi function $\psi(x)$}
\subsubsection{8.37\quad The function $\beta(x)$}
\subsubsection{8.38\quad The beta function (Euler's integral of the first kind): $\operatorname{B}(x,y)$}
\subsubsection{8.39\quad The incomplete beta function $\operatorname{B}_x(p,q)$}

\subsection{8.4--8.5\quad Bessel Functions and Functions Associated with Them}
\subsubsection{8.40\quad Definitions}
\subsubsection{8.41\quad Integral representations of the functions $J_{\nu}(z)$ and $N_{\nu}(z)$}
\subsubsection{8.42\quad Integral representations of the functions $H_{\nu}^{(1)}(z)$ and $H_{\nu}^{(2)}(z)$}
\subsubsection{8.43\quad Integral representations of the functions $I_{\nu}(z)$ and $K_{\nu}(z)$}
\subsubsection{8.44\quad Series representation}
\subsubsection{8.45\quad Asymptotic expansions of Bessel functions}
\subsubsection{8.46\quad Bessel functions of order equal to an integer plus one-half}
\subsubsection{8.47--8.48\quad Functional relations}
\subsubsection{8.49\quad Differential equations leading to Bessel functions}
\subsubsection{8.51--8.52\quad Series of Bessel functions}
\subsubsection{8.53\quad Expansion in products of Bessel functions}
\subsubsection{8.54\quad The zeros of Bessel functions}
\subsubsection{8.55\quad Struve functions}
\subsubsection{8.56\quad Thomson functions and their generalizations}
\subsubsection{8.57\quad Lommel functions}
\subsubsection{8.58\quad Anger and Weber functions $J_{\nu}(z)$ and $\mathbf{E}_{\nu}(z)$}
\subsubsection{8.59\quad Neumann's and Schl\"afli's polynomials: $O_{n}(z)$ and $S_{n}(z)$}

\subsection{8.6\quad Mathieu Functions}
\subsubsection{8.60\quad Mathieu's equation}
\subsubsection{8.61\quad Periodic Mathieu functions}
\subsubsection{8.62\quad Recursion relations for the coefficients $A_{2r}^{(2n)}$, $A_{2r+1}^{(2n+1)}$, $B_{2r+1}^{(2n+1)}$, $B_{2r+2}^{(2n+2)}$}
\subsubsection{8.63\quad Mathieu functions with a purely imaginary argument}
\subsubsection{8.64\quad Non-periodic solutions of Mathieu's equation}
\subsubsection{8.65\quad Mathieu functions for negative $q$}
\subsubsection{8.66\quad Representation of Mathieu functions as series of Bessel functions}
\subsubsection{8.67\quad The general theory}

\subsection{8.7--8.8\quad Associated Legendre Functions}
\subsubsection{8.70\quad Introduction}
\subsubsection{8.71\quad Integral representations}
\subsubsection{8.72\quad Asymptotic series for large values of $|\nu|$}
\subsubsection{8.73--8.74\quad Functional relations}
\subsubsection{8.75\quad Special cases and particular values}
\subsubsection{8.76\quad Derivatives with respect to the order}
\subsubsection{8.77\quad Series representation}
\subsubsection{8.78\quad The zeros of associated Legendre functions}
\subsubsection{8.79\quad Series of associated Legendre functions}
\subsubsection{8.81\quad Associated Legendre functions with integer indices}
\subsubsection{8.82--8.83\quad Legendre functions}
\subsubsection{8.84\quad Conical functions}
\subsubsection{8.85\quad Toroidal functions}

\subsection{8.9\quad Orthogonal Polynomials}
\subsubsection{8.90\quad Introduction}
\subsubsection{8.91\quad Legendre polynomials}
\subsubsection{8.919\quad Series of products of Legendre and Chebyshev polynomials}
\subsubsection{8.92\quad Series of Legendre polynomials}
\subsubsection{8.93\quad Gegenbauer polynomials $C_{n}^{\lambda}(t)$}
\subsubsection{8.94\quad The Chebyshev polynomials $T_{n}(x)$ and $U_{n}(x)$}
\subsubsection{8.95\quad The Hermite polynomials $H_{n}(x)$}
\subsubsection{8.96\quad Jacobi's polynomials}
\subsubsection{8.97\quad The Laguerre polynomials}

\subsection{9.1\quad Hypergeometric Functions}
\subsubsection{9.10\quad Definition}
\subsubsection{9.11\quad Integral representations}
\subsubsection{9.12\quad Representation of elementary functions in terms of a hypergeometric functions}
\subsubsection{9.13\quad Transformation formulas and the analytic continuation of functions defined by hypergeometric series}
\subsubsection{9.14\quad A generalized hypergeometric series}
\subsubsection{9.15\quad The hypergeometric differential equation}
\subsubsection{9.16\quad Riemann's differential equation}
\subsubsection{9.17\quad Representing the solutions to certain second-order differential equations using a Riemann scheme}
\subsubsection{9.18\quad Hypergeometric functions of two variables}
\subsubsection{9.19\quad A hypergeometric function of several variables}

\subsection{9.2\quad Confluent Hypergeometric Functions}
\subsubsection{9.20\quad Introduction}
\subsubsection{9.21\quad The functions $\Phi(\alpha,\gamma;z)$ and $\Psi(\alpha,\gamma;z)$}
\subsubsection{9.22--9.23\quad The Whittaker functions $M_{\lambda,\mu}(z)$ and $W_{\lambda,\mu}(z)$}
\subsubsection{9.24--9.25\quad Parabolic cylinder functions $D_{p}(z)$}
\subsubsection{9.26\quad Confluent hypergeometric series of two variables}

\subsection{9.3\quad Meijer's $G$-Function}
\subsubsection{9.30\quad Definition}
\subsubsection{9.31\quad Functional relations}
\subsubsection{9.32\quad A differential equation for the G-function}
\subsubsection{9.33\quad Series of G-functions}
\subsubsection{9.34\quad Connections with other special functions}

\subsection{9.4\quad MacRobert's $E$-Function}
\subsubsection{9.41\quad Representation by means of multiple integrals}
\subsubsection{9.42\quad Functional relations}

\subsection{9.5\quad Riemann's Zeta Functions $\zeta(z,q)$ and $\zeta(z)$, and the Functions $\Phi(z,s,v)$ and $\xi(s)$}
\subsubsection{9.51\quad Definition and integral representations}
\subsubsection{9.52\quad Representation as a series or as an infinite product}
\subsubsection{9.53\quad Functional relations}
\subsubsection{9.54\quad Singular points and zeros}
\subsubsection{9.55\quad The Lerch function $\Phi(z,s,v)$}
\subsubsection{9.56\quad The function $\xi(s)$}

\subsection{9.6\quad Bernoulli Numbers and Polynomials, Euler Numbers}
\subsubsection{9.61\quad Bernoulli numbers}
\subsubsection{9.62\quad Bernoulli polynomials}
\subsubsection{9.63\quad Euler numbers}
\subsubsection{9.64\quad The functions $\nu(x)$, $\nu(x,\alpha)$, $\mu(x,\beta)$, $\mu(x,\beta,\alpha)$, and $\lambda(x,y)$}
\subsubsection{9.65\quad Euler polynomials}

\subsection{9.7\quad Constants}
\subsubsection{9.71\quad Bernoulli numbers}
\subsubsection{9.72\quad Euler numbers}
\subsubsection{9.73\quad Euler's and Catalan's constants}
\subsubsection{9.74\quad Stirling numbers}

%% ============================================================
%% 10  Vector Field Theory
%% ============================================================
\section{10\quad Vector Field Theory}

\subsection{10.1--10.8\quad Vectors, Vector Operators, and Integral Theorems}
\subsubsection{10.11\quad Products of vectors}
\subsubsection{10.12\quad Properties of scalar product}
\subsubsection{10.13\quad Properties of vector product}
\subsubsection{10.14\quad Differentiation of vectors}
\subsubsection{10.21\quad Operators grad, div, and curl}
\subsubsection{10.31\quad Properties of the operator $\nabla$}
\subsubsection{10.41\quad Solenoidal fields}
\subsubsection{10.51--10.61\quad Orthogonal curvilinear coordinates}
\subsubsection{10.71--10.72\quad Vector integral theorems}
\subsubsection{10.81\quad Integral rate of change theorems}

%% ============================================================
%% 11  Algebraic Inequalities
%% ============================================================
\section{11\quad Algebraic Inequalities}

\subsection{11.1--11.3\quad General Algebraic Inequalities}
\subsubsection{11.11\quad Algebraic inequalities involving real numbers}
\subsubsection{11.21\quad Algebraic inequalities involving complex numbers}
\subsubsection{11.31\quad Inequalities for sets of complex numbers}

%% ============================================================
%% 12  Integral Inequalities
%% ============================================================
\section{12\quad Integral Inequalities}

\subsection{12.11\quad Mean Value Theorems}
\subsubsection{12.111\quad First mean value theorem}
\subsubsection{12.112\quad Second mean value theorem}
\subsubsection{12.113\quad First mean value theorem for infinite integrals}
\subsubsection{12.114\quad Second mean value theorem for infinite integrals}

\subsection{12.21\quad Differentiation of Definite Integral Containing a Parameter}
\subsubsection{12.211\quad Differentiation when limits are finite}
\subsubsection{12.212\quad Differentiation when a limit is infinite}

\subsection{12.31\quad Integral Inequalities}
\subsubsection{12.311\quad Cauchy-Schwarz-Buniakowsky inequality for integrals}
\subsubsection{12.312\quad H\"older's inequality for integrals}
\subsubsection{12.313\quad Minkowski's inequality for integrals}
\subsubsection{12.314\quad Chebyshev's inequality for integrals}
\subsubsection{12.315\quad Young's inequality for integrals}
\subsubsection{12.316\quad Steffensen's inequality for integrals}
\subsubsection{12.317\quad Gram's inequality for integrals}
\subsubsection{12.318\quad Ostrowski's inequality for integrals}

\subsection{12.41\quad Convexity and Jensen's Inequality}
\subsubsection{12.411\quad Jensen's inequality}
\subsubsection{12.412\quad Carleman's inequality for integrals}

\subsection{12.51\quad Fourier Series and Related Inequalities}
\subsubsection{12.511\quad Riemann-Lebesgue lemma}
\subsubsection{12.512\quad Dirichlet lemma}
\subsubsection{12.513\quad Parseval's theorem for trigonometric Fourier series}
\subsubsection{12.514\quad Integral representation of the $n^{\text{th}}$ partial sum}
\subsubsection{12.515\quad Generalized Fourier series}
\subsubsection{12.516\quad Bessel's inequality for generalized Fourier series}
\subsubsection{12.517\quad Parseval's theorem for generalized Fourier series}

%% ============================================================
%% 13  Matrices and Related Results
%% ============================================================
\section{13\quad Matrices and Related Results}

\subsection{13.11--13.12\quad Special Matrices}
\subsubsection{13.111\quad Diagonal matrix}
\subsubsection{13.112\quad Identity matrix and null matrix}
\subsubsection{13.113\quad Reducible and irreducible matrices}
\subsubsection{13.114\quad Equivalent matrices}
\subsubsection{13.115\quad Transpose of a matrix}
\subsubsection{13.116\quad Adjoint matrix}
\subsubsection{13.117\quad Inverse matrix}
\subsubsection{13.118\quad Trace of a matrix}
\subsubsection{13.119\quad Symmetric matrix}
\subsubsection{13.120\quad Skew-symmetric matrix}
\subsubsection{13.121\quad Triangular matrices}
\subsubsection{13.122\quad Orthogonal matrices}
\subsubsection{13.123\quad Hermitian transpose of a matrix}
\subsubsection{13.124\quad Hermitian matrix}
\subsubsection{13.125\quad Unitary matrix}
\subsubsection{13.126\quad Eigenvalues and eigenvectors}
\subsubsection{13.127\quad Nilpotent matrix}
\subsubsection{13.128\quad Idempotent matrix}
\subsubsection{13.129\quad Positive definite}
\subsubsection{13.130\quad Non-negative definite}
\subsubsection{13.131\quad Diagonally dominant}

\subsection{13.21\quad Quadratic Forms}
\subsubsection{13.211\quad Sylvester's law of inertia}
\subsubsection{13.212\quad Rank}
\subsubsection{13.213\quad Signature}
\subsubsection{13.214\quad Positive definite and semidefinite quadratic form}
\subsubsection{13.215\quad Basic theorems on quadratic forms}

\subsection{13.31\quad Differentiation of Matrices}

\subsection{13.41\quad The Matrix Exponential}
\subsubsection{13.411\quad Basic properties}

%% ============================================================
%% 14  Determinants
%% ============================================================
\section{14\quad Determinants}

\subsection{14.11\quad Expansion of Second- and Third-Order Determinants}
\subsection{14.12\quad Basic Properties}
\subsection{14.13\quad Minors and Cofactors of a Determinant}
\subsection{14.14\quad Principal Minors}
\subsection{14.15\textsuperscript{*}\quad Laplace Expansion of a Determinant}
\subsection{14.16\quad Jacobi's Theorem}
\subsection{14.17\quad Hadamard's Theorem}
\subsection{14.18\quad Hadamard's Inequality}
\subsection{14.21\quad Cramer's Rule}

\subsection{14.31\quad Some Special Determinants}
\subsubsection{14.311\quad Vandermonde's determinant (alternant)}
\subsubsection{14.312\quad Circulants}
\subsubsection{14.313\quad Jacobian determinant}
\subsubsection{14.314\quad Hessian determinants}
\subsubsection{14.315\quad Wronskian determinants}
\subsubsection{14.316\quad Properties}
\subsubsection{14.317\quad Gram-Kowalewski theorem on linear dependence}

%% ============================================================
%% 15  Norms
%% ============================================================
\section{15\quad Norms}

\subsection{15.1--15.9\quad Vector Norms}
\subsubsection{15.11\quad General Properties}
\subsubsection{15.21\quad Principal Vector Norms}
\subsubsection{15.211\quad The norm $\|\mathbf{x}\|_{1}$}
\subsubsection{15.212\quad The norm $\|\mathbf{x}\|_{2}$ (Euclidean or $L_{2}$ norm)}
\subsubsection{15.213\quad The norm $\|\mathbf{x}\|_{\infty}$}
\subsubsection{15.31\quad Matrix Norms}
\subsubsection{15.311\quad General properties}
\subsubsection{15.312\quad Induced norms}
\subsubsection{15.313\quad Natural norm of unit matrix}
\subsubsection{15.41\quad Principal Natural Norms}
\subsubsection{15.411\quad Maximum absolute column sum norm}
\subsubsection{15.412\quad Spectral norm}
\subsubsection{15.413\quad Maximum absolute row sum norm}
\subsubsection{15.51\quad Spectral Radius of a Square Matrix}
\subsubsection{15.511\quad Inequalities concerning matrix norms and the spectral radius}
\subsubsection{15.512\quad Deductions from Gerschgorin's theorem (see 15.814)}
\subsubsection{15.61\quad Inequalities Involving Eigenvalues of Matrices}
\subsubsection{15.611\quad Cayley-Hamilton theorem}
\subsubsection{15.612\quad Corollaries}
\subsubsection{15.71\quad Inequalities for the Characteristic Polynomial}
\subsubsection{15.711\quad Named and unnamed inequalities}
\subsubsection{15.712\quad Parodi's theorem}
\subsubsection{15.713\quad Corollary of Brauer's theorem}
\subsubsection{15.714\quad Ballieu's theorem}
\subsubsection{15.715\quad Routh-Hurwitz theorem}

\subsection{15.81--15.82\quad Named Theorems on Eigenvalues}
\subsubsection{15.811\quad Schur's inequalities}
\subsubsection{15.812\quad Sturmian separation theorem}
\subsubsection{15.813\quad Poincar\'e's separation theorem}
\subsubsection{15.814\quad Gerschgorin's theorem}
\subsubsection{15.815\quad Brauer's theorem}
\subsubsection{15.816\quad Perron's theorem}
\subsubsection{15.817\quad Frobenius theorem}
\subsubsection{15.818\quad Perron--Frobenius theorem}
\subsubsection{15.819\quad Wielandt's theorem}
\subsubsection{15.820\quad Ostrowski's theorem}
\subsubsection{15.821\quad First theorem due to Lyapunov}
\subsubsection{15.822\quad Second theorem due to Lyapunov}

\subsection{15.823\quad Hermitian matrices and diophantine relations involving circular functions of rational angles due to Calogero and Perelomov}

\subsection{15.91\quad Variational Principles}
\subsubsection{15.911\quad Rayleigh quotient}
\subsubsection{15.912\quad Basic theorems}

%% ============================================================
%% 16  Ordinary Differential Equations
%% ============================================================
\section{16\quad Ordinary Differential Equations}

\subsection{16.1--16.9\quad Results Relating to the Solution of Ordinary Differential Equations}

\subsubsection{16.11\quad First-Order Equations}
\subsubsection{16.111\quad Solution of a first-order equation}
\subsubsection{16.112\quad Cauchy problem}
\subsubsection{16.113\quad Approximate solution to an equation}
\subsubsection{16.114\quad Lipschitz continuity of a function}

\subsubsection{16.21\quad Fundamental Inequalities and Related Results}
\subsubsection{16.211\quad Gronwall's lemma}
\subsubsection{16.212\quad Comparison of approximate solutions of a differential equation}

\subsubsection{16.31\quad First-Order Systems}
\subsubsection{16.311\quad Solution of a system of equations}
\subsubsection{16.312\quad Cauchy problem for a system}
\subsubsection{16.313\quad Approximate solution to a system}
\subsubsection{16.314\quad Lipschitz continuity of a vector}
\subsubsection{16.315\quad Comparison of approximate solutions of a system}
\subsubsection{16.316\quad First-order linear differential equation}
\subsubsection{16.317\quad Linear systems of differential equations}

\subsubsection{16.41\quad Some Special Types of Elementary Differential Equations}
\subsubsection{16.411\quad Variables separable}
\subsubsection{16.412\quad Exact differential equations}
\subsubsection{16.413\quad Conditions for an exact equation}
\subsubsection{16.414\quad Homogeneous differential equations}

\subsubsection{16.51\quad Second-Order Equations}
\subsubsection{16.511\quad Adjoint and self-adjoint equations}
\subsubsection{16.512\quad Abel's identity}
\subsubsection{16.513\quad Lagrange identity}
\subsubsection{16.514\quad The Riccati equation}
\subsubsection{16.515\quad Solutions of the Riccati equation}
\subsubsection{16.516\quad Solution of a second-order linear differential equation}

\subsubsection{16.61--16.62\quad Oscillation and Non-Oscillation Theorems for Second-Order Equations}
\subsubsection{16.611\quad First basic comparison theorem}
\subsubsection{16.622\quad Second basic comparison theorem}
\subsubsection{16.623\quad Interlacing of zeros}
\subsubsection{16.624\quad Sturm separation theorem}
\subsubsection{16.625\quad Sturm comparison theorem}
\subsubsection{16.626\quad Szeg\"o's comparison theorem}
\subsubsection{16.627\quad Picone's identity}
\subsubsection{16.628\quad Sturm-Picone theorem}
\subsubsection{16.629\quad Oscillation on the half line}

\subsubsection{16.71\quad Two Related Comparison Theorems}
\subsubsection{16.711\quad Theorem 1}
\subsubsection{16.712\quad Theorem 2}

\subsubsection{16.81--16.82\quad Non-Oscillatory Solutions}
\subsubsection{16.811\quad Kneser's non-oscillation theorem}
\subsubsection{16.822\quad Comparison theorem for non-oscillation}
\subsubsection{16.823\quad Necessary and sufficient conditions for non-oscillation}

\subsubsection{16.91\quad Some Growth Estimates for Solutions of Second-Order Equations}
\subsubsection{16.911\quad Strictly increasing and decreasing solutions}
\subsubsection{16.912\quad General result on dominant and subdominant solutions}
\subsubsection{16.913\quad Estimate of dominant solution}
\subsubsection{16.914\quad A theorem due to Lyapunov}

\subsubsection{16.92\quad Boundedness Theorems}
\subsubsection{16.921\quad All solutions of the equation}
\subsubsection{16.922\quad If all solutions of the equation}
\subsubsection{16.923\quad If $a(x)\to \infty$ monotonically as $x\to \infty$, then all solutions of}
\subsubsection{16.924\quad Consider the equation}

\subsubsection{16.93\quad Growth of maxima of $|y|$}

%% ============================================================
%% 17  Fourier, Laplace, and Mellin Transforms
%% ============================================================
\section{17\quad Fourier, Laplace, and Mellin Transforms}

\subsection{17.1--17.4\quad Integral Transforms}
\subsubsection{17.11\quad Laplace transform}
\subsubsection{17.12\quad Basic properties of the Laplace transform}
\subsubsection{17.13\quad Table of Laplace transform pairs}
\subsubsection{17.21\quad Fourier transform}
\subsubsection{17.22\quad Basic properties of the Fourier transform}
\subsubsection{17.23\quad Table of Fourier transform pairs}
\subsubsection{17.24\quad Table of Fourier transform pairs for spherically symmetric functions}
\subsubsection{17.31\quad Fourier sine and cosine transforms}
\subsubsection{17.32\quad Basic properties of the Fourier sine and cosine transforms}
\subsubsection{17.33\quad Table of Fourier sine transforms}
\subsubsection{17.34\quad Table of Fourier cosine transforms}
\subsubsection{17.35\quad Relationships between transforms}
\subsubsection{17.41\quad Mellin transform}
\subsubsection{17.42\quad Basic properties of the Mellin transform}
\subsubsection{17.43\quad Table of Mellin transforms}

%% ============================================================
%% 18  The z-Transform
%% ============================================================
\section{18\quad The $z$-Transform}

\subsection{18.1--18.3\quad Definition, Bilateral, and Unilateral $z$-Transforms}
\subsubsection{18.1\quad Definitions}
\subsubsection{18.2\quad Bilateral $z$-transform}
\subsubsection{18.3\quad Unilateral $z$-transform}

\end{document}
