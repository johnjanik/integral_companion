%% Section 14 — Determinants
\section{14\quad Determinants}

\subsection{14.11\quad Expansion of Second- and Third-Order Determinants}

%% -------------------------------------------------------------------
\subsubsection{14.111\quad Second-order determinants}

The determinant of a $2\times 2$ matrix $A=\bigl(\begin{smallmatrix}a&b\\c&d\end{smallmatrix}\bigr)$ is $\det A=ad-bc$.
This is the signed area of the parallelogram spanned by the column vectors and is the simplest instance of the alternating multilinear form that defines all determinants.

\paragraph{Physics applications.}
\begin{enumerate}
\item \textbf{Torque and cross products in two dimensions.}%
  \index{torque!two-dimensional}%
  \index{cross product!two-dimensional determinant}%
  \index{angular momentum!two-dimensional}%
  The two-dimensional cross product $\mathbf{u}\times\mathbf{v}=u_{1}v_{2}-u_{2}v_{1}=\det(u_{i},v_{j})$ gives the signed area of the parallelogram spanned by $\mathbf{u}$ and $\mathbf{v}$.
  In planar mechanics, the torque about the origin due to a force $\mathbf{F}$ applied at position $\mathbf{r}$ is $\tau=\det\bigl(\begin{smallmatrix}r_{1}&F_{1}\\r_{2}&F_{2}\end{smallmatrix}\bigr)$, and its sign determines the sense of rotation.
  This connection between determinants and oriented areas pervades classical and quantum angular momentum theory.

\item \textbf{Jones matrices in polarisation optics.}%
  \index{Jones matrix}%
  \index{polarisation optics!Jones calculus}%
  \index{optical elements!determinant}%
  \index{birefringence|see{Jones matrix}}%
  A Jones matrix $J\in\mathrm{GL}(2,\mathbb{C})$ describes the transformation of the polarisation state of coherent light by an optical element.
  For lossless elements $|\det J|=1$, while $|\det J|<1$ indicates absorption.
  The determinant condition $\det J=e^{i\phi}$ characterises unitary (phase-only) elements such as wave plates.
  Cascading two elements gives $\det(J_{1}J_{2})=\det J_{1}\det J_{2}$, so the total loss is the product of individual losses.

\item \textbf{Stability of two-dimensional dynamical systems.}%
  \index{dynamical system!two-dimensional stability}%
  \index{trace-determinant plane}%
  \index{fixed point classification}%
  For the linear system $\dot{\mathbf{x}}=A\mathbf{x}$ with $A$ a $2\times 2$ real matrix, the eigenvalues are $\lambda_{\pm}=\frac{1}{2}(\mathrm{tr}\,A\pm\sqrt{(\mathrm{tr}\,A)^{2}-4\det A})$.
  The trace--determinant plane classifies fixed points: stable nodes ($\det A>0$, $\mathrm{tr}\,A<0$), saddles ($\det A<0$), and spirals ($(\mathrm{tr}\,A)^{2}<4\det A$).
  This classification is fundamental in nonlinear dynamics near equilibria.
\end{enumerate}

\paragraph{Mathematics applications.}
\begin{enumerate}
\item \textbf{Area of triangles and orientation.}%
  \index{triangle!signed area}%
  \index{orientation!determinant}%
  \index{computational geometry!orientation test}%
  The signed area of a triangle with vertices $(x_{1},y_{1})$, $(x_{2},y_{2})$, $(x_{3},y_{3})$ is
  $\tfrac{1}{2}\det\bigl(\begin{smallmatrix}x_{2}-x_{1}&x_{3}-x_{1}\\y_{2}-y_{1}&y_{3}-y_{1}\end{smallmatrix}\bigr)$.
  The sign determines the orientation (counterclockwise vs.\ clockwise), and the test $\det\gtrless 0$ is the fundamental orientation predicate in computational geometry, used in convex hull algorithms and Delaunay triangulation.

\item \textbf{M\"obius transformations and $\mathrm{PSL}(2,\mathbb{C})$.}%
  \index{M\"obius transformation}%
  \index{PSL(2,C)@$\mathrm{PSL}(2,\mathbb{C})$}%
  \index{fractional linear transformation|see{M\"obius transformation}}%
  The M\"obius transformation $z\mapsto(az+b)/(cz+d)$ is determined by $\bigl(\begin{smallmatrix}a&b\\c&d\end{smallmatrix}\bigr)$ up to scaling.
  The group of such transformations is $\mathrm{PSL}(2,\mathbb{C})$, the quotient of $2\times 2$ matrices of determinant~$1$ by $\{\pm I\}$.
  The requirement $ad-bc=1$ ensures invertibility and connects the determinant to conformal geometry of the Riemann sphere.

\item \textbf{Quadratic forms and conic classification.}%
  \index{conic sections!classification}%
  \index{quadratic form!discriminant}%
  \index{discriminant!conic}%
  The general conic $ax^{2}+2bxy+cy^{2}+dx+ey+f=0$ is classified by $\Delta=\det\bigl(\begin{smallmatrix}a&b\\b&c\end{smallmatrix}\bigr)=ac-b^{2}$: an ellipse when $\Delta>0$, a hyperbola when $\Delta<0$, and a parabola when $\Delta=0$.
  This discriminant is the simplest invariant of a quadratic form under rotation and is the starting point for the classification of quadrics in higher dimensions.
\end{enumerate}

%% -------------------------------------------------------------------
\subsubsection{14.112\quad Third-order determinants}

The determinant of a $3\times 3$ matrix can be expanded by the rule of Sarrus or by cofactor expansion along any row or column.
For $A=(a_{ij})$, the explicit formula is
$\det A=a_{11}(a_{22}a_{33}-a_{23}a_{32})-a_{12}(a_{21}a_{33}-a_{23}a_{31})+a_{13}(a_{21}a_{32}-a_{22}a_{31})$.

\paragraph{Physics applications.}
\begin{enumerate}
\item \textbf{Triple scalar product and volume.}%
  \index{triple scalar product}%
  \index{volume!parallelepiped}%
  \index{crystallography!unit cell volume}%
  The volume of the parallelepiped spanned by vectors $\mathbf{a}$, $\mathbf{b}$, $\mathbf{c}$ is $V=|\mathbf{a}\cdot(\mathbf{b}\times\mathbf{c})|=|\det(\mathbf{a},\mathbf{b},\mathbf{c})|$.
  In crystallography, the unit cell volume is $V=|\mathbf{a}_{1}\cdot(\mathbf{a}_{2}\times\mathbf{a}_{3})|$ where $\mathbf{a}_{i}$ are the lattice vectors.
  The reciprocal lattice vectors are $\mathbf{b}_{i}=\epsilon_{ijk}\mathbf{a}_{j}\times\mathbf{a}_{k}/V$, with each component involving a $2\times 2$ subdeterminant.

\item \textbf{Levi-Civita symbol and pseudotensors.}%
  \index{Levi-Civita symbol}%
  \index{pseudotensor}%
  \index{determinant!Levi-Civita relation}%
  \index{cross product!Levi-Civita}%
  The Levi-Civita symbol $\epsilon_{ijk}$ is the determinant of the $3\times 3$ matrix whose columns are $\mathbf{e}_{i}$, $\mathbf{e}_{j}$, $\mathbf{e}_{k}$.
  The identity $\epsilon_{ijk}\epsilon_{ilm}=\delta_{jl}\delta_{km}-\delta_{jm}\delta_{kl}$ reduces products of cross products to dot products, and the determinant of a $3\times 3$ matrix is $\det A=\epsilon_{ijk}a_{1i}a_{2j}a_{3k}$.
  This connection makes the Levi-Civita symbol the algebraic engine of three-dimensional vector analysis.

\item \textbf{Moment of inertia tensor eigenvalues.}%
  \index{moment of inertia tensor}%
  \index{principal axes!rigid body}%
  \index{characteristic polynomial!3x3}%
  The principal moments of inertia of a rigid body are the roots of the characteristic polynomial $\det(I-\lambda\mathbf{1})=0$, a cubic equation.
  The coefficients of this cubic are the three elementary symmetric functions of the eigenvalues: $\mathrm{tr}\,I$, the sum of $2\times 2$ principal minors, and $\det I$.
  Cardano's formula or trigonometric solution of the depressed cubic then yields the principal moments explicitly.
\end{enumerate}

\paragraph{Mathematics applications.}
\begin{enumerate}
\item \textbf{Cramer's rule for $3\times 3$ systems.}%
  \index{Cramer's rule!3x3}%
  \index{linear system!small}%
  \index{explicit solution formula}%
  For the system $A\mathbf{x}=\mathbf{b}$ with $A$ a $3\times 3$ matrix, Cramer's rule gives $x_{i}=\det A_{i}/\det A$, where $A_{i}$ is $A$ with column~$i$ replaced by $\mathbf{b}$.
  While numerically inferior to Gaussian elimination for large systems, the explicit formula is invaluable for symbolic computation and for proving existence and uniqueness when $\det A\neq 0$.

\item \textbf{Cayley--Hamilton theorem for $3\times 3$ matrices.}%
  \index{Cayley--Hamilton theorem!3x3}%
  \index{characteristic polynomial!3x3}%
  \index{matrix polynomial}%
  Every $3\times 3$ matrix satisfies its own characteristic equation $A^{3}-(\mathrm{tr}\,A)A^{2}+\tfrac{1}{2}[(\mathrm{tr}\,A)^{2}-\mathrm{tr}(A^{2})]A-(\det A)I=0$.
  This allows any polynomial in $A$ to be reduced to degree at most~$2$ and provides the matrix inverse $A^{-1}=\frac{1}{\det A}[A^{2}-(\mathrm{tr}\,A)A+\tfrac{1}{2}((\mathrm{tr}\,A)^{2}-\mathrm{tr}(A^{2}))I]$ when $\det A\neq 0$.
  The Cayley--Hamilton identity is the finite-dimensional prototype of functional calculus.

\item \textbf{The vector triple product identity.}%
  \index{triple product!vector}%
  \index{BAC-CAB rule}%
  \index{Grassmann identity|see{BAC-CAB rule}}%
  The identity $\mathbf{a}\times(\mathbf{b}\times\mathbf{c})=\mathbf{b}(\mathbf{a}\cdot\mathbf{c})-\mathbf{c}(\mathbf{a}\cdot\mathbf{b})$ (the BAC-CAB rule) is proved by expanding both sides using determinants.
  It is equivalent to the contraction identity for the Levi-Civita symbol and is used extensively in electromagnetic theory when simplifying expressions involving curls of curls, such as $\nabla\times(\nabla\times\mathbf{E})$.
\end{enumerate}


\subsection{14.12\quad Basic Properties}

%% -------------------------------------------------------------------
\subsubsection{14.121\quad Multilinearity and alternating property}

The determinant is the unique (up to normalisation) alternating multilinear function of the columns (or rows) of a matrix: it is linear in each column separately, changes sign when two columns are swapped, and satisfies $\det I=1$.
These three axioms suffice to derive every other property.

\paragraph{Physics applications.}
\begin{enumerate}
\item \textbf{Slater determinants and the Pauli exclusion principle.}%
  \index{Slater determinant}%
  \index{Pauli exclusion principle}%
  \index{fermion!antisymmetry}%
  \index{Hartree--Fock method|see{Slater determinant}}%
  The antisymmetric many-body wavefunction for $N$ fermions in orbitals $\phi_{1},\ldots,\phi_{N}$ is the Slater determinant
  \[
    \Psi(x_{1},\ldots,x_{N})=\frac{1}{\sqrt{N!}}
    \det\bigl[\phi_{i}(x_{j})\bigr]_{i,j=1}^{N}.
  \]
  The alternating property ensures $\Psi=0$ when any two particles occupy the same state (Pauli exclusion).
  The Hartree--Fock method approximates the ground state of an $N$-electron system by the single Slater determinant that minimises the energy functional.
  Configuration interaction and coupled-cluster methods systematically improve upon this by including linear combinations of multiple Slater determinants.

\item \textbf{Flux quantisation in superconductors.}%
  \index{flux quantisation}%
  \index{superconductor!order parameter}%
  \index{Ginzburg--Landau theory}%
  In the Ginzburg--Landau theory, the superconducting order parameter transforms under gauge transformations, and the requirement that the many-electron wavefunction (a Slater-like determinant) be single-valued leads to flux quantisation $\Phi=n\Phi_{0}$, where $\Phi_{0}=h/(2e)$ is the magnetic flux quantum.
  The alternating property of the determinant is essential for maintaining the correct fermionic statistics.

\item \textbf{Exterior algebra and differential forms.}%
  \index{exterior algebra}%
  \index{differential forms!wedge product}%
  \index{wedge product!determinant}%
  The wedge product $\omega^{1}\wedge\cdots\wedge\omega^{n}$ of $n$ one-forms evaluates on $n$ vectors to give a determinant.
  The alternating property of determinants becomes the antisymmetry of differential forms, and the multilinearity becomes the tensorial character.
  This is the mathematical language of electromagnetism (Faraday two-form $F=dA$), thermodynamics (contact forms), and general relativity (volume forms).
\end{enumerate}

\paragraph{Mathematics applications.}
\begin{enumerate}
\item \textbf{Orientation of manifolds.}%
  \index{orientation!manifold}%
  \index{volume form!orientation}%
  \index{orientability!determinant}%
  A smooth manifold $M$ is orientable if and only if it admits an atlas whose transition functions all have positive-determinant Jacobians.
  The determinant's alternating property means that reversing the order of two basis vectors changes the sign of the volume form, capturing the notion of ``handedness.''
  Non-orientable manifolds such as the M\"obius band and Klein bottle fail this condition.

\item \textbf{Leibniz formula and the symmetric group.}%
  \index{Leibniz formula!determinant}%
  \index{symmetric group!sign of permutation}%
  \index{permutation!signature}%
  The Leibniz formula $\det A=\sum_{\sigma\in S_{n}}\mathrm{sgn}(\sigma)\prod_{i=1}^{n}a_{i,\sigma(i)}$ expresses the determinant as a sum over all $n!$ permutations, weighted by their signs.
  This formula connects determinants to the representation theory of the symmetric group $S_{n}$ and shows that $\det$ is the character of the one-dimensional sign representation.
  It also provides the starting point for the combinatorial theory of determinants (path matrices, Lindstr\"om--Gessel--Viennot lemma).

\item \textbf{Characterisation by axioms.}%
  \index{determinant!axiomatic characterisation}%
  \index{alternating multilinear form}%
  \index{uniqueness of determinant}%
  The determinant is the unique alternating multilinear function $\det\colon(\mathbb{R}^{n})^{n}\to\mathbb{R}$ with $\det(e_{1},\ldots,e_{n})=1$.
  This axiomatic approach, due to Weierstrass, provides a coordinate-free definition and extends to determinants over commutative rings, where $\det$ is the unique natural transformation $\bigwedge^{n}\to\mathbf{1}$ from the $n$-th exterior power functor to the identity.
\end{enumerate}

%% -------------------------------------------------------------------
\subsubsection{14.122\quad Multiplicativity}

The product rule $\det(AB)=\det A\cdot\det B$ is the most computationally powerful property of determinants, reducing the determinant of a product to a product of determinants.

\paragraph{Physics applications.}
\begin{enumerate}
\item \textbf{Liouville's theorem and phase space volume.}%
  \index{Liouville's theorem!phase space}%
  \index{phase space!volume preservation}%
  \index{Hamiltonian mechanics!symplecticity}%
  \index{symplectic matrix!determinant}%
  Hamiltonian time evolution is a canonical (symplectic) transformation with $\det(\partial(q',p')/\partial(q,p))=1$.
  By multiplicativity, composing time steps preserves this unit determinant, so phase space volume is conserved (Liouville's theorem).
  This is the classical foundation of the ergodic hypothesis and of statistical mechanics.

\item \textbf{Renormalisation group and functional determinants.}%
  \index{renormalisation group}%
  \index{functional determinant}%
  \index{path integral!Gaussian}%
  In quantum field theory, one-loop contributions are Gaussian functional integrals yielding $(\det\mathcal{O})^{-1/2}$ for bosonic operators.
  Under a renormalisation group step that decomposes $\mathcal{O}=\mathcal{O}_{<}\mathcal{O}_{>}$, multiplicativity gives $\det\mathcal{O}=\det\mathcal{O}_{<}\det\mathcal{O}_{>}$, separating high- and low-energy contributions.
  The anomalous Jacobian $\det(\partial\phi'/\partial\phi)$ under field redefinitions produces the chiral anomaly.

\item \textbf{Transfer matrices in statistical mechanics.}%
  \index{transfer matrix!statistical mechanics}%
  \index{partition function!transfer matrix}%
  \index{Ising model!transfer matrix}%
  The partition function of the one-dimensional Ising model is $Z=\mathrm{tr}\,T^{N}$, where $T$ is the transfer matrix.
  The free energy per site in the thermodynamic limit is $f=-k_{B}T\ln\lambda_{\max}$, where $\lambda_{\max}$ is the largest eigenvalue, and $\det T=\lambda_{1}\lambda_{2}$ gives the product of eigenvalues.
  Multiplicativity $\det(T^{N})=(\det T)^{N}$ ensures consistent normalisation.
\end{enumerate}

\paragraph{Mathematics applications.}
\begin{enumerate}
\item \textbf{The group $\mathrm{GL}(n)$ and its subgroups.}%
  \index{general linear group}%
  \index{special linear group}%
  \index{determinant!group homomorphism}%
  The map $\det\colon\mathrm{GL}(n,\mathbb{F})\to\mathbb{F}^{\times}$ is a group homomorphism by multiplicativity, with kernel $\mathrm{SL}(n,\mathbb{F})$.
  The first isomorphism theorem gives $\mathrm{GL}(n)/\mathrm{SL}(n)\cong\mathbb{F}^{\times}$.
  This short exact sequence is the starting point for the theory of algebraic $K$-theory ($K_{1}$ of a ring is the abelianisation of its general linear group).

\item \textbf{Resultants and elimination theory.}%
  \index{resultant!determinant}%
  \index{elimination theory}%
  \index{Sylvester matrix}%
  The resultant $\mathrm{Res}(f,g)$ of two polynomials $f$ and $g$ is the determinant of the Sylvester matrix.
  By multiplicativity, $\mathrm{Res}(fg,h)=\mathrm{Res}(f,h)\mathrm{Res}(g,h)$, which is the key to proving that the resultant vanishes if and only if $f$ and $g$ share a common root.
  This connects determinant theory to algebraic geometry (intersection multiplicity).

\item \textbf{Determinant of block matrices.}%
  \index{block matrix!determinant}%
  \index{Schur complement}%
  \index{block triangular factorisation}%
  For a block matrix $M=\bigl(\begin{smallmatrix}A&B\\C&D\end{smallmatrix}\bigr)$ with $A$ invertible, $\det M=\det A\cdot\det(D-CA^{-1}B)$, where $D-CA^{-1}B$ is the Schur complement.
  This follows from multiplicativity applied to the block LU factorisation.
  The Schur complement formula is ubiquitous in statistics (conditional covariance), control theory (transfer functions), and numerical linear algebra (domain decomposition).
\end{enumerate}

\subsection{14.13\quad Minors and Cofactors of a Determinant}

%% -------------------------------------------------------------------
\subsubsection{14.131\quad Minors, cofactors, and cofactor expansion}

The $(i,j)$-minor $M_{ij}$ of an $n\times n$ matrix $A$ is the determinant of the $(n-1)\times(n-1)$ submatrix obtained by deleting row~$i$ and column~$j$.
The cofactor is $C_{ij}=(-1)^{i+j}M_{ij}$, and the determinant admits the cofactor (Laplace) expansion $\det A=\sum_{j=1}^{n}a_{ij}C_{ij}$ along any row~$i$.

\paragraph{Physics applications.}
\begin{enumerate}
\item \textbf{Green's functions via matrix inversion.}%
  \index{Green's function!matrix inversion}%
  \index{cofactor!inverse matrix}%
  \index{lattice models!Green's function}%
  For a tight-binding Hamiltonian $H$ on a lattice, the retarded Green's function is $G(E)=(EI-H)^{-1}$, with matrix elements $(G)_{ij}=C_{ji}/\det(EI-H)$.
  The cofactors $C_{ji}$ encode the amplitude for propagation from site~$j$ to site~$i$, and the poles of $G(E)$ are the eigenvalues of $H$.
  Recursive evaluation of minors (decimation) is the basis of the Green's function method for quasi-one-dimensional systems.

\item \textbf{Kirchhoff's matrix tree theorem.}%
  \index{Kirchhoff's theorem!matrix tree}%
  \index{spanning tree!determinant}%
  \index{electrical network!resistance}%
  \index{graph Laplacian!cofactor}%
  For an electrical network with graph Laplacian $L$, the number of spanning trees is any cofactor $C_{ii}$ of $L$ (all cofactors are equal since every row and column of $L$ sums to zero).
  The effective resistance between nodes $i$ and $j$ is $R_{ij}=C_{ij}^{(2)}/C_{11}$, where $C_{ij}^{(2)}$ involves a $2\times 2$ minor.
  This elegant connection between combinatorics and circuit theory was discovered by Kirchhoff in 1847.

\item \textbf{Sensitivity analysis in control systems.}%
  \index{control theory!sensitivity}%
  \index{cofactor!transfer function}%
  \index{signal flow graph}%
  In Mason's gain formula for signal flow graphs, the transfer function from input to output is $T=\sum_{k}P_{k}\Delta_{k}/\Delta$, where $\Delta=\det(I-A)$ is the graph determinant and $\Delta_{k}$ is the cofactor obtained by removing loops that touch path~$k$.
  Each cofactor quantifies the contribution of non-touching feedback loops, providing a systematic sensitivity analysis of the control system.
\end{enumerate}

\paragraph{Mathematics applications.}
\begin{enumerate}
\item \textbf{Adjugate matrix and the inverse.}%
  \index{adjugate matrix}%
  \index{inverse matrix!cofactor formula}%
  \index{classical adjoint|see{adjugate matrix}}%
  The adjugate (classical adjoint) of $A$ is $\mathrm{adj}(A)=(C_{ji})$, the transpose of the cofactor matrix.
  The identity $A\,\mathrm{adj}(A)=(\det A)\,I$ gives the inverse $A^{-1}=\mathrm{adj}(A)/\det A$ and is valid over any commutative ring, making it the basis for computing inverses symbolically and for proving that $A$ is invertible if and only if $\det A$ is a unit.

\item \textbf{Jacobi's formula for the derivative of a determinant.}%
  \index{Jacobi's formula}%
  \index{determinant!derivative}%
  \index{matrix calculus!determinant derivative}%
  Jacobi's formula $\frac{d}{dt}\det A(t)=\mathrm{tr}\bigl(\mathrm{adj}(A)\,\dot{A}\bigr)$ expresses the derivative of a determinant in terms of cofactors.
  When $A$ is invertible, this simplifies to $\frac{d}{dt}\det A=(\det A)\,\mathrm{tr}(A^{-1}\dot{A})$, which is fundamental in Riemannian geometry ($\frac{d}{dt}\sqrt{\det g}=\frac{1}{2}\sqrt{\det g}\,g^{ij}\dot{g}_{ij}$) and in the study of matrix differential equations.

\item \textbf{Lindstr\"om--Gessel--Viennot lemma.}%
  \index{Lindstr\"om--Gessel--Viennot lemma}%
  \index{non-intersecting lattice paths}%
  \index{combinatorics!determinantal}%
  The number of families of $n$ non-intersecting lattice paths from sources $\{s_{i}\}$ to sinks $\{t_{j}\}$ is $\det\bigl[e(s_{i},t_{j})\bigr]$, where $e(s,t)$ is the number of paths from $s$ to $t$.
  This lemma reduces a combinatorial counting problem to a determinant evaluation and is the key tool in the enumeration of plane partitions, Young tableaux, and tilings.
  The alternating sign in the cofactor expansion implements the inclusion-exclusion over path crossings.
\end{enumerate}

\subsection{14.14\quad Principal Minors}

%% -------------------------------------------------------------------
\subsubsection{14.141\quad Principal minors and positive definiteness}

A principal minor of an $n\times n$ matrix $A$ is the determinant of a submatrix obtained by deleting the same set of rows and columns.
The $k$-th leading principal minor is $\Delta_{k}=\det(a_{ij})_{1\leq i,j\leq k}$.
Sylvester's criterion states that a symmetric matrix is positive definite if and only if all leading principal minors are positive: $\Delta_{k}>0$ for $k=1,\ldots,n$.

\paragraph{Physics applications.}
\begin{enumerate}
\item \textbf{Thermodynamic stability conditions.}%
  \index{thermodynamic stability}%
  \index{Hessian!free energy}%
  \index{Le Chatelier's principle}%
  \index{principal minor!stability}%
  The conditions for local thermodynamic stability require that the Hessian matrix of the entropy (or free energy) with respect to extensive (or intensive) variables be negative (or positive) definite.
  By Sylvester's criterion, this reduces to positivity of the leading principal minors of the Hessian: $\partial^{2}F/\partial T^{2}<0$, $\det\bigl(\begin{smallmatrix}\partial^{2}F/\partial T^{2}&\partial^{2}F/\partial T\partial V\\\partial^{2}F/\partial V\partial T&\partial^{2}F/\partial V^{2}\end{smallmatrix}\bigr)>0$, etc.
  Violation of these conditions signals a phase transition or spinodal decomposition.

\item \textbf{Stability of mechanical equilibria.}%
  \index{mechanical equilibrium!stability}%
  \index{potential energy!Hessian}%
  \index{Sylvester's criterion!mechanics}%
  A mechanical equilibrium at $\mathbf{q}_{0}$ is stable if the Hessian of the potential energy $V_{ij}=\partial^{2}V/\partial q_{i}\partial q_{j}|_{\mathbf{q}_{0}}$ is positive definite.
  For a system with $n$ degrees of freedom, checking $n$ leading principal minors via Sylvester's criterion is often simpler than computing all $n$ eigenvalues.
  Failure of the $k$-th minor identifies the subspace in which the instability first occurs.

\item \textbf{Passivity of multiport networks.}%
  \index{passive network!principal minors}%
  \index{impedance matrix!positive real}%
  \index{multiport network}%
  A multiport electrical network is passive if and only if its impedance matrix $Z(\omega)$ satisfies $\mathrm{Re}\,Z\geq 0$ (positive semidefinite Hermitian part) for all frequencies $\omega>0$.
  By the principal minor criterion for positive semidefiniteness, every principal minor of $\mathrm{Re}\,Z(\omega)$ must be non-negative.
  This provides a hierarchy of necessary conditions that can be checked sequentially, from one-port to multi-port constraints.
\end{enumerate}

\paragraph{Mathematics applications.}
\begin{enumerate}
\item \textbf{Descartes' rule of signs for characteristic polynomials.}%
  \index{Descartes' rule of signs}%
  \index{characteristic polynomial!sign pattern}%
  \index{inertia!principal minors}%
  The coefficients of the characteristic polynomial $\det(\lambda I-A)=\sum_{k}(-1)^{k}e_{k}\lambda^{n-k}$ are the elementary symmetric polynomials $e_{k}$ of the eigenvalues, which are sums of $k\times k$ principal minors.
  Descartes' rule applied to the sign pattern of these sums bounds the number of positive and negative eigenvalues.
  This connects the principal minors to the inertia of the matrix (Sylvester's law).

\item \textbf{Totally positive matrices.}%
  \index{totally positive matrix}%
  \index{minor!positivity}%
  \index{oscillatory matrix}%
  A matrix is totally positive if every minor (not just principal minors) is non-negative.
  Totally positive matrices arise in spline theory, combinatorics (Jacobi--Trudi identity), and the theory of P\'olya frequency sequences.
  The Loewner--Whitney theorem characterises totally positive matrices as products of elementary bidiagonal matrices with positive entries, providing a useful parametrisation.

\item \textbf{Compound matrices and exterior powers.}%
  \index{compound matrix}%
  \index{exterior power!matrix}%
  \index{Cauchy--Binet formula}%
  The $k$-th compound matrix $C_{k}(A)$ has entries that are all $k\times k$ minors of $A$, indexed by the corresponding row and column index sets.
  By the Cauchy--Binet formula, $C_{k}(AB)=C_{k}(A)C_{k}(B)$, so the map $A\mapsto C_{k}(A)$ is a group homomorphism.
  The eigenvalues of $C_{k}(A)$ are all $\binom{n}{k}$ products of $k$ eigenvalues of $A$, connecting principal minors to spectral theory.
\end{enumerate}

\subsection{14.15\textsuperscript{*}\quad Laplace Expansion of a Determinant}

%% -------------------------------------------------------------------
\subsubsection{14.151\quad Generalised Laplace expansion}

The generalised Laplace expansion expresses the determinant as a sum over complementary $k\times k$ and $(n-k)\times(n-k)$ minors.
Choosing a set $S$ of $k$ rows, $\det A=\sum_{T}(-1)^{|S|+|T|}M_{S,T}\,M_{\bar{S},\bar{T}}$, where the sum ranges over all $\binom{n}{k}$ column subsets $T$, and bars denote complements.

\paragraph{Physics applications.}
\begin{enumerate}
\item \textbf{Pfaffian and BCS pairing.}%
  \index{Pfaffian}%
  \index{BCS theory!Pfaffian}%
  \index{pairing Hamiltonian}%
  \index{superconductivity!pairing}%
  For a $2n\times 2n$ antisymmetric matrix $A$, $\det A=(\mathrm{Pf}\,A)^{2}$, where the Pfaffian $\mathrm{Pf}\,A$ is computed via a Laplace-type expansion over perfect matchings.
  In BCS theory, the ground-state wavefunction of a superconductor involves a Pfaffian of the pairing matrix.
  The Pfaffian sign determines the topological invariant of a topological superconductor (Kitaev chain).

\item \textbf{Wick's theorem and Feynman diagrams.}%
  \index{Wick's theorem}%
  \index{Feynman diagram!combinatorics}%
  \index{Gaussian integral!Wick contraction}%
  Wick's theorem states that the expectation value $\langle\phi_{1}\cdots\phi_{2n}\rangle$ in a free field theory equals the sum over all pairings $\sum\prod\langle\phi_{i}\phi_{j}\rangle$, which is exactly $\mathrm{Pf}(G)$ where $G_{ij}=\langle\phi_{i}\phi_{j}\rangle$ is the propagator matrix.
  Each pairing corresponds to a Feynman diagram, and the Laplace expansion of the determinant/Pfaffian organises the combinatorics.
  For fermions, the sign of each Wick contraction is the signature of the corresponding permutation.

\item \textbf{Multi-electron integrals in quantum chemistry.}%
  \index{multi-electron integral}%
  \index{quantum chemistry!determinant expansion}%
  \index{Slater--Condon rules}%
  The Slater--Condon rules express matrix elements of one- and two-body operators between Slater determinants in terms of one- and two-electron integrals.
  These rules are derived by Laplace expansion of the overlap determinant between two Slater determinants differing in $k$ orbitals: the matrix element vanishes if $k>2$ (two-body operator) or $k>1$ (one-body operator).
  This is the computational backbone of configuration interaction methods.
\end{enumerate}

\paragraph{Mathematics applications.}
\begin{enumerate}
\item \textbf{Cauchy--Binet formula.}%
  \index{Cauchy--Binet formula}%
  \index{rectangular matrix!determinant}%
  \index{Gram determinant}%
  For an $m\times n$ matrix $A$ and an $n\times m$ matrix $B$ with $m\leq n$, the Cauchy--Binet formula gives $\det(AB)=\sum_{S}\det(A_{S})\det(B_{S})$, where the sum is over all $\binom{n}{m}$ subsets $S$ of columns of $A$ (rows of $B$).
  Setting $B=A^{T}$ yields $\det(AA^{T})=\sum_{S}(\det A_{S})^{2}\geq 0$, proving that Gram matrices are positive semidefinite.
  This formula generalises $\det(AB)=\det A\det B$ to rectangular matrices.

\item \textbf{Pl\"ucker coordinates and Grassmannians.}%
  \index{Pl\"ucker coordinates}%
  \index{Grassmannian}%
  \index{Pl\"ucker relations}%
  The Pl\"ucker embedding maps a $k$-dimensional subspace $V\subset\mathbb{R}^{n}$, represented by a $k\times n$ matrix of basis vectors, to the vector of all $\binom{n}{k}$ maximal minors (Pl\"ucker coordinates) in projective space.
  These coordinates satisfy the Pl\"ucker relations, quadratic equations derived from the Laplace expansion.
  This gives the Grassmannian $\mathrm{Gr}(k,n)$ the structure of a projective variety and is the foundation of the modern theory of scattering amplitudes (amplituhedron).

\item \textbf{Dodgson condensation.}%
  \index{Dodgson condensation}%
  \index{determinant!recursive evaluation}%
  \index{Lewis Carroll identity|see{Dodgson condensation}}%
  Dodgson (Lewis Carroll) condensation computes $\det A$ recursively via the identity $\det A\cdot\det A_{ij}^{ij}=\det A_{i}^{i}\det A_{j}^{j}-\det A_{i}^{j}\det A_{j}^{i}$, where superscripts and subscripts denote deleted rows and columns.
  This is a consequence of the Laplace expansion and the Desnanot--Jacobi identity.
  It provides an $O(n^{3})$ algorithm that is naturally parallelisable, and the intermediate quantities have combinatorial interpretations as weighted sums of non-intersecting lattice paths.
\end{enumerate}

\subsection{14.16\quad Jacobi's Theorem}

%% -------------------------------------------------------------------
\subsubsection{14.161\quad Jacobi's theorem on complementary minors}

Jacobi's theorem states that for an invertible matrix $A$, the $(I,J)$-minor of $A^{-1}$ is related to the complementary minor of $A$:
\[
  \det\bigl[(A^{-1})_{I,J}\bigr]=(-1)^{|I|+|J|}\frac{\det(A_{\bar{J},\bar{I}})}{\det A},
\]
where $\bar{I}$ and $\bar{J}$ are the complementary index sets.

\paragraph{Physics applications.}
\begin{enumerate}
\item \textbf{Schur complement and effective Hamiltonians.}%
  \index{Schur complement!effective Hamiltonian}%
  \index{L\"owdin partitioning}%
  \index{downfolding}%
  L\"owdin partitioning in quantum mechanics writes the effective Hamiltonian for a subspace $P$ as $H_{\mathrm{eff}}=H_{PP}-H_{PQ}(H_{QQ}-E)^{-1}H_{QP}$, which is a Schur complement.
  Jacobi's theorem relates the determinant of the effective Hamiltonian to the complementary minor of the full resolvent, providing a direct link between the full and reduced spectra.
  This partitioning is used in electronic structure theory, nuclear shell models, and effective field theories.

\item \textbf{Fluctuation--dissipation and response functions.}%
  \index{fluctuation--dissipation theorem}%
  \index{response function!minor}%
  \index{susceptibility!submatrix}%
  In linear response theory, the susceptibility matrix $\chi=-\beta(G^{-1})$ relates fluctuations to responses.
  A submatrix of the susceptibility corresponds, by Jacobi's theorem, to a complementary minor of the correlation matrix $G$, yielding the conditional response when some degrees of freedom are held fixed.
  This is the matrix analogue of the thermodynamic Maxwell relations.

\item \textbf{Network reduction and Kron reduction.}%
  \index{Kron reduction}%
  \index{network reduction!determinant}%
  \index{power grid!reduced model}%
  Kron reduction eliminates internal nodes from a network, replacing the full admittance matrix $Y$ by a reduced matrix $Y_{\mathrm{red}}=Y_{PP}-Y_{PQ}Y_{QQ}^{-1}Y_{QP}$ (a Schur complement).
  By Jacobi's theorem, the determinant of the reduced matrix relates to minors of the original.
  This technique is standard in power systems analysis, where networks with thousands of buses are reduced to equivalent models at boundary nodes.
\end{enumerate}

\paragraph{Mathematics applications.}
\begin{enumerate}
\item \textbf{Matrix inversion lemma (Woodbury identity).}%
  \index{Woodbury identity}%
  \index{matrix inversion lemma}%
  \index{Sherman--Morrison formula}%
  The Woodbury identity $(A+UCV)^{-1}=A^{-1}-A^{-1}U(C^{-1}+VA^{-1}U)^{-1}VA^{-1}$ is a consequence of Jacobi's theorem on complementary minors applied to the block matrix $\bigl(\begin{smallmatrix}A&U\\-V&C^{-1}\end{smallmatrix}\bigr)$.
  The special case of a rank-one update is the Sherman--Morrison formula.
  These identities are computationally essential when $A$ is large but $C$ is small, enabling $O(n^{2})$ updates instead of $O(n^{3})$ re-inversions.

\item \textbf{Complementary subspaces and duality.}%
  \index{complementary subspace}%
  \index{duality!determinant}%
  \index{Hodge star!algebraic analogue}%
  Jacobi's theorem provides an algebraic analogue of Hodge duality: the $k$-form data of a linear map (the $k\times k$ minors) determines the $(n-k)$-form data (the complementary minors) up to a sign and the total determinant.
  This is the matrix-theoretic shadow of the Hodge star operator $\star\colon\bigwedge^{k}V\to\bigwedge^{n-k}V$ and connects determinantal identities to differential geometry.

\item \textbf{Dodgson--Jacobi identity and cluster algebras.}%
  \index{cluster algebra!determinant}%
  \index{Desnanot--Jacobi identity}%
  \index{Pl\"ucker relation!Jacobi}%
  The Desnanot--Jacobi identity $\det A\cdot\det A_{ij}^{ij}=\det A_{i}^{i}\det A_{j}^{j}-\det A_{i}^{j}\det A_{j}^{i}$ is a special case of Jacobi's theorem on complementary minors.
  This identity is an exchange relation in the cluster algebra structure on the coordinate ring of the Grassmannian, connecting classical determinantal identities to the modern theory of Fomin and Zelevinsky.
\end{enumerate}

\subsection{14.17\quad Hadamard's Theorem}

%% -------------------------------------------------------------------
\subsubsection{14.171\quad Hadamard matrices}

A Hadamard matrix $H_{n}$ is an $n\times n$ matrix with entries $\pm 1$ satisfying $H_{n}H_{n}^{T}=nI$.
The Hadamard conjecture asserts that $H_{n}$ exists for every $n$ divisible by~$4$ (and for $n=1,2$).
The simplest construction is the Sylvester (Kronecker product) method: $H_{2^{k}}=H_{2}\otimes H_{2^{k-1}}$ with $H_{2}=\bigl(\begin{smallmatrix}1&1\\1&-1\end{smallmatrix}\bigr)$.

\paragraph{Physics applications.}
\begin{enumerate}
\item \textbf{Hadamard gate in quantum computing.}%
  \index{Hadamard gate}%
  \index{quantum computing!Hadamard}%
  \index{superposition!quantum}%
  \index{quantum circuit}%
  The Hadamard gate $H=\frac{1}{\sqrt{2}}\bigl(\begin{smallmatrix}1&1\\1&-1\end{smallmatrix}\bigr)$ creates an equal superposition $|0\rangle\mapsto(|0\rangle+|1\rangle)/\sqrt{2}$.
  Applying $H^{\otimes n}$ to $|0\rangle^{\otimes n}$ creates the uniform superposition over all $2^{n}$ computational basis states, the starting step of Grover's search and the quantum Fourier transform.
  The fact that $H$ is simultaneously a Hadamard matrix (up to normalisation) and a unitary gate is no coincidence: $H_{n}H_{n}^{T}=nI$ becomes $UU^{\dagger}=I$ after dividing by $\sqrt{n}$.

\item \textbf{Walsh--Hadamard transform in signal processing.}%
  \index{Walsh--Hadamard transform}%
  \index{signal processing!Walsh functions}%
  \index{CDMA!Walsh codes}%
  The Walsh--Hadamard transform $\mathbf{y}=H_{n}\mathbf{x}$ can be computed in $O(n\log n)$ operations using the butterfly structure of the Sylvester construction, analogous to the Cooley--Tukey FFT.
  In CDMA telecommunications, Walsh codes (rows of $H_{n}$) provide orthogonal spreading sequences.
  The entries $\pm 1$ ensure constant envelope, desirable for power amplifier linearity.

\item \textbf{Speckle patterns and Hadamard spectroscopy.}%
  \index{Hadamard spectroscopy}%
  \index{multiplexing!Hadamard}%
  \index{Fellgett advantage}%
  Hadamard transform spectroscopy uses a mask based on a Hadamard matrix row to encode spectral channels.
  The inverse transform recovers the spectrum from $n$ measurements, each integrating roughly half the channels.
  The multiplex (Fellgett) advantage gives a $\sqrt{n}$ improvement in signal-to-noise ratio over scanning spectrometers, because each measurement receives signal from $n/2$ channels simultaneously.
\end{enumerate}

\paragraph{Mathematics applications.}
\begin{enumerate}
\item \textbf{Combinatorial designs and error-correcting codes.}%
  \index{combinatorial design!Hadamard}%
  \index{error-correcting code!Hadamard}%
  \index{Reed--Muller code}%
  From an $n\times n$ Hadamard matrix one constructs a $(2n,n,n/2)$ Hadamard code (first-order Reed--Muller code) by using the rows and their negatives as codewords.
  This code achieves the Plotkin bound and is the basis of the Reed--Muller family.
  Hadamard matrices also yield symmetric balanced incomplete block designs (2-designs) with parameters $(4n-1,2n-1,n-1)$.

\item \textbf{Hadamard conjecture and Paley construction.}%
  \index{Hadamard conjecture}%
  \index{Paley construction}%
  \index{quadratic residue!Hadamard}%
  Paley's construction produces Hadamard matrices of order $q+1$ when $q\equiv 3\pmod{4}$ is a prime power, using the quadratic residue character of $\mathbb{F}_{q}$.
  The resulting matrix $H=(h_{ij})$ with $h_{ij}=\chi(i-j)$ (Jacobsthal matrix) plus a border of ones satisfies $HH^{T}=(q+1)I$.
  Despite intensive search, the Hadamard conjecture remains open; the smallest unresolved order is $n=668$.

\item \textbf{Spectral properties and flat polynomials.}%
  \index{Hadamard matrix!spectral property}%
  \index{flat polynomial}%
  \index{Littlewood conjecture}%
  The rows of a normalised Hadamard matrix $n^{-1/2}H_{n}$ form an orthonormal basis of $\mathbb{R}^{n}$ with all entries of equal absolute value.
  The existence question is related to the Littlewood conjecture on polynomials with $\pm 1$ coefficients having flat magnitude on the unit circle.
  The eigenvalues of $H_{n}$ all have absolute value $\sqrt{n}$, making $H_{n}$ a conference matrix when $n$ is even.
\end{enumerate}

\subsection{14.18\quad Hadamard's Inequality}

%% -------------------------------------------------------------------
\subsubsection{14.181\quad Hadamard's determinant inequality}

Hadamard's inequality states that for any $n\times n$ matrix $A$ with columns $\mathbf{a}_{1},\ldots,\mathbf{a}_{n}$,
\[
  |\det A|\leq\prod_{j=1}^{n}\|\mathbf{a}_{j}\|,
\]
with equality if and only if the columns are mutually orthogonal.
For positive definite matrices with entries $|a_{ij}|\leq 1$, this gives $|\det A|\leq n^{n/2}$ (the Hadamard bound), achieved by Hadamard matrices.

\paragraph{Physics applications.}
\begin{enumerate}
\item \textbf{Maximum entropy and covariance determinants.}%
  \index{maximum entropy!Gaussian}%
  \index{covariance matrix!determinant bound}%
  \index{entropy!multivariate Gaussian}%
  The differential entropy of a multivariate Gaussian with covariance $\Sigma$ is $h=\frac{1}{2}\ln\det(2\pi e\Sigma)$.
  Hadamard's inequality gives $\det\Sigma\leq\prod\sigma_{ii}^{2}$, with equality when variables are uncorrelated.
  Thus, among all distributions with given marginal variances, the product of independent Gaussians has the maximum entropy---a result central to information theory and statistical physics.

\item \textbf{MIMO channel capacity.}%
  \index{MIMO!channel capacity}%
  \index{channel capacity!determinant}%
  \index{wireless communications!Hadamard bound}%
  The capacity of a MIMO (multiple-input multiple-output) wireless channel is $C=\log_{2}\det(I+\frac{\mathrm{SNR}}{n_{t}}HH^{\dagger})$ bits per channel use.
  Hadamard's inequality shows that capacity is maximised when the columns of $H$ are orthogonal (no inter-antenna interference).
  This motivates beamforming and precoding strategies that attempt to orthogonalise the effective channel matrix.

\item \textbf{Experimental design and D-optimality.}%
  \index{D-optimality}%
  \index{experimental design!determinant}%
  \index{Fisher information!determinant}%
  \index{optimal design|see{D-optimality}}%
  A D-optimal experimental design maximises $\det(X^{T}X)$, where $X$ is the design matrix.
  By Hadamard's inequality, $\det(X^{T}X)\leq\prod_{j}\|x_{j}\|^{2}$, and the bound is achieved when design columns are orthogonal.
  This determinantal criterion maximises the volume of the confidence ellipsoid and is equivalent to maximising the determinant of the Fisher information matrix.
\end{enumerate}

\paragraph{Mathematics applications.}
\begin{enumerate}
\item \textbf{Maximum volume simplices.}%
  \index{simplex!maximum volume}%
  \index{Hadamard inequality!simplex}%
  \index{Hadamard bound}%
  The volume of the simplex with vertices at the origin and at $\mathbf{a}_{1},\ldots,\mathbf{a}_{n}$ is $V=|\det A|/n!$.
  Hadamard's inequality bounds this volume by the product of edge lengths divided by $n!$.
  Among all simplices inscribed in the unit cube $[0,1]^{n}$, the maximum volume is achieved when the vertex matrix is (up to affine transformation) a Hadamard matrix, connecting the Hadamard conjecture to discrete geometry.

\item \textbf{Gram determinant and geometric measure.}%
  \index{Gram determinant!volume}%
  \index{geometric measure!Gram}%
  \index{parallelotope volume}%
  For vectors $v_{1},\ldots,v_{k}\in\mathbb{R}^{n}$, the $k$-dimensional volume of the parallelotope they span is $\mathrm{vol}_{k}=\sqrt{\det G}$, where $G_{ij}=\langle v_{i},v_{j}\rangle$ is the Gram matrix.
  Hadamard's inequality applied to $G$ gives $\det G\leq\prod\|v_{i}\|^{2}$, recovering the fact that volume is maximised for orthogonal vectors.
  The ratio $\det G/\prod\|v_{i}\|^{2}$ measures the ``orthogonality defect'' and is used as a quality metric in lattice basis reduction (LLL algorithm).

\item \textbf{Coding theory and the Singleton bound.}%
  \index{coding theory!Hadamard bound}%
  \index{Singleton bound}%
  \index{MDS code}%
  For a linear code with generator matrix $G$, the minimum distance is related to the smallest number of linearly dependent columns.
  Hadamard's inequality bounds the number of codewords achievable with a given minimum distance, and codes meeting this bound (MDS codes) have the property that every square submatrix of $G$ is non-singular (every maximal minor is non-zero).
  The determinant bounds from Hadamard's inequality thus constrain the existence of optimal codes.
\end{enumerate}

\subsection{14.21\quad Cramer's Rule}

%% -------------------------------------------------------------------
\subsubsection{14.211\quad Cramer's rule for linear systems}

Cramer's rule solves the system $A\mathbf{x}=\mathbf{b}$ when $\det A\neq 0$ by
\[
  x_{i}=\frac{\det A_{i}(\mathbf{b})}{\det A},\qquad i=1,\ldots,n,
\]
where $A_{i}(\mathbf{b})$ is the matrix $A$ with its $i$-th column replaced by $\mathbf{b}$.

\paragraph{Physics applications.}
\begin{enumerate}
\item \textbf{Circuit analysis with mesh currents.}%
  \index{circuit analysis!Cramer's rule}%
  \index{mesh current method}%
  \index{impedance matrix}%
  Kirchhoff's voltage law for a network with $n$ meshes yields $Z\mathbf{I}=\mathbf{V}$, where $Z$ is the impedance matrix.
  Cramer's rule gives each mesh current as a ratio of determinants: $I_{k}=\det Z_{k}/\det Z$.
  For small networks ($n\leq 4$), this is practical and gives closed-form expressions showing how each current depends on all sources, useful for understanding mutual coupling.

\item \textbf{Scattering parameters from boundary conditions.}%
  \index{scattering parameter}%
  \index{boundary condition!linear system}%
  \index{transmission line!junction}%
  At a junction of $n$ transmission lines, continuity of voltage and current yields a linear system whose solution via Cramer's rule gives the scattering parameters $S_{ij}$ as ratios of determinants.
  The structure of these determinants reveals which geometric parameters affect each $S_{ij}$, guiding the design of microwave filters and impedance-matching networks.
  The condition $\det Z=0$ signals a resonance at which the system has a non-trivial solution with no external driving.

\item \textbf{Equilibrium concentrations in chemical kinetics.}%
  \index{chemical kinetics!equilibrium}%
  \index{steady-state concentration}%
  \index{King--Altman method}%
  The King--Altman method for enzyme kinetics expresses steady-state concentrations of enzyme intermediates as ratios of determinants of the rate-constant matrix.
  Each cofactor in the numerator is a sum over spanning trees of the kinetic graph (by Kirchhoff's matrix tree theorem), and the denominator is the sum of all such trees.
  This gives the Michaelis--Menten and more complex rate laws as determinantal expressions.
\end{enumerate}

\paragraph{Mathematics applications.}
\begin{enumerate}
\item \textbf{Birational geometry and rational solutions.}%
  \index{Cramer's rule!birational}%
  \index{rational solution}%
  \index{algebraic geometry!linear systems}%
  Cramer's rule shows that the solution of a parametric linear system $A(\mathbf{p})\mathbf{x}=\mathbf{b}(\mathbf{p})$ is a rational function of the parameters.
  This is the prototype of the general principle that solutions of algebraic equations are algebraic (or rational) functions of the coefficients.
  In algebraic geometry, Cramer's rule describes the birational map from the space of coefficients to the space of solutions.

\item \textbf{Interpolation formulas.}%
  \index{interpolation!Cramer's rule}%
  \index{Lagrange interpolation}%
  \index{Vandermonde system}%
  The coefficients of the interpolating polynomial of degree $n-1$ through $n$ points satisfy a Vandermonde system $V\mathbf{c}=\mathbf{y}$.
  Cramer's rule gives each coefficient $c_{k}$ as a ratio of Vandermonde-like determinants; the resulting formula is equivalent to the Lagrange interpolation formula.
  The explicit determinantal form reveals the condition number of the interpolation problem and motivates the use of orthogonal polynomial bases.

\item \textbf{Consistency and the Rouch\'e--Capelli theorem.}%
  \index{Rouch\'e--Capelli theorem}%
  \index{consistency!linear system}%
  \index{augmented matrix!rank}%
  The system $A\mathbf{x}=\mathbf{b}$ is consistent if and only if $\mathrm{rank}[A|\mathbf{b}]=\mathrm{rank}\,A$ (Rouch\'e--Capelli).
  When $\mathrm{rank}\,A=n$, Cramer's rule provides the unique solution.
  When $\mathrm{rank}\,A<n$ but the system is consistent, the general solution is a translate of the null space, parametrised using the cofactors of a maximal non-singular submatrix.
\end{enumerate}


\subsection{14.31\quad Some Special Determinants}

%% -------------------------------------------------------------------
\subsubsection{14.311\quad Vandermonde's determinant (alternant)}

The Vandermonde determinant of $x_{1},\ldots,x_{n}$ is
\[
  V(x_{1},\ldots,x_{n})=\det\bigl[x_{j}^{i-1}\bigr]_{i,j=1}^{n}
  =\prod_{1\leq i<j\leq n}(x_{j}-x_{i}).
\]
This is the prototypical alternating polynomial and vanishes if and only if two arguments coincide.

\paragraph{Physics applications.}
\begin{enumerate}
\item \textbf{Free fermion partition function and eigenvalue repulsion.}%
  \index{free fermion!partition function}%
  \index{eigenvalue repulsion}%
  \index{random matrix theory!Vandermonde}%
  \index{Coulomb gas!Vandermonde}%
  The joint probability density of eigenvalues $\lambda_{1},\ldots,\lambda_{N}$ of a random Hermitian matrix from the Gaussian Unitary Ensemble (GUE) is
  \[
    p(\lambda_{1},\ldots,\lambda_{N})\propto\prod_{i<j}|\lambda_{i}-\lambda_{j}|^{2}\,e^{-\sum\lambda_{k}^{2}/2}
    =|V(\lambda_{1},\ldots,\lambda_{N})|^{2}\,e^{-\sum\lambda_{k}^{2}/2}.
  \]
  The Vandermonde factor $|V|^{2}$ produces the eigenvalue repulsion characteristic of random matrices and is equivalent to a two-dimensional Coulomb gas at inverse temperature $\beta=2$.
  This distribution is also the squared norm of a Slater determinant of harmonic oscillator wavefunctions, establishing the connection between free fermions and random matrices.

\item \textbf{Quantum Hall effect and Laughlin wavefunction.}%
  \index{quantum Hall effect!Laughlin}%
  \index{Laughlin wavefunction}%
  \index{fractional quantum Hall effect}%
  The Laughlin wavefunction for the fractional quantum Hall state at filling $\nu=1/m$ is
  $\Psi(z_{1},\ldots,z_{N})=\prod_{i<j}(z_{i}-z_{j})^{m}\,e^{-\sum|z_{k}|^{2}/4\ell^{2}}$,
  where $z_{k}=x_{k}+iy_{k}$ are complex coordinates and $\ell$ is the magnetic length.
  For $m=1$ (integer quantum Hall), this is exactly the Vandermonde determinant of the lowest Landau level orbitals.
  The exponent $m$ introduces stronger correlations (fractional statistics), and the topological order of the state is encoded in the analytic structure of this generalised Vandermonde factor.

\item \textbf{Polynomial interpolation in spectroscopy.}%
  \index{spectroscopy!polynomial interpolation}%
  \index{Vandermonde matrix!interpolation}%
  \index{calibration curve}%
  Fitting a calibration curve through $n$ data points $(x_{i},y_{i})$ in spectroscopy requires solving the Vandermonde system $V\mathbf{c}=\mathbf{y}$.
  The condition number of $V$ grows exponentially with $n$ for equally spaced points, but the Vandermonde determinant $\prod(x_{j}-x_{i})$ reveals that the system is well-conditioned when the nodes are spread out (e.g., Chebyshev nodes).
  This motivates the use of orthogonal polynomial bases for stable calibration.

\item \textbf{Discrete Fourier transform as a Vandermonde matrix.}%
  \index{discrete Fourier transform!Vandermonde}%
  \index{DFT matrix!structure}%
  \index{roots of unity}%
  The DFT matrix $F_{n}$ with entries $F_{jk}=\omega^{jk}$, $\omega=e^{2\pi i/n}$, is a Vandermonde matrix with nodes at the $n$-th roots of unity.
  Its determinant is $\det F_{n}=V(1,\omega,\ldots,\omega^{n-1})=\prod_{0\leq j<k\leq n-1}(\omega^{k}-\omega^{j})=n^{n/2}e^{i\pi n(n-1)/4}$ (up to sign convention).
  The Vandermonde structure guarantees invertibility and underlies the FFT factorisation.
\end{enumerate}

\paragraph{Mathematics applications.}
\begin{enumerate}
\item \textbf{Schur polynomials and representation theory.}%
  \index{Schur polynomial}%
  \index{representation theory!symmetric group}%
  \index{Weyl character formula!Vandermonde}%
  The Schur polynomial $s_{\lambda}(x_{1},\ldots,x_{n})$ corresponding to a partition $\lambda$ is the ratio of two alternants:
  $s_{\lambda}=\det[x_{j}^{\lambda_{i}+n-i}]/\det[x_{j}^{n-i}]=a_{\lambda+\delta}/a_{\delta}$,
  where $\delta=(n-1,n-2,\ldots,0)$ and the denominator is the Vandermonde determinant.
  The Weyl character formula for $\mathrm{GL}(n)$ representations takes exactly this form, with Schur polynomials as the irreducible characters.
  This connects determinantal algebra to the deepest structures in combinatorics and representation theory.

\item \textbf{Newton's identities and symmetric functions.}%
  \index{Newton's identities}%
  \index{symmetric function!Vandermonde}%
  \index{power sum polynomial}%
  The Vandermonde determinant is the simplest alternating symmetric polynomial.
  Every alternating polynomial is divisible by $V$, and the quotient is a symmetric polynomial.
  Newton's identities relate the power sums $p_{k}=\sum x_{i}^{k}$ to the elementary symmetric polynomials $e_{k}$ via the determinantal formula $e_{k}=\frac{1}{k!}\det\bigl(\begin{smallmatrix}p_{1}&1&0&\cdots\\p_{2}&p_{1}&2&\cdots\\\vdots&&\ddots\\p_{k}&p_{k-1}&\cdots&p_{1}\end{smallmatrix}\bigr)$.

\item \textbf{Hermite interpolation and confluent Vandermonde.}%
  \index{Hermite interpolation}%
  \index{confluent Vandermonde matrix}%
  \index{divided difference!Vandermonde}%
  When interpolation nodes coalesce ($x_{i}\to x_{j}$), the Vandermonde matrix degenerates into the confluent Vandermonde matrix, whose rows involve derivatives of the monomial basis.
  The resulting system solves the Hermite interpolation problem (matching function values and derivatives).
  The confluent Vandermonde determinant is $\prod(x_{j}-x_{i})^{m_{i}m_{j}}$, where $m_{k}$ are the multiplicities, generalising the classical product formula.
\end{enumerate}

%% -------------------------------------------------------------------
\subsubsection{14.312\quad Circulants}

The circulant matrix $C=\mathrm{circ}(c_{0},c_{1},\ldots,c_{n-1})$ has $(i,j)$-entry $c_{(j-i)\bmod n}$.
Its determinant is
\[
  \det C=\prod_{k=0}^{n-1}p(\omega^{k}),\qquad
  p(z)=\sum_{j=0}^{n-1}c_{j}z^{j},\quad\omega=e^{2\pi i/n}.
\]

\paragraph{Physics applications.}
\begin{enumerate}
\item \textbf{DFT diagonalisation and Bloch's theorem.}%
  \index{DFT!circulant diagonalisation}%
  \index{Bloch's theorem}%
  \index{periodic boundary conditions}%
  \index{tight-binding model!circulant}%
  A circulant matrix is diagonalised by the DFT matrix: $C=F^{-1}\mathrm{diag}(\hat{c}_{0},\ldots,\hat{c}_{n-1})F$, where $\hat{c}_{k}=p(\omega^{k})$.
  In solid-state physics, the tight-binding Hamiltonian with periodic boundary conditions is circulant, and diagonalisation by the DFT is Bloch's theorem: the eigenstates are plane waves $\psi_{k}(j)=\omega^{jk}/\sqrt{n}$ with eigenvalues $\epsilon(k)=\sum_{j}t_{j}\omega^{jk}$ (the band structure).
  The determinant $\det(EI-H)=\prod_{k}(E-\epsilon(k))$ gives the spectral polynomial.

\item \textbf{Cyclic codes and error correction.}%
  \index{cyclic code!circulant}%
  \index{error-correcting code!cyclic}%
  \index{generator polynomial}%
  A cyclic code of length $n$ over $\mathbb{F}_{q}$ corresponds to an ideal in $\mathbb{F}_{q}[x]/(x^{n}-1)$, generated by a divisor $g(x)$ of $x^{n}-1$.
  The parity-check matrix is circulant, and its determinant (over the finite field) characterises the code's error-detection capability.
  The BCH bound on minimum distance is derived from the roots of $g(x)$ among the $n$-th roots of unity, mirroring the eigenvalue decomposition of the circulant.

\item \textbf{Discrete convolution and filtering.}%
  \index{discrete convolution!circulant}%
  \index{circular convolution}%
  \index{digital filter!circulant matrix}%
  Multiplication by a circulant matrix implements circular convolution: $C\mathbf{x}=\mathbf{c}\circledast\mathbf{x}$.
  The determinant condition $\det C\neq 0$ is equivalent to $p(\omega^{k})\neq 0$ for all $k$, meaning the frequency response has no zeros---the filter is invertible (deconvolution is possible).
  This is the discrete analogue of the Wiener--Khinchin condition for causal deconvolution.

\item \textbf{Normal modes of cyclic molecular chains.}%
  \index{normal mode!cyclic chain}%
  \index{cyclic molecule}%
  \index{benzene!normal modes}%
  The Hessian of a cyclic molecular chain (e.g., benzene) with nearest-neighbour force constants is circulant.
  The normal-mode frequencies are $\omega_{k}^{2}=f_{0}+2f_{1}\cos(2\pi k/n)$, where $f_{0}$ and $f_{1}$ are the diagonal and off-diagonal force constants.
  The degeneracies $\omega_{k}=\omega_{n-k}$ reflect the real-valuedness of the circulant (symmetry under complex conjugation of eigenvalues).
\end{enumerate}

\paragraph{Mathematics applications.}
\begin{enumerate}
\item \textbf{Resultants via circulants.}%
  \index{resultant!circulant}%
  \index{polynomial!common root}%
  \index{companion matrix!circulant}%
  The resultant of $x^{n}-1$ and $p(x)$ is $\prod_{k}p(\omega^{k})=\det\mathrm{circ}(c_{0},\ldots,c_{n-1})$, the determinant of the circulant.
  More generally, the resultant $\mathrm{Res}(f,g)=\det S$ (Sylvester matrix) can be block-diagonalised into circulant-like blocks when $f$ and $g$ have special structure (e.g., when $f=x^{n}-a$), reducing the resultant to a product over roots of unity.

\item \textbf{Number theory and the norm of algebraic integers.}%
  \index{algebraic integer!norm}%
  \index{cyclotomic field}%
  \index{circulant determinant!norm}%
  The norm of an element $\alpha=\sum c_{j}\zeta^{j}$ in the cyclotomic field $\mathbb{Q}(\zeta)$, $\zeta=e^{2\pi i/n}$, is $N(\alpha)=\prod_{k}\sigma_{k}(\alpha)=\prod_{k}p(\zeta^{k})$, which is the determinant of the circulant matrix of multiplication by $\alpha$ in the integral basis.
  This connects the circulant determinant to algebraic number theory and is used in computing class numbers of cyclotomic fields.

\item \textbf{Graph spectra of cycle graphs.}%
  \index{cycle graph!spectrum}%
  \index{graph spectrum!circulant}%
  \index{Cheeger constant}%
  The adjacency matrix of the cycle graph $C_{n}$ is the circulant $\mathrm{circ}(0,1,0,\ldots,0,1)$ with eigenvalues $\lambda_{k}=2\cos(2\pi k/n)$.
  More generally, the adjacency matrix of any circulant graph $\mathrm{Circ}(n;S)$ is a circulant, and its spectrum is given by the polynomial $p(\omega^{k})$ evaluated at roots of unity.
  The spectral gap $\lambda_{1}-\lambda_{2}$ determines the expansion properties of the graph (Cheeger inequality).
\end{enumerate}

%% -------------------------------------------------------------------
\subsubsection{14.313\quad Jacobian determinant}

The Jacobian determinant of a differentiable map $\mathbf{f}\colon\mathbb{R}^{n}\to\mathbb{R}^{n}$ at a point $\mathbf{x}$ is
\[
  J_{\mathbf{f}}(\mathbf{x})=\det\!\left(\frac{\partial f_{i}}{\partial x_{j}}\right).
\]
It measures the local volume distortion of the map: an infinitesimal volume element $d^{n}x$ is mapped to $|J_{\mathbf{f}}|\,d^{n}x$.

\paragraph{Physics applications.}
\begin{enumerate}
\item \textbf{Change of variables in multiple integrals.}%
  \index{change of variables!Jacobian}%
  \index{Jacobian determinant!integration}%
  \index{curvilinear coordinates}%
  \index{spherical coordinates!Jacobian}%
  The change-of-variables formula $\int f(\mathbf{x})\,d^{n}x=\int f(\mathbf{g}(\mathbf{u}))\,|J_{\mathbf{g}}(\mathbf{u})|\,d^{n}u$ is the workhorse of multivariate integration.
  For spherical coordinates in $\mathbb{R}^{3}$, $J=r^{2}\sin\theta$; for general curvilinear coordinates, $J=\sqrt{\det g}$ where $g_{ij}$ is the metric tensor.
  In general relativity, the invariant volume element $\sqrt{-\det g_{\mu\nu}}\,d^{4}x$ ensures that the Einstein--Hilbert action is coordinate-independent.

\item \textbf{Faddeev--Popov determinant in gauge theory.}%
  \index{Faddeev--Popov determinant}%
  \index{gauge fixing}%
  \index{ghost field}%
  \index{path integral!gauge theory}%
  In the quantisation of gauge theories, the Faddeev--Popov procedure inserts the determinant $\det(\delta G/\delta\alpha)$ (the Jacobian of the gauge-fixing condition $G$ with respect to gauge parameters $\alpha$) into the path integral to compensate for the redundant integration over gauge orbits.
  This determinant is represented by anticommuting ghost fields $c$, $\bar{c}$ with action $S_{\mathrm{ghost}}=\int\bar{c}\,(\delta G/\delta\alpha)\,c$.
  The ghost fields contribute to loop diagrams and are essential for the unitarity and renormalisability of non-Abelian gauge theories (Yang--Mills).

\item \textbf{Hamiltonian mechanics and canonical transformations.}%
  \index{canonical transformation!Jacobian}%
  \index{generating function!canonical}%
  \index{Poisson bracket!Jacobian}%
  A transformation $(q,p)\to(Q,P)$ is canonical if and only if the Jacobian matrix is symplectic: $J^{T}\Omega J=\Omega$, where $\Omega=\bigl(\begin{smallmatrix}0&I\\-I&0\end{smallmatrix}\bigr)$.
  Taking determinants gives $(\det J)^{2}=1$, so $\det J=\pm 1$.
  Canonical transformations thus preserve oriented phase-space volume (Liouville's theorem) and the Poisson bracket structure.

\item \textbf{Probability density transformation.}%
  \index{probability density!transformation}%
  \index{change of variables!probability}%
  \index{normalising flow}%
  If $\mathbf{X}$ is a random vector with density $p_{\mathbf{X}}$ and $\mathbf{Y}=\mathbf{g}(\mathbf{X})$ is a diffeomorphism, then $p_{\mathbf{Y}}(\mathbf{y})=p_{\mathbf{X}}(\mathbf{g}^{-1}(\mathbf{y}))\,|J_{\mathbf{g}^{-1}}(\mathbf{y})|$.
  In machine learning, normalising flows compose a sequence of invertible maps to transform a simple base density into a complex target, with the log-likelihood computed via the sum of log-Jacobian-determinants at each layer.
  Efficient architectures ensure that each Jacobian is triangular, making the determinant computable in $O(n)$ time.
\end{enumerate}

\paragraph{Mathematics applications.}
\begin{enumerate}
\item \textbf{Inverse function theorem.}%
  \index{inverse function theorem}%
  \index{Jacobian!invertibility}%
  \index{local diffeomorphism}%
  The inverse function theorem states that if $J_{\mathbf{f}}(\mathbf{x}_{0})\neq 0$, then $\mathbf{f}$ is a local diffeomorphism near $\mathbf{x}_{0}$ with Jacobian of the inverse $J_{\mathbf{f}^{-1}}=(J_{\mathbf{f}})^{-1}$.
  The proof via the contraction mapping principle gives quantitative bounds on the radius of invertibility in terms of $|J_{\mathbf{f}}|$ and the Lipschitz constant of $D\mathbf{f}$.
  This is the foundation of the implicit function theorem and the theory of smooth manifolds.

\item \textbf{Degree of a map and topology.}%
  \index{degree of a map}%
  \index{Brouwer degree}%
  \index{winding number!Jacobian}%
  The Brouwer degree of a smooth map $\mathbf{f}\colon M\to N$ between compact oriented manifolds is $\deg\mathbf{f}=\sum_{\mathbf{x}\in\mathbf{f}^{-1}(\mathbf{y})}\mathrm{sgn}\,J_{\mathbf{f}}(\mathbf{x})$ for any regular value $\mathbf{y}$.
  This integer is a topological invariant (independent of the regular value chosen) and generalises the winding number.
  The degree governs the existence of solutions to $\mathbf{f}(\mathbf{x})=\mathbf{y}$: if $\deg\mathbf{f}\neq 0$, every regular value has at least one preimage.

\item \textbf{Sard's theorem and critical values.}%
  \index{Sard's theorem}%
  \index{critical value}%
  \index{measure zero!critical values}%
  Sard's theorem states that the set of critical values $\{\mathbf{f}(\mathbf{x}):J_{\mathbf{f}}(\mathbf{x})=0\}$ has Lebesgue measure zero.
  Thus ``almost every'' value of a smooth map is a regular value, and the preimage $\mathbf{f}^{-1}(\mathbf{y})$ is a smooth submanifold for almost every $\mathbf{y}$.
  This theorem is the analytic foundation of transversality theory and Morse theory.
\end{enumerate}

%% -------------------------------------------------------------------
\subsubsection{14.314\quad Hessian determinants}

The Hessian matrix of a twice-differentiable function $f\colon\mathbb{R}^{n}\to\mathbb{R}$ is $H_{ij}=\partial^{2}f/\partial x_{i}\partial x_{j}$, and the Hessian determinant is $\det H$.
For $n=2$, $\det H=f_{xx}f_{yy}-f_{xy}^{2}$ classifies critical points: positive for extrema, negative for saddle points.

\paragraph{Physics applications.}
\begin{enumerate}
\item \textbf{Stability of equilibria in classical mechanics.}%
  \index{Hessian!stability}%
  \index{equilibrium!classification}%
  \index{Morse index!mechanics}%
  At an equilibrium $\nabla V=0$ of a potential energy $V(q_{1},\ldots,q_{n})$, the Hessian $H_{ij}=\partial^{2}V/\partial q_{i}\partial q_{j}$ determines stability.
  The number of negative eigenvalues (the Morse index) counts the number of unstable directions.
  If $\det H>0$ and the diagonal minors are all positive (Sylvester's criterion), the equilibrium is stable; if $\det H<0$, at least one direction is unstable.

\item \textbf{Gaussian beam optics and ray transfer matrices.}%
  \index{Gaussian beam!Hessian}%
  \index{ray transfer matrix}%
  \index{wavefront curvature}%
  \index{paraxial optics}%
  In paraxial optics, the phase of a Gaussian beam $\psi\propto\exp(ik\,\mathbf{r}^{T}Q^{-1}\mathbf{r}/2)$ has a Hessian proportional to $Q^{-1}$, the inverse complex beam parameter matrix.
  The determinant $\det Q$ determines the beam cross-sectional area, and the Hessian eigenvalues give the principal curvatures of the wavefront.
  Under propagation through an optical system described by a ray transfer (ABCD) matrix, $Q$ transforms as $Q'=(AQ+B)(CQ+D)^{-1}$, preserving $\det(\mathrm{Im}\,Q^{-1})>0$ (beam physicality).

\item \textbf{Saddle-point approximation (steepest descent).}%
  \index{saddle-point approximation}%
  \index{steepest descent method}%
  \index{Hessian!Gaussian integral}%
  The saddle-point (Laplace) approximation of $\int e^{-\lambda f(\mathbf{x})}\,d^{n}x$ as $\lambda\to\infty$ gives $(\frac{2\pi}{\lambda})^{n/2}|\det H|^{-1/2}e^{-\lambda f(\mathbf{x}_{0})}$, where $H$ is the Hessian at the critical point $\mathbf{x}_{0}$.
  The Hessian determinant controls the prefactor and thus the one-loop (semiclassical) correction in quantum mechanics and quantum field theory.
  For path integrals, the Hessian becomes a functional determinant (the fluctuation operator), connecting to the Fredholm determinant of Section~14.315.
\end{enumerate}

\paragraph{Mathematics applications.}
\begin{enumerate}
\item \textbf{Morse theory and topology of level sets.}%
  \index{Morse theory!Hessian}%
  \index{critical point!non-degenerate}%
  \index{CW decomposition}%
  A critical point of a smooth function $f$ is non-degenerate if $\det H\neq 0$ (i.e., the Hessian is non-singular).
  The Morse lemma states that near such a point, $f$ can be written as $f=f(\mathbf{x}_{0})-x_{1}^{2}-\cdots-x_{\lambda}^{2}+x_{\lambda+1}^{2}+\cdots+x_{n}^{2}$ in suitable coordinates, where $\lambda$ is the Morse index.
  The fundamental theorem of Morse theory builds the topology of $\{f\leq c\}$ by attaching a $\lambda$-handle at each critical point, yielding a CW decomposition of the manifold.

\item \textbf{Monge--Amp\`ere equation.}%
  \index{Monge--Amp\`ere equation}%
  \index{Hessian determinant!PDE}%
  \index{optimal transport!Monge--Amp\`ere}%
  The Monge--Amp\`ere equation $\det(\partial^{2}u/\partial x_{i}\partial x_{j})=f(\mathbf{x})$ prescribes the Hessian determinant as a given function.
  It arises in optimal transport (Brenier's theorem: the optimal map is the gradient of a convex function solving Monge--Amp\`ere), in affine differential geometry (affine spheres), and in the prescribed Gauss curvature problem.
  The theory of viscosity solutions (Caffarelli) gives existence and regularity under natural convexity assumptions.

\item \textbf{Convexity and the Hessian.}%
  \index{convexity!Hessian criterion}%
  \index{positive semidefinite!Hessian}%
  \index{second-order optimality}%
  A twice-differentiable function $f$ is convex if and only if its Hessian is positive semidefinite everywhere.
  The second-order sufficient condition for a local minimum at $\mathbf{x}_{0}$ is $\nabla f(\mathbf{x}_{0})=0$ and $H(\mathbf{x}_{0})\succ 0$ (positive definite), which implies $\det H>0$ and all principal minors positive.
  For constrained optimisation, the bordered Hessian determinant conditions replace Sylvester's criterion.
\end{enumerate}

%% -------------------------------------------------------------------
\subsubsection{14.315\quad Wronskian determinants}

The Wronskian of $n$ functions $y_{1},\ldots,y_{n}$ is
\[
  W(y_{1},\ldots,y_{n})(x)=\det\begin{pmatrix}
  y_{1} & y_{2} & \cdots & y_{n}\\
  y_{1}' & y_{2}' & \cdots & y_{n}'\\
  \vdots & \vdots & \ddots & \vdots\\
  y_{1}^{(n-1)} & y_{2}^{(n-1)} & \cdots & y_{n}^{(n-1)}
  \end{pmatrix}.
\]
If $y_{1},\ldots,y_{n}$ are solutions of an $n$-th order linear ODE, then $W\neq 0$ at one point implies linear independence.

\paragraph{Physics applications.}
\begin{enumerate}
\item \textbf{Abel's identity and conservation in quantum mechanics.}%
  \index{Abel's identity}%
  \index{Wronskian!Abel}%
  \index{probability current}%
  \index{continuity equation!quantum}%
  Abel's identity for a second-order ODE $y''+p(x)y'+q(x)y=0$ states that $W(y_{1},y_{2})(x)=W_{0}\exp(-\int^{x}p(t)\,dt)$.
  For the Schr\"odinger equation ($p=0$), the Wronskian $W=\psi_{1}\psi_{2}'-\psi_{2}\psi_{1}'$ is constant, which is the one-dimensional form of probability current conservation.
  The constancy of the Wronskian ensures that the probability current $j=\frac{\hbar}{2mi}W(\psi,\psi^{*})$ satisfies the continuity equation.

\item \textbf{Sturm--Liouville theory and eigenfunction expansions.}%
  \index{Sturm--Liouville theory}%
  \index{eigenfunction expansion}%
  \index{Green's function!Wronskian}%
  The Green's function for the Sturm--Liouville operator $Ly=-(py')'+qy$ on $[a,b]$ is constructed from two linearly independent solutions $y_{1}$, $y_{2}$ satisfying different boundary conditions:
  $G(x,\xi)=\frac{y_{1}(x_{<})y_{2}(x_{>})}{p(\xi)W(y_{1},y_{2})(\xi)}$.
  The Wronskian in the denominator ensures correct normalisation and encodes the self-adjointness of the operator.
  The eigenfunction expansion theorem (Sturm--Liouville) guarantees completeness of the eigenfunctions, generalising Fourier series.

\item \textbf{Transfer matrices and scattering in one dimension.}%
  \index{transfer matrix!scattering}%
  \index{scattering!one-dimensional}%
  \index{transmission coefficient}%
  For the one-dimensional Schr\"odinger equation, the transfer matrix $M$ relates the wavefunction and its derivative at two points: $\bigl(\begin{smallmatrix}\psi(b)\\\psi'(b)\end{smallmatrix}\bigr)=M\bigl(\begin{smallmatrix}\psi(a)\\\psi'(a)\end{smallmatrix}\bigr)$.
  The Wronskian constancy implies $\det M=1$ (unit determinant), which yields the relation $|t|^{2}+|r|^{2}=1$ between transmission and reflection coefficients (unitarity of scattering).
  For a sequence of barriers, $\det(M_{1}M_{2}\cdots M_{N})=1$ by multiplicativity, ensuring conservation of probability through any layered structure.

\item \textbf{Variation of parameters.}%
  \index{variation of parameters}%
  \index{Wronskian!variation of parameters}%
  \index{inhomogeneous ODE}%
  The particular solution of the inhomogeneous ODE $y^{(n)}+\cdots=f(x)$ is given by variation of parameters:
  $y_{p}(x)=\sum_{k=1}^{n}y_{k}(x)\int\frac{W_{k}(\xi)}{W(\xi)}f(\xi)\,d\xi$,
  where $W_{k}$ is the Wronskian with the $k$-th column replaced by $(0,\ldots,0,1)^{T}$.
  The non-vanishing of $W$ (guaranteed for a fundamental set) ensures the method produces a valid solution.
  This generalises the integrating-factor method to arbitrary-order linear ODEs.
\end{enumerate}

\paragraph{Mathematics applications.}
\begin{enumerate}
\item \textbf{Linear independence of analytic functions.}%
  \index{linear independence!Wronskian}%
  \index{analytic function!Wronskian}%
  \index{Bostan--Dumas theorem}%
  For analytic functions, $W(y_{1},\ldots,y_{n})\equiv 0$ on an interval implies linear dependence over $\mathbb{R}$---this is the Wronskian criterion for linear dependence.
  The converse fails for merely smooth functions (Peano's counterexample: $y_{1}=x^{2}$, $y_{2}=x|x|$ are linearly independent with $W\equiv 0$), but holds for solutions of linear ODEs with continuous coefficients.
  This distinction is important in the theory of differential Galois groups.

\item \textbf{Differential algebra and Picard--Vessiot theory.}%
  \index{Picard--Vessiot theory}%
  \index{differential Galois group}%
  \index{Wronskian!differential algebra}%
  The Wronskian is a differential algebraic invariant: if $y_{1},\ldots,y_{n}$ satisfy $y^{(n)}+a_{n-1}y^{(n-1)}+\cdots+a_{0}y=0$, then $W'=-a_{n-1}W$ (Abel's identity generalised).
  The Picard--Vessiot extension is the differential field generated by a fundamental set of solutions, and its differential Galois group is the algebraic subgroup of $\mathrm{GL}(n)$ preserving the Wronskian relations.
  This theory classifies which linear ODEs are solvable in terms of elementary functions, Liouvillian extensions, or algebraic functions.

\item \textbf{Oscillation theory and Sturm comparison.}%
  \index{oscillation theory}%
  \index{Sturm comparison theorem}%
  \index{zero counting!Wronskian}%
  Sturm's comparison theorem uses the Wronskian to compare solutions of two ODEs: if $y''+q_{1}y=0$ and $z''+q_{2}z=0$ with $q_{1}<q_{2}$, then between any two consecutive zeros of $y$, there is at least one zero of $z$.
  The proof uses $\frac{d}{dx}(zy'-yz')=(q_{2}-q_{1})yz$ (the Wronskian derivative) and the intermediate value theorem.
  Sturm oscillation theory extends this to count eigenvalues below a given level for Sturm--Liouville problems.
\end{enumerate}

%% -------------------------------------------------------------------
\subsubsection{14.316\quad Properties}

This subsection collects general properties shared by special determinants: behaviour under row and column operations, evaluation by recursion, and product formulas that arise from underlying algebraic or combinatorial structure.

\paragraph{Physics applications.}
\begin{enumerate}
\item \textbf{Determinantal point processes in quantum mechanics.}%
  \index{determinantal point process}%
  \index{fermion!correlation function}%
  \index{random matrix theory!correlation}%
  The $k$-point correlation function of free fermions at zero temperature is
  $\rho_{k}(x_{1},\ldots,x_{k})=\det[K(x_{i},x_{j})]_{i,j=1}^{k}$,
  where $K$ is the one-particle density matrix (projection kernel).
  This determinantal structure encodes the Pauli exclusion principle statistically: the joint probability of finding particles at $x_{1},\ldots,x_{k}$ factorises into a determinant, ensuring anti-bunching.
  Random matrix eigenvalue statistics are the prototypical determinantal point process with sine kernel $K(x,y)=\sin\pi(x-y)/[\pi(x-y)]$.

\item \textbf{Fredholm determinant and spectral zeta functions.}%
  \index{Fredholm determinant}%
  \index{spectral zeta function}%
  \index{functional determinant!regularised}%
  \index{quantum field theory!one-loop}%
  The Fredholm determinant of a trace-class operator $K$ on a Hilbert space is
  $\det(I-zK)=\exp\bigl(-\sum_{n=1}^{\infty}\frac{z^{n}}{n}\mathrm{tr}\,K^{n}\bigr)$.
  This infinite-dimensional generalisation of the finite determinant arises in the exact solution of quantum integrable models (e.g., the Tracy--Widom distribution for the largest eigenvalue of a random matrix).
  In quantum field theory, one-loop partition functions are regularised Fredholm determinants, computed via the spectral zeta function $\zeta_{A}(s)=\sum\lambda_{n}^{-s}$ with $\det A=\exp(-\zeta_{A}'(0))$.

\item \textbf{Permanents and bosonic systems.}%
  \index{permanent!boson}%
  \index{boson!permanent}%
  \index{boson sampling}%
  The permanent $\mathrm{perm}(A)=\sum_{\sigma}\prod_{i}a_{i\sigma(i)}$ (no sign factor) plays the role for bosons that the determinant plays for fermions.
  The $N$-boson wavefunction in orbitals $\phi_{1},\ldots,\phi_{N}$ is proportional to $\mathrm{perm}[\phi_{i}(x_{j})]$, which is symmetric under particle exchange.
  Computing the permanent is \#P-hard (Valiant's theorem), unlike the determinant which is in P; this complexity gap underlies the proposed computational advantage of boson sampling experiments.
\end{enumerate}

\paragraph{Mathematics applications.}
\begin{enumerate}
\item \textbf{Determinantal identities and Schur complements.}%
  \index{determinantal identity}%
  \index{Schur complement!identity}%
  \index{matrix determinant lemma}%
  The matrix determinant lemma $\det(A+\mathbf{u}\mathbf{v}^{T})=(1+\mathbf{v}^{T}A^{-1}\mathbf{u})\det A$ is the rank-one case of the more general identity $\det(A+UBV)=\det A\cdot\det(B^{-1}+VA^{-1}U)\cdot\det B$.
  These identities are proved by writing the augmented block matrix and taking the Schur complement.
  They enable efficient determinant updates in Monte Carlo simulations, where the matrix changes by a low-rank perturbation at each step.

\item \textbf{Cauchy determinant.}%
  \index{Cauchy determinant}%
  \index{partial fraction decomposition}%
  \index{Hilbert matrix|see{Cauchy determinant}}%
  The Cauchy determinant $\det\bigl[\frac{1}{x_{i}+y_{j}}\bigr]=\frac{\prod_{i<j}(x_{j}-x_{i})(y_{j}-y_{i})}{\prod_{i,j}(x_{i}+y_{j})}$ generalises the Vandermonde and appears in partial fraction decompositions, integrable systems (KP hierarchy), and random matrix theory (Cauchy ensemble).
  The Hilbert matrix $H_{ij}=1/(i+j-1)$ is a special case ($x_{i}=i-1/2$, $y_{j}=j-1/2$), and its determinant has the closed form $\det H_{n}=\prod_{k=1}^{n}(k-1)!^{4}/((2k-1)\cdot(2k-2)!^{2})$.

\item \textbf{Determinants and generating functions.}%
  \index{generating function!determinantal}%
  \index{Jacobi--Trudi identity}%
  \index{Young tableau!determinant}%
  The Jacobi--Trudi identity expresses the Schur polynomial as a determinant of complete homogeneous symmetric polynomials: $s_{\lambda}=\det[h_{\lambda_{i}-i+j}]$.
  This determinantal formula connects the theory of symmetric functions to the representation theory of $\mathrm{GL}(n)$ and $S_{n}$ and gives generating functions for the number of Young tableaux of a given shape.
  The dual Jacobi--Trudi identity uses elementary symmetric polynomials: $s_{\lambda'}=\det[e_{\lambda_{i}'-i+j}]$, where $\lambda'$ is the conjugate partition.
\end{enumerate}

%% -------------------------------------------------------------------
\subsubsection{14.317\quad Gram-Kowalewski theorem on linear dependence}

The Gram determinant (Gramian) of vectors $v_{1},\ldots,v_{k}$ in an inner product space is $G=\det[\langle v_{i},v_{j}\rangle]$.
The Gram--Kowalewski theorem states that $G=0$ if and only if $v_{1},\ldots,v_{k}$ are linearly dependent, and $G>0$ when the vectors are linearly independent (in a real inner product space).

\paragraph{Physics applications.}
\begin{enumerate}
\item \textbf{Linear independence of quantum states.}%
  \index{quantum state!linear independence}%
  \index{Gram matrix!quantum}%
  \index{overlap matrix}%
  \index{non-orthogonal basis!Gram}%
  In quantum mechanics, the overlap matrix $S_{ij}=\langle\phi_{i}|\phi_{j}\rangle$ of a set of (possibly non-orthogonal) basis functions is the Gram matrix.
  The condition $\det S>0$ ensures linear independence and is checked routinely in basis-set quantum chemistry.
  When $\det S$ is small, the basis is nearly linearly dependent, causing numerical instability (``basis set superposition error''); the L\"owdin orthogonalisation $|\tilde{\phi}\rangle=S^{-1/2}|\phi\rangle$ remedies this.

\item \textbf{Signal detection and matched subspace detectors.}%
  \index{matched subspace detector}%
  \index{Gram determinant!signal detection}%
  \index{GLRT!subspace}%
  In array signal processing, the generalised likelihood ratio test (GLRT) for detecting a signal in a $k$-dimensional subspace involves the ratio $\det(S_{\mathrm{signal}})/\det(S_{\mathrm{noise}})$ of Gram determinants.
  The Gram--Kowalewski theorem ensures that this ratio is well-defined (positive) when the signal vectors are linearly independent.
  The test statistic is related to the product of canonical correlations and to the volumes of projected parallelotopes.

\item \textbf{Strain and deformation in continuum mechanics.}%
  \index{strain tensor!Gram}%
  \index{deformation gradient}%
  \index{Cauchy--Green tensor}%
  The right Cauchy--Green deformation tensor $C=F^{T}F$, where $F$ is the deformation gradient, is the Gram matrix of the deformed basis vectors.
  Its determinant $\det C=(\det F)^{2}=J^{2}$ is the square of the volume ratio, and $\det C>0$ (guaranteed by Gram--Kowalewski for linearly independent columns of $F$) ensures that the deformation does not collapse the material to a lower-dimensional subspace.
  The principal stretches are $\sqrt{\lambda_{i}}$ where $\lambda_{i}$ are the eigenvalues of $C$.
\end{enumerate}

\paragraph{Mathematics applications.}
\begin{enumerate}
\item \textbf{Volume of $k$-dimensional parallelotopes.}%
  \index{parallelotope!volume}%
  \index{Gram determinant!volume}%
  \index{Pythagorean theorem!higher-dimensional}%
  The $k$-dimensional volume of the parallelotope spanned by $v_{1},\ldots,v_{k}\in\mathbb{R}^{n}$ ($k\leq n$) is $\mathrm{vol}_{k}=\sqrt{\det G}$.
  For $k=n$, this reduces to $|\det A|$ where $A=[v_{1}|\cdots|v_{n}]$.
  The formula $\det G=\sum_{|S|=k}(\det A_{S})^{2}$ (by Cauchy--Binet, where $A_{S}$ is the $k\times k$ submatrix of rows indexed by $S$) is the higher-dimensional Pythagorean theorem: the squared volume of a $k$-parallelotope in $\mathbb{R}^{n}$ equals the sum of squares of its projections onto all coordinate $k$-planes.

\item \textbf{Lattice theory and the geometry of numbers.}%
  \index{lattice!Gram matrix}%
  \index{geometry of numbers}%
  \index{Minkowski's theorem!lattice}%
  A lattice $\Lambda=\{n_{1}v_{1}+\cdots+n_{k}v_{k}:n_{i}\in\mathbb{Z}\}$ has fundamental volume $\mathrm{vol}(\Lambda)=\sqrt{\det G}$.
  Minkowski's theorem states that a convex symmetric body of volume greater than $2^{k}\mathrm{vol}(\Lambda)$ contains a non-zero lattice point.
  The LLL lattice basis reduction algorithm seeks a basis minimising $\det G$ (which is invariant under unimodular transformations but whose individual entries $G_{ij}$ can be reduced), and its efficiency is measured by the orthogonality defect $\det G/\prod\|v_{i}\|^{2}$.

\item \textbf{Reproducing kernel Hilbert spaces.}%
  \index{reproducing kernel Hilbert space}%
  \index{kernel!positive definite}%
  \index{Gram matrix!RKHS}%
  \index{Mercer's theorem}%
  In a reproducing kernel Hilbert space (RKHS) with kernel $K(x,y)$, the Gram matrix of evaluation functionals at points $x_{1},\ldots,x_{n}$ is $G_{ij}=K(x_{i},x_{j})$.
  By Mercer's theorem, $G$ is positive semidefinite, and $\det G\geq 0$ with equality if and only if the evaluation points are linearly dependent in the feature space.
  The Gram determinant appears in the power function for interpolation error bounds: $|f(x)-s_{n}(x)|\leq\|f\|_{\mathcal{H}}\sqrt{K(x,x)-\mathbf{k}^{T}G^{-1}\mathbf{k}}$.

\item \textbf{Hadamard--Fischer inequality.}%
  \index{Hadamard--Fischer inequality}%
  \index{positive definite!block inequality}%
  \index{Gram determinant!inequality}%
  For a positive definite matrix $A$ partitioned as $A=\bigl(\begin{smallmatrix}A_{11}&A_{12}\\A_{21}&A_{22}\end{smallmatrix}\bigr)$, the Hadamard--Fischer inequality states $\det A\leq\det A_{11}\det A_{22}$, with equality if and only if $A_{12}=0$.
  This refines Hadamard's inequality (which is the case of $1\times 1$ diagonal blocks) and bounds the Gram determinant of a full set in terms of Gram determinants of subsets.
  It is the determinantal counterpart of the subadditivity of entropy.
\end{enumerate}
