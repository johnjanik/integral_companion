%% ============================================================
%% 0  Introduction
%% ============================================================
\section{0\quad Introduction}

\subsection{0.1\quad Finite Sums}

%% -------------------------------------------------------------------
\subsubsection{0.11\quad Progressions}

Arithmetic and geometric progressions---the simplest closed-form sums---underpin
an extraordinary range of applications whenever discrete accumulation or
repeated multiplication is modelled.

\paragraph{Physics applications.}
\begin{enumerate}
\item \textbf{Quantum harmonic oscillator partition function.}%
  \index{partition function!quantum harmonic oscillator}%
  \index{geometric series!partition function}%
  \index{quantum harmonic oscillator}%
  \index{Planck distribution}%
  The canonical partition function of a quantum harmonic oscillator is
  the geometric series
  \[
    Z = \sum_{n=0}^{\infty} e^{-\beta\hbar\omega(n+1/2)}
      = \frac{e^{-\beta\hbar\omega/2}}{1-e^{-\beta\hbar\omega}},
  \]
  from which the Planck distribution, zero-point energy, and
  the entire thermodynamics of lattice vibrations (phonons) follow
  directly \cite{Pathria2011}.

\item \textbf{Signal processing and the Shannon sampling kernel.}%
  \index{Shannon sampling theorem}%
  \index{Dirichlet kernel}%
  \index{signal processing!sampling}%
  \index{discrete Fourier transform}%
  The finite geometric sum
  $\sum_{k=0}^{N-1}e^{i k\theta}=(1-e^{iN\theta})/(1-e^{i\theta})$
  gives the Dirichlet kernel, which controls spectral leakage in the
  discrete Fourier transform and appears in the proof of the Shannon
  sampling theorem.

\item \textbf{Geometric optics and thin-film interference.}%
  \index{thin-film interference}%
  \index{Fabry--P\'erot interferometer}%
  \index{geometric series!optics}%
  Each partial reflection in a Fabry--P\'{e}rot cavity contributes a
  factor~$r^{2}e^{i\delta}$; the total transmitted amplitude is a
  geometric series whose sum gives the Airy function describing the
  interference fringes used in laser cavity design and spectroscopy.

\item \textbf{Discrete compounding and present value.}%
  \index{present value!geometric series}%
  \index{annuity}%
  \index{financial mathematics}%
  The present value of an annuity paying~$C$ for~$n$ periods at rate~$r$
  is $C\,(1-(1+r)^{-n})/r$, a geometric sum.  This formula is the
  foundation of bond pricing, mortgage amortisation, and discounted cash
  flow analysis.
\end{enumerate}

\paragraph{Mathematics applications.}
\begin{enumerate}
\item \textbf{Analytic continuation and regularisation of divergent series.}%
  \index{analytic continuation!geometric series}%
  \index{zeta regularisation}%
  \index{Ramanujan summation}%
  The geometric series $\sum_{n=0}^{\infty}x^{n}=1/(1-x)$ for $|x|<1$
  provides the prototype for analytic continuation: evaluating the
  right-hand side at $x=-1$ gives $1/2$, matching the Abel/Ces\`{a}ro
  sum $1-1+1-1+\cdots=1/2$, which underlies zeta-regularised sums in
  physics.

\item \textbf{$p$-adic absolute value and non-archimedean analysis.}%
  \index{p-adic@$p$-adic numbers}%
  \index{non-archimedean analysis}%
  \index{Hensel's lemma}%
  Over the $p$-adic numbers $\mathbb{Q}_p$, the geometric series
  $\sum p^n$ converges to $1/(1-p)$, illustrating that convergence
  depends on the chosen absolute value.  This is the entry point to
  $p$-adic analysis and Hensel's lemma.

\item \textbf{Fractal geometry and self-similarity.}%
  \index{fractal!self-similarity}%
  \index{Hausdorff dimension}%
  \index{Cantor set}%
  The total length removed in constructing the Cantor set is the
  geometric series $\sum_{k=0}^{\infty}2^k/3^{k+1}=1$, while the
  Hausdorff dimension $\log 2/\log 3$ comes from the scaling ratio
  of the geometric progression of covering intervals.
\end{enumerate}

%% -------------------------------------------------------------------
\subsubsection{0.12\quad Sums of powers of natural numbers}

\paragraph{Physics applications.}
\begin{enumerate}
\item \textbf{Bernoulli numbers and the Casimir effect.}%
  \index{Bernoulli numbers!power sums}%
  \index{Casimir effect!zeta regularisation}%
  \index{zeta function!negative integers}%
  \index{Faulhaber's formula}%
  Faulhaber's formula expresses $\sum_{k=1}^{n}k^{p}$ as a polynomial
  in~$n$ with Bernoulli number coefficients.  The analytic continuation
  $\zeta(-p)=(-1)^{p}B_{p+1}/(p+1)$ relates these sums to the Riemann
  zeta function at negative integers, which appears in the zeta-regularised
  Casimir energy between conducting plates \cite{Elizalde1995}.

\item \textbf{Euler--Maclaurin summation in lattice simulations.}%
  \index{Euler--Maclaurin formula}%
  \index{lattice QCD}%
  \index{numerical integration!trapezoidal rule}%
  \index{lattice sums}%
  The Euler--Maclaurin formula bridges discrete sums and integrals:
  $\sum_{k=a}^{b}f(k)=\int_{a}^{b}f(x)\,dx+\tfrac{1}{2}[f(a)+f(b)]
  +\sum_{j=1}^{p}\frac{B_{2j}}{(2j)!}[f^{(2j-1)}(b)-f^{(2j-1)}(a)]
  +R_{p}$.
  This controls discretisation error in lattice QCD, numerical quadrature,
  and Madelung-constant calculations for crystal lattices.

\item \textbf{Debye model and low-temperature specific heat.}%
  \index{Debye model!power sums}%
  \index{specific heat!low temperature}%
  \index{phonon density of states}%
  In the Debye model, the phonon contribution to specific heat at low
  temperature involves sums $\sum n^{2}e^{-n}$ which, via
  Euler--Maclaurin or direct evaluation, lead to the $T^{3}$ law and the
  Debye function $D_{n}(x)=\frac{n}{x^{n}}\int_{0}^{x}\frac{t^{n}}{e^{t}-1}\,dt$.
\end{enumerate}

\paragraph{Mathematics applications.}
\begin{enumerate}
\item \textbf{Todd class and the Hirzebruch--Riemann--Roch theorem.}%
  \index{Todd class}%
  \index{Hirzebruch--Riemann--Roch theorem}%
  \index{algebraic topology!Todd genus}%
  The generating function $x/(1-e^{-x})=\sum_{k=0}^{\infty}(-1)^{k}B_{k}x^{k}/k!$
  defines the Todd class in algebraic topology.  The Hirzebruch--Riemann--Roch
  theorem computes the Euler characteristic of coherent sheaves on smooth
  projective varieties as an integral of the Todd class---Bernoulli numbers
  encode the topology of complex manifolds.

\item \textbf{Umbral calculus and finite operator methods.}%
  \index{umbral calculus}%
  \index{finite differences}%
  \index{Bernoulli polynomials!umbral calculus}%
  In the umbral calculus, $B^{n}$ is formally replaced by $B_{n}$
  (the $n$-th Bernoulli number), turning the identity
  $(B+1)^{n}=B^{n}$ into the recurrence for Bernoulli numbers.
  This technique extends to Appell polynomials and Sheffer sequences,
  providing a systematic framework for finite-difference identities.

\item \textbf{Analytic number theory: Euler--Maclaurin and $\zeta(s)$.}%
  \index{Riemann zeta function!Euler--Maclaurin computation}%
  \index{analytic continuation!Riemann zeta}%
  The Euler--Maclaurin formula applied to $f(x)=x^{-s}$ provides
  both the analytic continuation of $\zeta(s)$ to $\mathrm{Re}(s)>-2p$
  and efficient numerical computation of $\zeta(s)$ on the critical
  strip, as used in verification of the Riemann Hypothesis for
  trillions of zeros.
\end{enumerate}

%% -------------------------------------------------------------------
\subsubsection{0.13\quad Sums of reciprocals of natural numbers}

\paragraph{Physics applications.}
\begin{enumerate}
\item \textbf{Harmonic numbers and the coupon collector problem.}%
  \index{harmonic numbers!coupon collector}%
  \index{coupon collector problem}%
  \index{statistical mechanics!information}%
  The expected number of trials to collect all~$n$ distinct types is
  $nH_{n}=n\sum_{k=1}^{n}1/k\sim n\ln n$, where $H_{n}$ is the $n$-th
  harmonic number.  This arises in sampling theory, randomised algorithms,
  and the statistical mechanics of site-occupation models.

\item \textbf{Renormalisation group logarithms.}%
  \index{renormalisation group!logarithms}%
  \index{harmonic numbers!perturbation theory}%
  \index{QCD!anomalous dimensions}%
  In perturbative quantum field theory, harmonic sums
  $S_{1}(n)=\sum_{k=1}^{n}1/k$ and their nested generalisations appear
  as Mellin-space representations of splitting functions governing parton
  evolution in QCD \cite{Vermaseren1999}.

\item \textbf{Diffusion on networks.}%
  \index{random walk!expected return time}%
  \index{resistance distance}%
  \index{graph Laplacian}%
  The expected commute time between nodes $i$ and $j$ on a graph is
  proportional to the effective resistance, which for certain lattices
  involves partial harmonic sums.  The divergence of $H_{n}$ reflects
  the recurrence of the one-dimensional random walk.
\end{enumerate}

\paragraph{Mathematics applications.}
\begin{enumerate}
\item \textbf{Euler--Mascheroni constant.}%
  \index{Euler--Mascheroni constant!definition}%
  \index{harmonic numbers!asymptotic expansion}%
  \index{Stieltjes constants}%
  $\gamma=\lim_{n\to\infty}(H_{n}-\ln n)=0.57721\ldots$ is one of the
  fundamental constants of analysis.  It appears as the constant term in
  the Laurent expansion $\zeta(s)=1/(s-1)+\gamma+O(s-1)$ and generates
  the Stieltjes constants $\gamma_{k}$.

\item \textbf{Digamma function at positive integers.}%
  \index{digamma function!at integers}%
  \index{partial fractions!series summation}%
  The identity $\psi(n+1)=-\gamma+H_{n}$ connects the harmonic numbers
  to the digamma function, enabling the closed-form evaluation of any
  convergent series $\sum P(n)/Q(n)$ with rational terms via partial
  fractions (cf.\ G\&R~6.46).

\item \textbf{Mertens' theorems and prime distribution.}%
  \index{Mertens' theorems}%
  \index{prime numbers!sum of reciprocals}%
  $\sum_{p\leq x}1/p=\ln\ln x+M+O(1/\ln x)$, where $M$ is the
  Meissel--Mertens constant.  The divergence of the sum of prime
  reciprocals (Euler, 1737) was the first result connecting harmonic-type
  sums to the distribution of primes.
\end{enumerate}

%% -------------------------------------------------------------------
\subsubsection{0.14\quad Sums of products of reciprocals of natural numbers}

\paragraph{Physics applications.}
\begin{enumerate}
\item \textbf{Nested harmonic sums in higher-order QCD.}%
  \index{harmonic sums!nested}%
  \index{DGLAP splitting functions}%
  \index{QCD!higher-order corrections}%
  Products of reciprocals generate nested sums
  $S_{a,b,\ldots}(n)=\sum_{k=1}^{n}k^{-a}S_{b,\ldots}(k-1)$,
  which appear at two-loop and three-loop order in DGLAP splitting
  functions for deep inelastic scattering \cite{Vermaseren1999}.

\item \textbf{Perturbation theory in quantum mechanics.}%
  \index{perturbation theory!second order}%
  \index{energy denominators}%
  Second-order energy corrections
  $E_{n}^{(2)}=\sum_{m\neq n}|V_{mn}|^{2}/(E_{n}^{(0)}-E_{m}^{(0)})$
  produce sums of products of reciprocals when the unperturbed spectrum
  is harmonic or Coulombic, evaluated using partial-fraction identities
  from G\&R~0.14.

\item \textbf{Correlation functions in statistical mechanics.}%
  \index{correlation function!cumulant expansion}%
  \index{cluster expansion}%
  \index{virial coefficients}%
  Cluster and virial expansions express thermodynamic quantities as sums
  over products of pairwise interactions, leading to products of reciprocal
  powers when the interaction has power-law form.  The combinatorial
  structure mirrors that of multiple zeta values.
\end{enumerate}

\paragraph{Mathematics applications.}
\begin{enumerate}
\item \textbf{Multiple zeta values.}%
  \index{multiple zeta values}%
  \index{Zagier conjecture}%
  \index{Kontsevich integrals}%
  The multiple zeta values
  $\zeta(s_{1},\ldots,s_{k})=\sum_{n_{1}>\cdots>n_{k}\geq 1}
  n_{1}^{-s_{1}}\cdots n_{k}^{-s_{k}}$ generalise products of
  reciprocal sums.  They satisfy algebraic relations (stuffle and
  shuffle products) and appear in Kontsevich integrals, knot invariants,
  and periods of mixed Tate motives.

\item \textbf{Partial fraction decomposition.}%
  \index{partial fractions!Heaviside method}%
  \index{rational function integration}%
  Products of reciprocals $1/[n(n+1)\cdots(n+k)]$ are the discrete
  analogues of partial fractions, evaluated by telescoping.  This is the
  discrete prototype for the Heaviside cover-up method used in integration
  of rational functions (Section~2.1).

\item \textbf{Stirling numbers and combinatorial identities.}%
  \index{Stirling numbers!harmonic number identities}%
  \index{combinatorial identities}%
  Products of reciprocals arise in the expansion
  $\binom{x}{n}=\sum_{k}s(n,k)x^{k}/n!$, where $s(n,k)$ are Stirling
  numbers of the first kind.  Identities for sums of products of
  reciprocals underpin the theory of finite differences and the calculus
  of factorial powers.
\end{enumerate}

%% -------------------------------------------------------------------
\subsubsection{0.15\quad Sums of the binomial coefficients}

\paragraph{Physics applications.}
\begin{enumerate}
\item \textbf{Lattice paths and random walks.}%
  \index{random walk!lattice paths}%
  \index{binomial coefficients!lattice paths}%
  \index{diffusion!discrete}%
  \index{polymer physics!random walk models}%
  The number of paths of length~$n$ on $\mathbb{Z}^{d}$ returning to the
  origin is $\binom{n}{n/2}^{d}$ (suitably interpreted).  Binomial
  coefficient sums control the probability of return and the mean-square
  displacement $\langle r^{2}\rangle=n$ in discrete diffusion, with
  applications to polymer physics and Brownian motion.

\item \textbf{Catalan numbers and Dyck paths.}%
  \index{Catalan numbers}%
  \index{Dyck paths}%
  \index{parenthesisation}%
  \index{quantum gravity!planar maps}%
  $C_{n}=\binom{2n}{n}/(n+1)$ counts Dyck paths, non-crossing
  partitions, and planar binary trees.  In theoretical physics, Catalan
  numbers enumerate planar Feynman diagrams and triangulations of
  polygons relevant to $2d$ quantum gravity \cite{DiFrancesco1995}.

\item \textbf{Entropy of the binomial distribution.}%
  \index{binomial distribution!entropy}%
  \index{Stirling's approximation!binomial coefficient}%
  \index{information theory!binary entropy}%
  Using Stirling's approximation, $\ln\binom{n}{k}\approx nH(k/n)$
  where $H(p)=-p\ln p-(1-p)\ln(1-p)$ is the binary entropy.
  This asymptotic identity underlies the method of types in information
  theory and the derivation of the microcanonical ensemble in
  statistical mechanics.
\end{enumerate}

\paragraph{Mathematics applications.}
\begin{enumerate}
\item \textbf{Vandermonde's identity and hypergeometric foundations.}%
  \index{Vandermonde's identity}%
  \index{hypergeometric series!Chu--Vandermonde}%
  \index{Chu--Vandermonde identity}%
  The Chu--Vandermonde identity
  $\sum_{k}\binom{m}{k}\binom{n}{p-k}=\binom{m+n}{p}$ is equivalent
  to the evaluation ${}_2F_1(-n,b;c;1)=(c-b)_n/(c)_n$, the simplest
  non-trivial hypergeometric identity and the starting point for the
  Wilf--Zeilberger method of automatic proof.

\item \textbf{Central binomial coefficients and $\pi$.}%
  \index{central binomial coefficient}%
  \index{pi@$\pi$!series involving binomial coefficients}%
  \index{Ramanujan series!for $1/\pi$}%
  The asymptotic $\binom{2n}{n}\sim 4^{n}/\sqrt{\pi n}$ connects
  binomial sums to~$\pi$.  Ramanujan-type series
  $1/\pi=\sum_{n=0}^{\infty}\binom{2n}{n}^{3}a_{n}/b^{n}$ converge
  extremely rapidly and are used in modern record computations of~$\pi$.

\item \textbf{Generating functions and the binomial transform.}%
  \index{generating functions!binomial coefficients}%
  \index{binomial transform}%
  \index{Euler transform}%
  The binomial transform $b_{n}=\sum_{k=0}^{n}\binom{n}{k}a_{k}$
  is an involution on sequences, related to the Euler transform for
  accelerating alternating series.  It connects the ordinary and
  exponential generating functions via the Borel correspondence.
\end{enumerate}

\subsection{0.2\quad Numerical Series and Infinite Products}

%% -------------------------------------------------------------------
\subsubsection{0.21\quad The convergence of numerical series}

\paragraph{Physics applications.}
\begin{enumerate}
\item \textbf{Perturbation series and asymptotic convergence.}%
  \index{perturbation series!asymptotic}%
  \index{QED!perturbation series}%
  \index{Dyson's argument}%
  \index{Borel summability}%
  Most perturbation series in quantum field theory are asymptotic
  (divergent) rather than convergent.  Dyson's argument (1952) shows
  that the QED perturbation series has zero radius of convergence,
  yet its partial sums give predictions accurate to $10^{-12}$.
  Borel summability and resurgence theory provide rigorous meaning to
  such series.

\item \textbf{Convergence of lattice sums (Madelung constants).}%
  \index{Madelung constant}%
  \index{lattice sums!conditional convergence}%
  \index{Ewald summation}%
  The electrostatic energy of an ionic crystal involves the conditionally
  convergent Madelung sum $\sum'q_{j}/r_{j}$.  The order of summation
  matters: the Ewald summation technique splits the series into rapidly
  convergent parts in real and reciprocal space.

\item \textbf{Renormalisation and removal of divergences.}%
  \index{renormalisation!divergent series}%
  \index{ultraviolet divergence}%
  \index{infrared divergence}%
  In QFT, loop integrals produce divergent series that must be
  regularised (dimensional, cutoff, zeta) and renormalised.
  Understanding the convergence properties of the regulated series
  is essential for extracting finite physical predictions.
\end{enumerate}

\paragraph{Mathematics applications.}
\begin{enumerate}
\item \textbf{Absolute vs.\ conditional convergence.}%
  \index{conditional convergence}%
  \index{absolute convergence}%
  \index{Riemann rearrangement theorem}%
  The Riemann rearrangement theorem states that a conditionally convergent
  series can be rearranged to converge to any prescribed value.
  This motivates the distinction between absolute and conditional
  convergence, crucial for justifying term-by-term operations on series.

\item \textbf{Banach space completeness.}%
  \index{Banach space!completeness}%
  \index{Cauchy sequences}%
  \index{absolute convergence!completeness criterion}%
  A normed space is complete (Banach) if and only if every absolutely
  convergent series converges.  This characterisation is the basis for
  proving completeness of $L^{p}$ spaces, $C[a,b]$, and other function
  spaces central to analysis and PDE theory.

\item \textbf{Summability methods.}%
  \index{summability methods}%
  \index{Ces\`aro summation}%
  \index{Abel summation}%
  Ces\`{a}ro, Abel, and Borel summation extend the notion of convergence
  to assign values to divergent series.  A regular summability method
  (Silverman--Toeplitz theorem) must assign the usual sum to any
  convergent series, ensuring consistency.
\end{enumerate}

%% -------------------------------------------------------------------
\subsubsection{0.22\quad Convergence tests}

\paragraph{Physics applications.}
\begin{enumerate}
\item \textbf{Convergence of partition functions.}%
  \index{partition function!convergence}%
  \index{Hagedorn temperature}%
  \index{string theory!Hagedorn temperature}%
  The ratio test applied to $Z=\sum_{n}g(n)e^{-\beta E_{n}}$ determines
  the radius of convergence in $\beta$; the Hagedorn temperature in
  string theory is the value where the exponential growth of states
  overcomes the Boltzmann suppression and $Z$ diverges.

\item \textbf{Radius of convergence of virial expansions.}%
  \index{virial expansion!convergence}%
  \index{equation of state}%
  \index{Lee--Yang theorem}%
  The virial expansion $P/k_{B}T=\rho+B_{2}\rho^{2}+B_{3}\rho^{3}+\cdots$
  has a finite radius of convergence related to the closest singularity
  in the complex fugacity plane (Lee--Yang theorem).  The ratio and root
  tests estimate where the equation of state breaks down.

\item \textbf{Convergence of multipole expansions.}%
  \index{multipole expansion!convergence}%
  \index{electrostatics!multipole}%
  The multipole expansion of a potential
  $\phi(\mathbf{r})=\sum_{\ell=0}^{\infty}A_{\ell}r^{-(\ell+1)}P_{\ell}(\cos\theta)$
  converges for $r>r_{\max}$ (the radius of the smallest enclosing
  sphere).  The comparison test with a geometric series establishes the
  convergence rate.
\end{enumerate}

\paragraph{Mathematics applications.}
\begin{enumerate}
\item \textbf{Cauchy's root test and the Cauchy--Hadamard theorem.}%
  \index{root test}%
  \index{Cauchy--Hadamard theorem}%
  \index{radius of convergence}%
  The radius of convergence $R=1/\limsup|a_{n}|^{1/n}$ (Cauchy--Hadamard)
  generalises the root test to power series and is the basis for
  determining domains of analyticity of complex functions.

\item \textbf{Kummer's test and the hypergeometric series.}%
  \index{Kummer's test}%
  \index{hypergeometric series!convergence}%
  \index{Gauss's test}%
  For the hypergeometric series ${}_2F_1(a,b;c;1)$, Gauss showed
  convergence if and only if $\mathrm{Re}(c-a-b)>0$.  Kummer's and
  Raabe's tests handle the borderline cases and extend to generalised
  hypergeometric series ${}_pF_q$.

\item \textbf{Condensation test and number-theoretic series.}%
  \index{Cauchy condensation test}%
  \index{prime number series}%
  Cauchy's condensation test reduces $\sum f(n)$ to $\sum 2^{n}f(2^{n})$,
  providing quick proofs that $\sum 1/n$ diverges while
  $\sum 1/(n\ln^{2}n)$ converges.  These techniques extend to
  Dirichlet series and the convergence abscissa of $L$-functions.
\end{enumerate}

%% -------------------------------------------------------------------
\subsubsection{0.23--0.24\quad Examples of numerical series}

\paragraph{Physics applications.}
\begin{enumerate}
\item \textbf{Basel problem and quantum field theory.}%
  \index{Basel problem}%
  \index{zeta function!$\zeta(2)$}%
  \index{Casimir effect!one dimension}%
  Euler's result $\sum_{n=1}^{\infty}1/n^{2}=\pi^{2}/6$ ($=\zeta(2)$)
  appears in the one-dimensional Casimir energy calculation and in the
  blackbody radiation formula.  More generally, $\zeta(2k)$ gives rational
  multiples of $\pi^{2k}$ via Bernoulli numbers.

\item \textbf{Leibniz series and the Dirichlet beta function.}%
  \index{Leibniz series}%
  \index{Dirichlet beta function}%
  \index{Catalan's constant}%
  The alternating series $1-1/3+1/5-\cdots=\pi/4$ is $\beta(1)$, where
  $\beta(s)=\sum_{n=0}^{\infty}(-1)^{n}(2n+1)^{-s}$ is the Dirichlet
  beta function.  The value $\beta(2)=G$ (Catalan's constant) appears in
  combinatorics, hyperbolic geometry, and the Green's function of the
  two-dimensional lattice.

\item \textbf{Ap\'ery's constant and electron anomalous magnetic moment.}%
  \index{Ap\'ery's constant}%
  \index{zeta function!$\zeta(3)$}%
  \index{anomalous magnetic moment!electron}%
  $\zeta(3)=1.20205\ldots$ appears in the three-loop QED correction to
  the electron $g-2$ and in the free energy of the three-dimensional
  Ising model.  Ap\'{e}ry's 1978 proof that $\zeta(3)$ is irrational
  remains one of the landmarks of 20th-century number theory.
\end{enumerate}

\paragraph{Mathematics applications.}
\begin{enumerate}
\item \textbf{Euler's evaluation of $\zeta(2k)$.}%
  \index{Riemann zeta function!even integer values}%
  \index{Bernoulli numbers!$\zeta(2k)$}%
  $\zeta(2k)=(-1)^{k+1}(2\pi)^{2k}B_{2k}/[2(2k)!]$, connecting
  the series $\sum n^{-2k}$ to Bernoulli numbers and~$\pi$.  This
  family of identities is the simplest instance of the general
  theory of special values of $L$-functions.

\item \textbf{Irrationality and transcendence.}%
  \index{irrationality!$\zeta(3)$}%
  \index{transcendence!$\pi$, $e$}%
  \index{Lindemann--Weierstrass theorem}%
  Series representations provide irrationality proofs: Fourier's proof
  that $e$ is irrational uses the rapidly convergent series $e=\sum 1/n!$,
  while Ap\'{e}ry's proof for $\zeta(3)$ uses accelerated series.  The
  Lindemann--Weierstrass theorem (proving $\pi$ transcendental) relies on
  the exponential series.

\item \textbf{Acceleration of convergence.}%
  \index{series acceleration}%
  \index{Euler--Maclaurin formula!series acceleration}%
  \index{Richardson extrapolation}%
  Slowly convergent series are accelerated by the Euler transform,
  Richardson extrapolation, or the Levin $u$-transform.  These methods
  are essential in computational mathematics for evaluating special
  function values from their defining series.
\end{enumerate}

%% -------------------------------------------------------------------
\subsubsection{0.25\quad Infinite products}

\paragraph{Physics applications.}
\begin{enumerate}
\item \textbf{Euler product for $\zeta(s)$ and the prime number theorem.}%
  \index{Euler product!Riemann zeta}%
  \index{prime number theorem}%
  \index{Riemann hypothesis}%
  $\zeta(s)=\prod_{p\,\text{prime}}(1-p^{-s})^{-1}$ for $\mathrm{Re}(s)>1$
  connects the analytic properties of $\zeta$ to the distribution of
  primes.  The non-vanishing of $\zeta(1+it)$ (an infinite-product
  property) is the key step in the proof of the prime number theorem.

\item \textbf{Partition function as infinite product.}%
  \index{partition function!infinite product}%
  \index{Euler's partition identity}%
  \index{bosonic string theory!partition function}%
  \index{Dedekind eta function}%
  The generating function for integer partitions is
  $\prod_{n=1}^{\infty}(1-q^{n})^{-1}=\sum_{n=0}^{\infty}p(n)q^{n}$,
  related to the Dedekind eta function $\eta(\tau)=q^{1/24}\prod(1-q^{n})$.
  In bosonic string theory, $1/\eta(\tau)^{24}$ gives the one-loop
  partition function.

\item \textbf{Weierstrass factorisation and spectral determinants.}%
  \index{Weierstrass factorisation}%
  \index{spectral determinant}%
  \index{functional determinant!infinite product}%
  Spectral determinants $\det(\Delta-\lambda)=\prod_{n}(\lambda_{n}-\lambda)$
  are infinite products over eigenvalues.  Regularised via zeta functions,
  they compute path-integral measures in quantum mechanics and one-loop
  partition functions in QFT.

\item \textbf{Pentagonal number theorem and combinatorics.}%
  \index{pentagonal number theorem}%
  \index{Euler's pentagonal theorem}%
  \index{partition function!pentagonal}%
  Euler's pentagonal number theorem
  $\prod_{n=1}^{\infty}(1-q^{n})=\sum_{k=-\infty}^{\infty}(-1)^{k}q^{k(3k-1)/2}$
  is a prototype for Jacobi's triple product and Macdonald identities,
  which appear in affine Lie algebra character formulas.
\end{enumerate}

\paragraph{Mathematics applications.}
\begin{enumerate}
\item \textbf{Weierstrass product theorem.}%
  \index{Weierstrass product theorem}%
  \index{entire functions!canonical product}%
  \index{genus (entire function)}%
  Every entire function with prescribed zeros $\{a_{n}\}$ can be written as
  $e^{g(z)}\prod E_{p}(z/a_{n})$ with canonical factors $E_{p}$.  This
  theorem is the starting point for the Hadamard factorisation theorem
  and the theory of entire functions of finite order.

\item \textbf{Jacobi triple product.}%
  \index{Jacobi triple product}%
  \index{theta functions!product formula}%
  \index{modular forms}%
  $\sum_{n=-\infty}^{\infty}z^{n}q^{n^{2}}=\prod_{n=1}^{\infty}(1-q^{2n})(1+zq^{2n-1})(1+z^{-1}q^{2n-1})$
  connects theta-function series to infinite products and is foundational
  for the theory of modular forms, elliptic functions, and combinatorial
  identities.

\item \textbf{Blaschke products and Hardy spaces.}%
  \index{Blaschke product}%
  \index{Hardy spaces}%
  \index{inner functions}%
  A Blaschke product $B(z)=\prod(z-a_{n})/(1-\bar{a}_{n}z)$ converges
  in the unit disc whenever $\sum(1-|a_{n}|)<\infty$.  These are the
  inner functions in Hardy space $H^{2}$, and the factorisation
  $f=B\cdot S\cdot F$ (Blaschke, singular inner, outer) is the structure
  theorem for bounded analytic functions.
\end{enumerate}

%% -------------------------------------------------------------------
\subsubsection{0.26\quad Examples of infinite products}

\paragraph{Physics applications.}
\begin{enumerate}
\item \textbf{Virasoro characters and modular invariance.}%
  \index{Virasoro algebra!characters}%
  \index{modular invariance}%
  \index{conformal field theory!partition function}%
  Characters of Virasoro algebra representations take the form
  $\chi(q)=q^{h-c/24}\prod_{n=1}^{\infty}(1-q^{n})^{-1}$,
  and modular invariance of the partition function
  $Z=\sum|\chi_{i}|^{2}$ constrains the spectrum of $2d$ conformal
  field theories.

\item \textbf{Wallis product and quantum tunnelling.}%
  \index{Wallis product}%
  \index{quantum tunnelling!WKB}%
  $\pi/2=\prod_{n=1}^{\infty}4n^{2}/(4n^{2}-1)$ (Wallis, 1655) is the
  simplest infinite product for~$\pi$.  Its structure appears in WKB
  connection formulas for quantum tunnelling through multiple barriers,
  where products of transmission coefficients accumulate.

\item \textbf{Dielectric function as infinite product.}%
  \index{dielectric function!plasma}%
  \index{Drude model}%
  The frequency-dependent dielectric function of a plasma with multiple
  resonances can be written as a product over poles and zeros,
  $\varepsilon(\omega)=\varepsilon_{\infty}\prod(\omega^{2}-\omega_{L,j}^{2})/(\omega^{2}-\omega_{T,j}^{2})$
  (Lyddane--Sachs--Teller relation), directly using infinite-product
  representations.
\end{enumerate}

\paragraph{Mathematics applications.}
\begin{enumerate}
\item \textbf{Euler's sine product and $\zeta(2k)$.}%
  \index{Euler's sine product}%
  \index{Riemann zeta function!via sine product}%
  $\sin(\pi z)/(\pi z)=\prod_{n=1}^{\infty}(1-z^{2}/n^{2})$ gives
  $\zeta(2k)$ by expanding $\ln\sin(\pi z)$ and comparing coefficients
  with Newton's identities relating power sums to elementary symmetric
  functions.

\item \textbf{Gamma function reciprocal.}%
  \index{gamma function!Weierstrass product}%
  \index{entire functions!order one}%
  $1/\Gamma(z)=ze^{\gamma z}\prod_{n=1}^{\infty}(1+z/n)e^{-z/n}$
  exhibits $1/\Gamma$ as an entire function of order~1 and genus~1.
  This product is the prototype for understanding the growth and zero
  distribution of entire functions.

\item \textbf{Pochhammer symbols and rising factorials.}%
  \index{Pochhammer symbol}%
  \index{rising factorial}%
  \index{hypergeometric series!Pochhammer}%
  The Pochhammer symbol $(a)_{n}=a(a+1)\cdots(a+n-1)=\Gamma(a+n)/\Gamma(a)$
  is a finite product that forms the building blocks of hypergeometric
  series.  Infinite products of ratios of Pochhammer symbols arise in
  Ramanujan-type product formulas for special constants.
\end{enumerate}

\subsection{0.3\quad Functional Series}

%% -------------------------------------------------------------------
\subsubsection{0.30\quad Definitions and theorems}

\paragraph{Physics applications.}
\begin{enumerate}
\item \textbf{Uniform convergence and interchange of limits.}%
  \index{uniform convergence}%
  \index{interchange of limits}%
  \index{thermodynamic limit!interchange}%
  Physicists routinely interchange sums, integrals, and derivatives.
  Uniform convergence (Weierstrass $M$-test) is the standard criterion
  justifying these operations.  In statistical mechanics, the interchange
  $\lim_{N\to\infty}\sum=\int$ in the thermodynamic limit requires
  careful convergence analysis.

\item \textbf{Normal modes and eigenfunction expansions.}%
  \index{normal modes}%
  \index{eigenfunction expansion}%
  \index{Sturm--Liouville theory}%
  Every solution of a linear PDE (wave equation, heat equation,
  Schr\"{o}dinger equation) on a bounded domain expands in eigenfunctions
  of the associated Sturm--Liouville operator.  Convergence theorems
  (pointwise, $L^{2}$, uniform) determine when the expansion faithfully
  represents the solution.

\item \textbf{Born series in scattering theory.}%
  \index{Born series}%
  \index{scattering theory!Born approximation}%
  \index{Neumann series}%
  The Born series $\psi=\psi_{0}+G_{0}V\psi_{0}+G_{0}VG_{0}V\psi_{0}+\cdots$
  is a Neumann series for the resolvent $(1-G_{0}V)^{-1}$.  It converges
  when $\|G_{0}V\|<1$ (weak potential), and its radius of convergence
  determines the breakdown of perturbative scattering theory.
\end{enumerate}

\paragraph{Mathematics applications.}
\begin{enumerate}
\item \textbf{Weierstrass approximation theorem.}%
  \index{Weierstrass approximation theorem}%
  \index{Stone--Weierstrass theorem}%
  \index{Bernstein polynomials}%
  Every continuous function on $[a,b]$ is the uniform limit of
  polynomials.  The constructive proof via Bernstein polynomials
  produces an explicit functional series.  The Stone--Weierstrass
  generalisation applies to any separating subalgebra of $C(X)$.

\item \textbf{Runge's theorem and rational approximation.}%
  \index{Runge's theorem}%
  \index{rational approximation}%
  \index{Pad\'e approximants}%
  Every function holomorphic on a compact set $K\subset\mathbb{C}$
  with connected complement is a uniform limit of polynomials (Runge).
  Pad\'{e} approximants give optimal rational function series that often
  converge beyond the disc of convergence of the Taylor series.

\item \textbf{Equicontinuity and the Arzel\`a--Ascoli theorem.}%
  \index{Arzel\`a--Ascoli theorem}%
  \index{equicontinuity}%
  \index{compactness!in function spaces}%
  The Arzel\`{a}--Ascoli theorem characterises compact subsets of
  $C[a,b]$: bounded and equicontinuous families have convergent
  subsequences.  This underpins existence proofs for ODEs (Peano's
  theorem) and the direct method in the calculus of variations.
\end{enumerate}

%% -------------------------------------------------------------------
\subsubsection{0.31\quad Power series}

\paragraph{Physics applications.}
\begin{enumerate}
\item \textbf{Taylor expansion and linearisation.}%
  \index{Taylor expansion!linearisation}%
  \index{small oscillations}%
  \index{harmonic approximation}%
  Nearly every physical theory begins with a Taylor expansion:
  $V(x)\approx V(x_{0})+\tfrac{1}{2}V''(x_{0})(x-x_{0})^{2}$ gives the
  harmonic approximation for small oscillations.  Higher-order terms
  yield anharmonic corrections, treated perturbatively.

\item \textbf{Generating functions in statistical mechanics.}%
  \index{generating functions!statistical mechanics}%
  \index{grand canonical ensemble}%
  \index{fugacity expansion}%
  The grand partition function $\Xi(z,T)=\sum_{N=0}^{\infty}z^{N}Z_{N}(T)$
  is a power series in the fugacity~$z$, whose radius of convergence
  determines the phase structure (Lee--Yang theory).

\item \textbf{Multipole and virial expansions.}%
  \index{multipole expansion!power series}%
  \index{virial expansion}%
  \index{equation of state!power series}%
  Both the multipole expansion of electrostatic potentials (in $1/r$)
  and the virial expansion of the equation of state (in density~$\rho$)
  are power series whose coefficients encode the physics of interactions
  at successive orders.
\end{enumerate}

\paragraph{Mathematics applications.}
\begin{enumerate}
\item \textbf{Analytic functions and the identity theorem.}%
  \index{analytic functions!identity theorem}%
  \index{identity theorem}%
  \index{analytic continuation!uniqueness}%
  A function analytic on a connected domain is uniquely determined by its
  Taylor coefficients at any point.  The identity theorem---if two analytic
  functions agree on a set with an accumulation point, they are
  identical---is the foundation for analytic continuation.

\item \textbf{Radius of convergence and singularity analysis.}%
  \index{radius of convergence!singularity}%
  \index{Pringsheim's theorem}%
  \index{analytic combinatorics!transfer theorems}%
  By Pringsheim's theorem, a power series with non-negative coefficients
  has a singularity at $z=R$ (its radius of convergence).  In analytic
  combinatorics, the nature of this singularity (pole, branch point,
  essential) determines the asymptotic growth of the coefficients via
  transfer theorems \cite{FlajoletSedgewick2009}.

\item \textbf{Formal power series and algebraic combinatorics.}%
  \index{formal power series}%
  \index{algebraic combinatorics}%
  \index{composition of series}%
  The ring of formal power series $\mathbb{Q}[[x]]$ has rich algebraic
  structure: composition, inversion (Lagrange), and the plethystic
  exponential.  Formal series enumerate combinatorial objects (trees,
  graphs, permutations) without convergence concerns.
\end{enumerate}

%% -------------------------------------------------------------------
\subsubsection{0.32\quad Fourier series}

\paragraph{Physics applications.}
\begin{enumerate}
\item \textbf{Heat equation and Fourier's original problem.}%
  \index{heat equation!Fourier series solution}%
  \index{Fourier, Joseph!heat conduction}%
  \index{thermal diffusion}%
  Fourier's 1807 solution of the heat equation
  $\partial_{t}u=\kappa\partial_{x}^{2}u$ on $[0,L]$ as
  $u(x,t)=\sum a_{n}\sin(n\pi x/L)\,e^{-\kappa(n\pi/L)^{2}t}$
  was the historical origin of Fourier series and one of the most
  consequential developments in mathematical physics.

\item \textbf{Quantum mechanics on a circle.}%
  \index{quantum mechanics!particle on a circle}%
  \index{Bloch's theorem}%
  \index{band structure}%
  \index{Brillouin zone}%
  The energy eigenstates of a particle on a ring are $e^{in\theta}$,
  and any wave function expands as a Fourier series.  In solid-state
  physics, Bloch's theorem says that eigenstates in a periodic potential
  have the form $u_{k}(x)e^{ikx}$ with $u_{k}$ periodic---i.e., a
  Fourier series modulated by a plane wave.

\item \textbf{Signal processing and spectral analysis.}%
  \index{spectral analysis}%
  \index{signal processing!Fourier series}%
  \index{Gibbs phenomenon}%
  Fourier series decompose periodic signals into harmonics.  The Gibbs
  phenomenon (9\% overshoot at discontinuities) limits the accuracy of
  truncated Fourier representations and motivates windowing techniques
  and sigma-factor smoothing in digital signal processing.

\item \textbf{Crystallography and diffraction.}%
  \index{crystallography!structure factor}%
  \index{X-ray diffraction}%
  \index{Bragg's law}%
  The electron density in a crystal is a three-dimensional Fourier series
  $\rho(\mathbf{r})=\sum_{\mathbf{G}}F_{\mathbf{G}}e^{i\mathbf{G}\cdot\mathbf{r}}$
  summed over reciprocal lattice vectors.  The Fourier coefficients
  $F_{\mathbf{G}}$ (structure factors) are measured in X-ray diffraction
  experiments.
\end{enumerate}

\paragraph{Mathematics applications.}
\begin{enumerate}
\item \textbf{$L^{2}$ convergence and Parseval's theorem.}%
  \index{Parseval's theorem}%
  \index{L2 convergence@$L^2$ convergence}%
  \index{orthonormal basis}%
  $\sum|c_{n}|^{2}=\frac{1}{2\pi}\int_{0}^{2\pi}|f(\theta)|^{2}\,d\theta$
  (Parseval) expresses the fact that $\{e^{in\theta}\}$ is an orthonormal
  basis for $L^{2}([0,2\pi])$.  This is the prototype for all Hilbert
  space expansions.

\item \textbf{Pointwise convergence and Carleson's theorem.}%
  \index{Carleson's theorem}%
  \index{pointwise convergence!Fourier series}%
  \index{Dirichlet kernel}%
  Carleson's theorem (1966) shows that the Fourier series of an
  $L^{2}$ function converges pointwise almost everywhere---settling a
  question open since Fourier.  The proof introduced techniques
  (time-frequency analysis) that became foundational in harmonic analysis.

\item \textbf{Equidistribution and Weyl's theorem.}%
  \index{equidistribution}%
  \index{Weyl's equidistribution theorem}%
  \index{ergodic theory}%
  Weyl's criterion: the sequence $\{n\alpha\}$ is equidistributed
  mod~1 if and only if $\sum e^{2\pi i n\alpha}/N\to 0$ for all
  non-zero frequencies.  This connects Fourier analysis to ergodic
  theory and Diophantine approximation.
\end{enumerate}

%% -------------------------------------------------------------------
\subsubsection{0.33\quad Asymptotic series}

\paragraph{Physics applications.}
\begin{enumerate}
\item \textbf{WKB approximation in quantum mechanics.}%
  \index{WKB approximation}%
  \index{semiclassical mechanics!WKB}%
  \index{Bohr--Sommerfeld quantisation}%
  \index{tunnelling (quantum)!WKB}%
  The WKB series $\psi(x)\sim\exp\bigl[\frac{i}{\hbar}\sum_{n=0}^{\infty}
  \hbar^{n}S_{n}(x)\bigr]$ is asymptotic in $\hbar\to 0$: the leading
  terms give the Bohr--Sommerfeld quantisation condition and tunnelling
  rates, while the series diverges if summed to all orders.

\item \textbf{Stirling's series and statistical mechanics.}%
  \index{Stirling's series}%
  \index{statistical mechanics!Stirling}%
  \index{combinatorial approximation}%
  $\ln\Gamma(z)\sim z\ln z-z-\frac{1}{2}\ln z+\frac{1}{2}\ln(2\pi)
  +\sum_{k=1}^{\infty}\frac{B_{2k}}{2k(2k-1)z^{2k-1}}$
  is the prototypical asymptotic series.  Though divergent, truncating
  optimally gives exponentially small error (superasymptotic
  approximation), essential for high-precision thermodynamic calculations.

\item \textbf{Resurgence and non-perturbative physics.}%
  \index{resurgence}%
  \index{trans-series}%
  \index{instanton}%
  \index{non-perturbative effects}%
  Resurgence theory shows that the large-order behaviour of an asymptotic
  series encodes non-perturbative information (instantons, renormalons).
  The divergent perturbation series is the ``tip of the iceberg'' of a
  trans-series combining all saddle-point contributions.
\end{enumerate}

\paragraph{Mathematics applications.}
\begin{enumerate}
\item \textbf{Poincar\'e's definition and Watson's lemma.}%
  \index{Poincar\'e asymptotic expansion}%
  \index{Watson's lemma}%
  \index{Laplace method}%
  Poincar\'{e} (1886) formalised asymptotic series:
  $f(z)\sim\sum a_{n}z^{-n}$ means
  $z^{N}[f(z)-\sum_{n=0}^{N-1}a_{n}z^{-n}]\to a_{N}$.
  Watson's lemma derives such expansions for Laplace-type integrals
  $\int_{0}^{\infty}t^{\lambda-1}e^{-zt}\phi(t)\,dt$.

\item \textbf{Stokes phenomenon and exponential asymptotics.}%
  \index{Stokes phenomenon}%
  \index{exponential asymptotics}%
  \index{connection formulas}%
  As $\arg z$ varies, subdominant exponentials switch on/off across Stokes
  lines, changing the form of the asymptotic expansion.  The Stokes
  phenomenon explains the different connection formulas for Airy, Bessel,
  and other special functions in different sectors of the complex plane.

\item \textbf{Borel summation.}%
  \index{Borel summation}%
  \index{Borel transform}%
  \index{Gevrey class}%
  The Borel transform $\hat{f}(\zeta)=\sum a_{n}\zeta^{n}/n!$ often
  converges even when $\sum a_{n}z^{-n}$ diverges.  If $\hat{f}$ has
  no singularities on $[0,\infty)$, the Laplace integral
  $\int_{0}^{\infty}\hat{f}(\zeta)e^{-z\zeta}\,d\zeta$ recovers the
  original function---this is Borel summation, applicable to Gevrey-class
  asymptotic series.
\end{enumerate}

\subsection{0.4\quad Certain Formulas from Differential Calculus}

%% -------------------------------------------------------------------
\subsubsection{0.41\quad Differentiation of a definite integral with respect to a parameter}

\paragraph{Physics applications.}
\begin{enumerate}
\item \textbf{Feynman's trick (differentiation under the integral sign).}%
  \index{Feynman's trick}%
  \index{differentiation under the integral sign}%
  \index{Leibniz integral rule}%
  \index{parameter integrals}%
  Feynman's favourite technique: introduce a parameter into an integral,
  differentiate to simplify, then integrate back.  For example,
  $\int_{0}^{\infty}\frac{\sin x}{x}\,dx=\frac{\pi}{2}$ follows from
  differentiating $I(\alpha)=\int_{0}^{\infty}\frac{\sin x}{x}e^{-\alpha x}\,dx$
  with respect to~$\alpha$.

\item \textbf{Schwinger parametrisation in QFT.}%
  \index{Schwinger parametrisation}%
  \index{Feynman parameters}%
  \index{parameter differentiation!QFT}%
  Differentiating $\int_{0}^{\infty}\alpha^{n-1}e^{-\alpha m^{2}}\,d\alpha
  =\Gamma(n)/m^{2n}$ with respect to~$m^{2}$ generates the higher-power
  propagators needed in multi-loop calculations.  This is the
  parameter-differentiation approach to Feynman integrals.

\item \textbf{Hellmann--Feynman theorem.}%
  \index{Hellmann--Feynman theorem}%
  \index{quantum chemistry!forces}%
  \index{Born--Oppenheimer approximation}%
  If $H(\lambda)|\psi(\lambda)\rangle=E(\lambda)|\psi(\lambda)\rangle$,
  then $\partial E/\partial\lambda=\langle\psi|\partial H/\partial\lambda|\psi\rangle$.
  This is differentiation of the ``integral'' $E=\langle\psi|H|\psi\rangle$
  with respect to a parameter, and it gives forces on nuclei in the
  Born--Oppenheimer framework of quantum chemistry.

\item \textbf{Thermodynamic Maxwell relations.}%
  \index{Maxwell relations}%
  \index{thermodynamic potentials}%
  \index{equation of state!parameter derivatives}%
  Differentiating thermodynamic potentials (which are integrals over phase
  space or partition-function derivatives) with respect to parameters
  $(T,P,V,\mu)$ yields the Maxwell relations and equations of state.
\end{enumerate}

\paragraph{Mathematics applications.}
\begin{enumerate}
\item \textbf{Leibniz integral rule with variable limits.}%
  \index{Leibniz integral rule!variable limits}%
  \index{fundamental theorem of calculus!generalised}%
  $\frac{d}{d\alpha}\int_{a(\alpha)}^{b(\alpha)}f(x,\alpha)\,dx
  =\int_{a}^{b}\partial_{\alpha}f\,dx+f(b,\alpha)b'(\alpha)
  -f(a,\alpha)a'(\alpha)$.
  This generalises the fundamental theorem of calculus and is the key
  tool for deriving variational equations and optimal control conditions.

\item \textbf{Dominated convergence and measure-theoretic formulation.}%
  \index{dominated convergence theorem}%
  \index{measure theory!parameter integrals}%
  The rigorous justification for differentiation under the integral sign
  is Lebesgue's dominated convergence theorem: if $|\partial_{\alpha}f|$
  is bounded by an integrable function uniformly in~$\alpha$, the
  interchange is valid.

\item \textbf{Green's functions and parameter dependence.}%
  \index{Green's function!parameter dependence}%
  \index{resolvent!derivative}%
  The resolvent $R(\lambda)=(A-\lambda)^{-1}=\int(A-\lambda)^{-1}$ of an
  operator satisfies $R'(\lambda)=R(\lambda)^{2}$, a parameter
  differentiation identity.  This is used throughout spectral theory and
  perturbation theory for linear operators.
\end{enumerate}

%% -------------------------------------------------------------------
\subsubsection{0.42\quad The nth derivative of a product (Leibniz's rule)}

\paragraph{Physics applications.}
\begin{enumerate}
\item \textbf{Moyal star product and deformation quantisation.}%
  \index{Moyal star product}%
  \index{deformation quantisation}%
  \index{Wigner function}%
  \index{phase space!quantum mechanics}%
  The Moyal star product $f\star g=\sum_{n=0}^{\infty}\frac{1}{n!}
  \bigl(\frac{i\hbar}{2}\bigr)^{n}\{f,g\}_{n}$, where $\{f,g\}_{n}$
  involves $n$-th order bidifferential operators (a generalised Leibniz
  rule), implements quantum mechanics on phase space.  The Wigner function
  $W(x,p)$ evolves under the Moyal bracket $\{f,g\}_{\star}=
  (f\star g-g\star f)/(i\hbar)$.

\item \textbf{Pseudodifferential operators and quantum observables.}%
  \index{pseudodifferential operators}%
  \index{symbol calculus}%
  \index{Weyl quantisation}%
  The composition of pseudodifferential operators (the symbol calculus)
  uses the Leibniz rule for the product of symbols:
  $\sigma(AB)\sim\sum_{\alpha}\frac{1}{\alpha!}\partial_{\xi}^{\alpha}
  \sigma_{A}\,D_{x}^{\alpha}\sigma_{B}$.
  This asymptotic expansion underlies Weyl quantisation and microlocal
  analysis.

\item \textbf{Higher-order perturbation theory.}%
  \index{perturbation theory!higher order}%
  \index{Rayleigh--Schr\"odinger perturbation theory}%
  \index{Leibniz rule!perturbation series}%
  In Rayleigh--Schr\"{o}dinger perturbation theory, the $n$-th order
  correction to $\langle\psi|H|\psi\rangle$ requires derivatives of
  products of wavefunctions and operators.  The general Leibniz rule
  organises these corrections systematically.

\item \textbf{Electromagnetic multipole radiation.}%
  \index{multipole radiation}%
  \index{electromagnetic radiation!multipole}%
  \index{spherical harmonics!derivatives}%
  The $n$-th derivative of the product $r^{\ell}Y_{\ell}^{m}(\theta,\phi)
  \cdot f(r)$ (using Leibniz's rule) generates the coupling between
  angular momentum channels in multipole radiation theory.
\end{enumerate}

\paragraph{Mathematics applications.}
\begin{enumerate}
\item \textbf{Leibniz rule for fractional derivatives.}%
  \index{fractional calculus!Leibniz rule}%
  \index{Riemann--Liouville derivative}%
  \index{fractional Leibniz rule}%
  The Leibniz rule extends to fractional derivatives:
  $D^{\alpha}(fg)=\sum_{k=0}^{\infty}\binom{\alpha}{k}D^{\alpha-k}f\,D^{k}g$,
  where $\binom{\alpha}{k}=\Gamma(\alpha+1)/[\Gamma(k+1)\Gamma(\alpha-k+1)]$.
  This is fundamental in fractional calculus and anomalous diffusion models.

\item \textbf{Fa\`a di Bruno via Leibniz iteration.}%
  \index{Fa\`a di Bruno's formula!via Leibniz}%
  \index{Bell polynomials!Leibniz connection}%
  Iterating the Leibniz rule for $(fg)^{(n)}$ with specific choices of
  $f$ and~$g$ recovers Fa\`{a} di Bruno's formula for the $n$-th
  derivative of a composite function, expressed via partial Bell
  polynomials $B_{n,k}$.

\item \textbf{D-module theory and differential algebra.}%
  \index{D-module theory}%
  \index{differential algebra}%
  \index{Weyl algebra}%
  The Leibniz rule $[D,f]=f'$ (where $D=d/dx$) defines the Weyl
  algebra $A_{1}=\mathbb{C}\langle x,D\rangle/(Dx-xD-1)$, the
  simplest non-commutative ring.  $D$-module theory studies systems of
  linear PDEs through this algebraic structure.
\end{enumerate}

%% -------------------------------------------------------------------
\subsubsection{0.43\quad The nth derivative of a composite function}

Fa\`{a} di Bruno's formula gives the $n$-th derivative of a composite
function $f(g(x))$ in terms of the derivatives of $f$ and~$g$:
\[
  \frac{d^{n}}{dx^{n}}f(g(x))
  =\sum_{k=1}^{n}f^{(k)}(g(x))\,B_{n,k}\!\bigl(g'(x),g''(x),\ldots,
  g^{(n-k+1)}(x)\bigr),
\]
where $B_{n,k}$ are partial Bell polynomials.

\paragraph{Physics applications.}
\begin{enumerate}
\item \textbf{Connes--Kreimer Hopf algebra of renormalisation.}%
  \index{Connes--Kreimer Hopf algebra}%
  \index{renormalisation!combinatorial}%
  \index{Hopf algebra!Feynman diagrams}%
  \index{BPHZ renormalisation}%
  The combinatorial structure of BPHZ renormalisation is encoded in
  a Hopf algebra on rooted trees (Connes--Kreimer, 1998), whose
  antipode (the counterterm map) is governed by Fa\`{a} di Bruno's
  formula \cite{ConnesKreimer2000}.  The composition of counterterms
  at nested subdivergences follows the Bell-polynomial structure exactly.

\item \textbf{Cumulant expansion in statistical mechanics.}%
  \index{cumulants!expansion}%
  \index{linked cluster theorem}%
  \index{moment-cumulant relation}%
  The moment--cumulant relation
  $\langle e^{tX}\rangle=\exp[\sum_{n=1}^{\infty}\kappa_{n}t^{n}/n!]$
  is inverted by Fa\`{a} di Bruno's formula:
  $\kappa_{n}=\sum(-1)^{k-1}(k-1)!\,B_{n,k}(\mu_{1}',\ldots,\mu_{n-k+1}')$.
  This expansion is the mathematical basis for the linked cluster theorem
  in statistical mechanics and quantum field theory.

\item \textbf{Normal ordering and Wick's theorem.}%
  \index{normal ordering}%
  \index{Wick's theorem}%
  \index{creation and annihilation operators}%
  In quantum optics and QFT, expressing a function of the field operator
  in normal-ordered form requires repeated use of the chain rule for
  composites.  The combinatorics of contractions in Wick's theorem
  mirror the Bell-polynomial structure of Fa\`{a} di Bruno's formula.

\item \textbf{Formal group laws in algebraic topology.}%
  \index{formal group laws}%
  \index{cobordism theory}%
  \index{Lazard ring}%
  A formal group law $F(x,y)=x+y+\sum a_{ij}x^{i}y^{j}$ on a ring~$R$
  satisfies associativity conditions that, when composed and differentiated,
  require Fa\`{a} di Bruno's formula.  The universal formal group law
  (Lazard ring) classifies complex cobordism.
\end{enumerate}

\paragraph{Mathematics applications.}
\begin{enumerate}
\item \textbf{Bell polynomials and combinatorial species.}%
  \index{Bell polynomials}%
  \index{combinatorial species}%
  \index{set partitions!Bell polynomials}%
  The partial Bell polynomials $B_{n,k}$ count the number of ways to
  partition $\{1,\ldots,n\}$ into $k$ non-empty blocks with weights.
  They unify many combinatorial identities and are central to the theory
  of species of structures.

\item \textbf{Lagrange inversion formula.}%
  \index{Lagrange inversion formula}%
  \index{implicit function theorem!formal}%
  \index{tree enumeration}%
  The Lagrange inversion formula for the compositional inverse
  $f^{-1}$ is closely related to Fa\`{a} di Bruno's formula via
  the identity $[z^{n}]f^{-1}(z)=\frac{1}{n}[w^{n-1}](w/f(w))^{n}$.
  This enumerates labelled rooted trees (Cayley's formula $n^{n-1}$).

\item \textbf{Umbral calculus and Sheffer sequences.}%
  \index{Sheffer sequences}%
  \index{umbral calculus!Fa\`a di Bruno}%
  Fa\`{a} di Bruno's formula provides the connection constants between
  different Sheffer polynomial sequences.  The group of formal
  diffeomorphisms under composition is the Fa\`{a} di Bruno group,
  whose Lie algebra is related to the Virasoro algebra.
\end{enumerate}

%% -------------------------------------------------------------------
\subsubsection{0.44\quad Integration by substitution}

\paragraph{Physics applications.}
\begin{enumerate}
\item \textbf{Change of variables in path integrals.}%
  \index{path integral!change of variables}%
  \index{Jacobian!path integral}%
  \index{Faddeev--Popov procedure}%
  In functional integrals, the substitution
  $\mathcal{D}\phi'=|\det(\delta\phi'/\delta\phi)|\,\mathcal{D}\phi$
  introduces the Jacobian determinant.  The Faddeev--Popov ghost fields
  in gauge theory arise precisely from this determinant when fixing a
  gauge \cite{FaddeevPopov1967}.

\item \textbf{Canonical transformations in Hamiltonian mechanics.}%
  \index{canonical transformation}%
  \index{Hamiltonian mechanics}%
  \index{generating function!canonical}%
  \index{action-angle variables}%
  Canonical transformations $(q,p)\to(Q,P)$ preserve the symplectic
  form $dp\wedge dq=dP\wedge dQ$.  The integral $\oint p\,dq$ (action
  variable) is invariant under substitution, leading to action-angle
  variables that simplify integrable systems.

\item \textbf{Dimensional analysis and scaling.}%
  \index{dimensional analysis}%
  \index{Buckingham $\Pi$ theorem}%
  \index{renormalisation group!scaling}%
  The substitution $x=\lambda\tilde{x}$ (rescaling) in integrals
  reveals scaling dimensions.  The Buckingham~$\Pi$ theorem formalises
  this, and the renormalisation group extends scaling analysis to
  quantum field theory.

\item \textbf{Coordinate transformations in general relativity.}%
  \index{general relativity!coordinate transformations}%
  \index{metric tensor!transformation}%
  \index{covariance!general}%
  The principle of general covariance requires that physical laws be
  invariant under arbitrary coordinate substitutions.  The transformation
  of the volume element $\sqrt{|g|}\,d^{4}x$ under diffeomorphisms
  is the curved-spacetime version of the substitution rule.
\end{enumerate}

\paragraph{Mathematics applications.}
\begin{enumerate}
\item \textbf{Change of variables formula in $\mathbb{R}^{n}$.}%
  \index{change of variables!multiple integrals}%
  \index{Jacobian determinant}%
  \index{diffeomorphism}%
  For a $C^{1}$ diffeomorphism $\phi\colon U\to V$,
  $\int_{V}f(y)\,dy=\int_{U}f(\phi(x))\,|\det D\phi(x)|\,dx$.
  The absolute value of the Jacobian determinant measures local volume
  distortion.

\item \textbf{Euler substitutions for algebraic integrands.}%
  \index{Euler substitutions}%
  \index{algebraic integrands}%
  \index{rationalisation!of integrals}%
  The three Euler substitutions rationalise integrals containing
  $\sqrt{ax^{2}+bx+c}$, reducing them to integrals of rational functions
  (Section~2.2).  These are the prototypes for the uniformisation of
  algebraic curves.

\item \textbf{Measure-theoretic change of variables.}%
  \index{change of variables!measure theory}%
  \index{pushforward measure}%
  \index{Radon--Nikodym theorem}%
  The pushforward of a measure $\mu$ under a measurable map $T$ gives
  $\int f\,d(T_{*}\mu)=\int(f\circ T)\,d\mu$.  When $T$ is
  differentiable, the Radon--Nikodym derivative is $|\det DT|$, unifying
  the substitution rule with the abstract theory of measures.
\end{enumerate}
