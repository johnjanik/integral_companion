%% Section 13 — Matrices and Related Results
\section{13\quad Matrices and Related Results}

\subsection{13.11--13.12\quad Special Matrices}
\subsubsection{13.111\quad Diagonal matrix}

A diagonal matrix $D=\mathrm{diag}(d_{1},d_{2},\dots,d_{n})$ has entries
$D_{ij}=d_{i}\delta_{ij}$.  Diagonal matrices commute, and their algebra
is isomorphic to a direct product of fields.

\paragraph{Physics applications.}
\begin{enumerate}
\item \textbf{Normal modes of coupled oscillators.}%
  \index{diagonal matrix!normal modes}%
  \index{normal modes}%
  \index{coupled oscillators}%
  \index{decoupling|see{diagonal matrix}}%
  A system of $n$ coupled harmonic oscillators with mass matrix $M$ and
  stiffness matrix $K$ is diagonalised by the normal-mode transformation
  $\mathbf{q}=S\boldsymbol{\eta}$ such that $S^{T}KS=\mathrm{diag}
  (\omega_{1}^{2},\dots,\omega_{n}^{2})$ and $S^{T}MS=I$.
  The equations of motion decouple into independent oscillators
  $\ddot{\eta}_{k}+\omega_{k}^{2}\eta_{k}=0$, each with its own
  natural frequency.

\item \textbf{Quantum numbers and simultaneous observables.}%
  \index{diagonal matrix!quantum numbers}%
  \index{complete set of commuting observables}%
  \index{quantum numbers!diagonal representation}%
  A complete set of commuting observables (CSCO)
  $\{A_{1},\dots,A_{k}\}$ can be simultaneously diagonalised:
  $A_{i}|\mathbf{a}\rangle=a_{i}|\mathbf{a}\rangle$.  The eigenvalue
  tuple $(a_{1},\dots,a_{k})$ defines the quantum numbers that
  uniquely label states, such as $(n,\ell,m_{\ell},m_{s})$ for the
  hydrogen atom.

\item \textbf{Principal moments of inertia.}%
  \index{inertia tensor!diagonalisation}%
  \index{principal axes}%
  \index{rigid body dynamics!principal moments}%
  The inertia tensor $I_{ij}=\int\rho(\mathbf{r})(r^{2}\delta_{ij}
  -x_{i}x_{j})\,dV$ can be diagonalised by rotating to the principal
  axes, giving $I=\mathrm{diag}(I_{1},I_{2},I_{3})$.  Euler's
  equations $I_{1}\dot{\omega}_{1}-(I_{2}-I_{3})\omega_{2}\omega_{3}=N_{1}$
  (and cyclic) then describe rigid body rotation.
\end{enumerate}

\paragraph{Mathematics applications.}
\begin{enumerate}
\item \textbf{Spectral theory and functional calculus.}%
  \index{spectral theorem!diagonal form}%
  \index{functional calculus}%
  \index{matrix functions!via diagonalisation}%
  If $A=PDP^{-1}$ with $D=\mathrm{diag}(\lambda_{1},\dots,\lambda_{n})$,
  then $f(A)=Pf(D)P^{-1}=P\,\mathrm{diag}(f(\lambda_{1}),\dots,
  f(\lambda_{n}))P^{-1}$ for any function $f$ analytic on the spectrum.
  This is the finite-dimensional version of the spectral theorem and
  provides the foundation for the matrix exponential, logarithm, and
  square root.

\item \textbf{Simultaneous diagonalisation and commutativity.}%
  \index{simultaneous diagonalisation}%
  \index{commuting matrices}%
  \index{abelian algebra}%
  A set of matrices $\{A_{1},\dots,A_{k}\}$ can be simultaneously
  diagonalised by a single invertible matrix if and only if they are all
  diagonalisable and mutually commute: $[A_{i},A_{j}]=0$ for all
  $i,j$.  This characterises maximal abelian subalgebras of
  $\mathrm{Mat}_{n}(\mathbb{C})$.
\end{enumerate}

%% -------------------------------------------------------------------
\subsubsection{13.112\quad Identity matrix and null matrix}

The identity matrix $I_{n}$ has entries $(I_{n})_{ij}=\delta_{ij}$
and serves as the multiplicative identity in $\mathrm{Mat}_{n}$.
The null (zero) matrix $0_{n}$ has all entries zero and serves as the
additive identity.

\paragraph{Physics applications.}
\begin{enumerate}
\item \textbf{Completeness relation in quantum mechanics.}%
  \index{identity matrix!completeness relation}%
  \index{completeness relation}%
  \index{resolution of the identity}%
  \index{projection operator|see{idempotent matrix}}%
  The resolution of the identity
  $I=\sum_{n}|n\rangle\langle n|$ (discrete) or
  $I=\int|x\rangle\langle x|\,dx$ (continuous) is the operator
  version of the identity matrix.  Inserting completeness relations
  between operators is the fundamental technique for evaluating
  matrix elements, transition amplitudes, and path integrals.

\item \textbf{Gauge identity and symmetry generators.}%
  \index{identity matrix!gauge theory}%
  \index{gauge group!identity element}%
  \index{Lie group!identity element}%
  The identity matrix is the identity element of every matrix Lie
  group $G\subset\mathrm{GL}(n)$.  A continuous symmetry
  transformation near the identity takes the form
  $g=I+i\epsilon^{a}T_{a}+O(\epsilon^{2})$, where $T_{a}$ are
  the generators of the Lie algebra $\mathfrak{g}$.

\item \textbf{Null matrix and vacuum state.}%
  \index{null matrix}%
  \index{vacuum state}%
  \index{annihilation operator}%
  In the Fock space representation of quantum field theory, the
  annihilation operator $a$ satisfies $a|0\rangle=0$; in matrix
  representation on a truncated basis the action on the vacuum
  produces the null vector.  The null matrix arises as the
  representation of the zero operator on any subspace annihilated
  by all lowering operators.
\end{enumerate}

\paragraph{Mathematics applications.}
\begin{enumerate}
\item \textbf{Ring theory: identity and zero elements.}%
  \index{matrix ring}%
  \index{ring with identity}%
  \index{zero element!matrix ring}%
  The set $\mathrm{Mat}_{n}(R)$ of $n\times n$ matrices over a
  ring $R$ is itself a ring with identity $I_{n}$ and zero element
  $0_{n}$.  The study of ideals in matrix rings leads to the
  Artin--Wedderburn structure theorem for semisimple rings.

\item \textbf{Augmented matrices and affine transformations.}%
  \index{augmented matrix}%
  \index{affine transformation}%
  \index{homogeneous coordinates}%
  Affine transformations $\mathbf{x}\mapsto A\mathbf{x}+\mathbf{b}$
  are represented as linear maps in homogeneous coordinates:
  $\begin{pmatrix}A&\mathbf{b}\\0&1\end{pmatrix}$, where the
  identity in the bottom-right corner preserves the augmentation.
  This embeds the affine group into $\mathrm{GL}(n+1)$.
\end{enumerate}

%% -------------------------------------------------------------------
\subsubsection{13.113\quad Reducible and irreducible matrices}

A matrix representation is \emph{reducible} if it can be brought
to block upper-triangular form by a similarity transformation, and
\emph{irreducible} if no such transformation exists.

\paragraph{Physics applications.}
\begin{enumerate}
\item \textbf{Irreducible representations of symmetry groups.}%
  \index{irreducible representation}%
  \index{symmetry group!representations}%
  \index{Wigner--Eckart theorem}%
  \index{selection rules|see{irreducible representation}}%
  In quantum mechanics, the state space of a system with symmetry
  group $G$ decomposes into irreducible representations (irreps).
  Selection rules for transitions follow from the Wigner--Eckart
  theorem: the matrix element $\langle\alpha'j'm'|T^{(k)}_{q}
  |\alpha jm\rangle$ vanishes unless the Clebsch--Gordan coefficient
  $\langle jm;kq|j'm'\rangle\neq 0$.

\item \textbf{Block diagonalisation in molecular spectroscopy.}%
  \index{reducible representation!molecular spectroscopy}%
  \index{character table}%
  \index{molecular vibrations}%
  The reducible representation of a molecular point group on the
  $3N$-dimensional displacement space is decomposed into irreps
  using the character projection formula
  $n_{\Gamma}=\frac{1}{|G|}\sum_{R}\chi_{\Gamma}(R)^{*}\chi(R)$.
  Each irrep labels a symmetry species of vibrational mode, and
  only modes transforming as the appropriate irrep are infrared or
  Raman active.

\item \textbf{Irreducibility and ergodicity in Markov chains.}%
  \index{Markov chain!irreducible}%
  \index{transition matrix}%
  \index{ergodic theorem!Markov chains}%
  A Markov chain with transition matrix $P$ is irreducible if every
  state communicates with every other.  The Perron--Frobenius theorem
  then guarantees a unique stationary distribution
  $\boldsymbol{\pi}P=\boldsymbol{\pi}$, ensuring ergodicity.
  This is the mathematical foundation of Google's PageRank algorithm.
\end{enumerate}

\paragraph{Mathematics applications.}
\begin{enumerate}
\item \textbf{Maschke's theorem and complete reducibility.}%
  \index{Maschke's theorem}%
  \index{complete reducibility}%
  \index{semisimple algebra}%
  Maschke's theorem states that every representation of a finite group
  over a field of characteristic zero (or coprime to $|G|$) is
  completely reducible: every invariant subspace has an invariant
  complement.  This fails for modular representations (characteristic
  dividing $|G|$), where indecomposable but reducible modules appear.

\item \textbf{Schur's lemma.}%
  \index{Schur's lemma}%
  \index{intertwining operator}%
  \index{irreducible representation!endomorphism}%
  Schur's lemma states that any linear map commuting with all
  matrices of an irreducible representation over $\mathbb{C}$ is
  a scalar multiple of the identity: $\mathrm{End}_{G}(V)=\mathbb{C}$.
  This fundamental result underlies the orthogonality relations for
  characters and the classification of representations.
\end{enumerate}

%% -------------------------------------------------------------------
\subsubsection{13.114\quad Equivalent matrices}

Two matrices $A$ and $B$ are \emph{equivalent} if $B=PAQ$ for
invertible $P$ and $Q$; they are \emph{similar} if $B=P^{-1}AP$.
Similar matrices represent the same linear map in different bases.

\paragraph{Physics applications.}
\begin{enumerate}
\item \textbf{Change of basis in quantum mechanics.}%
  \index{equivalent matrices!change of basis}%
  \index{unitary transformation!change of basis}%
  \index{Schr\"odinger and Heisenberg pictures}%
  The transformation between Schr\"{o}dinger and Heisenberg pictures
  is a time-dependent similarity transformation
  $A_{H}(t)=e^{iHt/\hbar}A_{S}\,e^{-iHt/\hbar}$.  The physics
  (eigenvalues, expectation values, transition probabilities) is
  invariant because similar matrices share the same spectrum.

\item \textbf{Normal forms in control theory.}%
  \index{canonical forms!control theory}%
  \index{controllability canonical form}%
  \index{state-space representation}%
  A linear time-invariant system $\dot{\mathbf{x}}=A\mathbf{x}
  +B\mathbf{u}$ can be transformed to controllability canonical form
  via $\bar{A}=T^{-1}AT$, where $T$ is built from the
  controllability matrix $\mathcal{C}=[B,AB,\dots,A^{n-1}B]$.
  Controllability is a similarity invariant.

\item \textbf{Tensor transformations in relativity.}%
  \index{tensor transformation law}%
  \index{Lorentz transformation!similarity}%
  \index{covariance!physical laws}%
  A rank-2 tensor transforms as $T'^{\mu\nu}=\Lambda^{\mu}{}_{\alpha}
  \Lambda^{\nu}{}_{\beta}T^{\alpha\beta}$ under Lorentz transformations.
  The equivalence class of a tensor under these transformations encodes
  the physical content; the trace $T^{\mu}{}_{\mu}$, determinant, and
  eigenvalues of the mixed tensor $T^{\mu}{}_{\nu}$ are Lorentz
  invariants.
\end{enumerate}

\paragraph{Mathematics applications.}
\begin{enumerate}
\item \textbf{Jordan normal form.}%
  \index{Jordan normal form!equivalence}%
  \index{similarity invariants}%
  \index{elementary divisors}%
  Every matrix over $\mathbb{C}$ is similar to a unique (up to
  block ordering) Jordan normal form
  $J=\mathrm{diag}(J_{n_{1}}(\lambda_{1}),\dots,
  J_{n_{k}}(\lambda_{k}))$.
  The Jordan form is the complete similarity invariant: two matrices
  are similar if and only if they have the same Jordan form.

\item \textbf{Rational canonical form and invariant factors.}%
  \index{rational canonical form}%
  \index{invariant factors}%
  \index{Smith normal form}%
  The rational canonical form, constructed from the invariant factors
  of $xI-A$, is the canonical form for similarity over any field.
  The Smith normal form of $xI-A$ over $\mathbb{F}[x]$ computes
  these invariant factors and determines the module structure
  $\mathbb{F}[x]/(f_{1})\oplus\cdots\oplus\mathbb{F}[x]/(f_{k})$.
\end{enumerate}

%% -------------------------------------------------------------------
\subsubsection{13.115\quad Transpose of a matrix}

The transpose $A^{T}$ satisfies $(A^{T})_{ij}=A_{ji}$, and
$(AB)^{T}=B^{T}A^{T}$.

\paragraph{Physics applications.}
\begin{enumerate}
\item \textbf{Reciprocity in network theory.}%
  \index{transpose!reciprocity}%
  \index{reciprocity!network theory}%
  \index{impedance matrix!symmetric}%
  For a reciprocal electrical network, the impedance matrix satisfies
  $Z=Z^{T}$, expressing the reciprocity theorem: the transfer
  impedance from port $i$ to port $j$ equals that from $j$ to $i$.
  This symmetry follows from the time-reversal invariance of
  Maxwell's equations in passive media.

\item \textbf{Adjoint operators in quantum mechanics.}%
  \index{transpose!adjoint operator}%
  \index{Hermitian adjoint}%
  \index{bra-ket notation!transpose}%
  In a real vector space, the transpose is the adjoint: $\langle
  A\mathbf{x},\mathbf{y}\rangle=\langle\mathbf{x},A^{T}\mathbf{y}
  \rangle$.  In quantum mechanics over $\mathbb{C}$, the adjoint
  involves both transposition and complex conjugation:
  $A^{\dagger}=\bar{A}^{T}$, ensuring
  $\langle A^{\dagger}\psi|\phi\rangle=\langle\psi|A\phi\rangle$.

\item \textbf{Strain and stress tensors in continuum mechanics.}%
  \index{strain tensor!symmetry}%
  \index{stress tensor!symmetry}%
  \index{continuum mechanics!transpose}%
  The infinitesimal strain tensor
  $\varepsilon_{ij}=\frac{1}{2}(\partial_{i}u_{j}+\partial_{j}u_{i})$
  is symmetric by construction ($\varepsilon=\varepsilon^{T}$).
  Angular momentum conservation requires the Cauchy stress tensor to be
  symmetric: $\sigma_{ij}=\sigma_{ji}$, i.e., $\sigma=\sigma^{T}$.
\end{enumerate}

\paragraph{Mathematics applications.}
\begin{enumerate}
\item \textbf{Bilinear forms and duality.}%
  \index{bilinear form!transpose}%
  \index{dual space}%
  \index{transpose!bilinear form}%
  A bilinear form $B(\mathbf{x},\mathbf{y})=\mathbf{x}^{T}A\mathbf{y}$
  satisfies $B(\mathbf{x},\mathbf{y})=B'(\mathbf{y},\mathbf{x})$
  where $B'$ is the form with matrix $A^{T}$.  The transpose is the
  matrix of the dual map $A^{*}\colon V^{*}\to V^{*}$ in the dual
  basis, so transposition is the coordinate expression of duality.

\item \textbf{Left and right null spaces.}%
  \index{null space!left and right}%
  \index{fundamental subspaces}%
  \index{rank-nullity theorem}%
  The four fundamental subspaces of $A\in\mathbb{R}^{m\times n}$ are
  $\mathrm{Col}(A)$, $\mathrm{Null}(A)$, $\mathrm{Col}(A^{T})$,
  $\mathrm{Null}(A^{T})$, with the orthogonal decompositions
  $\mathbb{R}^{n}=\mathrm{Col}(A^{T})\oplus\mathrm{Null}(A)$ and
  $\mathbb{R}^{m}=\mathrm{Col}(A)\oplus\mathrm{Null}(A^{T})$.
\end{enumerate}

%% -------------------------------------------------------------------
\subsubsection{13.116\quad Adjoint matrix}

The classical adjoint (adjugate) $\mathrm{adj}(A)$ has entries
$(\mathrm{adj}(A))_{ij}=(-1)^{i+j}M_{ji}$, where $M_{ji}$ is the
$(j,i)$-minor.  It satisfies $A\,\mathrm{adj}(A)=\det(A)\,I$.

\paragraph{Physics applications.}
\begin{enumerate}
\item \textbf{Cramer's rule in circuit analysis.}%
  \index{adjugate matrix!Cramer's rule}%
  \index{Cramer's rule}%
  \index{circuit analysis}%
  Kirchhoff's laws for an electrical network give a linear system
  $Z\mathbf{I}=\mathbf{V}$, and Cramer's rule yields
  $I_{k}=\det(Z_{k})/\det(Z)$ where $Z_{k}$ replaces the $k$th
  column by $\mathbf{V}$.  The numerator is a cofactor expansion
  involving $\mathrm{adj}(Z)$, and the formula is practical for
  symbolic analysis of small networks.

\item \textbf{Inverse of the metric tensor.}%
  \index{adjugate matrix!metric tensor}%
  \index{metric tensor!inverse}%
  \index{general relativity!inverse metric}%
  In general relativity, the inverse metric $g^{\mu\nu}$ is computed
  as $g^{\mu\nu}=\mathrm{adj}(g)^{\mu\nu}/\det(g)$.  For a
  $4\times 4$ metric, the adjugate provides the explicit formula for
  $g^{\mu\nu}$ without recourse to Gauss--Jordan elimination.

\item \textbf{Resolvent and Green's functions.}%
  \index{resolvent!adjugate}%
  \index{Green's function!matrix}%
  \index{poles of resolvent}%
  The resolvent $(zI-A)^{-1}=\mathrm{adj}(zI-A)/\det(zI-A)$ expresses
  the Green's function as a ratio of a matrix polynomial to the
  characteristic polynomial.  Poles occur at the eigenvalues, and
  the residues are the spectral projections.
\end{enumerate}

\paragraph{Mathematics applications.}
\begin{enumerate}
\item \textbf{Cayley--Hamilton theorem.}%
  \index{Cayley--Hamilton theorem}%
  \index{characteristic polynomial}%
  \index{adjugate matrix!Cayley--Hamilton}%
  The identity $\mathrm{adj}(zI-A)=\sum_{k=0}^{n-1}B_{k}z^{k}$
  with $B_{k}$ satisfying the Faddeev--LeVerrier recursion provides
  a constructive proof of the Cayley--Hamilton theorem: substituting
  $z=A$ into $\det(zI-A)=0$ yields $p(A)=0$.

\item \textbf{Compound matrices and exterior algebra.}%
  \index{compound matrix}%
  \index{exterior algebra!adjugate}%
  \index{adjugate matrix!exterior algebra}%
  The adjugate of an $n\times n$ matrix is the transpose of the
  $(n-1)$th compound matrix $C_{n-1}(A)$, whose entries are
  $(n-1)\times(n-1)$ minors.  This connects the adjugate to the
  action of $A$ on $\bigwedge^{n-1}V$, the $(n-1)$th exterior power.
\end{enumerate}

%% -------------------------------------------------------------------
\subsubsection{13.117\quad Inverse matrix}

The inverse $A^{-1}$ exists if and only if $\det A\neq 0$, and
satisfies $AA^{-1}=A^{-1}A=I$.

\paragraph{Physics applications.}
\begin{enumerate}
\item \textbf{Solving linear systems in computational physics.}%
  \index{inverse matrix!linear systems}%
  \index{LU decomposition}%
  \index{numerical linear algebra}%
  In practice, $A\mathbf{x}=\mathbf{b}$ is solved not by computing
  $A^{-1}$ but by LU decomposition $A=LU$ (or Cholesky $A=LL^{T}$
  for positive definite systems), reducing the cost from $O(n^{3})$
  for inversion to $O(n^{3})$ with a smaller constant and better
  numerical stability.

\item \textbf{Transfer matrices in statistical mechanics.}%
  \index{transfer matrix}%
  \index{partition function!transfer matrix}%
  \index{Ising model!transfer matrix}%
  The partition function of the Ising model on a strip is
  $Z=\mathrm{tr}(T^{N})$ where $T$ is the transfer matrix.
  Correlation functions involve $T^{-1}$, and the correlation
  length $\xi=-1/\ln(\lambda_{2}/\lambda_{1})$ is determined by
  the ratio of the two largest eigenvalues.

\item \textbf{Scattering matrix and its inverse.}%
  \index{S-matrix!inverse}%
  \index{scattering theory!inverse}%
  \index{time reversal!S-matrix}%
  The scattering matrix $S$ is unitary ($S^{-1}=S^{\dagger}$),
  expressing conservation of probability (flux).  Time-reversal
  invariance implies $S=S^{T}$, so that $S^{-1}=\bar{S}$.
  The inverse scattering problem---reconstructing the potential from
  $S$---is central to soliton theory and quantum inverse problems.
\end{enumerate}

\paragraph{Mathematics applications.}
\begin{enumerate}
\item \textbf{General linear group.}%
  \index{general linear group}%
  \index{matrix group}%
  \index{GL($n$)!structure}%
  The set of invertible $n\times n$ matrices forms the general linear
  group $\mathrm{GL}(n,\mathbb{F})$, the largest matrix Lie group.
  It is an open dense subset of $\mathrm{Mat}_{n}$ (complement of the
  hypersurface $\det A=0$) and has two connected components (for
  $\mathbb{F}=\mathbb{R}$) distinguished by $\mathrm{sgn}(\det A)$.

\item \textbf{Sherman--Morrison--Woodbury formula.}%
  \index{Sherman--Morrison formula}%
  \index{Woodbury formula}%
  \index{rank-one update}%
  The formula $(A+UCV)^{-1}=A^{-1}-A^{-1}U(C^{-1}+VA^{-1}U)^{-1}
  VA^{-1}$ updates the inverse after a low-rank perturbation.
  This is indispensable in statistics (updating regression after
  adding data), optimisation (quasi-Newton methods), and numerical
  methods (bordered systems).
\end{enumerate}

%% -------------------------------------------------------------------
\subsubsection{13.118\quad Trace of a matrix}

The trace $\mathrm{tr}(A)=\sum_{i}A_{ii}$ is a similarity invariant
and equals the sum of eigenvalues: $\mathrm{tr}(A)=\sum_{i}\lambda_{i}$.
It satisfies $\mathrm{tr}(AB)=\mathrm{tr}(BA)$ (cyclic property).

\paragraph{Physics applications.}
\begin{enumerate}
\item \textbf{Density matrix and expectation values.}%
  \index{trace!density matrix}%
  \index{density matrix!trace}%
  \index{expectation value!trace formula}%
  In quantum statistical mechanics, the expectation value of an
  observable $A$ in a mixed state $\rho$ is
  $\langle A\rangle=\mathrm{tr}(\rho A)$, and the partition function
  is $Z=\mathrm{tr}(e^{-\beta H})$.  The normalisation
  $\mathrm{tr}(\rho)=1$ and positivity $\rho\geq 0$ define a valid
  density matrix.

\item \textbf{Wilson loops in gauge theory.}%
  \index{Wilson loop}%
  \index{trace!gauge invariance}%
  \index{lattice gauge theory}%
  \index{confinement|see{Wilson loop}}%
  The Wilson loop $W(C)=\mathrm{tr}\,\mathcal{P}\exp\!\left(ig
  \oint_{C}A_{\mu}\,dx^{\mu}\right)$ is a gauge-invariant observable
  because the trace is invariant under cyclic permutations
  (conjugation).  The area-law versus perimeter-law behaviour of
  $\langle W(C)\rangle$ diagnoses confinement in lattice gauge theory.

\item \textbf{Trace anomaly and the energy-momentum tensor.}%
  \index{trace anomaly}%
  \index{energy-momentum tensor!trace}%
  \index{conformal anomaly}%
  For a classically conformal field theory, the trace of the
  energy-momentum tensor $T^{\mu}{}_{\mu}=0$ at the classical level.
  Quantum corrections produce the trace anomaly
  $\langle T^{\mu}{}_{\mu}\rangle=\frac{\beta(g)}{2g}F_{\mu\nu}
  F^{\mu\nu}$, which governs the running of the coupling constant.
\end{enumerate}

\paragraph{Mathematics applications.}
\begin{enumerate}
\item \textbf{Trace as a linear functional and Killing form.}%
  \index{trace!Killing form}%
  \index{Killing form}%
  \index{Lie algebra!semisimple}%
  The Killing form $B(X,Y)=\mathrm{tr}(\mathrm{ad}_{X}\circ
  \mathrm{ad}_{Y})$ is a bilinear form on a Lie algebra built from
  the trace.  Cartan's criterion states that a Lie algebra is
  semisimple if and only if the Killing form is non-degenerate.

\item \textbf{Newton's identities and symmetric functions.}%
  \index{Newton's identities}%
  \index{symmetric functions!power sums}%
  \index{characteristic polynomial!from traces}%
  The power sums $p_{k}=\mathrm{tr}(A^{k})=\sum_{i}\lambda_{i}^{k}$
  determine the characteristic polynomial via Newton's identities:
  $ke_{k}=\sum_{j=1}^{k}(-1)^{j-1}e_{k-j}p_{j}$, where $e_{k}$
  are the elementary symmetric polynomials of the eigenvalues.
\end{enumerate}

%% -------------------------------------------------------------------
\subsubsection{13.119\quad Symmetric matrix}

A real matrix is symmetric if $A=A^{T}$.  Every real symmetric matrix
is diagonalisable by an orthogonal matrix: $A=Q\Lambda Q^{T}$ with
$\Lambda$ real diagonal.

\paragraph{Physics applications.}
\begin{enumerate}
\item \textbf{Moment of inertia tensor.}%
  \index{symmetric matrix!inertia tensor}%
  \index{inertia tensor!symmetric}%
  \index{Euler equations!rigid body}%
  The inertia tensor $I_{ij}$ is symmetric by construction.
  Diagonalising it yields the principal moments $I_{1}\leq I_{2}
  \leq I_{3}$ and the principal axes, which determine the stability
  of free rotation: rotation about the axis of intermediate moment
  is unstable (tennis racket theorem).

\item \textbf{Elastic stiffness tensor.}%
  \index{elastic stiffness!symmetric}%
  \index{Hooke's law!matrix form}%
  \index{Voigt notation}%
  Hooke's law $\sigma_{ij}=C_{ijkl}\varepsilon_{kl}$ in Voigt
  notation becomes $\boldsymbol{\sigma}=C\boldsymbol{\varepsilon}$
  with $C$ a $6\times 6$ symmetric matrix (from the symmetry of
  the strain energy density $U=\frac{1}{2}\varepsilon_{ij}C_{ijkl}
  \varepsilon_{kl}$).  The 21 independent components reduce further
  with crystal symmetry.

\item \textbf{Covariance matrix in data analysis.}%
  \index{covariance matrix}%
  \index{principal component analysis|see{eigenvalues}}%
  \index{error ellipse}%
  The covariance matrix $\Sigma_{ij}=\mathrm{Cov}(X_{i},X_{j})$ is
  symmetric and positive semidefinite.  Its eigenvectors define the
  principal components, and the eigenvalues give the variance along
  each principal direction, forming the error ellipse (or ellipsoid)
  of a multivariate Gaussian distribution.
\end{enumerate}

\paragraph{Mathematics applications.}
\begin{enumerate}
\item \textbf{Spectral theorem for symmetric matrices.}%
  \index{spectral theorem!real symmetric}%
  \index{orthogonal diagonalisation}%
  \index{eigenvalues!real symmetric}%
  Every real symmetric $n\times n$ matrix has $n$ real eigenvalues
  and an orthonormal basis of eigenvectors.  The spectral decomposition
  $A=\sum_{i}\lambda_{i}\mathbf{q}_{i}\mathbf{q}_{i}^{T}$ is the
  finite-dimensional prototype of the spectral theorem for
  self-adjoint operators on Hilbert spaces.

\item \textbf{Rayleigh quotient and variational characterisation.}%
  \index{Rayleigh quotient}%
  \index{min-max theorem}%
  \index{Courant--Fischer theorem}%
  The eigenvalues of a symmetric matrix satisfy the Courant--Fischer
  min-max characterisation: $\lambda_{k}=\min_{\dim W=k}\max_{
  \mathbf{x}\in W\setminus\{0\}}\frac{\mathbf{x}^{T}A\mathbf{x}}
  {\mathbf{x}^{T}\mathbf{x}}$.  This variational principle is the
  basis for the Rayleigh--Ritz method in finite element analysis.
\end{enumerate}

%% -------------------------------------------------------------------
\subsubsection{13.120\quad Skew-symmetric matrix}

A real matrix is skew-symmetric if $A^{T}=-A$.  Its eigenvalues
are purely imaginary (or zero), and every skew-symmetric matrix of
even order has $\det A=(\mathrm{Pf}\,A)^{2}$, where $\mathrm{Pf}$
is the Pfaffian.

\paragraph{Physics applications.}
\begin{enumerate}
\item \textbf{Angular velocity and infinitesimal rotations.}%
  \index{skew-symmetric matrix!angular velocity}%
  \index{angular velocity!skew-symmetric}%
  \index{infinitesimal rotation}%
  \index{Lie algebra!$\mathfrak{so}(3)$}%
  An infinitesimal rotation $R=I+\epsilon\,\Omega+O(\epsilon^{2})$
  requires $\Omega^{T}=-\Omega$.  The $3\times 3$ skew-symmetric
  matrix $\Omega_{ij}=-\varepsilon_{ijk}\omega_{k}$ encodes the
  angular velocity vector $\boldsymbol{\omega}$, and the Lie algebra
  $\mathfrak{so}(3)$ consists of all such matrices.

\item \textbf{Electromagnetic field tensor.}%
  \index{electromagnetic field tensor}%
  \index{Faraday tensor}%
  \index{skew-symmetric matrix!electromagnetism}%
  The Faraday tensor $F_{\mu\nu}=-F_{\nu\mu}$ is an antisymmetric
  $4\times 4$ matrix encoding both $\mathbf{E}$ and $\mathbf{B}$:
  $F_{0i}=E_{i}/c$ and $F_{ij}=-\varepsilon_{ijk}B_{k}$.  The two
  Lorentz invariants are $F_{\mu\nu}F^{\mu\nu}=2(B^{2}-E^{2}/c^{2})$
  and $\varepsilon^{\mu\nu\rho\sigma}F_{\mu\nu}F_{\rho\sigma}
  \propto\mathbf{E}\cdot\mathbf{B}$.

\item \textbf{Symplectic structure in Hamiltonian mechanics.}%
  \index{symplectic matrix}%
  \index{Hamiltonian mechanics!symplectic}%
  \index{Poisson bracket!matrix form}%
  Hamilton's equations $\dot{z}^{i}=J^{ij}\partial H/\partial z^{j}$
  use the symplectic matrix
  $J=\begin{pmatrix}0&I\\-I&0\end{pmatrix}$ with $J^{T}=-J$.
  The Poisson bracket is
  $\{f,g\}=(\nabla f)^{T}J(\nabla g)$, and canonical
  transformations preserve $J$.
\end{enumerate}

\paragraph{Mathematics applications.}
\begin{enumerate}
\item \textbf{Pfaffian and matchings.}%
  \index{Pfaffian}%
  \index{perfect matching}%
  \index{dimer problem}%
  For a $2n\times 2n$ skew-symmetric matrix, the Pfaffian satisfies
  $(\mathrm{Pf}\,A)^{2}=\det A$.  In combinatorics, the number of
  perfect matchings of a planar graph equals the Pfaffian of the
  skew-adjacency matrix (Kasteleyn's theorem), solving the dimer
  problem on lattices.

\item \textbf{Lie algebra $\mathfrak{so}(n)$.}%
  \index{Lie algebra!$\mathfrak{so}(n)$}%
  \index{orthogonal group!Lie algebra}%
  \index{dimension formula!$\mathfrak{so}(n)$}%
  The Lie algebra $\mathfrak{so}(n)$ of the orthogonal group
  $\mathrm{SO}(n)$ consists of all $n\times n$ real skew-symmetric
  matrices, with the commutator as Lie bracket.  Its dimension is
  $n(n-1)/2$, which counts the independent rotation planes in
  $\mathbb{R}^{n}$.
\end{enumerate}

%% -------------------------------------------------------------------
\subsubsection{13.121\quad Triangular matrices}

An upper triangular matrix $U$ has $U_{ij}=0$ for $i>j$; lower
triangular $L$ has $L_{ij}=0$ for $i<j$.  The eigenvalues of a
triangular matrix are its diagonal entries.

\paragraph{Physics applications.}
\begin{enumerate}
\item \textbf{LU decomposition in computational physics.}%
  \index{triangular matrix!LU decomposition}%
  \index{LU decomposition!physics}%
  \index{Gaussian elimination}%
  Gaussian elimination factors $A=LU$, reducing the solution of
  $A\mathbf{x}=\mathbf{b}$ to forward and back substitution, each
  costing $O(n^{2})$.  With partial pivoting ($PA=LU$), this is
  the standard algorithm for solving dense linear systems arising
  in finite element analysis and circuit simulation.

\item \textbf{Cholesky decomposition and least-squares.}%
  \index{Cholesky decomposition}%
  \index{least squares!Cholesky}%
  \index{positive definite!Cholesky}%
  A positive definite matrix $A$ has a unique factorisation
  $A=LL^{T}$ with $L$ lower triangular and positive diagonal.
  This is computationally half the cost of LU and is the preferred
  method for normal equations $A^{T}A\mathbf{x}=A^{T}\mathbf{b}$ in
  least-squares fitting of experimental data.

\item \textbf{QR algorithm for eigenvalue computation.}%
  \index{QR algorithm}%
  \index{Schur decomposition}%
  \index{eigenvalue computation}%
  The QR algorithm iteratively computes the Schur form $A=QTQ^{*}$
  with $T$ upper triangular, whose diagonal entries are the
  eigenvalues.  This is the workhorse algorithm for dense eigenvalue
  problems in quantum chemistry and structural mechanics.
\end{enumerate}

\paragraph{Mathematics applications.}
\begin{enumerate}
\item \textbf{Borel subgroup and flag varieties.}%
  \index{Borel subgroup}%
  \index{flag variety}%
  \index{upper triangular matrices!Borel}%
  The group of invertible upper triangular matrices is the standard
  Borel subgroup $B\subset\mathrm{GL}(n)$.  The quotient
  $\mathrm{GL}(n)/B$ is the complete flag variety
  $\mathrm{Fl}(1,2,\dots,n;\mathbb{F}^{n})$, parametrising nested
  sequences of subspaces $V_{1}\subset V_{2}\subset\cdots\subset V_{n}$.

\item \textbf{Nilpotent radical and solvable Lie algebras.}%
  \index{nilpotent radical}%
  \index{solvable Lie algebra}%
  \index{Lie's theorem}%
  Lie's theorem states that every representation of a solvable Lie
  algebra over $\mathbb{C}$ can be put in upper triangular form.
  The strictly upper triangular matrices form the nilpotent radical of
  the Borel subalgebra, and they generate the unipotent subgroup.
\end{enumerate}

%% -------------------------------------------------------------------
\subsubsection{13.122\quad Orthogonal matrices}

An orthogonal matrix satisfies $Q^{T}Q=QQ^{T}=I$ and $\det Q=\pm 1$.
The set of orthogonal matrices with $\det Q=+1$ forms the special
orthogonal group $\mathrm{SO}(n)$.

\paragraph{Physics applications.}
\begin{enumerate}
\item \textbf{Rotation group and rigid body dynamics.}%
  \index{orthogonal matrix!rotation}%
  \index{rotation group SO(3)}%
  \index{Euler angles}%
  \index{rigid body!rotation matrix}%
  Every rotation in $\mathbb{R}^{3}$ is represented by a matrix
  $R\in\mathrm{SO}(3)$ parametrised by Euler angles
  $(\phi,\theta,\psi)$: $R=R_{z}(\phi)R_{x}(\theta)R_{z}(\psi)$.
  The orthogonality $R^{T}R=I$ preserves lengths and angles, as
  required for rigid body motion.  The topology of $\mathrm{SO}(3)
  \cong\mathbb{RP}^{3}$ (with $\pi_{1}=\mathbb{Z}_{2}$) leads
  to the ``plate trick'' and the need for spin-$\tfrac{1}{2}$
  representations.

\item \textbf{Normal modes and orthogonal transformations.}%
  \index{orthogonal matrix!normal modes}%
  \index{normal mode analysis}%
  \index{phonons!orthogonal diagonalisation}%
  The normal-mode transformation $\mathbf{q}=Q\boldsymbol{\eta}$
  with $Q$ orthogonal simultaneously diagonalises the kinetic and
  potential energy matrices for a system with $M=I$ (mass-weighted
  coordinates).  Each column of $Q$ is a normal-mode eigenvector,
  and the transformation preserves the total energy.

\item \textbf{Lorentz group and pseudo-orthogonal matrices.}%
  \index{Lorentz group}%
  \index{pseudo-orthogonal matrix}%
  \index{SO(3,1)|see{Lorentz group}}%
  The Lorentz group $\mathrm{SO}(3,1)$ consists of matrices preserving
  the Minkowski metric $\eta=\mathrm{diag}(-1,1,1,1)$:
  $\Lambda^{T}\eta\Lambda=\eta$.  Boosts are pseudo-rotations
  in spacetime, with rapidity playing the role of angle.
\end{enumerate}

\paragraph{Mathematics applications.}
\begin{enumerate}
\item \textbf{Orthogonal group as a compact Lie group.}%
  \index{orthogonal group!compact Lie group}%
  \index{Haar measure}%
  \index{compact Lie group!O($n$)}%
  $\mathrm{O}(n)$ is compact (closed and bounded in
  $\mathrm{Mat}_{n}(\mathbb{R})$) and hence admits a unique
  normalised Haar measure.  Integration over $\mathrm{O}(n)$ arises
  in random matrix theory: the circular orthogonal ensemble (COE)
  uses Haar-distributed orthogonal matrices.

\item \textbf{Stiefel and Grassmann manifolds.}%
  \index{Stiefel manifold}%
  \index{Grassmann manifold}%
  \index{orthogonal group!quotients}%
  The Stiefel manifold $V_{k}(\mathbb{R}^{n})=\mathrm{O}(n)/
  \mathrm{O}(n-k)$ parametrises orthonormal $k$-frames, and the
  Grassmannian $\mathrm{Gr}(k,n)=\mathrm{O}(n)/(\mathrm{O}(k)
  \times\mathrm{O}(n-k))$ parametrises $k$-dimensional subspaces.
  These spaces appear in optimisation on manifolds and in the
  topology of vector bundles.
\end{enumerate}

%% -------------------------------------------------------------------
\subsubsection{13.123\quad Hermitian transpose of a matrix}

The Hermitian transpose (conjugate transpose) is
$(A^{\dagger})_{ij}=\overline{A_{ji}}$.  It satisfies
$(AB)^{\dagger}=B^{\dagger}A^{\dagger}$ and reduces to the ordinary
transpose for real matrices.

\paragraph{Physics applications.}
\begin{enumerate}
\item \textbf{Adjoint operators in quantum mechanics.}%
  \index{Hermitian transpose!quantum mechanics}%
  \index{adjoint operator}%
  \index{Dirac notation!adjoint}%
  In Dirac notation, $\langle\psi|=(|\psi\rangle)^{\dagger}$,
  so taking the Hermitian transpose converts kets to bras.  An
  operator satisfies $\langle\phi|A|\psi\rangle^{*}=\langle\psi|
  A^{\dagger}|\phi\rangle$, and physical observables require
  $A^{\dagger}=A$.

\item \textbf{Creation and annihilation operators.}%
  \index{creation operator}%
  \index{annihilation operator!adjoint}%
  \index{Fock space}%
  The creation operator $a^{\dagger}$ is the Hermitian transpose of
  the annihilation operator $a$, satisfying $[a,a^{\dagger}]=1$.
  In matrix representation on the number basis,
  $(a)_{mn}=\sqrt{m}\,\delta_{m,n+1}$ and
  $(a^{\dagger})_{mn}=\sqrt{n}\,\delta_{m+1,n}$, which are indeed
  transposes of each other (all entries being real).

\item \textbf{Charge conjugation and CPT.}%
  \index{charge conjugation!Hermitian transpose}%
  \index{CPT theorem}%
  \index{Dirac equation!charge conjugation}%
  In the Dirac equation, the charge-conjugation operation involves
  complex conjugation of the spinor and the relation
  $\gamma^{\mu*}=B\gamma^{\mu}B^{-1}$ for a matrix $B$.  The
  interplay between complex conjugation, transposition, and Hermitian
  conjugation is at the heart of the CPT theorem.
\end{enumerate}

\paragraph{Mathematics applications.}
\begin{enumerate}
\item \textbf{$*$-algebras and $C^{*}$-algebras.}%
  \index{$*$-algebra}%
  \index{$C^{*}$-algebra}%
  \index{involution!Hermitian transpose}%
  The Hermitian transpose defines an involution $A\mapsto A^{\dagger}$
  on $\mathrm{Mat}_{n}(\mathbb{C})$, making it a $*$-algebra.  The
  $C^{*}$-identity $\|A^{\dagger}A\|=\|A\|^{2}$ is the defining
  axiom of a $C^{*}$-algebra, the abstract framework for quantum
  mechanics (Gelfand--Naimark theorem).

\item \textbf{Polar decomposition.}%
  \index{polar decomposition}%
  \index{Hermitian transpose!polar decomposition}%
  \index{singular values!polar decomposition}%
  Every matrix $A$ admits a polar decomposition $A=UP$ where $U$ is
  unitary and $P=(A^{\dagger}A)^{1/2}$ is positive semidefinite.
  The singular values of $A$ are the eigenvalues of $P$, and the
  decomposition generalises the polar form $z=|z|e^{i\theta}$ of
  a complex number.
\end{enumerate}

%% -------------------------------------------------------------------
\subsubsection{13.124\quad Hermitian matrix}

A Hermitian matrix satisfies $A^{\dagger}=A$.  Its eigenvalues are
real, and eigenvectors corresponding to distinct eigenvalues are
orthogonal.

\paragraph{Physics applications.}
\begin{enumerate}
\item \textbf{Quantum observables and the measurement postulate.}%
  \index{Hermitian matrix!observables}%
  \index{quantum measurement}%
  \index{Born rule}%
  \index{spectral theorem!quantum mechanics}%
  Every physical observable in quantum mechanics is represented by a
  Hermitian operator.  The spectral theorem $A=\sum_{\lambda}\lambda
  \,P_{\lambda}$ decomposes it into projections onto eigenspaces.
  Upon measurement, the probability of outcome $\lambda$ is
  $p_{\lambda}=\mathrm{tr}(\rho\,P_{\lambda})$ (Born rule), and
  the reality of eigenvalues ensures that measurement outcomes are
  real numbers.

\item \textbf{Pauli matrices and spin-$\frac{1}{2}$.}%
  \index{Pauli matrices}%
  \index{spin-$\frac{1}{2}$!Pauli matrices}%
  \index{Hermitian matrix!Pauli}%
  The Pauli matrices $\sigma_{x}=\begin{pmatrix}0&1\\1&0\end{pmatrix}$,
  $\sigma_{y}=\begin{pmatrix}0&-i\\i&0\end{pmatrix}$,
  $\sigma_{z}=\begin{pmatrix}1&0\\0&-1\end{pmatrix}$ are
  Hermitian, traceless, and satisfy $\sigma_{i}\sigma_{j}=
  \delta_{ij}I+i\varepsilon_{ijk}\sigma_{k}$.  They form a basis
  for $\mathfrak{su}(2)$ and represent spin angular momentum
  $S_{i}=\tfrac{\hbar}{2}\sigma_{i}$.

\item \textbf{Hamiltonian matrix in tight-binding models.}%
  \index{Hamiltonian matrix!tight-binding}%
  \index{tight-binding model}%
  \index{band structure}%
  \index{Hermitian matrix!band structure}%
  In the tight-binding approximation, the Hamiltonian is an
  $N\times N$ Hermitian matrix $H_{mn}=\langle m|H|n\rangle$ with
  hopping integrals $t_{mn}=H_{mn}$ for nearest neighbours and
  on-site energies $\varepsilon_{m}=H_{mm}$.  The eigenvalues
  $E_{k}$ give the electronic band structure, and the Hermiticity
  ensures real energy bands.

\item \textbf{Density matrix formalism.}%
  \index{density matrix!Hermitian}%
  \index{von Neumann entropy}%
  \index{mixed state}%
  The density matrix $\rho=\sum_{i}p_{i}|\psi_{i}\rangle
  \langle\psi_{i}|$ is Hermitian, positive semidefinite, and has
  unit trace.  The von Neumann entropy $S=-\mathrm{tr}(\rho\ln\rho)
  =-\sum_{i}\lambda_{i}\ln\lambda_{i}$ measures the mixedness
  of the state and vanishes for pure states ($\rho^{2}=\rho$).
\end{enumerate}

\paragraph{Mathematics applications.}
\begin{enumerate}
\item \textbf{Spectral theorem for Hermitian matrices.}%
  \index{spectral theorem!Hermitian}%
  \index{unitary diagonalisation}%
  \index{eigenvalues!Hermitian matrix}%
  Every $n\times n$ Hermitian matrix is unitarily diagonalisable:
  $A=U\Lambda U^{\dagger}$ with $\Lambda=\mathrm{diag}(\lambda_{1},
  \dots,\lambda_{n})$ real and $U$ unitary.  This is the
  finite-dimensional case of the spectral theorem for self-adjoint
  operators, the cornerstone of functional analysis.

\item \textbf{Weyl's inequalities and eigenvalue perturbation.}%
  \index{Weyl's inequalities}%
  \index{eigenvalue perturbation}%
  \index{Hermitian matrix!perturbation}%
  For Hermitian $A$ and $B$ with eigenvalues $\alpha_{1}\leq\cdots
  \leq\alpha_{n}$ and $\beta_{1}\leq\cdots\leq\beta_{n}$, the
  eigenvalues $\gamma_{k}$ of $A+B$ satisfy $\alpha_{i}+\beta_{j}
  \leq\gamma_{i+j-1}\leq\alpha_{i}+\beta_{n-j+1}$ (Weyl).  These
  inequalities bound the sensitivity of the spectrum to perturbations
  and underpin numerical error analysis.

\item \textbf{Random Hermitian matrices and the Gaussian Unitary Ensemble.}%
  \index{random matrix theory!GUE}%
  \index{Gaussian Unitary Ensemble}%
  \index{Wigner semicircle law}%
  The GUE consists of Hermitian matrices with Gaussian-distributed
  entries.  The Wigner semicircle law states that the empirical
  spectral distribution converges to $\rho(\lambda)=\frac{1}{2\pi}
  \sqrt{4-\lambda^{2}}$ as $n\to\infty$, a universal result with
  applications from nuclear physics to number theory.
\end{enumerate}

%% -------------------------------------------------------------------
\subsubsection{13.125\quad Unitary matrix}

A unitary matrix satisfies $U^{\dagger}U=UU^{\dagger}=I$ and
$|\det U|=1$.  The set of $n\times n$ unitary matrices forms the
unitary group $\mathrm{U}(n)$.

\paragraph{Physics applications.}
\begin{enumerate}
\item \textbf{Time evolution in quantum mechanics.}%
  \index{unitary matrix!time evolution}%
  \index{time evolution operator}%
  \index{Schr\"odinger equation!unitary evolution}%
  \index{probability conservation!unitarity}%
  The time evolution operator $U(t)=e^{-iHt/\hbar}$ is unitary for
  Hermitian $H$, ensuring conservation of probability:
  $\langle\psi(t)|\psi(t)\rangle=\langle\psi(0)|U^{\dagger}U
  |\psi(0)\rangle=\langle\psi(0)|\psi(0)\rangle$.  Every closed
  quantum system evolves unitarily.

\item \textbf{Quantum gates and quantum computing.}%
  \index{quantum gates}%
  \index{quantum computing!unitary}%
  \index{Hadamard gate}%
  \index{CNOT gate}%
  Quantum logic gates are unitary matrices.  The Hadamard gate
  $H=\frac{1}{\sqrt{2}}\begin{pmatrix}1&1\\1&-1\end{pmatrix}$
  creates superposition, and the CNOT gate (a $4\times 4$ unitary
  matrix) creates entanglement.  Any $n$-qubit unitary can be
  decomposed into a product of one- and two-qubit gates (universality
  theorem).

\item \textbf{Scattering matrix (S-matrix).}%
  \index{S-matrix!unitary}%
  \index{scattering theory!unitarity}%
  \index{optical theorem}%
  The S-matrix $S=I+2iT$ relating in-states to out-states is unitary:
  $S^{\dagger}S=I$.  This unitarity condition implies the optical
  theorem $\mathrm{Im}\,T_{ii}=\sum_{f}|T_{fi}|^{2}$, relating the
  total cross section to the imaginary part of the forward scattering
  amplitude.

\item \textbf{CKM and PMNS matrices in particle physics.}%
  \index{CKM matrix}%
  \index{PMNS matrix}%
  \index{CP violation!unitary matrix}%
  The Cabibbo--Kobayashi--Maskawa (CKM) matrix for quarks and the
  Pontecorvo--Maki--Nakagawa--Sakata (PMNS) matrix for neutrinos are
  $3\times 3$ unitary matrices parametrising flavour mixing.  A single
  complex phase in the CKM matrix accounts for CP violation.
\end{enumerate}

\paragraph{Mathematics applications.}
\begin{enumerate}
\item \textbf{Unitary group as a compact Lie group.}%
  \index{unitary group!compact Lie group}%
  \index{maximal torus}%
  \index{Weyl group!unitary}%
  $\mathrm{U}(n)$ is a compact, connected Lie group of dimension
  $n^{2}$.  Its maximal torus consists of diagonal unitary matrices
  $\mathrm{diag}(e^{i\theta_{1}},\dots,e^{i\theta_{n}})$, and the
  Weyl group is $S_{n}$ (permutations).  The representation theory
  of $\mathrm{U}(n)$ is governed by highest-weight theory and
  Young diagrams.

\item \textbf{Singular value decomposition.}%
  \index{singular value decomposition}%
  \index{SVD|see{singular value decomposition}}%
  \index{unitary matrix!SVD}%
  Every matrix $A\in\mathbb{C}^{m\times n}$ has a singular value
  decomposition $A=U\Sigma V^{\dagger}$ with $U\in\mathrm{U}(m)$,
  $V\in\mathrm{U}(n)$, and $\Sigma$ diagonal with non-negative real
  entries.  The SVD provides the best rank-$k$ approximation
  (Eckart--Young theorem) and is the computational backbone of
  principal component analysis, data compression, and pseudoinverse
  computation.
\end{enumerate}

%% -------------------------------------------------------------------
\subsubsection{13.126\quad Eigenvalues and eigenvectors}

The eigenvalue equation $A\mathbf{v}=\lambda\mathbf{v}$ with
$\mathbf{v}\neq\mathbf{0}$ defines the eigenvalues (roots of
$\det(A-\lambda I)=0$) and the corresponding eigenvectors.

\paragraph{Physics applications.}
\begin{enumerate}
\item \textbf{Energy levels and stationary states.}%
  \index{eigenvalues!energy levels}%
  \index{stationary states}%
  \index{Schr\"odinger equation!eigenvalue problem}%
  \index{quantum mechanics!eigenvalue problem}%
  The time-independent Schr\"{o}dinger equation
  $H|\psi\rangle=E|\psi\rangle$ is an eigenvalue problem whose
  eigenvalues $E_{n}$ are the allowed energy levels.  The
  eigenstates $|\psi_{n}\rangle$ form a complete orthonormal set,
  and the general state is $|\psi(t)\rangle=\sum_{n}c_{n}\,
  e^{-iE_{n}t/\hbar}|\psi_{n}\rangle$.

\item \textbf{Normal modes and resonance frequencies.}%
  \index{normal modes!eigenvalue problem}%
  \index{resonance frequencies}%
  \index{vibration!eigenvalue problem}%
  The generalised eigenvalue problem $(K-\omega^{2}M)\mathbf{u}=0$
  for a structure with stiffness $K$ and mass $M$ gives the natural
  frequencies $\omega_{k}$ and mode shapes $\mathbf{u}_{k}$.
  Resonance occurs when an external driving frequency matches an
  eigenfrequency, leading to large-amplitude response.

\item \textbf{Stability analysis of dynamical systems.}%
  \index{stability analysis!eigenvalues}%
  \index{linearisation}%
  \index{Lyapunov stability}%
  The stability of a fixed point $\mathbf{x}_{0}$ of
  $\dot{\mathbf{x}}=\mathbf{f}(\mathbf{x})$ is determined by the
  eigenvalues of the Jacobian $J_{ij}=\partial f_{i}/\partial x_{j}
  |_{\mathbf{x}_{0}}$.  If all eigenvalues have negative real part,
  the fixed point is asymptotically stable; if any has positive real
  part, it is unstable.

\item \textbf{Principal component analysis (PCA).}%
  \index{principal component analysis}%
  \index{eigenvalues!PCA}%
  \index{dimensionality reduction}%
  PCA computes the eigenvectors and eigenvalues of the covariance
  matrix $\Sigma$.  The eigenvectors (principal components) define
  directions of maximal variance, and the eigenvalues quantify the
  variance along each direction.  Retaining the top $k$ components
  gives the optimal $k$-dimensional approximation of the data.
\end{enumerate}

\paragraph{Mathematics applications.}
\begin{enumerate}
\item \textbf{Characteristic polynomial and algebraic multiplicity.}%
  \index{characteristic polynomial!eigenvalues}%
  \index{algebraic multiplicity}%
  \index{geometric multiplicity}%
  The characteristic polynomial $p(\lambda)=\det(\lambda I-A)$ has
  degree $n$ with roots $\lambda_{1},\dots,\lambda_{n}$ (counted
  with algebraic multiplicity).  The algebraic multiplicity of
  $\lambda$ is its multiplicity as a root; the geometric
  multiplicity $\dim\ker(A-\lambda I)$ satisfies $1\leq g\leq a$.
  Diagonalisability requires $g=a$ for every eigenvalue.

\item \textbf{Perron--Frobenius theorem.}%
  \index{Perron--Frobenius theorem}%
  \index{non-negative matrix}%
  \index{spectral radius!Perron--Frobenius}%
  A non-negative irreducible matrix $A\geq 0$ has a unique largest
  eigenvalue $\lambda_{\max}>0$ (the Perron root) with a positive
  eigenvector.  This theorem governs population dynamics (Leslie
  matrix), web page ranking (PageRank), and convergence of iterative
  methods.

\item \textbf{Spectral graph theory.}%
  \index{spectral graph theory}%
  \index{graph Laplacian}%
  \index{Fiedler vector}%
  The eigenvalues of the adjacency matrix and the graph Laplacian
  $L=D-A$ encode graph properties: the number of zero eigenvalues
  of $L$ counts connected components, and the second-smallest
  eigenvalue (algebraic connectivity, or Fiedler value) measures
  how well-connected the graph is.  The corresponding Fiedler vector
  is used for spectral graph partitioning.
\end{enumerate}

%% -------------------------------------------------------------------
\subsubsection{13.127\quad Nilpotent matrix}

A matrix $N$ is nilpotent if $N^{k}=0$ for some positive integer $k$.
The smallest such $k$ is the \emph{index of nilpotency}.  All
eigenvalues of a nilpotent matrix are zero.

\paragraph{Physics applications.}
\begin{enumerate}
\item \textbf{Raising and lowering operators in angular momentum.}%
  \index{nilpotent matrix!raising operator}%
  \index{raising operator}%
  \index{lowering operator}%
  \index{angular momentum!ladder operators}%
  The raising operator $J_{+}=J_{x}+iJ_{y}$ restricted to a
  finite-dimensional spin-$j$ representation satisfies
  $J_{+}^{2j+1}=0$: it is nilpotent of index $2j+1$.  The matrix
  representation in the $|j,m\rangle$ basis has entries
  $(J_{+})_{m',m}=\hbar\sqrt{j(j+1)-m(m+1)}\,\delta_{m',m+1}$,
  a strictly upper triangular (hence nilpotent) matrix.

\item \textbf{Grassmann variables and fermionic coherent states.}%
  \index{Grassmann algebra!nilpotent}%
  \index{fermionic coherent states}%
  \index{path integral!fermionic}%
  Grassmann variables $\theta$ satisfy $\theta^{2}=0$, the algebraic
  analogue of nilpotency.  In the path integral formulation of
  fermionic quantum field theory, integration over Grassmann variables
  replaces the trace over Fock space:
  $\mathrm{tr}(e^{-\beta H})=\int e^{-S[\bar{\theta},\theta]}
  \,d\bar{\theta}\,d\theta$.

\item \textbf{BRST operator in gauge theory.}%
  \index{BRST symmetry}%
  \index{nilpotent matrix!BRST}%
  \index{ghost fields}%
  The BRST operator $Q$ satisfies $Q^{2}=0$ (nilpotent of index 2).
  Physical states are defined as the cohomology of $Q$:
  $|\text{phys}\rangle\in\ker Q/\mathrm{im}\,Q$.
  This nilpotency is essential for the consistency of gauge-fixed
  quantum field theories and the decoupling of ghost fields.
\end{enumerate}

\paragraph{Mathematics applications.}
\begin{enumerate}
\item \textbf{Jordan canonical form.}%
  \index{Jordan canonical form!nilpotent}%
  \index{Jordan block}%
  \index{nilpotent matrix!Jordan form}%
  Every nilpotent matrix is similar to a direct sum of Jordan blocks
  $J_{k}(0)$ (with zeros on the diagonal and ones on the
  superdiagonal).  The partition giving the sizes of the Jordan
  blocks uniquely determines the similarity class and the
  \emph{nilpotent orbit} in $\mathrm{Mat}_{n}$.

\item \textbf{Nilpotent Lie algebras and the lower central series.}%
  \index{nilpotent Lie algebra}%
  \index{lower central series}%
  \index{Engel's theorem}%
  Engel's theorem states that a Lie algebra $\mathfrak{g}$ is
  nilpotent if and only if every $\mathrm{ad}_{X}$ ($X\in
  \mathfrak{g}$) is a nilpotent endomorphism.  The Heisenberg
  algebra, with $[x,y]=z$ and all other brackets zero, is the
  prototypical nilpotent Lie algebra and plays a fundamental role
  in quantum mechanics.
\end{enumerate}

%% -------------------------------------------------------------------
\subsubsection{13.128\quad Idempotent matrix}

A matrix $P$ is idempotent if $P^{2}=P$.  Its eigenvalues are 0
and 1, and $\mathrm{tr}(P)=\mathrm{rank}(P)$.

\paragraph{Physics applications.}
\begin{enumerate}
\item \textbf{Projection operators in quantum mechanics.}%
  \index{idempotent matrix!projection operator}%
  \index{projection operator!quantum}%
  \index{measurement!projection}%
  \index{L\"uders rule}%
  The projection onto an eigenspace $P_{\lambda}=|\lambda\rangle
  \langle\lambda|$ is idempotent and Hermitian.  Measurement of an
  observable $A=\sum_{\lambda}\lambda\,P_{\lambda}$ projects the
  state via the L\"{u}ders rule: $\rho\mapsto P_{\lambda}\rho\,
  P_{\lambda}/\mathrm{tr}(P_{\lambda}\rho)$ upon obtaining
  outcome $\lambda$.

\item \textbf{Projection operators in regression.}%
  \index{hat matrix}%
  \index{least squares!projection}%
  \index{residual projection}%
  In linear regression $\mathbf{y}=X\boldsymbol{\beta}+
  \boldsymbol{\varepsilon}$, the hat matrix $H=X(X^{T}X)^{-1}X^{T}$
  is idempotent and symmetric, projecting $\mathbf{y}$ onto the
  column space of $X$.  The residual projection $I-H$ is also
  idempotent: $\hat{\boldsymbol{\varepsilon}}=(I-H)\mathbf{y}$.

\item \textbf{Density matrix of a pure state.}%
  \index{density matrix!pure state}%
  \index{idempotent matrix!pure state}%
  \index{purity!idempotent condition}%
  A density matrix $\rho$ represents a pure state if and only if
  $\rho^{2}=\rho$ (idempotent).  In this case $\rho=|\psi\rangle
  \langle\psi|$ is a rank-one projector, and the purity
  $\mathrm{tr}(\rho^{2})=1$ is maximal.  Mixed states satisfy
  $\rho^{2}\neq\rho$ and $\mathrm{tr}(\rho^{2})<1$.
\end{enumerate}

\paragraph{Mathematics applications.}
\begin{enumerate}
\item \textbf{Direct sum decomposition.}%
  \index{idempotent matrix!direct sum}%
  \index{direct sum decomposition}%
  \index{complementary subspace}%
  An idempotent $P$ decomposes $V=\mathrm{im}(P)\oplus\ker(P)$,
  and $I-P$ is the complementary idempotent.  A set of idempotents
  $\{P_{1},\dots,P_{k}\}$ with $\sum P_{i}=I$ and $P_{i}P_{j}=
  \delta_{ij}P_{i}$ gives a complete orthogonal decomposition of
  the identity, the matrix version of a partition of unity.

\item \textbf{$K$-theory and idempotents over rings.}%
  \index{$K$-theory!idempotents}%
  \index{projective module}%
  \index{idempotent matrix!$K$-theory}%
  A finitely generated projective module over a ring $R$ is the image
  of an idempotent $e\in\mathrm{Mat}_{n}(R)$.  The Grothendieck
  group $K_{0}(R)$ is built from equivalence classes of idempotents,
  providing a bridge between linear algebra and algebraic topology.
\end{enumerate}

%% -------------------------------------------------------------------
\subsubsection{13.129\quad Positive definite}

A Hermitian matrix $A$ is positive definite if
$\mathbf{x}^{\dagger}A\mathbf{x}>0$ for all
$\mathbf{x}\neq\mathbf{0}$, equivalently if all eigenvalues are
strictly positive.

\paragraph{Physics applications.}
\begin{enumerate}
\item \textbf{Kinetic energy and mass matrices.}%
  \index{positive definite!kinetic energy}%
  \index{mass matrix!positive definite}%
  \index{kinetic energy!quadratic form}%
  The kinetic energy $T=\frac{1}{2}\dot{\mathbf{q}}^{T}M
  \dot{\mathbf{q}}$ requires the mass matrix $M$ to be positive
  definite so that $T>0$ for any nonzero velocity.  This is a
  physical requirement: kinetic energy cannot be negative.  Positive
  definiteness of $M$ also guarantees that the generalised eigenvalue
  problem $K\mathbf{u}=\omega^{2}M\mathbf{u}$ has real positive
  eigenfrequencies.

\item \textbf{Metric tensor in Riemannian geometry.}%
  \index{positive definite!metric tensor}%
  \index{Riemannian metric!positive definite}%
  \index{proper distance}%
  A Riemannian metric $g_{ij}$ is a positive definite symmetric
  tensor field: $ds^{2}=g_{ij}\,dx^{i}dx^{j}>0$ for any nonzero
  displacement.  This ensures a well-defined notion of distance.
  In general relativity, the metric is only required to be
  non-degenerate (signature $(-,+,+,+)$), not positive definite.

\item \textbf{Fisher information matrix.}%
  \index{Fisher information matrix}%
  \index{Cram\'er--Rao bound}%
  \index{statistical estimation!positive definite}%
  The Fisher information matrix
  $\mathcal{I}_{ij}(\theta)=-E[\partial^{2}\ln L/\partial\theta_{i}
  \partial\theta_{j}]$ is positive semidefinite (positive definite
  when the parameters are identifiable).  The Cram\'{e}r--Rao bound
  $\mathrm{Cov}(\hat{\theta})\geq\mathcal{I}^{-1}$ states that no
  unbiased estimator can have covariance smaller than the inverse
  Fisher information.
\end{enumerate}

\paragraph{Mathematics applications.}
\begin{enumerate}
\item \textbf{Cholesky decomposition and inner products.}%
  \index{Cholesky decomposition!positive definite}%
  \index{inner product!positive definite matrix}%
  \index{Gram matrix}%
  A Hermitian matrix is positive definite if and only if it has a
  (unique) Cholesky factorisation $A=LL^{\dagger}$ with $L$ lower
  triangular and positive diagonal entries.  Equivalently, $A$
  defines an inner product $\langle\mathbf{x},\mathbf{y}\rangle_{A}
  =\mathbf{x}^{\dagger}A\mathbf{y}$, and every Gram matrix of a
  linearly independent set is positive definite.

\item \textbf{Sylvester's criterion.}%
  \index{Sylvester's criterion}%
  \index{leading principal minors}%
  \index{positive definite!criterion}%
  A Hermitian matrix is positive definite if and only if all leading
  principal minors are positive: $\Delta_{k}=\det(A_{k\times k})>0$
  for $k=1,\dots,n$.  This provides a computationally efficient test
  ($O(n^{3})$) without computing eigenvalues.
\end{enumerate}

%% -------------------------------------------------------------------
\subsubsection{13.130\quad Non-negative definite}

A Hermitian matrix $A$ is non-negative definite (positive semidefinite)
if $\mathbf{x}^{\dagger}A\mathbf{x}\geq 0$ for all $\mathbf{x}$,
equivalently if all eigenvalues are non-negative.

\paragraph{Physics applications.}
\begin{enumerate}
\item \textbf{Density matrices and quantum states.}%
  \index{non-negative definite!density matrix}%
  \index{density matrix!positivity}%
  \index{quantum state!positivity}%
  A valid density matrix must be positive semidefinite ($\rho\geq 0$)
  with $\mathrm{tr}(\rho)=1$.  Positivity ensures non-negative
  probabilities: $p_{\lambda}=\langle\lambda|\rho|\lambda\rangle
  \geq 0$ for every state $|\lambda\rangle$.  Entanglement witnesses
  are operators $W$ such that $\mathrm{tr}(W\rho)<0$ for some
  entangled states, detecting violation of positivity under partial
  transpose.

\item \textbf{Noise covariance in signal processing.}%
  \index{covariance matrix!positive semidefinite}%
  \index{noise covariance}%
  \index{Wiener filter}%
  The noise covariance matrix $R_{nn}=E[\mathbf{n}\mathbf{n}^{\dagger}]$
  is positive semidefinite by construction.  The Wiener filter
  minimising mean-square error involves $R_{nn}^{-1}$ (when positive
  definite) or $R_{nn}^{+}$ (Moore--Penrose pseudoinverse when
  singular), and the semidefiniteness ensures the error is bounded
  below by zero.

\item \textbf{Correlation matrices in finance.}%
  \index{correlation matrix}%
  \index{portfolio optimisation}%
  \index{Markowitz model}%
  The asset return correlation matrix $C_{ij}=\mathrm{Corr}(R_{i},
  R_{j})$ must be positive semidefinite.  In the Markowitz
  mean-variance model, the portfolio variance
  $\sigma_{p}^{2}=\mathbf{w}^{T}C\mathbf{w}\geq 0$ is non-negative
  by the semidefiniteness of $C$, and the efficient frontier is
  a quadratic optimisation problem over the simplex.
\end{enumerate}

\paragraph{Mathematics applications.}
\begin{enumerate}
\item \textbf{Semidefinite programming.}%
  \index{semidefinite programming}%
  \index{convex optimisation!semidefinite}%
  \index{linear matrix inequality}%
  Semidefinite programming (SDP) optimises a linear objective subject
  to a linear matrix inequality $A_{0}+\sum_{i}x_{i}A_{i}\succeq 0$.
  The cone of positive semidefinite matrices is a self-dual convex
  cone, and interior-point methods solve SDPs in polynomial time.
  Applications range from combinatorial optimisation (Max-Cut) to
  quantum information (entanglement detection).

\item \textbf{Reproducing kernel Hilbert spaces.}%
  \index{reproducing kernel Hilbert space}%
  \index{kernel matrix!positive semidefinite}%
  \index{Mercer's theorem}%
  A kernel $k(x,y)$ defines a reproducing kernel Hilbert space if and
  only if the kernel matrix $K_{ij}=k(x_{i},x_{j})$ is positive
  semidefinite for all finite sets $\{x_{1},\dots,x_{n}\}$ (Mercer's
  condition).  This is the foundation of kernel methods in machine
  learning (support vector machines, Gaussian processes).
\end{enumerate}

%% -------------------------------------------------------------------
\subsubsection{13.131\quad Diagonally dominant}

A matrix $A$ is (strictly) diagonally dominant if $|A_{ii}|>\sum_{j
\neq i}|A_{ij}|$ for every row $i$.

\paragraph{Physics applications.}
\begin{enumerate}
\item \textbf{Convergence of iterative solvers.}%
  \index{diagonally dominant!iterative methods}%
  \index{Jacobi iteration}%
  \index{Gauss--Seidel iteration}%
  The Jacobi and Gauss--Seidel iterative methods are guaranteed to
  converge for strictly diagonally dominant systems.  In computational
  physics, large sparse systems arising from finite difference
  discretisations of elliptic PDEs (e.g., the Laplace equation with
  a 5-point stencil) are often diagonally dominant, ensuring rapid
  convergence.

\item \textbf{Stability of finite difference schemes.}%
  \index{finite difference!diagonal dominance}%
  \index{stability!numerical}%
  \index{von Neumann stability analysis}%
  Implicit time-stepping schemes (e.g., backward Euler, Crank--Nicolson)
  for parabolic PDEs produce diagonally dominant linear systems when the
  time step satisfies a CFL-type condition.  Diagonal dominance implies
  the non-singularity of the system matrix and bounds the growth of
  numerical errors.

\item \textbf{Network equilibrium in electrical circuits.}%
  \index{nodal analysis!diagonal dominance}%
  \index{electrical network}%
  \index{admittance matrix}%
  The nodal admittance matrix $Y$ of a passive electrical network is
  diagonally dominant (with equality for floating networks).  The
  diagonal entry $Y_{ii}$ is the sum of all admittances connected to
  node $i$, and $|Y_{ii}|\geq\sum_{j\neq i}|Y_{ij}|$ by
  construction, ensuring a unique voltage solution.
\end{enumerate}

\paragraph{Mathematics applications.}
\begin{enumerate}
\item \textbf{Gershgorin circle theorem.}%
  \index{Gershgorin circle theorem}%
  \index{eigenvalue localisation}%
  \index{diagonally dominant!Gershgorin}%
  Every eigenvalue of $A$ lies in at least one Gershgorin disc
  $D_{i}=\{z\in\mathbb{C}:|z-A_{ii}|\leq\sum_{j\neq i}|A_{ij}|\}$.
  For a strictly diagonally dominant matrix, no disc contains the
  origin, so $A$ is non-singular.  The theorem provides cheap
  eigenvalue bounds without computing the characteristic polynomial.

\item \textbf{$M$-matrices and monotonicity.}%
  \index{$M$-matrix}%
  \index{monotone matrix}%
  \index{diagonally dominant!$M$-matrix}%
  A $Z$-matrix (non-positive off-diagonal entries) that is also
  diagonally dominant is an $M$-matrix: $A^{-1}\geq 0$
  (entrywise non-negative).  $M$-matrices arise in the
  discretisation of elliptic operators and guarantee the
  maximum principle at the discrete level.
\end{enumerate}

%% -------------------------------------------------------------------
\subsection{13.21\quad Quadratic Forms}
\subsubsection{13.211\quad Sylvester's law of inertia}

Sylvester's law of inertia states that for any real symmetric matrix
$A$, the number of positive, negative, and zero eigenvalues is
invariant under congruence transformations $A\mapsto S^{T}AS$.

\paragraph{Physics applications.}
\begin{enumerate}
\item \textbf{Metric signature in special and general relativity.}%
  \index{Sylvester's law!metric signature}%
  \index{metric signature}%
  \index{Minkowski metric}%
  \index{Lorentzian manifold}%
  The Minkowski metric $\eta_{\mu\nu}=\mathrm{diag}(-1,+1,+1,+1)$
  has signature $(1,3)$ (one negative, three positive eigenvalues).
  Sylvester's law guarantees that no coordinate transformation can
  change this signature: the distinction between time and space is
  an invariant of the metric.  In general relativity, the signature
  of $g_{\mu\nu}$ is $(1,3)$ everywhere on a Lorentzian manifold.

\item \textbf{Stability of equilibria via the Hessian.}%
  \index{Hessian matrix!stability}%
  \index{stability!Hessian}%
  \index{second derivative test}%
  The Hessian $H_{ij}=\partial^{2}V/\partial q_{i}\partial q_{j}$
  of the potential energy at an equilibrium determines stability.
  By Sylvester's law, the inertia $(n_{+},n_{-},n_{0})$ of $H$ is
  a congruence invariant: a minimum requires $n_{+}=n$ (positive
  definite), while a saddle has $n_{-}\geq 1$, independent of
  the choice of generalised coordinates.

\item \textbf{Classification of conic sections and quadrics.}%
  \index{conic sections!classification}%
  \index{quadric surfaces}%
  \index{Sylvester's law!conics}%
  The general conic $\mathbf{x}^{T}A\mathbf{x}+\mathbf{b}^{T}
  \mathbf{x}+c=0$ is classified by the inertia of $A$:
  $(2,0)$ for ellipses, $(1,1)$ for hyperbolas, and rank-deficient
  cases for parabolas.  In three dimensions, the inertia of the
  $3\times 3$ matrix classifies quadric surfaces (ellipsoids,
  hyperboloids, paraboloids, cones, cylinders).
\end{enumerate}

\paragraph{Mathematics applications.}
\begin{enumerate}
\item \textbf{Classification of bilinear forms.}%
  \index{bilinear form!classification}%
  \index{congruence!classification}%
  \index{Sylvester's law!bilinear forms}%
  Over $\mathbb{R}$, every symmetric bilinear form is congruent to
  $\mathrm{diag}(1,\dots,1,-1,\dots,-1,0,\dots,0)$ with signature
  $(p,q)$.  The pair $(p,q)$ is the complete invariant for real
  symmetric bilinear forms under congruence.  Over $\mathbb{C}$,
  only the rank matters.

\item \textbf{Morse theory and critical points.}%
  \index{Morse theory!index}%
  \index{critical point!index}%
  \index{Sylvester's law!Morse index}%
  The Morse index of a non-degenerate critical point of a smooth
  function $f$ is the number of negative eigenvalues of the
  Hessian, which by Sylvester's law is a well-defined integer
  independent of the choice of local coordinates.  The Morse
  inequalities relate these indices to the Betti numbers of the
  manifold.
\end{enumerate}

%% -------------------------------------------------------------------
\subsubsection{13.212\quad Rank}

The rank of a matrix is the dimension of its column (or row) space,
equivalently the number of nonzero singular values.

\paragraph{Physics applications.}
\begin{enumerate}
\item \textbf{Degeneracy and constraint counting.}%
  \index{rank!constraint counting}%
  \index{degeneracy!rank deficiency}%
  \index{degrees of freedom}%
  In a system of linear constraints $A\mathbf{x}=\mathbf{b}$,
  the number of independent constraints is $\mathrm{rank}(A)$ and
  the number of free parameters (degrees of freedom) is
  $n-\mathrm{rank}(A)$.  Rank deficiency signals degeneracy---for
  example, a degenerate eigenvalue in quantum mechanics or a
  gauge symmetry in field theory.

\item \textbf{Rank of the stress-energy tensor.}%
  \index{stress-energy tensor!rank}%
  \index{electromagnetic stress tensor}%
  \index{rank!stress tensor}%
  The electromagnetic stress-energy tensor $T^{\mu\nu}$ is a
  $4\times 4$ symmetric matrix.  For a null electromagnetic field
  ($\mathbf{E}\perp\mathbf{B}$, $|\mathbf{E}|=|\mathbf{B}|$), the
  rank of $T^{\mu\nu}$ drops to 1 (all energy flows in one null
  direction), while for a general field the rank is 4.

\item \textbf{Schmidt rank and entanglement.}%
  \index{Schmidt decomposition}%
  \index{entanglement!Schmidt rank}%
  \index{rank!entanglement}%
  A bipartite quantum state $|\psi\rangle\in H_{A}\otimes H_{B}$
  has a Schmidt decomposition $|\psi\rangle=\sum_{k=1}^{r}
  \sqrt{p_{k}}\,|a_{k}\rangle|b_{k}\rangle$ with Schmidt rank
  $r=\mathrm{rank}(\rho_{A})$.  The state is entangled if and
  only if $r>1$.
\end{enumerate}

\paragraph{Mathematics applications.}
\begin{enumerate}
\item \textbf{Rank-nullity theorem.}%
  \index{rank-nullity theorem!rank}%
  \index{dimension formula}%
  \index{kernel and image}%
  For $A\in\mathbb{F}^{m\times n}$, $\mathrm{rank}(A)+
  \mathrm{nullity}(A)=n$.  This dimension formula is the
  finite-dimensional case of the first isomorphism theorem:
  $V/\ker(T)\cong\mathrm{im}(T)$.

\item \textbf{Low-rank approximation and compressed sensing.}%
  \index{low-rank approximation}%
  \index{Eckart--Young theorem}%
  \index{compressed sensing}%
  The Eckart--Young theorem states that the best rank-$k$
  approximation to $A$ (in Frobenius or operator norm) is
  $A_{k}=\sum_{i=1}^{k}\sigma_{i}\mathbf{u}_{i}\mathbf{v}_{i}^{T}$
  from the SVD.  Low-rank structure is exploited in compressed
  sensing, matrix completion (Netflix problem), and tensor networks
  in quantum many-body physics.
\end{enumerate}

%% -------------------------------------------------------------------
\subsubsection{13.213\quad Signature}

The signature $(p,q)$ of a real symmetric matrix is the number of
positive and negative eigenvalues.  It is a congruence invariant
by Sylvester's law.

\paragraph{Physics applications.}
\begin{enumerate}
\item \textbf{Spacetime signature and causal structure.}%
  \index{signature!spacetime}%
  \index{causal structure}%
  \index{light cone}%
  The signature $(1,3)$ of the Minkowski metric determines the
  causal structure of spacetime: the light cone
  $\eta_{\mu\nu}\,dx^{\mu}\,dx^{\nu}=0$ separates timelike,
  spacelike, and null directions.  A Euclidean signature $(0,4)$
  (Wick rotation $t\to i\tau$) converts the Lorentzian path
  integral to a statistical-mechanics partition function.

\item \textbf{Signature change and cosmological models.}%
  \index{signature change}%
  \index{Hartle--Hawking state}%
  \index{quantum cosmology}%
  The Hartle--Hawking no-boundary proposal for the wave function of
  the universe involves a transition from Euclidean signature $(0,4)$
  to Lorentzian $(1,3)$.  The signature of the metric is the
  fundamental distinction between space and time, and its possible
  change at the Planck scale is a topic of quantum gravity research.

\item \textbf{Indefinite inner products in BRST quantisation.}%
  \index{indefinite inner product}%
  \index{BRST quantisation!indefinite metric}%
  \index{Gupta--Bleuler quantisation}%
  The Gupta--Bleuler and BRST methods for quantising gauge fields use
  an indefinite-metric state space (signature $(p,q)$ with both $p$
  and $q$ nonzero).  Physical states form a positive-definite
  subspace, and the ghosts (negative-norm states) decouple from
  physical amplitudes.
\end{enumerate}

\paragraph{Mathematics applications.}
\begin{enumerate}
\item \textbf{Clifford algebras and signature.}%
  \index{Clifford algebra!signature}%
  \index{signature!Clifford algebra}%
  \index{Bott periodicity}%
  The Clifford algebra $\mathrm{Cl}(p,q)$ generated by
  $\gamma_{i}\gamma_{j}+\gamma_{j}\gamma_{i}=2g_{ij}$ with
  $g=\mathrm{diag}(\underbrace{+1}_{p},\underbrace{-1}_{q})$
  depends on the signature $(p,q)$.  The isomorphism class of
  $\mathrm{Cl}(p,q)$ exhibits Bott periodicity with period~8,
  connecting to the classification of real division algebras and
  topological $K$-theory.

\item \textbf{Witt group and quadratic form theory.}%
  \index{Witt group}%
  \index{quadratic form!Witt equivalence}%
  \index{hyperbolic plane}%
  Two quadratic forms are Witt-equivalent if they become isometric
  after adding hyperbolic planes $\begin{pmatrix}0&1\\1&0
  \end{pmatrix}$.  The Witt group $W(\mathbb{F})$ classifies
  non-degenerate symmetric bilinear forms modulo hyperbolic forms
  and is a fundamental invariant in algebraic number theory.
\end{enumerate}

%% -------------------------------------------------------------------
\subsubsection{13.214\quad Positive definite and semidefinite quadratic form}

A quadratic form $Q(\mathbf{x})=\mathbf{x}^{T}A\mathbf{x}$ is
positive definite if $Q(\mathbf{x})>0$ for all
$\mathbf{x}\neq\mathbf{0}$, and positive semidefinite if
$Q(\mathbf{x})\geq 0$.

\paragraph{Physics applications.}
\begin{enumerate}
\item \textbf{Potential energy and stability.}%
  \index{quadratic form!potential energy}%
  \index{stability!positive definite}%
  \index{small oscillations}%
  Near a stable equilibrium, the potential energy
  $V\approx\frac{1}{2}\mathbf{q}^{T}K\mathbf{q}$ is a positive
  definite quadratic form ($K>0$), ensuring a restoring force for
  any displacement.  The eigenvalues of $M^{-1}K$ give the squared
  normal-mode frequencies $\omega_{k}^{2}$, all positive.

\item \textbf{Thermodynamic stability conditions.}%
  \index{thermodynamic stability}%
  \index{quadratic form!thermodynamics}%
  \index{Le Chatelier principle}%
  The stability of a thermodynamic equilibrium requires the second
  variation of the entropy to be negative definite (equivalently,
  the Hessian of the internal energy with respect to extensive
  variables is positive definite).  This gives the conditions
  $C_{V}>0$ (thermal stability) and $(\partial P/\partial V)_{T}<0$
  (mechanical stability).

\item \textbf{Electromagnetic energy density.}%
  \index{energy density!electromagnetic}%
  \index{permittivity tensor!positive definite}%
  \index{permeability tensor}%
  The electromagnetic energy density $u=\frac{1}{2}(\mathbf{E}
  \cdot\mathbf{D}+\mathbf{B}\cdot\mathbf{H})=\frac{1}{2}
  (\mathbf{E}^{T}\varepsilon\mathbf{E}+\mathbf{B}^{T}
  \mu^{-1}\mathbf{B})$ is positive definite when the permittivity
  tensor $\varepsilon$ and inverse permeability $\mu^{-1}$ are
  positive definite, which holds for passive media.
\end{enumerate}

\paragraph{Mathematics applications.}
\begin{enumerate}
\item \textbf{Convexity and optimisation.}%
  \index{convexity!positive definite Hessian}%
  \index{quadratic programming}%
  \index{optimisation!convex}%
  A twice-differentiable function is (strictly) convex if and only if
  its Hessian is positive semidefinite (definite) everywhere.
  Quadratic programming minimises
  $\frac{1}{2}\mathbf{x}^{T}Q\mathbf{x}+\mathbf{c}^{T}\mathbf{x}$
  subject to linear constraints; for $Q\succ 0$ the problem has a
  unique global minimum.

\item \textbf{Lattices and number theory.}%
  \index{lattice!positive definite form}%
  \index{number theory!quadratic forms}%
  \index{Minkowski's theorem}%
  A positive definite quadratic form $Q(\mathbf{n})=\mathbf{n}^{T}A
  \mathbf{n}$ with $\mathbf{n}\in\mathbb{Z}^{k}$ defines a lattice
  in $\mathbb{R}^{k}$.  The minimum $\min_{\mathbf{n}\neq 0}
  Q(\mathbf{n})$ is the squared length of the shortest lattice
  vector, a central quantity in the geometry of numbers and the basis
  of lattice-based cryptography.
\end{enumerate}

%% -------------------------------------------------------------------
\subsubsection{13.215\quad Basic theorems on quadratic forms}

This subsubsection collects the principal structural results on
real and complex quadratic forms, including diagonalisation,
canonical forms, and invariant characterisations.

\paragraph{Physics applications.}
\begin{enumerate}
\item \textbf{Diagonalisation of the Hamiltonian for coupled systems.}%
  \index{quadratic form!Hamiltonian}%
  \index{Bogoliubov transformation}%
  \index{coupled oscillators!quadratic form}%
  A quadratic Hamiltonian $H=\frac{1}{2}\mathbf{z}^{T}\mathcal{H}
  \mathbf{z}$ (where $\mathbf{z}=(q_{1},\dots,q_{n},p_{1},\dots,
  p_{n})^{T}$) is diagonalised by a symplectic (canonical)
  transformation $\mathbf{z}=S\boldsymbol{\zeta}$ satisfying
  $S^{T}JS=J$.  The resulting normal-mode Hamiltonian
  $H=\sum_{k}\frac{\omega_{k}}{2}(\zeta_{k}^{2}+\pi_{k}^{2})$
  decouples into independent oscillators.  In the quantum case,
  this is the Bogoliubov transformation.

\item \textbf{Index of a quadratic form and the Morse lemma.}%
  \index{Morse lemma}%
  \index{quadratic form!index}%
  \index{saddle point!index}%
  The Morse lemma states that near a non-degenerate critical point,
  a smooth function can be put in the form
  $f=f_{0}-y_{1}^{2}-\cdots-y_{\lambda}^{2}+y_{\lambda+1}^{2}
  +\cdots+y_{n}^{2}$, where $\lambda$ is the index (number of
  negative squares).  In the path integral, saddle points with
  different indices contribute with different phases to the
  semiclassical approximation.

\item \textbf{Williamson's theorem and quantum uncertainty.}%
  \index{Williamson's theorem}%
  \index{symplectic eigenvalues}%
  \index{uncertainty relation!covariance matrix}%
  Williamson's theorem states that any positive definite $2n\times 2n$
  matrix can be brought to the form $S^{T}AS=\mathrm{diag}(d_{1},
  \dots,d_{n},d_{1},\dots,d_{n})$ by a symplectic transformation.
  The symplectic eigenvalues $d_{k}$ characterise Gaussian quantum
  states, and the uncertainty relation becomes $d_{k}\geq\hbar/2$.
\end{enumerate}

\paragraph{Mathematics applications.}
\begin{enumerate}
\item \textbf{Simultaneous diagonalisation of two quadratic forms.}%
  \index{simultaneous diagonalisation!quadratic forms}%
  \index{generalised eigenvalue problem}%
  \index{pencil of quadratic forms}%
  If $A$ is positive definite and $B$ is symmetric, there exists an
  invertible $S$ such that $S^{T}AS=I$ and $S^{T}BS=\Lambda$
  (diagonal).  This reduces the generalised eigenvalue problem
  $B\mathbf{v}=\lambda A\mathbf{v}$ to an ordinary one.  The
  pencil $\det(B-\lambda A)=0$ defines the eigenvalues.

\item \textbf{Representation numbers and theta functions.}%
  \index{theta function!quadratic form}%
  \index{representation numbers}%
  \index{modular form!quadratic form}%
  The number of representations $r_{Q}(n)=\#\{\mathbf{m}\in
  \mathbb{Z}^{k}:Q(\mathbf{m})=n\}$ is encoded by the theta
  function $\Theta_{Q}(\tau)=\sum_{\mathbf{m}}e^{2\pi i\tau\,
  Q(\mathbf{m})}$, which is a modular form.  Jacobi's four-square
  theorem $r_{4}(n)=8\sum_{4\nmid d|n}d$ is a classical application.
\end{enumerate}

%% -------------------------------------------------------------------
\subsection{13.31\quad Differentiation of Matrices}

Matrix differentiation extends ordinary calculus to matrix-valued
functions.  For a matrix $A(t)$ depending on a parameter $t$,
$dA/dt$ has entries $(dA/dt)_{ij}=dA_{ij}/dt$.

\paragraph{Physics applications.}
\begin{enumerate}
\item \textbf{Equations of motion for density matrices.}%
  \index{matrix differentiation!density matrix}%
  \index{von Neumann equation}%
  \index{Lindblad equation}%
  \index{Liouville--von Neumann equation|see{von Neumann equation}}%
  The von Neumann equation $i\hbar\,d\rho/dt=[H,\rho]$ governs the
  time evolution of the density matrix for a closed system.  For open
  systems, the Lindblad master equation adds dissipative terms:
  $d\rho/dt=-\frac{i}{\hbar}[H,\rho]+\sum_{k}\left(L_{k}\rho
  L_{k}^{\dagger}-\frac{1}{2}\{L_{k}^{\dagger}L_{k},\rho\}\right)$.

\item \textbf{Matrix Riccati equation in control theory.}%
  \index{Riccati equation!matrix}%
  \index{optimal control}%
  \index{LQR controller}%
  The linear-quadratic regulator (LQR) problem leads to the matrix
  Riccati equation $\dot{P}=-PA-A^{T}P-Q+PBR^{-1}B^{T}P$ for the
  cost-to-go matrix $P(t)$.  The steady-state solution (algebraic
  Riccati equation with $\dot{P}=0$) gives the optimal feedback gain
  $K=R^{-1}B^{T}P$.

\item \textbf{Gradient descent on matrix manifolds.}%
  \index{matrix differentiation!gradient descent}%
  \index{optimisation!matrix manifold}%
  \index{Riemannian gradient}%
  Training neural networks requires derivatives with respect to weight
  matrices: $\partial\mathcal{L}/\partial W$.  On structured matrix
  manifolds (e.g., the Stiefel manifold of orthonormal frames), the
  Riemannian gradient projects the Euclidean gradient onto the tangent
  space, and retraction maps ensure iterates remain on the manifold.

\item \textbf{Jacobi's formula for the determinant.}%
  \index{Jacobi's formula}%
  \index{determinant!derivative}%
  \index{matrix differentiation!determinant}%
  Jacobi's formula $\frac{d}{dt}\det A(t)=\det A(t)\,\mathrm{tr}
  \!\left(A^{-1}\frac{dA}{dt}\right)$ gives the rate of change of
  the determinant.  In general relativity, this yields
  $\partial_{\mu}\sqrt{-g}=\frac{1}{2}\sqrt{-g}\,g^{\alpha\beta}
  \partial_{\mu}g_{\alpha\beta}$, essential for deriving the
  covariant divergence.
\end{enumerate}

\paragraph{Mathematics applications.}
\begin{enumerate}
\item \textbf{Matrix calculus identities.}%
  \index{matrix calculus}%
  \index{chain rule!matrices}%
  \index{Kronecker product!derivative}%
  Key identities include $d\,\mathrm{tr}(AXB)=A^{T}B^{T}\,dX$
  (in the sense of Frechet derivatives) and
  $\frac{\partial}{\partial X}\mathrm{tr}(X^{T}AX)=(A+A^{T})X$.
  The vec operator and Kronecker product linearise matrix equations:
  $\mathrm{vec}(AXB)=(B^{T}\otimes A)\,\mathrm{vec}(X)$.

\item \textbf{Derivative of the matrix exponential.}%
  \index{matrix exponential!derivative}%
  \index{Wilcox formula}%
  \index{Duhamel's formula}%
  The derivative of the matrix exponential is
  $\frac{d}{dt}e^{A(t)}=\int_{0}^{1}e^{sA}\frac{dA}{dt}
  e^{(1-s)A}\,ds$ (Duhamel/Wilcox formula), which differs from the
  scalar case $\frac{d}{dt}e^{a(t)}=\dot{a}e^{a}$ because $A$ and
  $dA/dt$ need not commute.  When they do commute, the scalar formula
  is recovered.

\item \textbf{Perturbation theory for eigenvalues.}%
  \index{eigenvalue perturbation!derivative}%
  \index{Hellmann--Feynman theorem}%
  \index{matrix differentiation!eigenvalues}%
  For a Hermitian matrix $A(\epsilon)$ with non-degenerate eigenvalue
  $\lambda(\epsilon)$, the Hellmann--Feynman theorem gives
  $d\lambda/d\epsilon=\mathbf{v}^{\dagger}(dA/d\epsilon)\mathbf{v}$,
  where $\mathbf{v}$ is the normalised eigenvector.  Second-order
  perturbation theory involves the resolvent $(A-\lambda I)^{-1}$
  restricted to the orthogonal complement of $\mathbf{v}$.
\end{enumerate}

%% -------------------------------------------------------------------
\subsection{13.41\quad The Matrix Exponential}
\subsubsection{13.411\quad Basic properties}

The matrix exponential is defined by $e^{A}=\sum_{k=0}^{\infty}
A^{k}/k!$ and satisfies $e^{A}e^{B}=e^{A+B}$ when $[A,B]=0$.
The inverse is $e^{-A}$, and $\det(e^{A})=e^{\mathrm{tr}(A)}$.

\paragraph{Physics applications.}
\begin{enumerate}
\item \textbf{Time evolution operator in quantum mechanics.}%
  \index{matrix exponential!time evolution}%
  \index{time evolution operator!matrix exponential}%
  \index{Dyson series}%
  \index{time-ordered exponential}%
  For a time-independent Hamiltonian, the time evolution operator is
  $U(t)=e^{-iHt/\hbar}$.  For time-dependent $H(t)$, the Dyson
  series gives the time-ordered exponential
  $U(t)=\mathcal{T}\exp\!\left(-\frac{i}{\hbar}\int_{0}^{t}
  H(t')\,dt'\right)$, which reduces to the ordinary exponential
  when $[H(t_{1}),H(t_{2})]=0$ for all $t_{1},t_{2}$.

\item \textbf{Lie group--Lie algebra correspondence.}%
  \index{matrix exponential!Lie group}%
  \index{Lie group!exponential map}%
  \index{exponential map}%
  \index{one-parameter subgroup}%
  The matrix exponential maps the Lie algebra $\mathfrak{g}$ to the
  Lie group $G$: $\exp\colon\mathfrak{g}\to G$.  Every one-parameter
  subgroup of a matrix Lie group has the form $g(t)=e^{tX}$ for some
  $X\in\mathfrak{g}$.  For example, $e^{t\,\Omega}\in\mathrm{SO}(3)$
  for skew-symmetric $\Omega\in\mathfrak{so}(3)$ is a rotation by
  angle $t|\boldsymbol{\omega}|$ about the axis
  $\boldsymbol{\omega}/|\boldsymbol{\omega}|$.

\item \textbf{Baker--Campbell--Hausdorff formula.}%
  \index{Baker--Campbell--Hausdorff formula}%
  \index{BCH formula|see{Baker--Campbell--Hausdorff formula}}%
  \index{Zassenhaus formula}%
  When $[A,B]\neq 0$, the product $e^{A}e^{B}=e^{C}$ is given by the
  BCH formula: $C=A+B+\frac{1}{2}[A,B]+\frac{1}{12}([A,[A,B]]
  +[B,[B,A]])+\cdots$, an infinite series of nested commutators.
  This formula is essential in quantum optics (disentangling
  exponentials of boson operators) and in numerical methods
  (splitting methods for differential equations).

\item \textbf{Rotation matrices via the exponential map.}%
  \index{Rodrigues' formula!exponential}%
  \index{rotation matrix!exponential}%
  \index{matrix exponential!rotation}%
  Rodrigues' rotation formula
  $e^{\theta\,\hat{n}\cdot\mathbf{J}}=I+\sin\theta\,[\hat{n}]_{\times}
  +(1-\cos\theta)\,[\hat{n}]_{\times}^{2}$
  (where $[\hat{n}]_{\times}$ is the skew-symmetric matrix for
  the cross product with $\hat{n}$) gives the finite rotation by
  angle $\theta$ about axis $\hat{n}$ as a closed-form matrix
  exponential.  This is fundamental in robotics, computer graphics,
  and spacecraft attitude dynamics.
\end{enumerate}

\paragraph{Mathematics applications.}
\begin{enumerate}
\item \textbf{Solution of linear ODEs.}%
  \index{matrix exponential!linear ODE}%
  \index{linear ODE!matrix exponential}%
  \index{fundamental matrix}%
  The solution of $\dot{\mathbf{x}}=A\mathbf{x}$ with initial
  condition $\mathbf{x}(0)=\mathbf{x}_{0}$ is
  $\mathbf{x}(t)=e^{At}\mathbf{x}_{0}$.  The matrix exponential
  $e^{At}$ is the fundamental matrix (state transition matrix), and
  its computation is one of the ``nineteen dubious ways'' surveyed
  by Moler and Van Loan, each with distinct numerical trade-offs.

\item \textbf{Surjectivity of the exponential map.}%
  \index{exponential map!surjectivity}%
  \index{matrix logarithm}%
  \index{GL($n$)!exponential map}%
  The exponential map $\exp\colon\mathfrak{gl}(n,\mathbb{C})
  \to\mathrm{GL}(n,\mathbb{C})$ is surjective: every invertible
  complex matrix has a logarithm.  Over $\mathbb{R}$, surjectivity
  fails: a real matrix with a negative real eigenvalue of odd
  algebraic multiplicity has no real logarithm.  The exponential map
  for compact Lie groups is always surjective (by the maximal torus
  theorem).

\item \textbf{Trotter product formula.}%
  \index{Trotter product formula}%
  \index{operator splitting}%
  \index{Lie--Trotter formula|see{Trotter product formula}}%
  The Trotter product formula $e^{A+B}=\lim_{n\to\infty}
  (e^{A/n}e^{B/n})^{n}$ holds for bounded operators and extends to
  unbounded self-adjoint operators (Trotter--Kato theorem).  This
  is the mathematical basis for Suzuki--Trotter decompositions in
  quantum Monte Carlo simulations and for operator splitting in
  numerical PDEs.
\end{enumerate}
