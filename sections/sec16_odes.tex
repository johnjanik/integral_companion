%% ============================================================
%% 16  Ordinary Differential Equations
%% ============================================================
\section{16\quad Ordinary Differential Equations}

\subsection{16.1--16.9\quad Results Relating to the Solution of Ordinary Differential Equations}

%% -------------------------------------------------------------------
\subsubsection{16.11\quad First-Order Equations}

A first-order ordinary differential equation $y'=f(x,y)$ relates the
derivative of an unknown function to the independent variable and the
function itself.  The theory of such equations---existence, uniqueness,
and continuous dependence on initial data---is the foundation of the
entire subject.  The entries in G\&R~16.111--16.114 formalise these
ideas: the solution concept, the initial value (Cauchy) problem,
approximation methods, and the Lipschitz condition that guarantees
uniqueness.

\subsubsection{16.111\quad Solution of a first-order equation}

\paragraph{Physics applications.}
\begin{enumerate}
\item \textbf{Radioactive decay and exponential processes.}%
  \index{radioactive decay!first-order ODE}%
  \index{exponential decay}%
  \index{half-life}%
  The simplest first-order ODE $dN/dt=-\lambda N$ models radioactive
  decay: $N(t)=N_{0}e^{-\lambda t}$.  The half-life
  $t_{1/2}=\ln 2/\lambda$ follows immediately.  The same equation
  governs RC circuit discharge, Beer--Lambert absorption, and first-order
  chemical kinetics.

\item \textbf{Newton's law of cooling.}%
  \index{Newton's law of cooling}%
  \index{heat transfer!lumped capacitance}%
  \index{Biot number}%
  $dT/dt=-h(T-T_{\mathrm{env}})$ gives exponential relaxation to
  ambient temperature.  The validity of this lumped-capacitance model
  requires $\mathrm{Bi}=hL/k\ll 1$ (Biot number), linking the ODE
  solution to heat transfer theory.

\item \textbf{Population dynamics and the logistic equation.}%
  \index{logistic equation}%
  \index{population dynamics}%
  \index{carrying capacity}%
  The logistic equation $dP/dt=rP(1-P/K)$ is a nonlinear first-order
  ODE with exact solution
  $P(t)=K/[1+(K/P_{0}-1)e^{-rt}]$, exhibiting sigmoidal growth
  toward the carrying capacity $K$.  This models population saturation,
  epidemic curves, and chemical autocatalysis.
\end{enumerate}

\paragraph{Mathematics applications.}
\begin{enumerate}
\item \textbf{Integral curves and the flow of a vector field.}%
  \index{integral curves}%
  \index{flow of a vector field!ODE}%
  \index{phase portrait!first-order}%
  A solution $y(x)$ is an integral curve of the direction field
  $f(x,y)$.  The collection of all solutions defines a flow
  $\phi_{t}:\mathbb{R}\to\mathbb{R}$, a one-parameter group of
  diffeomorphisms (when $f$ is smooth).  Phase portraits visualise
  the qualitative behaviour of solutions.

\item \textbf{Picard--Lindel\"of existence and uniqueness theorem.}%
  \index{Picard--Lindel\"of theorem}%
  \index{existence and uniqueness!first-order ODE}%
  \index{Banach fixed-point theorem!Picard iteration}%
  If $f(x,y)$ is continuous in a rectangle about $(x_{0},y_{0})$ and
  Lipschitz continuous in $y$, then the initial value problem
  $y'=f(x,y)$, $y(x_{0})=y_{0}$ has a unique local solution.  The
  proof constructs the solution as a fixed point of the Picard integral
  operator $Ty(x)=y_{0}+\int_{x_{0}}^{x}f(t,y(t))\,dt$ using the
  Banach contraction mapping theorem.

\item \textbf{Peano's existence theorem.}%
  \index{Peano existence theorem}%
  \index{existence without uniqueness}%
  \index{Arzel\`a--Ascoli theorem}%
  If $f(x,y)$ is merely continuous (not necessarily Lipschitz), then
  a solution still exists (Peano, 1890), but may not be unique.
  The classical example $y'=y^{2/3}$, $y(0)=0$ admits both $y\equiv 0$
  and $y=(x/3)^{3}$ as solutions.  The proof uses the Arzel\`{a}--Ascoli
  compactness theorem on the sequence of Euler polygonal approximations.
\end{enumerate}

%% -------------------------------------------------------------------
\subsubsection{16.112\quad Cauchy problem}

\paragraph{Physics applications.}
\begin{enumerate}
\item \textbf{Initial value problems in classical mechanics.}%
  \index{Cauchy problem!classical mechanics}%
  \index{initial conditions!position and velocity}%
  \index{determinism!Laplacian}%
  Newton's second law $m\ddot{x}=F(x,\dot{x},t)$, written as a
  first-order system, is a Cauchy problem: given position and velocity
  at time $t_{0}$, the trajectory is determined for all future (and
  past) times.  This is the mathematical expression of Laplacian
  determinism in classical physics.

\item \textbf{Well-posedness in geophysical fluid dynamics.}%
  \index{well-posedness!Hadamard}%
  \index{weather prediction!initial conditions}%
  \index{sensitive dependence!chaos}%
  Hadamard's notion of well-posedness---existence, uniqueness, and
  continuous dependence on initial data---is essential for weather
  prediction.  Lorenz's discovery (1963) that atmospheric equations
  exhibit sensitive dependence on initial conditions does not violate
  well-posedness but limits practical prediction horizons, motivating
  ensemble forecasting methods.
\end{enumerate}

\paragraph{Mathematics applications.}
\begin{enumerate}
\item \textbf{Continuous dependence on initial data.}%
  \index{continuous dependence!initial data}%
  \index{Gronwall's lemma!continuous dependence}%
  \index{structural stability}%
  If $y'=f(x,y)$ satisfies a Lipschitz condition with constant $L$,
  then two solutions $y_{1}$, $y_{2}$ with initial data differing by
  $\delta$ satisfy $|y_{1}(x)-y_{2}(x)|\leq\delta e^{L|x-x_{0}|}$.
  This exponential bound, proved via Gronwall's lemma
  (G\&R~16.211), quantifies both the stability of the Cauchy problem
  and the growth of perturbations.

\item \textbf{Smooth dependence on parameters.}%
  \index{smooth dependence!on parameters}%
  \index{sensitivity analysis!ODE}%
  \index{variational equation}%
  If $f(x,y;\mu)$ is smooth in a parameter $\mu$, then the solution
  $y(x;\mu)$ is also smooth in $\mu$, and the sensitivity
  $\partial y/\partial\mu$ satisfies the variational equation
  $z'=f_{y}z+f_{\mu}$, a linear ODE along the reference solution.
  This underpins sensitivity analysis and optimal control theory.
\end{enumerate}

%% -------------------------------------------------------------------
\subsubsection{16.113\quad Approximate solution to an equation}

\paragraph{Physics applications.}
\begin{enumerate}
\item \textbf{Euler's method and molecular dynamics.}%
  \index{Euler method}%
  \index{molecular dynamics!time stepping}%
  \index{symplectic integrator}%
  Euler's method $y_{n+1}=y_{n}+hf(x_{n},y_{n})$ is the simplest
  numerical scheme.  In molecular dynamics, the Verlet (St\"{o}rmer)
  integrator---a symplectic variant---preserves the Hamiltonian structure,
  preventing artificial energy drift over billions of time steps in
  $N$-body simulations.

\item \textbf{Perturbation methods in celestial mechanics.}%
  \index{perturbation methods!celestial mechanics}%
  \index{Poincar\'e--Lindstedt method}%
  \index{secular terms}%
  When exact solutions are unavailable, perturbation expansions
  $y=y_{0}+\varepsilon y_{1}+\varepsilon^{2}y_{2}+\cdots$ give
  approximate solutions.  The Poincar\'{e}--Lindstedt method removes
  secular terms (spurious growth) by simultaneously expanding the
  frequency, a technique essential in planetary orbit calculations.
\end{enumerate}

\paragraph{Mathematics applications.}
\begin{enumerate}
\item \textbf{Picard iteration as successive approximation.}%
  \index{Picard iteration}%
  \index{successive approximation}%
  \index{contraction mapping!Picard}%
  The Picard iterates $y_{n+1}(x)=y_{0}+\int_{x_{0}}^{x}f(t,y_{n}(t))\,dt$
  converge uniformly to the exact solution under the Lipschitz condition.
  The rate of convergence is geometric: the error after $n$ iterations
  is $O(L^{n}|x-x_{0}|^{n}/n!)$, where $L$ is the Lipschitz constant.

\item \textbf{Error analysis and order of convergence.}%
  \index{order of convergence!numerical methods}%
  \index{Runge--Kutta methods}%
  \index{local truncation error}%
  A numerical method has order $p$ if the local truncation error is
  $O(h^{p+1})$ and the global error is $O(h^{p})$.  Euler's method has
  order 1, the classical Runge--Kutta method has order 4, and adaptive
  methods (Dormand--Prince) embed pairs of different orders to estimate
  and control the error.
\end{enumerate}

%% -------------------------------------------------------------------
\subsubsection{16.114\quad Lipschitz continuity of a function}

\paragraph{Physics applications.}
\begin{enumerate}
\item \textbf{Bounded force fields and physical regularity.}%
  \index{Lipschitz continuity!physical interpretation}%
  \index{force field!bounded gradient}%
  \index{regularity!physical systems}%
  The Lipschitz condition $|f(x,y_{1})-f(x,y_{2})|\leq L|y_{1}-y_{2}|$
  means that the ``force'' $f$ does not change too abruptly.  In mechanical
  systems, bounded stiffness (spring constant) guarantees Lipschitz
  continuity.  Singularities such as the Coulomb potential $V\sim 1/r$
  violate Lipschitz continuity at $r=0$, requiring regularisation or
  collision handling in $N$-body codes.

\item \textbf{Finite propagation speed.}%
  \index{finite propagation speed}%
  \index{Lipschitz continuity!signal propagation}%
  \index{causality!Lipschitz bounds}%
  In relativistic systems, Lipschitz bounds on the right-hand side of
  evolution equations ensure finite propagation speed of disturbances,
  consistent with the causality requirement that information cannot
  travel faster than light.
\end{enumerate}

\paragraph{Mathematics applications.}
\begin{enumerate}
\item \textbf{Lipschitz vs.\ H\"older and Sobolev regularity.}%
  \index{Lipschitz continuity!vs.\ H\"older}%
  \index{H\"older continuity}%
  \index{Sobolev embedding}%
  A Lipschitz function is H\"{o}lder continuous with exponent 1 and
  is differentiable almost everywhere (Rademacher's theorem).  In
  Sobolev space language, $\mathrm{Lip}(\Omega)=W^{1,\infty}(\Omega)$,
  the space of functions with essentially bounded first derivatives.

\item \textbf{Gronwall-type estimates and the Lipschitz constant.}%
  \index{Lipschitz constant!role in estimates}%
  \index{Gronwall's lemma!Lipschitz constant}%
  \index{error propagation!ODE}%
  The Lipschitz constant $L$ controls every quantitative estimate in
  ODE theory: the radius of convergence of Picard iteration
  ($\sim 1/L$), the exponential divergence rate of nearby trajectories
  ($\sim e^{Lt}$), and the constants in Gronwall-type inequalities.
  Computing tight Lipschitz bounds is essential for validated
  numerics and interval arithmetic ODE solvers.
\end{enumerate}

%% -------------------------------------------------------------------
\subsubsection{16.21\quad Fundamental Inequalities and Related Results}
\subsubsection{16.211\quad Gronwall's lemma}

Gronwall's lemma (also Gronwall--Bellman inequality) states that if
$u(t)\leq\alpha(t)+\int_{a}^{t}\beta(s)u(s)\,ds$ with $u,\beta\geq 0$
and $\alpha$ non-decreasing, then
$u(t)\leq\alpha(t)\exp\!\bigl(\int_{a}^{t}\beta(s)\,ds\bigr)$.
This is the single most important tool in ODE theory for bounding
solutions and proving uniqueness, continuous dependence, and stability.

\paragraph{Physics applications.}
\begin{enumerate}
\item \textbf{Stability of dynamical systems.}%
  \index{Gronwall's lemma!stability}%
  \index{Lyapunov stability!Gronwall bound}%
  \index{orbital stability}%
  Gronwall's lemma provides the fundamental estimate showing that
  small perturbations to initial conditions or forcing terms grow at
  most exponentially.  In orbital mechanics, this bounds the divergence
  of nearby orbits and gives rigorous meaning to the notion that
  circular orbits are Lyapunov stable under perturbation.

\item \textbf{Error bounds for numerical integrators.}%
  \index{numerical integration!error bounds}%
  \index{Gronwall's lemma!numerical methods}%
  \index{global error!accumulation}%
  The global error of a numerical ODE solver is bounded using Gronwall's
  lemma: if the local truncation error is $O(h^{p+1})$ per step, then
  after $N=T/h$ steps the global error is $O(h^{p})$, because Gronwall's
  exponential factor $e^{LT}$ bounds the accumulation of local errors.
  This is the standard technique for proving convergence of Euler and
  Runge--Kutta methods.

\item \textbf{Continuous dependence in control theory.}%
  \index{continuous dependence!control theory}%
  \index{robustness!control systems}%
  \index{model uncertainty}%
  In robust control, Gronwall-type estimates quantify how much the
  system trajectory can deviate when the plant model is uncertain.
  The exponential bound $e^{LT}$ shows that the sensitivity grows
  with both the Lipschitz constant of the dynamics and the time
  horizon, motivating feedback to reduce effective $L$.
\end{enumerate}

\paragraph{Mathematics applications.}
\begin{enumerate}
\item \textbf{Uniqueness of solutions.}%
  \index{uniqueness!via Gronwall}%
  \index{Gronwall's lemma!uniqueness proof}%
  If two solutions $y_{1}$, $y_{2}$ of $y'=f(x,y)$ satisfy the same
  initial condition, then $u=|y_{1}-y_{2}|$ satisfies
  $u(t)\leq\int_{0}^{t}Lu(s)\,ds$.  Gronwall's lemma gives
  $u(t)\leq 0\cdot e^{Lt}=0$, hence $y_{1}\equiv y_{2}$.  This is the
  cleanest proof of uniqueness in the Picard--Lindel\"{o}f theorem.

\item \textbf{Nonlinear generalisations and Bihari's inequality.}%
  \index{Bihari's inequality}%
  \index{Gronwall's lemma!nonlinear generalisation}%
  \index{comparison principle}%
  Bihari's inequality generalises Gronwall to
  $u(t)\leq\alpha+\int_{a}^{t}\beta(s)\omega(u(s))\,ds$ with $\omega$
  nonlinear and non-decreasing, yielding
  $\Omega(u(t))\leq\Omega(\alpha)+\int_{a}^{t}\beta(s)\,ds$ where
  $\Omega(v)=\int_{1}^{v}d\xi/\omega(\xi)$.  This handles
  super-exponential growth and is used in blow-up analysis for
  nonlinear ODEs.
\end{enumerate}

%% -------------------------------------------------------------------
\subsubsection{16.212\quad Comparison of approximate solutions of a differential equation}

\paragraph{Physics applications.}
\begin{enumerate}
\item \textbf{Validated numerics and interval methods.}%
  \index{validated numerics}%
  \index{interval arithmetic!ODE}%
  \index{computer-assisted proofs}%
  Comparison of approximate solutions gives rigorous error bounds: if
  $\tilde{y}$ is an approximate solution with residual
  $\tilde{y}'-f(x,\tilde{y})=\delta(x)$, then
  $|y(x)-\tilde{y}(x)|\leq\|\delta\|e^{LT}/L$.  This is the basis
  of validated ODE solvers (VNODE, CAPD) that produce guaranteed
  enclosures, used in computer-assisted proofs of chaotic dynamics
  (e.g., Tucker's proof of the Lorenz attractor).

\item \textbf{Model comparison in pharmacokinetics.}%
  \index{pharmacokinetics!model comparison}%
  \index{compartment models!ODE}%
  \index{drug concentration!time course}%
  In pharmacokinetics, different compartment models (one-compartment
  vs.\ two-compartment) yield different approximate solutions for drug
  concentration.  Comparison theorems bound the discrepancy between
  models, informing clinical decisions about dosing intervals and
  therapeutic windows.
\end{enumerate}

\paragraph{Mathematics applications.}
\begin{enumerate}
\item \textbf{A posteriori error estimates.}%
  \index{a posteriori error estimate!ODE}%
  \index{defect!approximate solution}%
  \index{backward error analysis}%
  The defect (residual) of an approximate solution measures how well it
  satisfies the equation.  A posteriori error estimates use the defect
  and Gronwall's lemma to bound the true error without knowing the exact
  solution.  This is complementary to backward error analysis, where the
  approximate solution is shown to be the exact solution of a nearby
  problem.

\item \textbf{Shadowing lemma in dynamical systems.}%
  \index{shadowing lemma}%
  \index{pseudo-orbit}%
  \index{hyperbolic dynamical systems}%
  The shadowing lemma guarantees that every pseudo-orbit (approximate
  solution with bounded defect per step) of a hyperbolic dynamical
  system is uniformly close to a true orbit.  This justifies long-time
  numerical simulations of chaotic systems: individual trajectories are
  unreliable, but the computed orbit shadows a genuine one.
\end{enumerate}

%% -------------------------------------------------------------------
\subsubsection{16.31\quad First-Order Systems}

The theory of a single first-order equation extends to systems
$\mathbf{y}'=\mathbf{f}(x,\mathbf{y})$, where
$\mathbf{y}\in\mathbb{R}^{n}$ is a vector.  Every higher-order ODE
reduces to a first-order system (write $y_{1}=y$, $y_{2}=y'$, \ldots),
so the system formulation is the natural general framework.  Linear
systems $\mathbf{y}'=A(x)\mathbf{y}+\mathbf{g}(x)$ have a particularly
clean theory based on the matrix exponential and fundamental matrices.

\subsubsection{16.311\quad Solution of a system of equations}
\subsubsection{16.312\quad Cauchy problem for a system}
\subsubsection{16.313\quad Approximate solution to a system}
\subsubsection{16.314\quad Lipschitz continuity of a vector}
\subsubsection{16.315\quad Comparison of approximate solutions of a system}

\paragraph{Physics applications.}
\begin{enumerate}
\item \textbf{Coupled oscillators and normal modes.}%
  \index{coupled oscillators!system of ODEs}%
  \index{normal modes!eigenvalue problem}%
  \index{phonons}%
  A chain of $n$ masses connected by springs yields $m_{i}\ddot{x}_{i}
  =k_{i+1}(x_{i+1}-x_{i})-k_{i}(x_{i}-x_{i-1})$, a linear system
  whose eigenvalues give the normal mode frequencies.  In the
  infinite limit, this becomes the wave equation; the normal modes
  become phonons in solid-state physics.

\item \textbf{Predator--prey dynamics (Lotka--Volterra).}%
  \index{Lotka--Volterra equations}%
  \index{predator--prey model}%
  \index{ecological dynamics}%
  The Lotka--Volterra system $\dot{x}=\alpha x-\beta xy$,
  $\dot{y}=\delta xy-\gamma y$ is a nonlinear first-order system
  with a conserved quantity $H=\delta x-\gamma\ln x+\beta y-\alpha\ln y$,
  giving closed orbits in phase space.  Extensions include competition,
  mutualism, and food-web models in ecology.

\item \textbf{Epidemiological models (SIR).}%
  \index{SIR model}%
  \index{epidemiology!ODE model}%
  \index{basic reproduction number $R_0$}%
  The SIR model $\dot{S}=-\beta SI$, $\dot{I}=\beta SI-\gamma I$,
  $\dot{R}=\gamma I$ is a three-dimensional first-order system.  The
  basic reproduction number $R_{0}=\beta S_{0}/\gamma$ determines
  whether an epidemic occurs ($R_{0}>1$) or dies out ($R_{0}<1$),
  a threshold phenomenon central to public health policy.

\item \textbf{Orbital mechanics and the two-body problem.}%
  \index{two-body problem!system of ODEs}%
  \index{orbital mechanics}%
  \index{Kepler problem!first-order system}%
  The Kepler problem $\ddot{\mathbf{r}}=-GM\mathbf{r}/|\mathbf{r}|^{3}$,
  written as a first-order system in $(\mathbf{r},\mathbf{v})$, has
  exact solutions (conic sections) and conserved quantities (energy,
  angular momentum, Laplace--Runge--Lenz vector).  The existence and
  uniqueness theory for systems guarantees determinism of the two-body
  problem away from collision.
\end{enumerate}

\paragraph{Mathematics applications.}
\begin{enumerate}
\item \textbf{Reduction of higher-order ODEs to systems.}%
  \index{reduction to first-order system}%
  \index{higher-order ODE!reduction}%
  \index{state space!ODE}%
  An $n$th-order ODE $y^{(n)}=F(x,y,y',\ldots,y^{(n-1)})$ is
  equivalent to the first-order system $y_{k}'=y_{k+1}$ for
  $k=1,\ldots,n-1$, $y_{n}'=F(x,y_{1},\ldots,y_{n})$.  This
  reduction shows that all of ODE theory reduces to the study of
  first-order systems.

\item \textbf{Picard--Lindel\"of theorem for systems.}%
  \index{Picard--Lindel\"of theorem!systems}%
  \index{Lipschitz continuity!vector-valued}%
  \index{existence and uniqueness!systems}%
  The existence--uniqueness theorem extends to systems: if
  $\mathbf{f}(x,\mathbf{y})$ is Lipschitz in $\mathbf{y}$ (in any
  norm on $\mathbb{R}^{n}$), the Cauchy problem has a unique local
  solution.  The Lipschitz constant is now the operator norm of the
  Jacobian matrix $\partial\mathbf{f}/\partial\mathbf{y}$, connecting
  ODE theory to matrix analysis.

\item \textbf{Flow maps and one-parameter groups.}%
  \index{flow map!ODE}%
  \index{one-parameter group!diffeomorphisms}%
  \index{dynamical systems!continuous}%
  The solution map $\phi_{t}(\mathbf{y}_{0})=\mathbf{y}(t)$ satisfies
  $\phi_{0}=\mathrm{id}$ and $\phi_{t+s}=\phi_{t}\circ\phi_{s}$
  (for autonomous systems), making it a one-parameter group of
  diffeomorphisms.  This is the starting point of the modern theory
  of dynamical systems.
\end{enumerate}

%% -------------------------------------------------------------------
\subsubsection{16.316\quad First-order linear differential equation}

\paragraph{Physics applications.}
\begin{enumerate}
\item \textbf{RC and RL circuits.}%
  \index{RC circuit!first-order linear ODE}%
  \index{RL circuit}%
  \index{time constant}%
  The voltage across a capacitor in an RC circuit satisfies
  $RC\,dV/dt+V=V_{\mathrm{in}}(t)$, a first-order linear ODE with
  integrating factor $e^{t/RC}$.  The time constant $\tau=RC$
  characterises the transient response.  The analogous RL circuit
  has $\tau=L/R$.

\item \textbf{Mixing problems and compartment models.}%
  \index{mixing problems}%
  \index{compartment models!first-order}%
  \index{dilution!first-order ODE}%
  A tank with inflow rate $r_{\mathrm{in}}$, concentration $c_{\mathrm{in}}$,
  and outflow rate $r_{\mathrm{out}}$ satisfies $dQ/dt=r_{\mathrm{in}}c_{\mathrm{in}}-r_{\mathrm{out}}Q/V(t)$,
  a first-order linear ODE in the amount $Q$ of solute.
  Chains of compartments model drug metabolism, tracer transport
  in the environment, and chemical reactor networks.
\end{enumerate}

\paragraph{Mathematics applications.}
\begin{enumerate}
\item \textbf{Integrating factor method.}%
  \index{integrating factor!first-order linear}%
  \index{variation of constants!first-order}%
  The general solution of $y'+p(x)y=q(x)$ is
  $y(x)=e^{-P(x)}\!\left[C+\int q(x)e^{P(x)}\,dx\right]$ where
  $P(x)=\int p(x)\,dx$.  The integrating factor $\mu=e^{P}$ converts
  the left-hand side into the exact derivative $d(\mu y)/dx$.  This
  is the prototype for variation of constants (Lagrange).

\item \textbf{Bernoulli and Riccati reductions.}%
  \index{Bernoulli equation}%
  \index{Riccati equation!linearisation}%
  \index{nonlinear ODE!reduction to linear}%
  The Bernoulli equation $y'+p(x)y=q(x)y^{n}$ reduces to a linear
  equation via $v=y^{1-n}$.  More generally, the Riccati equation
  $y'=a(x)+b(x)y+c(x)y^{2}$ linearises to a second-order equation
  $u''-[b+(c'/c)]u'+acu=0$ via $y=-u'/(cu)$, connecting first-order
  nonlinear and second-order linear theories (see G\&R~16.514).
\end{enumerate}

%% -------------------------------------------------------------------
\subsubsection{16.317\quad Linear systems of differential equations}

\paragraph{Physics applications.}
\begin{enumerate}
\item \textbf{Small oscillations and modal analysis.}%
  \index{small oscillations}%
  \index{modal analysis}%
  \index{eigenfrequencies}%
  \index{mass and stiffness matrices}%
  The linearised equations of motion near equilibrium take the form
  $M\ddot{\mathbf{q}}+C\dot{\mathbf{q}}+K\mathbf{q}=\mathbf{f}(t)$,
  equivalently a $2n$-dimensional linear system
  $\dot{\mathbf{x}}=A\mathbf{x}+\mathbf{b}$.  The eigenvalues of
  $A$ give the natural frequencies and damping rates; the eigenvectors
  give the mode shapes.  This is the foundation of structural dynamics,
  vibration analysis, and seismic engineering.

\item \textbf{Electrical network analysis.}%
  \index{electrical networks!state-space}%
  \index{state-space model}%
  \index{SPICE simulation}%
  A network of resistors, capacitors, and inductors yields a linear
  system $\dot{\mathbf{x}}=A\mathbf{x}+B\mathbf{u}$,
  $\mathbf{y}=C\mathbf{x}+D\mathbf{u}$ in the state-space formulation.
  Circuit simulators (SPICE) solve this system numerically, using the
  matrix exponential $e^{At}$ for the homogeneous response and
  convolution for the driven response.

\item \textbf{Quantum mechanics: time evolution and the Schr\"odinger equation.}%
  \index{Schr\"odinger equation!time evolution}%
  \index{time-evolution operator}%
  \index{matrix exponential!quantum mechanics}%
  The time-dependent Schr\"{o}dinger equation
  $i\hbar\,d|\psi\rangle/dt=H|\psi\rangle$ is a linear system in
  Hilbert space.  For a time-independent Hamiltonian, the solution is
  $|\psi(t)\rangle=e^{-iHt/\hbar}|\psi(0)\rangle$, the matrix
  exponential of $-iH/\hbar$.  The eigenvalues of $H$ are the
  energy levels, and the eigenvectors are the stationary states.
\end{enumerate}

\paragraph{Mathematics applications.}
\begin{enumerate}
\item \textbf{Matrix exponential and fundamental matrix.}%
  \index{matrix exponential!ODE systems}%
  \index{fundamental matrix}%
  \index{Peano--Baker series}%
  For $\mathbf{y}'=A\mathbf{y}$ with constant $A$, the solution is
  $\mathbf{y}(t)=e^{At}\mathbf{y}_{0}$ where
  $e^{At}=\sum_{k=0}^{\infty}(At)^{k}/k!$.  For non-constant $A(t)$,
  the fundamental matrix $\Phi(t)$ satisfies $\Phi'=A(t)\Phi$,
  $\Phi(0)=I$, and is given by the Peano--Baker series (the
  time-ordered exponential).

\item \textbf{Jordan normal form and solution structure.}%
  \index{Jordan normal form!ODE solution}%
  \index{generalised eigenvectors}%
  \index{polynomial$\times$exponential solutions}%
  When $A$ is not diagonalisable, the Jordan form $A=PJP^{-1}$ gives
  solutions involving $t^{k}e^{\lambda t}$ terms from Jordan blocks.
  Each Jordan block of size $m$ for eigenvalue $\lambda$ contributes
  $m$ linearly independent solutions
  $e^{\lambda t}\mathbf{v}$, $e^{\lambda t}(t\mathbf{v}+\mathbf{w})$,
  \ldots, where $\mathbf{v},\mathbf{w},\ldots$ are generalised
  eigenvectors.

\item \textbf{Variation of parameters for systems.}%
  \index{variation of parameters!systems}%
  \index{Duhamel's formula}%
  \index{Green's matrix}%
  The inhomogeneous system $\mathbf{y}'=A(t)\mathbf{y}+\mathbf{g}(t)$
  has solution $\mathbf{y}(t)=\Phi(t)\mathbf{y}_{0}+\int_{0}^{t}\Phi(t)\Phi^{-1}(s)\mathbf{g}(s)\,ds$
  (Duhamel's formula).  The kernel $\Phi(t)\Phi^{-1}(s)$ is the
  Green's matrix of the system, the matrix analogue of the scalar
  Green's function.

\item \textbf{Floquet theory for periodic systems.}%
  \index{Floquet theory}%
  \index{periodic coefficients!ODE}%
  \index{monodromy matrix}%
  \index{Floquet multipliers}%
  If $A(t+T)=A(t)$, then the fundamental matrix satisfies
  $\Phi(t+T)=\Phi(t)M$ where $M=\Phi(T)$ is the monodromy matrix.
  The eigenvalues of $M$ (Floquet multipliers) determine stability:
  all $|\mu_{k}|<1$ gives asymptotic stability, any $|\mu_{k}|>1$
  gives instability.  Floquet theory applies to parametric resonance
  (Mathieu equation), Bloch waves in crystals, and periodic orbits
  in celestial mechanics.
\end{enumerate}

%% -------------------------------------------------------------------
\subsubsection{16.41\quad Some Special Types of Elementary Differential Equations}
\subsubsection{16.411\quad Variables separable}

\paragraph{Physics applications.}
\begin{enumerate}
\item \textbf{Free-fall and terminal velocity.}%
  \index{free-fall!separable ODE}%
  \index{terminal velocity}%
  \index{drag force!velocity-dependent}%
  The equation $m\,dv/dt=mg-bv^{2}$ for fall with quadratic drag
  separates as $dv/(mg-bv^{2})=dt/m$.  Integration gives
  $v(t)=v_{\mathrm{term}}\tanh(gt/v_{\mathrm{term}})$ with
  $v_{\mathrm{term}}=\sqrt{mg/b}$, a result used in skydiving
  calculations and atmospheric science.

\item \textbf{Barometric formula and isothermal atmospheres.}%
  \index{barometric formula}%
  \index{isothermal atmosphere}%
  \index{hydrostatic equilibrium}%
  The hydrostatic equation $dP/dz=-\rho g=-Pg/(RT)$ for an
  isothermal atmosphere is separable: $dP/P=-g\,dz/(RT)$, giving
  $P(z)=P_{0}e^{-gz/(RT)}$.  This exponential pressure profile
  is the starting point for atmospheric physics and altimetry.
\end{enumerate}

\paragraph{Mathematics applications.}
\begin{enumerate}
\item \textbf{Quadrature and implicit solutions.}%
  \index{quadrature!separable ODE}%
  \index{implicit solution}%
  \index{singular solutions!separable ODE}%
  A separable equation $g(y)\,dy=f(x)\,dx$ reduces to two
  independent integrations: $\int g(y)\,dy=\int f(x)\,dx+C$.  The
  solution may be implicit rather than explicit, and values where
  $g(y)=0$ must be checked separately as they may yield singular
  solutions (envelopes) not captured by the general solution.

\item \textbf{Autonomous equations and phase line analysis.}%
  \index{autonomous equation!first-order}%
  \index{phase line}%
  \index{equilibria!classification}%
  An autonomous equation $y'=f(y)$ is separable with $dx=dy/f(y)$.
  The qualitative behaviour is determined by the zeros of $f$ (equilibria):
  $f'(y^{*})<0$ gives stable equilibrium, $f'(y^{*})>0$ gives unstable.
  The phase line (one-dimensional phase portrait) provides a complete
  qualitative picture without solving the equation.
\end{enumerate}

%% -------------------------------------------------------------------
\subsubsection{16.412\quad Exact differential equations}
\subsubsection{16.413\quad Conditions for an exact equation}

\paragraph{Physics applications.}
\begin{enumerate}
\item \textbf{Thermodynamic state functions and exact differentials.}%
  \index{exact differential!thermodynamics}%
  \index{state function}%
  \index{Maxwell relations!thermodynamics}%
  \index{entropy!exact differential}%
  In thermodynamics, $dU=T\,dS-P\,dV$ is an exact differential because
  $U$ is a state function.  The condition
  $(\partial T/\partial V)_{S}=-(\partial P/\partial S)_{V}$
  (a Maxwell relation) is precisely the exactness condition
  $\partial M/\partial y=\partial N/\partial x$ for $M\,dx+N\,dy=0$.
  Heat $\delta Q=T\,dS$ is exact only when expressed in terms of entropy;
  the distinction between exact and inexact differentials is fundamental
  to the second law.

\item \textbf{Conservative force fields and potential energy.}%
  \index{conservative force!exact differential}%
  \index{potential energy!exact differential}%
  \index{work!path independence}%
  A force field $\mathbf{F}=(M,N)$ is conservative if and only if
  $M\,dx+N\,dy$ is an exact differential, i.e.,
  $\partial M/\partial y=\partial N/\partial x$.  Then
  $\mathbf{F}=-\nabla V$ for a potential $V$, and work is path-independent.
  The failure of exactness characterises non-conservative forces
  (friction, magnetic forces on moving charges).
\end{enumerate}

\paragraph{Mathematics applications.}
\begin{enumerate}
\item \textbf{Poincar\'e lemma and simply connected domains.}%
  \index{Poincar\'e lemma!exact differentials}%
  \index{simply connected domain}%
  \index{closed form!exact}%
  On a simply connected domain, the exactness condition
  $\partial M/\partial y=\partial N/\partial x$ (closedness) implies
  the existence of a potential $F$ with $dF=M\,dx+N\,dy$ (exactness).
  On multiply connected domains, the integrability obstruction is
  measured by de Rham cohomology $H^{1}$, and periods around holes
  give topological invariants.

\item \textbf{Integrating factors and Lie symmetries.}%
  \index{integrating factor!existence}%
  \index{Lie symmetry!integrating factor}%
  \index{symmetry group!ODE}%
  If $M\,dx+N\,dy=0$ is not exact, an integrating factor $\mu(x,y)$
  makes $\mu M\,dx+\mu N\,dy=0$ exact.  The existence of $\mu$ is
  guaranteed (locally), but finding it requires solving a PDE.
  Lie's theory of symmetry groups provides a systematic method:
  each one-parameter symmetry of the ODE yields an integrating factor,
  and conversely \cite{Olver1993}.
\end{enumerate}

%% -------------------------------------------------------------------
\subsubsection{16.414\quad Homogeneous differential equations}

\paragraph{Physics applications.}
\begin{enumerate}
\item \textbf{Dimensional analysis and scaling laws.}%
  \index{dimensional analysis!homogeneous ODE}%
  \index{scaling laws}%
  \index{self-similar solutions}%
  A homogeneous ODE $y'=g(y/x)$ is invariant under the scaling
  $x\to\lambda x$, $y\to\lambda y$.  This scale invariance is the
  mathematical expression of dimensional analysis: if an ODE involves
  only dimensionless combinations $y/x$, the solution must be
  self-similar.  Self-similar solutions describe blast waves (Taylor--Sedov),
  boundary layer profiles (Blasius), and gravitational collapse.

\item \textbf{Polar coordinates and spiral trajectories.}%
  \index{polar coordinates!homogeneous ODE}%
  \index{spiral trajectories}%
  \index{pursuit curves}%
  The substitution $y=vx$ in a homogeneous equation yields a separable
  equation in $v$.  Geometrically, the solutions are curves whose slope
  depends only on the angle $\theta=\arctan(y/x)$, producing logarithmic
  spirals, pursuit curves, and other scale-invariant trajectories.
\end{enumerate}

\paragraph{Mathematics applications.}
\begin{enumerate}
\item \textbf{Substitution $y=vx$ and reduction to quadrature.}%
  \index{substitution $y=vx$!homogeneous ODE}%
  \index{reduction to quadrature}%
  The substitution $y=vx$ reduces $y'=g(y/x)$ to $v+xv'=g(v)$, hence
  $dv/(g(v)-v)=dx/x$, a separable equation solvable by quadrature.
  The back-substitution $v=y/x$ gives the solution in original variables.

\item \textbf{Generalised homogeneity and M\"obius transformations.}%
  \index{generalised homogeneity!ODE}%
  \index{M\"obius transformation!ODE}%
  \index{projective geometry!ODE}%
  The equation $y'=(ay+bx+c)/(dy+ex+f)$ reduces to a homogeneous
  equation by translating to eliminate the constants (if $ae-bd\neq 0$)
  or by a linear substitution (if $ae-bd=0$).  This connects to
  projective geometry: the general linear-fractional ODE is covariant
  under M\"{o}bius transformations of the $(x,y)$-plane.
\end{enumerate}

%% -------------------------------------------------------------------
\subsubsection{16.51\quad Second-Order Equations}
\subsubsection{16.511\quad Adjoint and self-adjoint equations}

The general second-order linear ODE
$a_{0}(x)y''+a_{1}(x)y'+a_{2}(x)y=0$ can be written in self-adjoint
(Sturm--Liouville) form $[p(x)y']'+q(x)y=0$ by multiplication by
an appropriate integrating factor.  The self-adjoint form is the natural
setting for the oscillation and spectral theory that follows in
G\&R~16.6--16.9.

\paragraph{Physics applications.}
\begin{enumerate}
\item \textbf{Sturm--Liouville problems and quantum mechanics.}%
  \index{Sturm--Liouville problem!quantum mechanics}%
  \index{self-adjoint operator!quantum}%
  \index{Schr\"odinger equation!Sturm--Liouville}%
  \index{spectral theorem!physical interpretation}%
  The time-independent Schr\"{o}dinger equation
  $-\frac{\hbar^{2}}{2m}\psi''+V(x)\psi=E\psi$ is a Sturm--Liouville
  problem with eigenvalue $E$.  Self-adjointness of the Hamiltonian
  guarantees real eigenvalues (observable energies), orthogonal
  eigenfunctions (quantum states), and completeness (any state is a
  superposition of energy eigenstates).

\item \textbf{Vibrating strings and membranes.}%
  \index{vibrating string!Sturm--Liouville}%
  \index{normal modes!Sturm--Liouville}%
  \index{eigenvalue problem!vibrating string}%
  The spatial part of the wave equation $[T(x)X']'+\omega^{2}\rho(x)X=0$
  for a non-uniform string is a Sturm--Liouville problem.  The eigenvalues
  $\omega_{n}^{2}$ are the squared natural frequencies, and the
  eigenfunctions $X_{n}(x)$ are the mode shapes.  Self-adjointness
  guarantees orthogonality of modes, the basis of Fourier analysis in
  acoustics.

\item \textbf{Heat conduction in non-uniform media.}%
  \index{heat conduction!Sturm--Liouville}%
  \index{thermal diffusivity!variable}%
  \index{eigenfunction expansion!heat equation}%
  Separation of the heat equation $\rho c\,\partial T/\partial t
  =\nabla\cdot(k\nabla T)$ in one dimension yields the
  Sturm--Liouville problem $[k(x)X']'+\lambda\rho(x)c(x)X=0$.
  The eigenfunction expansion $T(x,t)=\sum c_{n}X_{n}(x)e^{-\lambda_{n}t}$
  gives the transient temperature distribution.
\end{enumerate}

\paragraph{Mathematics applications.}
\begin{enumerate}
\item \textbf{Self-adjointness and the spectral theorem.}%
  \index{self-adjoint operator!spectral theorem}%
  \index{spectral theorem!ODE}%
  \index{eigenfunction expansion!Sturm--Liouville}%
  A regular Sturm--Liouville operator $Ly=-[p(x)y']'-q(x)y$ on
  $[a,b]$ with separated boundary conditions has a discrete spectrum
  of real simple eigenvalues $\lambda_{1}<\lambda_{2}<\cdots\to\infty$,
  and the eigenfunctions form a complete orthonormal basis of
  $L^{2}([a,b];w)$ where $w$ is the weight function.  This is the
  infinite-dimensional analogue of the spectral theorem for symmetric
  matrices.

\item \textbf{Green's functions for second-order operators.}%
  \index{Green's function!second-order ODE}%
  \index{boundary value problem!Green's function}%
  \index{self-adjointness!Green's function symmetry}%
  The Green's function $G(x,\xi)$ for $Ly=f$ with homogeneous
  boundary conditions satisfies $LG=\delta(x-\xi)$ and gives the
  solution as $y(x)=\int_{a}^{b}G(x,\xi)f(\xi)\,d\xi$.
  Self-adjointness implies the symmetry $G(x,\xi)=G(\xi,x)$,
  the ODE analogue of the reciprocity principle in physics.
\end{enumerate}

%% -------------------------------------------------------------------
\subsubsection{16.512\quad Abel's identity}

Abel's identity states that for a second-order linear ODE
$y''+p(x)y'+q(x)y=0$, the Wronskian of any two solutions $y_{1},y_{2}$
satisfies $W(x)=W(x_{0})\exp\!\left(-\int_{x_{0}}^{x}p(t)\,dt\right)$.

\paragraph{Physics applications.}
\begin{enumerate}
\item \textbf{Conservation laws and Liouville's theorem.}%
  \index{Abel's identity!Liouville's theorem}%
  \index{Liouville's theorem!phase space}%
  \index{phase space volume!conservation}%
  Abel's identity is the one-dimensional case of Liouville's theorem:
  the Wronskian is the phase-space volume element, and its exponential
  change $\exp(-\int p\,dx)$ reflects dissipation ($p>0$) or growth
  ($p<0$).  For Hamiltonian systems ($p=0$), the Wronskian is constant,
  corresponding to conservation of phase-space volume.

\item \textbf{Probability current in quantum mechanics.}%
  \index{probability current!Wronskian}%
  \index{Wronskian!quantum mechanics}%
  \index{transmission coefficient}%
  For the Schr\"{o}dinger equation $\psi''+k^{2}(x)\psi=0$ (with
  $p=0$), the Wronskian $W[\psi,\psi^{*}]=2i\,\mathrm{Im}(\psi^{*}\psi')$
  is proportional to the probability current $j$.
  Abel's identity ($W=\mathrm{const}$) gives conservation of
  probability current, the basis for computing transmission and
  reflection coefficients in quantum scattering.
\end{enumerate}

\paragraph{Mathematics applications.}
\begin{enumerate}
\item \textbf{Linear independence and the Wronskian.}%
  \index{Wronskian!linear independence}%
  \index{linear independence!solutions}%
  \index{Abel's identity!Wronskian non-vanishing}%
  Abel's identity shows that the Wronskian of two solutions either
  vanishes identically or never vanishes.  Thus $W(x_{0})\neq 0$ at one
  point implies $W(x)\neq 0$ everywhere, proving linear independence.
  Conversely, $W\equiv 0$ implies linear dependence---the solutions
  are proportional.

\item \textbf{Reduction of order.}%
  \index{reduction of order}%
  \index{d'Alembert's method}%
  \index{second solution!from first}%
  Given one solution $y_{1}$, Abel's identity yields the second
  solution as $y_{2}(x)=y_{1}(x)\int\frac{W_{0}}{y_{1}^{2}(x)}
  \exp\!\left(-\int p\,dx\right)dx$.  This is d'Alembert's reduction of
  order method, fundamental for constructing second solutions of
  Bessel, Legendre, and hypergeometric equations near singular points.
\end{enumerate}

%% -------------------------------------------------------------------
\subsubsection{16.513\quad Lagrange identity}

The Lagrange identity for the operator $L[y]=(py')'+qy$ is
$uL[v]-vL[u]=[p(uv'-vu')]'$, where the right-hand side is the
derivative of the bilinear concomitant.

\paragraph{Physics applications.}
\begin{enumerate}
\item \textbf{Reciprocity in Green's functions.}%
  \index{Lagrange identity!reciprocity}%
  \index{Green's function!reciprocity}%
  \index{reciprocity principle}%
  Integrating the Lagrange identity over $[a,b]$ and applying boundary
  conditions gives Green's identity, which proves the symmetry
  $G(x,\xi)=G(\xi,x)$ of the Green's function for self-adjoint
  operators.  This symmetry is the mathematical basis of the reciprocity
  principle: in acoustics, the response at $x$ due to a source at $\xi$
  equals the response at $\xi$ due to a source at $x$.

\item \textbf{Quantum mechanical scattering matrix symmetry.}%
  \index{scattering matrix!symmetry}%
  \index{time-reversal symmetry!S-matrix}%
  \index{Lagrange identity!scattering}%
  In one-dimensional scattering, the Lagrange identity for the
  Schr\"{o}dinger equation yields relations between the transmission
  and reflection amplitudes.  For real potentials (time-reversal
  symmetric), it gives $|t|^{2}+|r|^{2}=1$ (unitarity of the
  $S$-matrix) and the symmetry of the transmission coefficient for
  left- and right-incidence.
\end{enumerate}

\paragraph{Mathematics applications.}
\begin{enumerate}
\item \textbf{Green's formula and boundary terms.}%
  \index{Green's formula!Lagrange identity}%
  \index{bilinear concomitant}%
  \index{boundary conditions!self-adjoint}%
  Integration of the Lagrange identity gives Green's formula
  $\int_{a}^{b}(uLv-vLu)\,dx=[p(uv'-vu')]_{a}^{b}$.  The boundary
  term vanishes for self-adjoint boundary conditions (separated or
  periodic), establishing the symmetry
  $\langle u,Lv\rangle=\langle Lu,v\rangle$.

\item \textbf{Eigenvalue comparison and interlacing.}%
  \index{eigenvalue interlacing!Lagrange identity}%
  \index{minimax principle!ODE}%
  The Lagrange identity is the starting point for proving that the
  eigenvalues of a Sturm--Liouville problem with Dirichlet conditions
  on $[a,b]$ interlace with those on any subinterval $[a,c]\subset[a,b]$.
  This is the ODE counterpart of the Cauchy interlacing theorem for
  matrices \cite{Teschl2012}.
\end{enumerate}

%% -------------------------------------------------------------------
\subsubsection{16.514\quad The Riccati equation}
\subsubsection{16.515\quad Solutions of the Riccati equation}

The Riccati equation $y'=a(x)+b(x)y+c(x)y^{2}$ is the simplest
first-order ODE that is not solvable by quadrature in general.  It
occupies a central position in ODE theory because it linearises to a
second-order equation and connects to projective geometry, optimal control,
and matrix analysis.

\paragraph{Physics applications.}
\begin{enumerate}
\item \textbf{Optimal control and the matrix Riccati equation.}%
  \index{Riccati equation!optimal control}%
  \index{matrix Riccati equation}%
  \index{linear-quadratic regulator}%
  \index{Kalman filter!Riccati equation}%
  The linear-quadratic regulator (LQR) problem---minimising
  $J=\int_{0}^{\infty}(\mathbf{x}^{T}Q\mathbf{x}+\mathbf{u}^{T}R\mathbf{u})\,dt$
  subject to $\dot{\mathbf{x}}=A\mathbf{x}+B\mathbf{u}$---leads to
  the algebraic matrix Riccati equation $A^{T}P+PA-PBR^{-1}B^{T}P+Q=0$.
  The optimal feedback gain is $K=R^{-1}B^{T}P$.  The Kalman filter
  (dual problem) involves the same Riccati equation with transposed
  matrices.  These are the most important equations in modern
  control theory \cite{Anderson2007}.

\item \textbf{WKB approximation and the quantum Riccati equation.}%
  \index{WKB approximation!Riccati equation}%
  \index{semiclassical approximation}%
  \index{eikonal equation}%
  The substitution $\psi=\exp(\int w\,dx)$ in the Schr\"{o}dinger
  equation $\psi''+k^{2}(x)\psi=0$ gives the Riccati equation
  $w'+w^{2}+k^{2}=0$.  The WKB approximation is a systematic
  expansion $w=\sum_{n}\hbar^{n}w_{n}$ of this Riccati equation
  in powers of $\hbar$, with the leading term $w_{0}=\pm ik(x)$
  giving the classical (eikonal) approximation.

\item \textbf{Impedance and wave propagation.}%
  \index{impedance!Riccati equation}%
  \index{wave propagation!layered media}%
  \index{reflection coefficient!Riccati}%
  The input impedance $Z(x)$ of a non-uniform transmission line
  satisfies a Riccati equation $dZ/dx=-i\omega L-i\omega CZ^{2}$
  (in appropriate normalisation).  The reflection coefficient
  $r=(Z-Z_{0})/(Z+Z_{0})$ also satisfies a Riccati equation,
  used in the design of impedance-matching networks and anti-reflection
  coatings in optics.
\end{enumerate}

\paragraph{Mathematics applications.}
\begin{enumerate}
\item \textbf{Linearisation and the cross-ratio.}%
  \index{Riccati equation!linearisation}%
  \index{cross-ratio!Riccati equation}%
  \index{projective line}%
  The substitution $y=-u'/(cu)$ transforms the Riccati equation into
  the second-order linear equation $u''-[b+(c'/c)]u'+acu=0$.
  Conversely, the ratio $y=u_{1}/u_{2}$ of two solutions of a
  second-order equation satisfies a Riccati equation.  The cross-ratio
  of four particular solutions is constant---the Riccati equation
  preserves the projective structure of the line.

\item \textbf{Differential Galois theory.}%
  \index{differential Galois theory}%
  \index{Riccati equation!integrability}%
  \index{Liouvillian solutions}%
  A second-order linear ODE is solvable in terms of
  Liouvillian functions (exponentials, integrals, algebraic functions)
  if and only if its Riccati equation has an algebraic solution.  The
  differential Galois group---an algebraic group acting on the solution
  space---measures the ``complexity'' of the equation: solvability
  corresponds to the Galois group being a solvable group
  \cite{vanderPut2003}.
\end{enumerate}

%% -------------------------------------------------------------------
\subsubsection{16.516\quad Solution of a second-order linear differential equation}

\paragraph{Physics applications.}
\begin{enumerate}
\item \textbf{The harmonic oscillator and its generalisations.}%
  \index{harmonic oscillator!second-order ODE}%
  \index{damped harmonic oscillator}%
  \index{driven oscillator!resonance}%
  The equation $m\ddot{x}+c\dot{x}+kx=F(t)$ (damped driven harmonic
  oscillator) is the prototype second-order linear ODE.  Its solution
  exhibits underdamping ($c^{2}<4mk$), critical damping ($c^{2}=4mk$),
  and overdamping ($c^{2}>4mk$), and the driven response shows
  resonance when the driving frequency matches the natural frequency.
  The harmonic oscillator is ubiquitous: RLC circuits, acoustic
  resonators, molecular vibrations, and the quantum harmonic oscillator
  all share this equation.

\item \textbf{Bessel, Legendre, and hypergeometric equations.}%
  \index{Bessel equation!second-order linear}%
  \index{Legendre equation}%
  \index{hypergeometric equation}%
  \index{special functions!from second-order ODEs}%
  The classical special functions of G\&R sections 8--9 are solutions
  of specific second-order linear ODEs: Bessel's equation
  $x^{2}y''+xy'+(x^{2}-\nu^{2})y=0$, Legendre's equation
  $(1-x^{2})y''-2xy'+\ell(\ell+1)y=0$, and the hypergeometric
  equation $x(1-x)y''+[c-(a+b+1)x]y'-aby=0$.  The Frobenius method
  constructs power series solutions around regular singular points.
\end{enumerate}

\paragraph{Mathematics applications.}
\begin{enumerate}
\item \textbf{Frobenius method and regular singular points.}%
  \index{Frobenius method}%
  \index{regular singular point}%
  \index{indicial equation}%
  At a regular singular point $x_{0}$, the Frobenius method seeks
  solutions $y=\sum_{n=0}^{\infty}a_{n}(x-x_{0})^{n+r}$, where the
  exponent $r$ satisfies the indicial equation.  When the two roots
  differ by an integer, logarithmic terms may appear in the second
  solution.  This method generates all classical special functions
  and their series representations.

\item \textbf{Monodromy and the Riemann--Hilbert problem.}%
  \index{monodromy!second-order ODE}%
  \index{Riemann--Hilbert problem}%
  \index{Fuchsian equations}%
  For a Fuchsian equation (all singular points regular), analytic
  continuation of solutions around singular points defines the
  monodromy representation $\pi_{1}(\mathbb{C}\setminus\{z_{k}\})
  \to\mathrm{GL}(2,\mathbb{C})$.  The Riemann--Hilbert problem asks
  whether every monodromy representation arises from a Fuchsian
  equation (yes for $n\leq 3$ singular points; subtle for $n\geq 4$).
\end{enumerate}

%% -------------------------------------------------------------------
\subsubsection{16.61--16.62\quad Oscillation and Non-Oscillation Theorems for Second-Order Equations}

The oscillation theory of $y''+q(x)y=0$ studies whether solutions have
infinitely many zeros (oscillatory) or finitely many (non-oscillatory) on
a half-line $[a,\infty)$.  The sign and size of $q(x)$ determine the
behaviour: roughly, $q>0$ promotes oscillation, $q<0$ promotes
non-oscillation.  The results in G\&R~16.611--16.629 develop this
theory systematically through comparison theorems, starting with Sturm's
foundational work.

\subsubsection{16.611\quad First basic comparison theorem}
\subsubsection{16.622\quad Second basic comparison theorem}
\subsubsection{16.623\quad Interlacing of zeros}

\paragraph{Physics applications.}
\begin{enumerate}
\item \textbf{Zeros of Bessel functions and drum modes.}%
  \index{Bessel functions!zeros}%
  \index{drum modes!nodal lines}%
  \index{zeros!Bessel functions}%
  The modes of a circular drum are labelled by the zeros
  $j_{\nu,k}$ of $J_{\nu}(x)$.  Sturm comparison with
  $y''+y=0$ (whose solutions have zeros at spacing $\pi$) gives
  bounds on the spacing of Bessel zeros: for large $x$,
  $j_{\nu,k+1}-j_{\nu,k}\to\pi$, and the comparison theorems
  give rigorous monotonicity and interlacing results
  \cite{Watson1944}.

\item \textbf{WKB turning points and connection formulas.}%
  \index{WKB approximation!turning points}%
  \index{turning point!oscillation}%
  \index{connection formulas}%
  \index{Airy function!turning point}%
  Near a turning point $x_{0}$ where $k^{2}(x)$ changes sign, the
  WKB approximation breaks down.  The oscillation theorems show that
  solutions oscillate for $k^{2}>0$ and decay for $k^{2}<0$.  The
  Airy function provides the local connection between these regimes,
  and the Stokes phenomenon describes the exponentially small switching
  between dominant and subdominant behaviours.

\item \textbf{Quantum tunnelling and barrier penetration.}%
  \index{quantum tunnelling}%
  \index{barrier penetration}%
  \index{oscillation!quantum mechanics}%
  A quantum particle encountering a potential barrier $V(x)>E$ has a
  non-oscillatory wavefunction in the classically forbidden region and
  an oscillatory one outside.  The comparison theorems bound the
  rate of decay inside the barrier, and the connection formulas give
  the transmission coefficient
  $T\sim\exp\!\left(-2\int_{x_{1}}^{x_{2}}\kappa(x)\,dx\right)$.
\end{enumerate}

\paragraph{Mathematics applications.}
\begin{enumerate}
\item \textbf{Sturm comparison theorem (classical form).}%
  \index{Sturm comparison theorem!classical}%
  \index{comparison theorem!oscillation}%
  \index{zeros!interlacing}%
  If $q_{1}(x)\leq q_{2}(x)$ on $[a,b]$, then between consecutive
  zeros of any solution of $y''+q_{1}y=0$, every solution of
  $y''+q_{2}y=0$ has at least one zero.  Equivalently: more positive
  $q$ means more oscillation (closer zeros).

\item \textbf{Comparison with constant-coefficient equations.}%
  \index{comparison!with constant coefficients}%
  \index{oscillation frequency!bounds}%
  If $\alpha^{2}\leq q(x)\leq\beta^{2}$, comparison with $y''+\alpha^{2}y=0$
  and $y''+\beta^{2}y=0$ shows that the distance between consecutive zeros
  of solutions of $y''+q(x)y=0$ lies in $[\pi/\beta,\pi/\alpha]$.  This gives
  quantitative zero-spacing bounds without solving the equation.
\end{enumerate}

%% -------------------------------------------------------------------
\subsubsection{16.624\quad Sturm separation theorem}

\paragraph{Physics applications.}
\begin{enumerate}
\item \textbf{Nodal structure of quantum eigenstates.}%
  \index{Sturm separation theorem!quantum nodes}%
  \index{nodal theorem}%
  \index{quantum number!nodes}%
  The Sturm separation theorem implies that the zeros of linearly
  independent solutions interlace.  For Sturm--Liouville eigenvalue
  problems, this yields the oscillation theorem: the $n$th eigenfunction
  has exactly $n-1$ zeros in the interior of the domain.  In quantum
  mechanics, this is the nodal theorem: the ground state has no
  interior nodes, the first excited state has one, etc.

\item \textbf{Spectral gaps and band structure.}%
  \index{spectral gaps!oscillation}%
  \index{band structure!Hill's equation}%
  \index{Bloch wave!oscillation count}%
  For periodic potentials (Hill's equation), the oscillation count
  of Bloch solutions determines the band index.  The Sturm separation
  theorem ensures that bands do not cross, and the number of zeros
  per period increases by one from band to band.
\end{enumerate}

\paragraph{Mathematics applications.}
\begin{enumerate}
\item \textbf{Disconjugacy and the separation theorem.}%
  \index{disconjugacy}%
  \index{Sturm separation theorem!disconjugacy}%
  An equation is disconjugate on $[a,b]$ if no non-trivial solution
  has two zeros.  Sturm's separation theorem shows this is equivalent
  to the existence of a positive solution on $[a,b]$, which in turn
  is equivalent to the first eigenvalue being positive---connecting
  qualitative, analytic, and spectral properties.

\item \textbf{Pr\"ufer substitution and rotation number.}%
  \index{Pr\"ufer substitution}%
  \index{rotation number}%
  \index{polar coordinates!ODE}%
  The Pr\"{u}fer substitution $y=r\sin\theta$, $y'=r\cos\theta$
  transforms $y''+q(x)y=0$ into the system $\theta'=\cos^{2}\theta
  +q\sin^{2}\theta$, $r'/r=\frac{1}{2}(1-q)\sin 2\theta$.  The angle
  $\theta(x)$ monotonically counts zeros (each zero adds $\pi$ to
  $\theta$), and the rotation number $\lim_{x\to\infty}\theta(x)/x$
  encodes the oscillation rate.
\end{enumerate}

%% -------------------------------------------------------------------
\subsubsection{16.625\quad Sturm comparison theorem}
\subsubsection{16.626\quad Szeg\"o's comparison theorem}

\paragraph{Physics applications.}
\begin{enumerate}
\item \textbf{Frequency bounds for variable media.}%
  \index{Sturm comparison theorem!frequency bounds}%
  \index{variable media!oscillation}%
  \index{inhomogeneous string}%
  For a vibrating string with variable density $\rho(x)$, the Sturm
  comparison theorem bounds the eigenfrequencies $\omega_{n}$ between
  those of uniform strings with $\rho_{\min}$ and $\rho_{\max}$.
  Szeg\"{o}'s refinement gives sharper bounds using integral averages
  of $\rho$ rather than pointwise extremes.

\item \textbf{Semiclassical eigenvalue estimates.}%
  \index{semiclassical approximation!eigenvalue bounds}%
  \index{Szeg\"o comparison theorem}%
  \index{Bohr--Sommerfeld quantisation}%
  Szeg\"{o}'s comparison theorem, which compares solutions based on
  averages of the coefficient function rather than pointwise bounds,
  connects to the Bohr--Sommerfeld quantisation rule
  $\int_{x_{1}}^{x_{2}}k(x)\,dx=(n+\tfrac{1}{2})\pi$.  It provides
  rigorous error estimates for semiclassical eigenvalue approximations.
\end{enumerate}

\paragraph{Mathematics applications.}
\begin{enumerate}
\item \textbf{Comparison via Pr\"ufer angles.}%
  \index{Pr\"ufer substitution!comparison}%
  \index{Sturm comparison!Pr\"ufer proof}%
  The Pr\"{u}fer formulation gives a transparent proof of the Sturm
  comparison theorem: if $q_{1}\leq q_{2}$, then the Pr\"{u}fer angle
  $\theta_{2}$ for the more oscillatory equation increases at least as
  fast as $\theta_{1}$, so zeros of the second equation interlace
  with (and occur at least as often as) zeros of the first.

\item \textbf{Averaging and Szeg\"o's extension.}%
  \index{Szeg\"o comparison!averaging}%
  \index{integral comparison}%
  Szeg\"{o}'s theorem replaces the pointwise condition $q_{1}\leq q_{2}$
  with an integral condition: if $\int_{a}^{x}q_{1}\leq\int_{a}^{x}q_{2}$
  for all $x$, then comparison still holds.  This is strictly weaker than
  Sturm's condition and is useful when $q_{2}-q_{1}$ oscillates in sign
  but has positive running average.
\end{enumerate}

%% -------------------------------------------------------------------
\subsubsection{16.627\quad Picone's identity}
\subsubsection{16.628\quad Sturm--Picone theorem}

Picone's identity is
$\frac{d}{dx}\left[\frac{y}{v}(pv'y-qy'v)\right]
=(p-q)(y')^{2}+(P-Q)\left(\frac{y}{v}\right)^{2}v'^{2}
+q\left(y'-\frac{v'}{v}y\right)^{2}$,
where $y$ and $v$ are solutions of different self-adjoint equations
$[py']'+Py=0$ and $[qv']'+Qv=0$.

\paragraph{Physics applications.}
\begin{enumerate}
\item \textbf{Comparison of different physical systems.}%
  \index{Picone's identity!physical comparison}%
  \index{Sturm--Picone theorem!applications}%
  \index{stiffness comparison}%
  The Sturm--Picone theorem generalises the Sturm comparison theorem
  to the self-adjoint form $[p(x)y']'+q(x)y=0$ with variable $p$.
  This allows comparison of systems with different stiffness profiles
  (variable $p$) as well as different restoring forces ($q$): for
  instance, comparing oscillations of beams with different
  cross-sectional profiles.

\item \textbf{Spectral bounds for Sturm--Liouville operators.}%
  \index{spectral bounds!Sturm--Picone}%
  \index{eigenvalue bounds!comparison}%
  The Sturm--Picone theorem gives eigenvalue comparison: if $p\leq P$
  and $q\leq Q$, then the $n$th eigenvalue of $[Py']'+Qy+\lambda y=0$
  is no larger than that of $[py']'+qy+\lambda y=0$.  This is used to
  bound eigenvalues of complicated operators by comparison with simpler
  ones.
\end{enumerate}

\paragraph{Mathematics applications.}
\begin{enumerate}
\item \textbf{Picone's identity as a Lagrangian tool.}%
  \index{Picone's identity!variational}%
  \index{Lagrangian!Picone identity}%
  \index{Rayleigh quotient!comparison}%
  Integrating Picone's identity over $[a,b]$ relates boundary terms
  to an integral of non-negative quantities, yielding the Sturm--Picone
  comparison theorem directly.  The identity can also be used to derive
  Rayleigh quotient bounds on eigenvalues and to prove
  Hardy-type inequalities.

\item \textbf{Extensions to half-linear and $p$-Laplacian equations.}%
  \index{half-linear equations}%
  \index{$p$-Laplacian!oscillation}%
  \index{Picone's identity!nonlinear extensions}%
  Picone's identity has been extended to half-linear equations
  $(|y'|^{p-2}y')'+q(x)|y|^{p-2}y=0$ (the eigenvalue equation of the
  $p$-Laplacian), providing comparison and oscillation theorems for
  nonlinear operators.  This has applications to the regularity theory
  of quasilinear elliptic PDEs.
\end{enumerate}

%% -------------------------------------------------------------------
\subsubsection{16.629\quad Oscillation on the half line}

\paragraph{Physics applications.}
\begin{enumerate}
\item \textbf{Scattering states vs.\ bound states.}%
  \index{scattering states!oscillation}%
  \index{bound states!non-oscillation}%
  \index{essential spectrum!oscillation}%
  In quantum mechanics, oscillatory solutions on $[0,\infty)$ correspond
  to scattering states (continuous spectrum, $E>0$ for short-range
  potentials), while non-oscillatory solutions correspond to bound
  states (discrete spectrum, $E<0$).  The oscillation criteria on the
  half line determine the threshold between discrete and continuous
  spectrum.

\item \textbf{Stability of the hydrogen atom.}%
  \index{hydrogen atom!stability}%
  \index{critical coupling!oscillation}%
  \index{inverse-square potential}%
  For the radial Schr\"{o}dinger equation with potential
  $V(r)=-g/r^{2}$, Kneser-type oscillation criteria show that
  solutions oscillate (infinitely many bound states) if and only if
  $g>1/4$.  For the Coulomb potential $V=-e^{2}/r$, the centrifugal
  term ensures non-oscillation at $r=0$ for each angular momentum
  $\ell$, giving a discrete spectrum (hydrogen energy levels).
\end{enumerate}

\paragraph{Mathematics applications.}
\begin{enumerate}
\item \textbf{Hille's oscillation criteria.}%
  \index{Hille's oscillation criterion}%
  \index{oscillation!half-line criteria}%
  \index{critical constant $1/4$}%
  For $y''+q(x)y=0$ on $[1,\infty)$, Hille (1948) showed:
  (i)~if $\limsup_{x\to\infty}x\int_{x}^{\infty}q(t)\,dt>1$, then
  solutions oscillate;
  (ii)~if $x\int_{x}^{\infty}q(t)\,dt\leq 1/4$ for large $x$, then
  solutions are non-oscillatory.  The critical constant $1/4$ is sharp,
  as shown by the Euler equation $y''+\frac{1}{4x^{2}}y=0$ with
  solution $y=\sqrt{x}\ln x$ (non-oscillatory but borderline).

\item \textbf{Limit-point and limit-circle classification.}%
  \index{limit-point!limit-circle}%
  \index{Weyl's classification}%
  \index{essential self-adjointness}%
  Weyl's limit-point/limit-circle classification determines whether a
  Sturm--Liouville operator is essentially self-adjoint on the
  half-line.  In the limit-point case (typical for $q(x)\to+\infty$
  or slowly), no boundary condition is needed at $\infty$; in the
  limit-circle case, a boundary condition at $\infty$ is required.
  Oscillation criteria help determine the classification: if all
  solutions are $L^{2}$ near $\infty$, the equation is limit-circle
  \cite{Teschl2012}.
\end{enumerate}

%% -------------------------------------------------------------------
\subsubsection{16.71\quad Two Related Comparison Theorems}
\subsubsection{16.711\quad Theorem 1}
\subsubsection{16.712\quad Theorem 2}

\paragraph{Physics applications.}
\begin{enumerate}
\item \textbf{Envelope estimates for wave amplitudes.}%
  \index{envelope!wave amplitude}%
  \index{amplitude bounds!comparison}%
  \index{energy method!amplitude estimates}%
  Comparison theorems for solutions of different equations provide
  envelope bounds on wave amplitudes.  If the medium parameters change
  slowly (adiabatically), the amplitude of a wave governed by
  $y''+q(x)y=0$ can be bounded by comparing with constant-coefficient
  equations above and below.  This gives rigorous WKB-type amplitude
  estimates $|y|\sim q^{-1/4}$ without the full asymptotic machinery.

\item \textbf{Bounding solutions in stability analysis.}%
  \index{stability analysis!comparison bounds}%
  \index{Lyapunov methods!comparison}%
  \index{parametric excitation!bounds}%
  In the stability analysis of linear systems with time-varying
  coefficients (e.g., the Mathieu equation for parametric excitation),
  comparison theorems bound the growth or decay of solutions.  If
  $q(x)\geq q_{\min}>0$, all solutions are bounded, while if
  $q(x)$ takes negative values, comparison with the worst-case
  constant equation gives growth rate estimates.
\end{enumerate}

\paragraph{Mathematics applications.}
\begin{enumerate}
\item \textbf{Differential inequalities and maximum principles.}%
  \index{differential inequalities}%
  \index{maximum principle!ODE}%
  \index{sub- and supersolutions}%
  The comparison theorems of G\&R~16.71 are instances of the general
  theory of differential inequalities: if $y''+q_{1}y\leq 0$ and
  $u''+q_{2}u=0$ with $q_{1}\geq q_{2}$, then $y$ oscillates at
  least as fast as $u$.  This is the ODE analogue of the maximum
  principle for elliptic PDEs.

\item \textbf{Sturm--Liouville eigenvalue monotonicity.}%
  \index{eigenvalue monotonicity}%
  \index{potential perturbation!eigenvalues}%
  \index{domain monotonicity!eigenvalues}%
  Comparison theorems imply that eigenvalues of $-y''+q(x)y=\lambda y$
  are monotone in $q$: increasing the potential $q$ increases all
  eigenvalues.  Similarly, eigenvalues are monotonically decreasing in
  the length of the interval (domain monotonicity).  These monotonicity
  results are proved by counting zeros using comparison.
\end{enumerate}

%% -------------------------------------------------------------------
\subsubsection{16.81--16.82\quad Non-Oscillatory Solutions}
\subsubsection{16.811\quad Kneser's non-oscillation theorem}

Kneser's theorem states that for $y''+q(x)y=0$ on $[1,\infty)$:
if $x^{2}q(x)\leq 1/4$ for all large $x$, then the equation is
non-oscillatory; if $x^{2}q(x)\geq c>1/4$ for all large $x$, then
it is oscillatory.

\paragraph{Physics applications.}
\begin{enumerate}
\item \textbf{Long-range vs.\ short-range potentials.}%
  \index{Kneser's theorem!potentials}%
  \index{long-range potential}%
  \index{short-range potential}%
  \index{Coulomb potential!oscillation}%
  Kneser's theorem with $q(x)=E-V(x)$ distinguishes long-range and
  short-range potentials in quantum scattering.  For a potential
  decaying as $V(x)\sim -g/x^{2}$, the critical coupling $g=1/4$
  separates the regime of finitely many bound states ($g<1/4$) from
  infinitely many ($g>1/4$).  The Coulomb potential $V=-e^{2}/r$ is
  long-range but the effective potential $V_{\mathrm{eff}}=
  -e^{2}/r+\ell(\ell+1)\hbar^{2}/(2mr^{2})$ satisfies Kneser's
  condition for non-oscillation at $r\to\infty$ when $E<0$.

\item \textbf{Overdamped systems and exponential decay.}%
  \index{overdamped systems}%
  \index{exponential decay!non-oscillatory}%
  \index{critical damping}%
  Non-oscillatory solutions correspond physically to overdamped or
  critically damped behaviour.  For a harmonic oscillator with increasing
  damping, the transition from oscillatory to non-oscillatory is the
  critical damping point.  Kneser-type criteria generalise this to
  variable-coefficient systems.
\end{enumerate}

\paragraph{Mathematics applications.}
\begin{enumerate}
\item \textbf{The Euler equation as the critical case.}%
  \index{Euler equation!critical oscillation}%
  \index{critical constant $1/4$!Kneser}%
  \index{slowly varying solutions}%
  The Euler equation $y''+\frac{c}{x^{2}}y=0$ has solutions
  $y=x^{(1\pm\sqrt{1-4c})/2}$.  The critical case $c=1/4$ gives
  $y=\sqrt{x}$ and $y=\sqrt{x}\ln x$---non-oscillatory but with
  the slowest possible decay.  This is the boundary between power-law
  solutions ($c<1/4$) and oscillatory solutions ($c>1/4$).

\item \textbf{Non-oscillation and disconjugacy on $[0,\infty)$.}%
  \index{disconjugacy!half-line}%
  \index{non-oscillation!characterisations}%
  \index{Hartman's theorem}%
  Hartman's theorem characterises non-oscillation of $y''+q(x)y=0$ on
  $[a,\infty)$ by the existence of a solution $y>0$ on $[a,\infty)$
  (equivalently, a solution of the Riccati equation $w'+w^{2}+q=0$
  on $[a,\infty)$).  This connects non-oscillation to the Riccati
  equation theory of G\&R~16.514.
\end{enumerate}

%% -------------------------------------------------------------------
\subsubsection{16.822\quad Comparison theorem for non-oscillation}
\subsubsection{16.823\quad Necessary and sufficient conditions for non-oscillation}

\paragraph{Physics applications.}
\begin{enumerate}
\item \textbf{Stability boundaries for variable-coefficient systems.}%
  \index{stability boundary!non-oscillation}%
  \index{parametric resonance!stability chart}%
  \index{Mathieu equation!stability}%
  The transition from non-oscillatory to oscillatory behaviour
  corresponds to a stability boundary.  For the Mathieu equation
  $y''+(a-2q\cos 2x)y=0$, the stability chart (Strutt diagram)
  delineates regions of stable (bounded, possibly oscillatory) and
  unstable (exponentially growing) solutions.  Non-oscillation criteria
  determine the stable regions for the associated Hill equation.

\item \textbf{Sub-barrier behaviour and evanescent waves.}%
  \index{evanescent waves}%
  \index{sub-barrier!non-oscillatory}%
  \index{total internal reflection}%
  In regions where $q(x)<0$ (classically forbidden, sub-barrier),
  solutions are non-oscillatory and exponentially decaying.
  Non-oscillation criteria quantify the decay rate, relevant for
  tunnel diode design, evanescent wave coupling in fibre optics,
  and total internal reflection.
\end{enumerate}

\paragraph{Mathematics applications.}
\begin{enumerate}
\item \textbf{Necessary and sufficient conditions.}%
  \index{non-oscillation!necessary and sufficient}%
  \index{Leighton--Wintner theorem}%
  \index{integral conditions!oscillation}%
  The Leighton--Wintner theorem gives a sufficient condition for
  oscillation: if $\int_{a}^{\infty}q(x)\,dx=+\infty$, then
  $y''+q(x)y=0$ is oscillatory.  Combining this with Kneser's
  non-oscillation criterion provides sharp necessary and sufficient
  conditions for many classes of coefficient functions.

\item \textbf{Riccati equation and non-oscillation.}%
  \index{Riccati equation!non-oscillation}%
  \index{comparison!non-oscillatory solutions}%
  Non-oscillation on $[a,\infty)$ is equivalent to the existence of a
  solution of the Riccati inequality $w'+w^{2}+q(x)\leq 0$ on
  $[a,\infty)$.  Comparison theorems for non-oscillation then reduce to
  comparison of the corresponding Riccati equations, providing a unified
  framework linking the oscillation theory of G\&R~16.6 with the Riccati
  theory of G\&R~16.5.
\end{enumerate}

%% -------------------------------------------------------------------
\subsubsection{16.91\quad Some Growth Estimates for Solutions of Second-Order Equations}
\subsubsection{16.911\quad Strictly increasing and decreasing solutions}
\subsubsection{16.912\quad General result on dominant and subdominant solutions}
\subsubsection{16.913\quad Estimate of dominant solution}

The asymptotic behaviour of solutions of $y''+q(x)y=0$ as $x\to\infty$
is characterised by the dominant and subdominant solutions.  If $q(x)<0$
for large $x$, one solution grows and one decays; the growing one is
\emph{dominant} and the decaying one is \emph{subdominant}.  The ratio of
any two linearly independent solutions diverges, and the dominant solution
is the one selected by generic initial conditions.

\paragraph{Physics applications.}
\begin{enumerate}
\item \textbf{Tunnelling wavefunctions and asymptotic decay.}%
  \index{tunnelling!subdominant solution}%
  \index{wavefunction!asymptotic decay}%
  \index{bound state!exponential tail}%
  The bound-state wavefunction of the Schr\"{o}dinger equation must be
  the subdominant solution as $x\to\infty$ (otherwise it would be
  non-normalisable).  The quantisation condition arises from matching
  the subdominant solution at $+\infty$ with the subdominant at
  $-\infty$ through the oscillatory region---this is the essence of the
  WKB quantisation rule and the exact quantisation via Stokes graphs.

\item \textbf{Amplification in parametrically excited systems.}%
  \index{parametric excitation!growth}%
  \index{Mathieu equation!growth estimates}%
  \index{dominant solution!parametric resonance}%
  In the unstable regions of the Mathieu equation, the dominant
  solution grows exponentially.  Growth estimates bound the Floquet
  exponent $\mu$ (the rate of exponential growth per period),
  critical for determining the onset of parametric instability in
  mechanical systems, Faraday waves, and Paul traps for ions.

\item \textbf{Stokes phenomenon in asymptotic analysis.}%
  \index{Stokes phenomenon}%
  \index{asymptotic series!Stokes lines}%
  \index{dominant/subdominant!Stokes switching}%
  The Stokes phenomenon is the sudden switching of the coefficient
  of the subdominant solution as a Stokes line is crossed in the
  complex plane.  This is intimately connected to growth estimates:
  the subdominant solution is exponentially smaller than the dominant,
  so its coefficient is ambiguous to the accuracy of the asymptotic
  expansion of the dominant solution.
\end{enumerate}

\paragraph{Mathematics applications.}
\begin{enumerate}
\item \textbf{Dichotomy and exponential splitting.}%
  \index{exponential dichotomy}%
  \index{dominant/subdominant!dichotomy}%
  \index{stable and unstable manifolds}%
  The existence of dominant and subdominant solutions is an instance of
  exponential dichotomy: the solution space splits into subspaces of
  exponentially growing and decaying solutions.  For systems
  $\mathbf{y}'=A(x)\mathbf{y}$, exponential dichotomy is the key
  hypothesis for the existence of bounded solutions of
  inhomogeneous equations (roughness theorem).

\item \textbf{Asymptotic integration (Levinson's theorem).}%
  \index{Levinson's theorem!asymptotic integration}%
  \index{asymptotic integration}%
  \index{perturbation!asymptotic}%
  Levinson's theorem (1948) states that if $A(x)\to A_{0}$ as
  $x\to\infty$ and the eigenvalues of $A_{0}$ have distinct real parts,
  then the system $\mathbf{y}'=A(x)\mathbf{y}$ has a fundamental matrix
  asymptotic to $e^{A_{0}x}$.  This provides the rigorous foundation
  for the WKB approximation and the asymptotic classification of
  solutions into dominant and subdominant.

\item \textbf{Liouville--Green (LG) approximation.}%
  \index{Liouville--Green approximation}%
  \index{WKB approximation!rigorous}%
  \index{phase-amplitude method}%
  For $y''+\lambda^{2}q(x)y=0$ with $q>0$ and $\lambda\to\infty$, the
  Liouville--Green approximation gives
  $y\sim q^{-1/4}\exp\!\left(\pm\lambda\int q^{1/2}\,dx\right)$ with
  rigorous error bounds $O(1/\lambda)$.  The growth estimate is
  controlled by $\int q^{1/2}\,dx$, the ``optical path length'' through
  the medium.
\end{enumerate}

%% -------------------------------------------------------------------
\subsubsection{16.914\quad A theorem due to Lyapunov}

Lyapunov's inequality states that if $y''+q(x)y=0$ has a non-trivial
solution vanishing at both $x=a$ and $x=b$ ($a<b$), then
$\int_{a}^{b}q(x)\,dx>\frac{4}{b-a}$.

\paragraph{Physics applications.}
\begin{enumerate}
\item \textbf{Lower bounds on eigenvalues.}%
  \index{Lyapunov's inequality!eigenvalue bounds}%
  \index{eigenvalue!lower bound}%
  \index{quantum well!eigenvalue estimate}%
  For the eigenvalue problem $y''+\lambda q(x)y=0$, $y(a)=y(b)=0$,
  Lyapunov's inequality gives
  $\lambda_{1}\int_{a}^{b}q(x)\,dx>4/(b-a)$, hence a lower bound on
  the first eigenvalue.  For a quantum well of width $L$, this gives
  $E_{1}>4\hbar^{2}/(2mL^{2})\cdot(1/\int_{0}^{L}1\,dx)
  =2\hbar^{2}/(mL^{2})$, within a factor of $\pi^{2}/2$ of the exact
  value.

\item \textbf{Stability criteria for Hill's equation.}%
  \index{Hill's equation!Lyapunov stability}%
  \index{stability criterion!Lyapunov inequality}%
  \index{periodic orbit!stability}%
  For Hill's equation $y''+[a+q(x)]y=0$ with $q$ periodic of period $T$,
  Lyapunov's inequality applied to each half-period gives stability
  criteria: if $\int_{0}^{T}q(x)\,dx$ is too small, no solution can
  have two zeros in one period, ensuring stability.  This provides
  simple, computable stability tests for periodic orbits.
\end{enumerate}

\paragraph{Mathematics applications.}
\begin{enumerate}
\item \textbf{Sharpness and generalisations.}%
  \index{Lyapunov's inequality!sharpness}%
  \index{Lyapunov's inequality!generalisations}%
  \index{de La Vall\'ee-Poussin inequality}%
  The constant $4/(b-a)$ in Lyapunov's inequality is sharp, attained
  in the limit by the constant-coefficient equation $y''+\pi^{2}/(b-a)^{2}y=0$.
  Generalisations replace $4/(b-a)$ with larger constants involving
  higher moments of $q$ or weighted integrals, and extend to systems,
  higher-order equations, and fractional differential operators.

\item \textbf{Disconjugacy and de La Vall\'ee-Poussin criterion.}%
  \index{de La Vall\'ee-Poussin!disconjugacy}%
  \index{disconjugacy!Lyapunov inequality}%
  The contrapositive of Lyapunov's inequality gives a disconjugacy
  criterion: if $\int_{a}^{b}q^{+}(x)\,dx\leq 4/(b-a)$, then no
  non-trivial solution has two zeros in $[a,b]$.  This is a key tool
  in boundary value problem theory, where disconjugacy ensures
  unique solvability of two-point boundary value problems.
\end{enumerate}

%% -------------------------------------------------------------------
\subsubsection{16.92\quad Boundedness Theorems}
\subsubsection{16.921\quad All solutions of the equation}
\subsubsection{16.922\quad If all solutions of the equation}
\subsubsection{16.923\quad If $a(x)\to \infty$ monotonically as $x\to \infty$, then all solutions of}
\subsubsection{16.924\quad Consider the equation}

The boundedness theorems address the question: under what conditions on
$q(x)$ are all solutions of $y''+q(x)y=0$ bounded as $x\to\infty$?
This is a more delicate question than oscillation, as oscillatory solutions
may still be unbounded.

\paragraph{Physics applications.}
\begin{enumerate}
\item \textbf{Stability of oscillations with varying frequency.}%
  \index{bounded oscillations}%
  \index{variable frequency!stability}%
  \index{adiabatic invariant}%
  For $y''+\omega^{2}(x)y=0$ with slowly varying $\omega(x)$, the
  adiabatic invariant $E(x)/\omega(x)$ (energy divided by frequency)
  is approximately constant, giving amplitude
  $|y|\sim\omega^{-1/2}$.  This is bounded if $\omega\to\infty$
  (solutions actually decay) and unbounded if $\omega\to 0$.
  The boundedness theorems make this precise when $\omega$ varies
  non-monotonically.

\item \textbf{Quantum mechanics: normalisation and scattering.}%
  \index{normalisation!boundedness}%
  \index{scattering!bounded solutions}%
  \index{Jost solutions}%
  Bounded solutions of the Schr\"{o}dinger equation on $[0,\infty)$
  at energy $E>0$ correspond to scattering states.  The Jost solution
  $f(k,x)\sim e^{ikx}$ as $x\to\infty$ is bounded, and its behaviour
  at $x=0$ determines the scattering phase shift $\delta(k)$.
  Boundedness criteria determine which energies belong to the
  absolutely continuous spectrum.

\item \textbf{Suppression of parametric resonance.}%
  \index{parametric resonance!suppression}%
  \index{increasing stiffness!boundedness}%
  \index{WKB!boundedness}%
  The result that all solutions of $y''+a(x)y=0$ are bounded when
  $a(x)\to\infty$ monotonically (G\&R~16.923) explains why a stiffening
  spring suppresses unbounded growth: the increasing natural frequency
  prevents resonance accumulation.  The amplitude decreases as
  $a^{-1/4}$ (WKB estimate), confirmed by the rigorous boundedness
  theorem.
\end{enumerate}

\paragraph{Mathematics applications.}
\begin{enumerate}
\item \textbf{Energy method and Sonin--P\'olya theorem.}%
  \index{Sonin--P\'olya theorem}%
  \index{energy method!boundedness}%
  \index{amplitude monotonicity}%
  The Sonin--P\'{o}lya theorem states that the successive maxima of
  $|y|$ for $y''+q(x)y=0$ are non-increasing when $q(x)$ is
  non-decreasing.  This is proved by the energy method: define
  $E=y'^{2}+q(x)y^{2}$; then $E'=q'y^{2}\geq 0$ when $q'\geq 0$,
  but the maxima of $|y|$ are $|y_{\max}|=\sqrt{E/q}$, which decreases
  when $q$ grows faster than $E$.

\item \textbf{Wintner's boundedness theorem.}%
  \index{Wintner's theorem!boundedness}%
  \index{bounded solutions!sufficient conditions}%
  \index{integral conditions!boundedness}%
  Wintner's theorem gives conditions on $q$ ensuring all solutions are
  bounded: if $q(x)>0$ for large $x$ and $\int^{\infty}|q'|/q^{3/2}<\infty$,
  then all solutions are bounded and behave like $q^{-1/4}\sin$ or $\cos$
  of $\int q^{1/2}\,dx$.  The condition $|q'|/q^{3/2}\in L^{1}$ quantifies
  ``slowly varying $q$'' and is the rigorous version of the WKB
  validity condition.

\item \textbf{Cesaro means and generalised boundedness.}%
  \index{Ces\`aro means!boundedness}%
  \index{generalised boundedness}%
  \index{Hartman--Wintner theorem}%
  Hartman and Wintner showed that if $q(x)\to+\infty$ and $q$ has
  bounded variation on each interval $[n,n+1]$, then solutions are
  bounded.  More refined results use Ces\`{a}ro means of $q$: even if
  $q$ oscillates, its average growth determines boundedness of solutions.
\end{enumerate}

%% -------------------------------------------------------------------
\subsubsection{16.93\quad Growth of maxima of $|y|$}

\paragraph{Physics applications.}
\begin{enumerate}
\item \textbf{Amplitude modulation and beats.}%
  \index{amplitude modulation!ODE}%
  \index{beats!envelope growth}%
  \index{envelope!growth of maxima}%
  The successive maxima of $|y|$ form the ``envelope'' of the
  oscillation.  In physical systems with slowly varying parameters, the
  envelope evolves on a slow time scale.  Beating between two close
  frequencies produces a sinusoidal envelope
  $A(t)=2|\cos(\Delta\omega\,t/2)|$, while parametric driving can
  produce exponentially growing envelopes in unstable regimes.

\item \textbf{Seismic wave amplification.}%
  \index{seismic waves!amplitude growth}%
  \index{site amplification}%
  \index{impedance contrast!amplification}%
  Seismic waves propagating upward through layers of decreasing
  impedance $\rho c$ are amplified: the maxima of $|y|$ grow as
  $(\rho c)^{-1/2}$.  Growth-of-maxima estimates quantify site
  amplification factors, critical for earthquake engineering and
  building codes.
\end{enumerate}

\paragraph{Mathematics applications.}
\begin{enumerate}
\item \textbf{Pr\"ufer analysis of amplitude growth.}%
  \index{Pr\"ufer substitution!amplitude}%
  \index{amplitude function!Pr\"ufer}%
  \index{growth rate!maxima}%
  In the Pr\"{u}fer substitution $y=r\sin\theta$, the amplitude $r(x)$
  satisfies $(\ln r)'=\frac{1}{2}(1-q)\sin 2\theta$.  The maxima of
  $|y|$ occur when $\theta=\pi/2+n\pi$ (where $y'=0$), and their
  growth is controlled by the integral $\int(1-q)\sin 2\theta\,dx$
  between successive maxima.  Averaging gives envelope growth
  proportional to $q^{-1/4}$ for slowly varying $q$.

\item \textbf{Asymptotic distribution of maxima.}%
  \index{asymptotic distribution!maxima}%
  \index{Sonin--P\'olya theorem!growth}%
  \index{Liouville--Green!amplitude}%
  For $y''+q(x)y=0$ with $q(x)\to+\infty$, the Sonin--P\'{o}lya
  theorem guarantees that successive maxima of $|y|$ are non-increasing.
  The Liouville--Green approximation refines this: the $n$th maximum
  is approximately $q(x_{n})^{-1/4}$ where $x_{n}$ is the location of
  the $n$th maximum.  The spacing between consecutive maxima is
  approximately $\pi/q(x_{n})^{1/2}$, decreasing as $q$ grows.
\end{enumerate}
