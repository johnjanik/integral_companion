%% Section 18 — The z-Transform
\section{18\quad The $z$-Transform}

The $z$-transform is the discrete-time counterpart of the Laplace
transform.  Given a sequence $\{x[n]\}$, it associates a function of
the complex variable $z$ whose analytic properties encode the
asymptotic, stability, and spectral characteristics of the original
sequence.  The transform was popularised in engineering by Ragazzini
and Zadeh in the 1950s for sampled-data control systems, but its
mathematical roots lie in the theory of generating functions developed
by Euler and de~Moivre.  In the language of G\&R, the $z$-transform
translates the discrete sums and series of Sections~0--1 into the
complex-variable framework of Sections~8--9, providing a bridge between
combinatorial identities and contour integral representations.

\subsection{18.1--18.3\quad Definition, Bilateral, and Unilateral $z$-Transforms}

%% -------------------------------------------------------------------
\subsubsection{18.1\quad Definitions}

The $z$-transform of a sequence $\{x[n]\}_{n=-\infty}^{\infty}$ is the
Laurent series
\[
  X(z)=\mathcal{Z}\{x[n]\}=\sum_{n=-\infty}^{\infty}x[n]\,z^{-n},
\]
which converges in an annular region of the complex plane called the
\emph{region of convergence} (ROC): $r_{-}<|z|<r_{+}$.  The ROC
determines uniqueness: two distinct sequences may share the same
algebraic expression for $X(z)$ but differ in their ROC.  The inverse
$z$-transform recovers the sequence via a contour integral in the ROC:
\[
  x[n]=\frac{1}{2\pi i}\oint_{\mathcal{C}}X(z)\,z^{n-1}\,dz,
\]
where $\mathcal{C}$ is a simple closed contour encircling the origin
within the ROC.  When $z=e^{i\omega}$ lies on the unit circle and
belongs to the ROC, $X(e^{i\omega})$ reduces to the discrete-time
Fourier transform (DTFT) of $\{x[n]\}$.  The fundamental properties of
the $z$-transform---linearity, time shifting
($\mathcal{Z}\{x[n-k]\}=z^{-k}X(z)$), convolution
($\mathcal{Z}\{x*y\}=X(z)Y(z)$), and the initial and final value
theorems---mirror those of the Laplace transform (Section~17.11) under
the substitution $z=e^{sT}$, where $T$ is the sampling period.

\paragraph{Physics applications.}
\begin{enumerate}
\item \textbf{Discrete-time signal processing and sampling.}%
  \index{z-transform@$z$-transform!definition}%
  \index{sampling theorem!z-transform@$z$-transform}%
  \index{discrete-time Fourier transform}%
  \index{Shannon--Nyquist theorem}%
  When a continuous-time signal $x(t)$ is sampled at intervals $T$, the
  sequence $x[n]=x(nT)$ has a $z$-transform related to the Laplace
  transform of $x(t)$ by $X(z)\big|_{z=e^{sT}}=\frac{1}{T}
  \sum_{k=-\infty}^{\infty}X_{L}(s+ik\omega_{s})$, where
  $\omega_{s}=2\pi/T$.  The Shannon--Nyquist theorem---requiring
  $\omega_{s}>2\omega_{\max}$ to avoid aliasing---is the condition for
  the DTFT $X(e^{i\omega})$ to faithfully represent the continuous
  spectrum.  Every modern ADC (analog-to-digital converter) implicitly
  invokes this mapping.

\item \textbf{Digital filter design: IIR and FIR filters.}%
  \index{digital filter!IIR}%
  \index{digital filter!FIR}%
  \index{transfer function!discrete-time}%
  \index{pole-zero diagram}%
  A causal linear time-invariant (LTI) discrete-time system is
  characterised by its transfer function $H(z)=Y(z)/X(z)$, a rational
  function of $z$.  An infinite impulse response (IIR) filter has poles
  inside the unit circle for stability; a finite impulse response (FIR)
  filter has all poles at $z=0$ (all-zero design).  Filter design
  methods---bilinear transform, impulse invariance, windowed
  sinc---all operate in the $z$-domain, placing poles and zeros to
  sculpt the frequency response $H(e^{i\omega})$.

\item \textbf{Discrete-time control systems and plant discretisation.}%
  \index{discrete-time control}%
  \index{zero-order hold}%
  \index{Jury stability criterion}%
  \index{root locus!discrete-time}%
  In digital control, continuous plant dynamics $G(s)$ are discretised
  via zero-order hold to obtain the pulse transfer function $G(z)$.
  Stability requires all closed-loop poles to lie inside the unit disk
  $|z|<1$---the discrete analogue of the left half-plane condition for
  $s$.  The Jury stability criterion and the discrete root locus
  technique are the $z$-domain counterparts of the Routh--Hurwitz
  criterion and the $s$-domain root locus.

\item \textbf{Numerical ODE solver stability analysis.}%
  \index{numerical methods!ODE stability}%
  \index{A-stability}%
  \index{root locus!numerical methods}%
  \index{multistep methods!characteristic polynomial}%
  The stability of a linear multistep method for
  $y'=\lambda y$ is analysed by substituting the trial solution
  $y_{n}=z^{n}$ into the difference equation, yielding a characteristic
  polynomial $\rho(z)-h\lambda\sigma(z)=0$.  The method is
  zero-stable if all roots of $\rho(z)$ satisfy $|z|\leq 1$ (with
  simple roots on the unit circle).  The boundary locus method plots
  $h\lambda=\rho(e^{i\theta})/\sigma(e^{i\theta})$ to determine the
  region of absolute stability---a direct application of
  $z$-transform analysis to numerical analysis.

\item \textbf{Lattice vibrations and phonon dispersion.}%
  \index{lattice vibrations}%
  \index{phonon!dispersion relation}%
  \index{tight-binding model}%
  In a one-dimensional monatomic lattice with nearest-neighbour
  coupling, the equation of motion $m\ddot{u}_{n}=\kappa(u_{n+1}
  -2u_{n}+u_{n-1})$ has solutions $u_{n}(t)=Ae^{i(\omega t-qna)}$.
  Substituting the $z$-transform ansatz $U(z)=\sum u_{n}z^{-n}$
  converts the difference equation into
  $(z+z^{-1}-2+m\omega^{2}/\kappa)U(z)=0$, yielding the dispersion
  relation $\omega^{2}=(4\kappa/m)\sin^{2}(qa/2)$.  The same algebraic
  structure appears in tight-binding electronic band theory.
\end{enumerate}

\paragraph{Mathematics applications.}
\begin{enumerate}
\item \textbf{Generating functions and analytic combinatorics.}%
  \index{generating function!ordinary}%
  \index{analytic combinatorics}%
  \index{singularity analysis}%
  \index{z-transform@$z$-transform!generating function equivalence}%
  The ordinary generating function $f(x)=\sum_{n=0}^{\infty}a_{n}x^{n}$
  is $X(z)$ evaluated at $z=1/x$ (up to index convention).  The
  singularity analysis programme of Flajolet and
  Sedgewick~\cite{FlajoletSedgewick2009} extracts asymptotics of
  $a_{n}$ from the location and type of the dominant singularity of
  $f(x)$: a simple pole at $x=1/\alpha$ gives $a_{n}\sim C\alpha^{n}$,
  an algebraic singularity $(1-\alpha x)^{-\beta}$ gives
  $a_{n}\sim C\alpha^{n}n^{\beta-1}/\Gamma(\beta)$.  In the
  $z$-transform language, these are statements about the ROC boundary.

\item \textbf{Laurent series and residue calculus.}%
  \index{Laurent series!z-transform@$z$-transform}%
  \index{residue theorem!inverse $z$-transform}%
  \index{partial fraction expansion}%
  The inverse $z$-transform via contour integration is a direct
  application of the residue theorem (Section~8):
  $x[n]=\sum_{k}\mathrm{Res}[X(z)z^{n-1},z_{k}]$ where the sum is
  over poles inside $\mathcal{C}$.  For rational $X(z)$, partial
  fraction decomposition reduces the inversion to a table lookup of
  standard $z$-transform pairs, exactly paralleling the Laplace
  inversion technique of Section~17.12.

\item \textbf{Difference equations and linear recurrences.}%
  \index{difference equations!z-transform@$z$-transform solution}%
  \index{linear recurrence!characteristic equation}%
  \index{Fibonacci sequence!generating function}%
  The $z$-transform converts a linear constant-coefficient difference
  equation $\sum_{k=0}^{N}a_{k}x[n-k]=\sum_{k=0}^{M}b_{k}u[n-k]$ into
  the algebraic equation $A(z)X(z)=B(z)U(z)+\text{(initial conditions)}$,
  where $A(z)=\sum a_{k}z^{-k}$.  The Fibonacci recurrence
  $F_{n}=F_{n-1}+F_{n-2}$ yields
  $F(z)=z/(z^{2}-z-1)$; partial fractions give the Binet formula
  $F_{n}=(\varphi^{n}-\hat{\varphi}^{n})/\sqrt{5}$ with
  $\varphi=(1+\sqrt{5})/2$.

\item \textbf{Discrete convolution and polynomial multiplication.}%
  \index{convolution!discrete}%
  \index{polynomial multiplication!z-transform@$z$-transform}%
  \index{Cauchy product}%
  The convolution theorem
  $\mathcal{Z}\{x*y\}=X(z)Y(z)$ shows that discrete convolution
  corresponds to polynomial (or formal power series) multiplication.
  This is the algebraic foundation of the fast Fourier transform
  (FFT): evaluate $X$ and $Y$ at roots of unity, multiply pointwise,
  and interpolate, achieving $O(n\log n)$ complexity for polynomial
  multiplication.
\end{enumerate}

%% -------------------------------------------------------------------
\subsubsection{18.2\quad Bilateral $z$-transform}

The bilateral (or two-sided) $z$-transform
\[
  X(z)=\sum_{n=-\infty}^{\infty}x[n]\,z^{-n}
\]
treats sequences defined on all of $\mathbb{Z}$.  Its ROC is an open
annulus $r_{-}<|z|<r_{+}$, and the transform is analytic in that
region.  Distinct ROCs for the same algebraic expression correspond to
different sequences: for example, $X(z)=z/(z-a)$ with ROC $|z|>|a|$
gives the causal sequence $x[n]=a^{n}u[n]$, whereas the same $X(z)$
with ROC $|z|<|a|$ gives the anticausal sequence $x[n]=-a^{n}u[-n-1]$
(where $u[n]$ is the unit step).  The bilateral transform is the
natural setting for non-causal systems, two-sided convolutions, and
sequences arising from doubly-infinite lattice models.  Key properties
include the time-reversal rule $\mathcal{Z}\{x[-n]\}=X(1/z)$ (with
inverted ROC) and the multiplication property
$\mathcal{Z}\{a^{n}x[n]\}=X(z/a)$ (ROC scaled by $|a|$).  The
bilateral transform also admits a Parseval-type relation:
\[
  \sum_{n=-\infty}^{\infty}x[n]\overline{y[n]}
  =\frac{1}{2\pi i}\oint_{\mathcal{C}}X(z)\,\bar{Y}(1/\bar{z})\,
  \frac{dz}{z},
\]
which provides an inner product on $\ell^{2}(\mathbb{Z})$ expressed as
a contour integral.

\paragraph{Physics applications.}
\begin{enumerate}
\item \textbf{Discrete Green's functions on infinite lattices.}%
  \index{Green's function!discrete}%
  \index{lattice Green's function}%
  \index{random walk!lattice}%
  \index{bilateral $z$-transform!Green's function}%
  The lattice Green's function $G[n]$ satisfying
  $(E-H_{\text{lat}})G[n]=\delta[n]$ on a one-dimensional tight-binding
  chain is most naturally computed via the bilateral $z$-transform,
  which handles the two-sided spatial extent.  The poles of
  $G(z)=(z^{2}-(E/t)z+1)^{-1}$ at $|z|=1$ give the continuous spectrum
  (band), while poles off the unit circle correspond to evanescent
  (bound) states.  In the random walk interpretation, $G[n]$ is the
  expected number of visits to site~$n$ starting from the origin.

\item \textbf{Non-causal Wiener filtering.}%
  \index{Wiener filter!non-causal}%
  \index{bilateral $z$-transform!Wiener filter}%
  \index{power spectral density}%
  \index{minimum mean-square error}%
  The optimal non-causal Wiener filter for estimating a signal $s[n]$
  from noisy observations $x[n]=s[n]+w[n]$ is
  $H_{\mathrm{opt}}(z)=S_{sx}(z)/S_{xx}(z)$, where $S_{sx}$ and
  $S_{xx}$ are the cross- and auto-power spectral densities expressed as
  bilateral $z$-transforms of correlation sequences.  Because the filter
  has both causal and anticausal components, its impulse response
  $h[n]$ extends to $n<0$, requiring the bilateral framework.
  Restricting to causal filters leads to the Wiener--Hopf factorisation
  problem.

\item \textbf{LFSR sequences and stream cipher cryptanalysis.}%
  \index{LFSR!z-transform@$z$-transform analysis}%
  \index{stream cipher!LFSR}%
  \index{feedback polynomial}%
  \index{Berlekamp--Massey algorithm}%
  A linear feedback shift register (LFSR) with characteristic
  polynomial $p(z)=1+c_{1}z^{-1}+\cdots+c_{L}z^{-L}$ generates a
  periodic binary sequence whose $z$-transform is
  $S(z)=q(z)/p(z)$, a rational function.  The Berlekamp--Massey
  algorithm recovers the minimal polynomial $p(z)$ from $2L$ output
  bits---this is essentially a Pad\'{e} approximation problem in the
  $z$-domain.  In stream cipher cryptanalysis, the algebraic structure
  of $S(z)$ reveals the LFSR length and taps, motivating the use of
  nonlinear combiners to destroy the rational structure.

\item \textbf{Two-sided lattice models in statistical mechanics.}%
  \index{transfer matrix!z-transform@$z$-transform}%
  \index{Ising model!transfer matrix}%
  \index{partition function!lattice}%
  \index{bilateral $z$-transform!statistical mechanics}%
  The partition function of the one-dimensional Ising model
  $Z=\sum_{\{s_{n}\}}\exp\bigl(\beta J\sum_{n}s_{n}s_{n+1}
  +\beta h\sum_{n}s_{n}\bigr)$ can be written as a trace of transfer
  matrices $Z=\mathrm{tr}(\mathbf{T}^{N})$.  For infinite chains, the
  bilateral $z$-transform of the spin--spin correlation function
  $\langle s_{0}s_{n}\rangle$ is a rational function of $z$ whose poles
  yield the correlation length $\xi=-1/\ln|\lambda_{2}/\lambda_{1}|$,
  where $\lambda_{1,2}$ are the transfer matrix eigenvalues.

\item \textbf{Scattering on discrete structures.}%
  \index{scattering matrix!discrete}%
  \index{transmission line!discrete model}%
  \index{impedance matching!discrete}%
  A layered medium or periodically loaded transmission line is modelled
  by a discrete scattering problem: the wave amplitudes $a[n]$ and
  $b[n]$ (forward and backward) satisfy a recursion governed by the
  local reflection coefficient $\Gamma[n]$.  The bilateral
  $z$-transform of the scattering matrix converts this recursion into a
  matrix polynomial equation, whose factorisation yields the
  Schur--Levinson algorithm for layer stripping and impedance
  reconstruction from reflection data.
\end{enumerate}

\paragraph{Mathematics applications.}
\begin{enumerate}
\item \textbf{Doubly-infinite Laurent series and function theory.}%
  \index{Laurent series!doubly-infinite}%
  \index{annulus of convergence}%
  \index{analytic continuation!z-transform@$z$-transform}%
  The bilateral $z$-transform is a Laurent series
  $X(z)=\sum_{n=-\infty}^{\infty}c_{n}z^{-n}$ converging in an annulus.
  The theory of Laurent series in complex analysis (Section~8) guarantees
  that $X(z)$ is analytic in the ROC and that the coefficients are
  recovered by $c_{n}=(2\pi i)^{-1}\oint X(z)z^{n-1}\,dz$.  Different
  annuli of convergence yield different analytic continuations of the same
  formal series, providing a clean framework for distinguishing causal
  and anticausal sequences.

\item \textbf{Spectral theory of bilateral shift operators.}%
  \index{shift operator!bilateral}%
  \index{spectral theory!shift operator}%
  \index{Hardy space!bilateral}%
  The bilateral shift $S:\ell^{2}(\mathbb{Z})\to\ell^{2}(\mathbb{Z})$
  defined by $(Sx)[n]=x[n-1]$ is a unitary operator whose spectral
  decomposition is $S=\int_{0}^{2\pi}e^{i\theta}\,dE_{\theta}$.
  The bilateral $z$-transform intertwines $S$ with multiplication by
  $z^{-1}$ on $L^{2}(\mathbb{T})$ (the unit circle), providing the
  concrete spectral representation.  The spectral theorem for unitary
  operators on $\ell^{2}(\mathbb{Z})$ is thus equivalent to the
  Fourier series on $\mathbb{T}$.

\item \textbf{Toeplitz operators and Wiener--Hopf factorisation.}%
  \index{Toeplitz operator}%
  \index{Wiener--Hopf factorisation}%
  \index{Szeg\"o's theorem}%
  \index{Toeplitz determinant}%
  A Toeplitz matrix $T_{N}=(t_{j-k})_{j,k=0}^{N-1}$ has symbol
  $t(z)=\sum_{n=-\infty}^{\infty}t_{n}z^{-n}$, the bilateral
  $z$-transform of the sequence $\{t_{n}\}$.  Szeg\H{o}'s strong limit
  theorem gives $\det T_{N}\sim G^{N}\cdot E$ as $N\to\infty$, where
  $G=\exp\bigl((2\pi)^{-1}\int_{0}^{2\pi}\ln t(e^{i\theta})\,
  d\theta\bigr)$ and the constant $E$ involves the Wiener--Hopf
  factors $t(z)=t_{+}(z)t_{-}(z)$ (splitting into causal and anticausal
  parts).  This asymptotic formula appears in random matrix theory and
  the two-dimensional Ising model.

\item \textbf{Discrete Hilbert transform and analytic signals.}%
  \index{Hilbert transform!discrete}%
  \index{analytic signal!discrete}%
  \index{Hardy space!$H^{2}$ of the disk}%
  The projection of $X(z)=\sum_{n=-\infty}^{\infty}c_{n}z^{-n}$ onto
  the causal part $X_{+}(z)=\sum_{n=0}^{\infty}c_{n}z^{-n}$ is realised
  by the discrete Hilbert transform on the unit circle.  The space of
  causal sequences with finite energy is the Hardy space
  $H^{2}(\mathbb{D})$: functions analytic inside the unit disk with
  square-integrable boundary values.  The Riesz projection theorem
  guarantees that this splitting is bounded on $L^{2}(\mathbb{T})$.
\end{enumerate}

%% -------------------------------------------------------------------
\subsubsection{18.3\quad Unilateral $z$-transform}

The unilateral (or one-sided) $z$-transform restricts the summation to
$n\geq 0$:
\[
  X^{+}(z)=\sum_{n=0}^{\infty}x[n]\,z^{-n}.
\]
This is the standard form in engineering applications, since causal
systems produce output only for $n\geq 0$.  The ROC of $X^{+}(z)$ is
always the exterior of a disk, $|z|>r_{+}$, and $X^{+}(z)\to x[0]$
as $|z|\to\infty$.  The chief advantage over the bilateral transform is
the natural incorporation of initial conditions: the time-advance
property reads
\[
  \mathcal{Z}^{+}\{x[n+1]\}=z\,X^{+}(z)-z\,x[0],
  \qquad
  \mathcal{Z}^{+}\{x[n+2]\}=z^{2}X^{+}(z)-z^{2}x[0]-z\,x[1],
\]
and in general $\mathcal{Z}^{+}\{x[n+k]\}=z^{k}X^{+}(z)
-\sum_{m=0}^{k-1}z^{k-m}x[m]$.  The initial value theorem
$x[0]=\lim_{z\to\infty}X^{+}(z)$ and the final value theorem
$\lim_{n\to\infty}x[n]=\lim_{z\to 1}(z-1)X^{+}(z)$ (when the limit
exists and $(z-1)X^{+}(z)$ has no poles on or outside the unit circle)
are the discrete counterparts of the Laplace-domain initial and final
value theorems.

\paragraph{Physics applications.}
\begin{enumerate}
\item \textbf{Recurrent neural network analysis and training.}%
  \index{recurrent neural network!z-transform@$z$-transform}%
  \index{backpropagation through time}%
  \index{vanishing gradient!z-transform@$z$-transform interpretation}%
  \index{echo state network}%
  A linear recurrent neural network layer with hidden state
  $\mathbf{h}[n]=\mathbf{A}\mathbf{h}[n-1]+\mathbf{B}\mathbf{x}[n]$
  and output $\mathbf{y}[n]=\mathbf{C}\mathbf{h}[n]$ has transfer
  function $\mathbf{H}(z)=\mathbf{C}(z\mathbf{I}-\mathbf{A})^{-1}
  \mathbf{B}$, a matrix-valued rational function.  The eigenvalues of
  $\mathbf{A}$ are the poles; the vanishing gradient problem corresponds
  to $|\lambda_{\max}(\mathbf{A})|<1$ (all poles strictly inside the
  unit disk), which causes exponential decay of gradients during
  backpropagation through time.  Echo state networks operate near the
  ``edge of stability'' $|\lambda_{\max}|\approx 1$ to maintain long
  memory while preserving stability.

\item \textbf{Digital PID controllers and discretisation.}%
  \index{PID controller!discrete}%
  \index{Tustin transformation}%
  \index{anti-windup}%
  \index{z-transform@$z$-transform!control system design}%
  The continuous PID controller $C(s)=K_{p}+K_{i}/s+K_{d}s$ is
  discretised to $C(z)$ via the Tustin (bilinear) transformation
  $s=\frac{2}{T}\frac{z-1}{z+1}$, yielding
  $C(z)=K_{p}+\frac{K_{i}T}{2}\frac{z+1}{z-1}
  +\frac{2K_{d}}{T}\frac{z-1}{z+1}$.
  The unilateral $z$-transform naturally handles the integrator state
  (initial condition of the running sum) and facilitates anti-windup
  analysis by tracking the pole at $z=1$ from the integral term.  Modern
  embedded controllers implement $C(z)$ directly as a difference equation
  in fixed-point arithmetic.

\item \textbf{Kalman filter state estimation.}%
  \index{Kalman filter!z-transform@$z$-transform}%
  \index{state estimation!discrete}%
  \index{Riccati equation!discrete}%
  The discrete-time Kalman filter for the state-space system
  $\mathbf{x}[n+1]=\mathbf{F}\mathbf{x}[n]+\mathbf{G}\mathbf{w}[n]$,
  $\mathbf{y}[n]=\mathbf{H}\mathbf{x}[n]+\mathbf{v}[n]$ produces an
  optimal state estimate $\hat{\mathbf{x}}[n|n]$ whose error dynamics
  are governed by the transfer function
  $(z\mathbf{I}-(\mathbf{I}-\mathbf{K}\mathbf{H})\mathbf{F})^{-1}
  \mathbf{K}$.  The steady-state Kalman gain $\mathbf{K}$ is obtained
  from the discrete algebraic Riccati equation, and stability requires
  that $(\mathbf{I}-\mathbf{K}\mathbf{H})\mathbf{F}$ have all
  eigenvalues inside the unit disk---a spectral condition naturally
  expressed in the $z$-domain.

\item \textbf{Numerical stability of finite-difference schemes.}%
  \index{finite-difference scheme!stability}%
  \index{von Neumann stability analysis}%
  \index{CFL condition}%
  \index{amplification factor}%
  Von Neumann stability analysis of a finite-difference scheme for a PDE
  inserts the Fourier mode $u_{j}^{n}=g^{n}e^{ik j\Delta x}$ into the
  scheme, where $g$ is the amplification factor.  Interpreting $g$ as
  $z$ in the temporal direction, stability requires $|z|\leq 1$ for all
  spatial wave numbers $k$.  For the explicit FTCS scheme applied to the
  diffusion equation, this yields the CFL condition $\alpha=D\Delta
  t/(\Delta x)^{2}\leq 1/2$.  Implicit schemes (Crank--Nicolson) are
  analysed similarly: the resulting $z$-domain transfer function
  $g(k)=(1-\alpha\sin^{2}(k\Delta x/2))/(1+\alpha\sin^{2}(k\Delta x/2))$
  satisfies $|g|\leq 1$ unconditionally.

\item \textbf{Quantum computing: discrete-time quantum walks.}%
  \index{quantum walk!discrete-time}%
  \index{Hadamard coin}%
  \index{z-transform@$z$-transform!quantum walk}%
  A discrete-time quantum walk on the integer lattice uses a coin
  operator $\mathbf{C}$ (e.g., the Hadamard matrix) and a conditional
  shift $\mathbf{S}$, giving the evolution operator $\mathbf{U}=
  \mathbf{S}(\mathbf{C}\otimes\mathbf{I})$.  The $z$-transform of the
  position amplitude $\psi[n,t]$ in the spatial variable~$n$ converts
  the walk dynamics into a matrix recurrence in the time variable,
  whose spectral analysis reveals the ballistic spreading
  $\sigma(t)\sim t$ (contrasting with the diffusive $\sigma\sim\sqrt{t}$
  of the classical random walk).
\end{enumerate}

\paragraph{Mathematics applications.}
\begin{enumerate}
\item \textbf{Probability generating functions and branching processes.}%
  \index{probability generating function}%
  \index{branching process}%
  \index{extinction probability}%
  \index{z-transform@$z$-transform!probability theory}%
  The probability generating function (PGF) of a non-negative
  integer-valued random variable $N$ is $G(z)=\mathbb{E}[z^{N}]
  =\sum_{k=0}^{\infty}p_{k}z^{k}$, which is the unilateral
  $z$-transform of $\{p_{k}\}$ evaluated at $1/z$.  For a
  Galton--Watson branching process with offspring PGF $G(z)$, the
  PGF of the $n$th generation size is the $n$-fold iterate
  $G_{n}(z)=G(G(\cdots G(z)\cdots))$.  The extinction probability
  is the smallest fixed point of $G(z)=z$ in $[0,1]$, and criticality
  ($G'(1)=1$) separates subcritical from supercritical regimes.  Moments
  are computed by differentiation: $\mathbb{E}[N]=G'(1)$,
  $\mathrm{Var}(N)=G''(1)+G'(1)-[G'(1)]^{2}$.

\item \textbf{Operational calculus for difference equations with
  initial conditions.}%
  \index{operational calculus!discrete}%
  \index{initial value problem!difference equations}%
  \index{z-transform@$z$-transform!initial conditions}%
  The unilateral $z$-transform provides an operational calculus for
  linear difference equations exactly analogous to the Laplace transform
  for ODEs (Section~16).  The initial conditions appear explicitly in
  the transformed equation, and the particular solution is obtained by
  algebraic manipulation followed by inversion.  For the second-order
  recurrence $x[n+2]+ax[n+1]+bx[n]=f[n]$ with $x[0]=c_{0}$,
  $x[1]=c_{1}$, the transform yields
  $X^{+}(z)=\frac{F^{+}(z)+(z^{2}+az)c_{0}+zc_{1}}{z^{2}+az+b}$,
  and the solution is recovered by partial fractions and table lookup.

\item \textbf{Moment generating properties and asymptotic enumeration.}%
  \index{moment generating function!discrete}%
  \index{asymptotic enumeration}%
  \index{Darboux's theorem!transfer}%
  For a combinatorial sequence $\{a_{n}\}$ counted by the generating
  function $A(z)=\sum a_{n}z^{n}$, the Darboux transfer theorem
  extracts asymptotics from the behaviour of $A(z)$ near its dominant
  singularity.  If $A(z)\sim C(1-z/\rho)^{-\alpha}$ as $z\to\rho$
  (with $\rho$ the radius of convergence), then
  $a_{n}\sim C\rho^{-n}n^{\alpha-1}/\Gamma(\alpha)$.  This ``transfer''
  from singularity type to coefficient asymptotics is the combinatorial
  analogue of Tauberian theorems for the Laplace transform and is the
  main tool of analytic combinatorics~\cite{FlajoletSedgewick2009}.

\item \textbf{Generating functions for orthogonal polynomial sequences.}%
  \index{orthogonal polynomials!generating function}%
  \index{Chebyshev polynomials!generating function}%
  \index{three-term recurrence}%
  \index{z-transform@$z$-transform!orthogonal polynomials}%
  Classical orthogonal polynomials satisfy three-term recurrences
  $p_{n+1}(x)=(A_{n}x+B_{n})p_{n}(x)-C_{n}p_{n-1}(x)$, and their
  generating functions $G(x,z)=\sum_{n=0}^{\infty}p_{n}(x)z^{n}$ are
  unilateral $z$-transforms in disguise.  For Chebyshev polynomials of
  the first kind, $\sum_{n=0}^{\infty}T_{n}(x)z^{n}
  =(1-xz)/(1-2xz+z^{2})$, a rational function in $z$ whose poles at
  $z=x\pm\sqrt{x^{2}-1}$ encode the asymptotic behaviour of $T_{n}(x)$
  for large $n$.  The $z$-transform framework unifies recurrence
  relations, generating functions, and asymptotic analysis of special
  function sequences catalogued throughout G\&R.

\item \textbf{Discrete Laplace and Borel transforms.}%
  \index{Borel summation!discrete}%
  \index{discrete Laplace transform}%
  \index{formal power series!summability}%
  The unilateral $z$-transform can be viewed as a discrete Laplace
  transform via the substitution $z=e^{s}$:
  $X^{+}(e^{s})=\sum_{n=0}^{\infty}x[n]e^{-ns}$, which is a Dirichlet
  series when $x[n]=a(n)$ is an arithmetic function.  The Borel
  summation method assigns values to divergent formal power series
  $\sum a_{n}z^{n}$ by computing $\sum(a_{n}/n!)z^{n}$ (convergent by
  construction) and then applying a Laplace-type integral.  The discrete
  analogue uses the $z$-transform to resum divergent sequences, connecting
  to the theory of asymptotic series and moment problems.
\end{enumerate}
