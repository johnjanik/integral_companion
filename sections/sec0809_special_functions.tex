%% Sections 8–9 — Special Functions
\section{8--9\quad Special Functions}

\subsection{8.1\quad Elliptic Integrals and Functions}
\subsubsection{8.11\quad Elliptic integrals}
\subsubsection{8.12\quad Functional relations between elliptic integrals}
\subsubsection{8.13\quad Elliptic functions}
\subsubsection{8.14\quad Jacobian elliptic functions}
\subsubsection{8.15\quad Properties of Jacobian elliptic functions and functional relationships between them}
\subsubsection{8.16\quad The Weierstrass function $\wp(u)$}
\subsubsection{8.17\quad The functions $\zeta(u)$ and $\sigma(u)$}
\subsubsection{8.18--8.19\quad Theta functions}

\subsection{8.2\quad The Exponential Integral Function and Functions Generated by It}
\subsubsection{8.21\quad The exponential integral function $\operatorname{Ei}(x)$}
\subsubsection{8.22\quad The hyperbolic sine integral $\operatorname{shi} x$ and the hyperbolic cosine integral $\operatorname{chi} x$}
\subsubsection{8.23\quad The sine integral and the cosine integral: $\operatorname{si} x$ and $\operatorname{ci} x$}
\subsubsection{8.24\quad The logarithm integral $\operatorname{li}(x)$}
\subsubsection{8.25\quad The probability integral $\Phi(x)$, the Fresnel integrals $S(x)$ and $C(x)$, the error function $\operatorname{erf}(x)$, and the complementary error function $\operatorname{erfc}(x)$}
\subsubsection{8.26\quad Lobachevskiy's function $L(x)$}

\subsection{8.3\quad Euler's Integrals of the First and Second Kinds}
\subsubsection{8.31\quad The gamma function (Euler's integral of the second kind): $\Gamma(z)$}
\subsubsection{8.32\quad Representation of the gamma function as series and products}
\subsubsection{8.33\quad Functional relations involving the gamma function}
\subsubsection{8.34\quad The logarithm of the gamma function}
\subsubsection{8.35\quad The incomplete gamma function}
\subsubsection{8.36\quad The psi function $\psi(x)$}
\subsubsection{8.37\quad The function $\beta(x)$}
\subsubsection{8.38\quad The beta function (Euler's integral of the first kind): $\operatorname{B}(x,y)$}
\subsubsection{8.39\quad The incomplete beta function $\operatorname{B}_x(p,q)$}

\subsection{8.4--8.5\quad Bessel Functions and Functions Associated with Them}
\subsubsection{8.40\quad Definitions}
\subsubsection{8.41\quad Integral representations of the functions $J_{\nu}(z)$ and $N_{\nu}(z)$}
\subsubsection{8.42\quad Integral representations of the functions $H_{\nu}^{(1)}(z)$ and $H_{\nu}^{(2)}(z)$}
\subsubsection{8.43\quad Integral representations of the functions $I_{\nu}(z)$ and $K_{\nu}(z)$}
\subsubsection{8.44\quad Series representation}
\subsubsection{8.45\quad Asymptotic expansions of Bessel functions}
\subsubsection{8.46\quad Bessel functions of order equal to an integer plus one-half}
\subsubsection{8.47--8.48\quad Functional relations}
\subsubsection{8.49\quad Differential equations leading to Bessel functions}
\subsubsection{8.51--8.52\quad Series of Bessel functions}
\subsubsection{8.53\quad Expansion in products of Bessel functions}
\subsubsection{8.54\quad The zeros of Bessel functions}
\subsubsection{8.55\quad Struve functions}
\subsubsection{8.56\quad Thomson functions and their generalizations}
\subsubsection{8.57\quad Lommel functions}
\subsubsection{8.58\quad Anger and Weber functions $J_{\nu}(z)$ and $\mathbf{E}_{\nu}(z)$}
\subsubsection{8.59\quad Neumann's and Schl\"afli's polynomials: $O_{n}(z)$ and $S_{n}(z)$}

\subsection{8.6\quad Mathieu Functions}
\subsubsection{8.60\quad Mathieu's equation}
\subsubsection{8.61\quad Periodic Mathieu functions}
\subsubsection{8.62\quad Recursion relations for the coefficients $A_{2r}^{(2n)}$, $A_{2r+1}^{(2n+1)}$, $B_{2r+1}^{(2n+1)}$, $B_{2r+2}^{(2n+2)}$}
\subsubsection{8.63\quad Mathieu functions with a purely imaginary argument}
\subsubsection{8.64\quad Non-periodic solutions of Mathieu's equation}
\subsubsection{8.65\quad Mathieu functions for negative $q$}
\subsubsection{8.66\quad Representation of Mathieu functions as series of Bessel functions}
\subsubsection{8.67\quad The general theory}

\subsection{8.7--8.8\quad Associated Legendre Functions}
\subsubsection{8.70\quad Introduction}
\subsubsection{8.71\quad Integral representations}
\subsubsection{8.72\quad Asymptotic series for large values of $|\nu|$}
\subsubsection{8.73--8.74\quad Functional relations}
\subsubsection{8.75\quad Special cases and particular values}
\subsubsection{8.76\quad Derivatives with respect to the order}
\subsubsection{8.77\quad Series representation}
\subsubsection{8.78\quad The zeros of associated Legendre functions}
\subsubsection{8.79\quad Series of associated Legendre functions}
\subsubsection{8.81\quad Associated Legendre functions with integer indices}
\subsubsection{8.82--8.83\quad Legendre functions}
\subsubsection{8.84\quad Conical functions}
\subsubsection{8.85\quad Toroidal functions}

\subsection{8.9\quad Orthogonal Polynomials}
\subsubsection{8.90\quad Introduction}
\subsubsection{8.91\quad Legendre polynomials}
\subsubsection{8.919\quad Series of products of Legendre and Chebyshev polynomials}
\subsubsection{8.92\quad Series of Legendre polynomials}
\subsubsection{8.93\quad Gegenbauer polynomials $C_{n}^{\lambda}(t)$}
\subsubsection{8.94\quad The Chebyshev polynomials $T_{n}(x)$ and $U_{n}(x)$}
\subsubsection{8.95\quad The Hermite polynomials $H_{n}(x)$}
\subsubsection{8.96\quad Jacobi's polynomials}
\subsubsection{8.97\quad The Laguerre polynomials}

\subsection{9.1\quad Hypergeometric Functions}
\subsubsection{9.10\quad Definition}
\subsubsection{9.11\quad Integral representations}
\subsubsection{9.12\quad Representation of elementary functions in terms of a hypergeometric functions}
\subsubsection{9.13\quad Transformation formulas and the analytic continuation of functions defined by hypergeometric series}
\subsubsection{9.14\quad A generalized hypergeometric series}
\subsubsection{9.15\quad The hypergeometric differential equation}
\subsubsection{9.16\quad Riemann's differential equation}
\subsubsection{9.17\quad Representing the solutions to certain second-order differential equations using a Riemann scheme}
\subsubsection{9.18\quad Hypergeometric functions of two variables}
\subsubsection{9.19\quad A hypergeometric function of several variables}

\subsection{9.2\quad Confluent Hypergeometric Functions}
\subsubsection{9.20\quad Introduction}
\subsubsection{9.21\quad The functions $\Phi(\alpha,\gamma;z)$ and $\Psi(\alpha,\gamma;z)$}
\subsubsection{9.22--9.23\quad The Whittaker functions $M_{\lambda,\mu}(z)$ and $W_{\lambda,\mu}(z)$}
\subsubsection{9.24--9.25\quad Parabolic cylinder functions $D_{p}(z)$}
\subsubsection{9.26\quad Confluent hypergeometric series of two variables}

\subsection{9.3\quad Meijer's $G$-Function}
\subsubsection{9.30\quad Definition}
\subsubsection{9.31\quad Functional relations}
\subsubsection{9.32\quad A differential equation for the G-function}
\subsubsection{9.33\quad Series of G-functions}
\subsubsection{9.34\quad Connections with other special functions}

\subsection{9.4\quad MacRobert's $E$-Function}
\subsubsection{9.41\quad Representation by means of multiple integrals}
\subsubsection{9.42\quad Functional relations}

\subsection{9.5\quad Riemann's Zeta Functions $\zeta(z,q)$ and $\zeta(z)$, and the Functions $\Phi(z,s,v)$ and $\xi(s)$}
\subsubsection{9.51\quad Definition and integral representations}
\subsubsection{9.52\quad Representation as a series or as an infinite product}
\subsubsection{9.53\quad Functional relations}
\subsubsection{9.54\quad Singular points and zeros}
\subsubsection{9.55\quad The Lerch function $\Phi(z,s,v)$}
\subsubsection{9.56\quad The function $\xi(s)$}

\subsection{9.6\quad Bernoulli Numbers and Polynomials, Euler Numbers}
\subsubsection{9.61\quad Bernoulli numbers}
\subsubsection{9.62\quad Bernoulli polynomials}
\subsubsection{9.63\quad Euler numbers}
\subsubsection{9.64\quad The functions $\nu(x)$, $\nu(x,\alpha)$, $\mu(x,\beta)$, $\mu(x,\beta,\alpha)$, and $\lambda(x,y)$}
\subsubsection{9.65\quad Euler polynomials}

\subsection{9.7\quad Constants}
\subsubsection{9.71\quad Bernoulli numbers}
\subsubsection{9.72\quad Euler numbers}
\subsubsection{9.73\quad Euler's and Catalan's constants}
\subsubsection{9.74\quad Stirling numbers}
