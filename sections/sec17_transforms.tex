%% Section 17 — Fourier, Laplace, and Mellin Transforms
\section{17\quad Fourier, Laplace, and Mellin Transforms}

\subsection{17.1--17.4\quad Integral Transforms}

%% -------------------------------------------------------------------
\subsubsection{17.11\quad Laplace transform}

The Laplace transform converts a function $f(t)$ defined for $t\geq 0$ into
a function of a complex variable $s$ via
$\mathcal{L}\{f\}(s)=F(s)=\int_{0}^{\infty}e^{-st}f(t)\,dt$.
The integral converges in a half-plane $\operatorname{Re}s>\sigma_{0}$,
where $\sigma_{0}$ is the abscissa of convergence.  The inverse transform
is given by the Bromwich integral
$f(t)=\frac{1}{2\pi i}\int_{\gamma-i\infty}^{\gamma+i\infty}e^{st}F(s)\,ds$
along a vertical contour in the region of convergence.  The Laplace
transform is the principal tool for reducing linear ordinary differential
equations with constant coefficients to algebraic equations, and for
analysing the stability and transient response of dynamical systems.

\paragraph{Physics applications.}
\begin{enumerate}
\item \textbf{Control theory and transfer functions.}%
  \index{Laplace transform!control theory}%
  \index{transfer function}%
  \index{feedback systems!stability}%
  \index{Bode plot}%
  The Laplace transform converts a linear time-invariant system
  $\sum a_{k}y^{(k)}=\sum b_{k}u^{(k)}$ into the transfer function
  $H(s)=Y(s)/U(s)=B(s)/A(s)$, a rational function of~$s$ whose poles
  determine stability (all poles in $\operatorname{Re}s<0$) and whose
  frequency response $H(i\omega)$ is displayed in Bode and Nyquist plots.
  The entire classical theory of PID control, root locus, and state-space
  methods rests on this transformation.

\item \textbf{Circuit analysis and impedance.}%
  \index{Laplace transform!circuit analysis}%
  \index{impedance!Laplace domain}%
  \index{RLC circuits}%
  In the $s$-domain, resistors have impedance $R$, capacitors $1/(sC)$,
  and inductors $sL$.  Kirchhoff's laws become algebraic equations in~$s$,
  and the transient response to an arbitrary input is obtained by
  partial-fraction expansion and inverse transformation.  The natural
  frequencies of an RLC circuit are the poles of the impedance function.

\item \textbf{Radioactive decay chains.}%
  \index{radioactive decay!Laplace transform}%
  \index{Bateman equations}%
  \index{nuclear physics!decay chains}%
  The Bateman equations $dN_{i}/dt=-\lambda_{i}N_{i}+\lambda_{i-1}N_{i-1}$
  for a decay chain $A\to B\to C\to\cdots$ are solved by Laplace
  transform: $N_{i}(s)$ involves partial fractions with poles at
  $s=-\lambda_{k}$, and the inverse transform gives the classic Bateman
  solution as a sum of exponentials.

\item \textbf{Viscoelasticity and the standard linear solid.}%
  \index{viscoelasticity!Laplace transform}%
  \index{creep compliance}%
  \index{relaxation modulus}%
  \index{standard linear solid}%
  The constitutive equations of linear viscoelasticity (Maxwell, Kelvin--Voigt,
  standard linear solid) become algebraic relations between stress and
  strain in the Laplace domain.  The creep compliance $J(t)$ and relaxation
  modulus $G(t)$ are related by $\hat{J}(s)\hat{G}(s)=1/s^{2}$, a simple
  algebraic identity in the $s$-domain that is a convolution equation in
  the time domain.

\item \textbf{Moment generating functions and probability.}%
  \index{moment generating function}%
  \index{Laplace transform!probability}%
  \index{exponential distribution!Laplace transform}%
  The moment generating function $M_{X}(t)=\mathbb{E}[e^{tX}]$ is
  essentially the two-sided Laplace transform of the probability density
  evaluated at $-t$.  For non-negative random variables, $M_{X}(-s)$
  is the Laplace transform of the density.  Moments are recovered as
  $\mathbb{E}[X^{n}]=M_{X}^{(n)}(0)=(-1)^{n}F^{(n)}(0)$.  The
  convolution theorem then proves that the moment generating function of
  a sum of independent random variables is the product of individual
  moment generating functions.
\end{enumerate}

\paragraph{Mathematics applications.}
\begin{enumerate}
\item \textbf{Operational calculus and the Heaviside method.}%
  \index{operational calculus}%
  \index{Heaviside!operational method}%
  \index{Mikusinski operational calculus}%
  Heaviside's operational calculus---treating $d/dt$ as an algebraic
  quantity $p$---is made rigorous by the Laplace transform: $p$
  becomes~$s$, and the operational rules (partial fractions, expansion
  theorems) follow from the properties of the transform.  Mikusi\'{n}ski's
  algebraic approach constructs a field of operators by Cauchy quotients
  of convolution rings, providing an alternative rigorous foundation.

\item \textbf{Tauberian theorems and asymptotic analysis.}%
  \index{Tauberian theorems!Laplace transform}%
  \index{asymptotic analysis!Laplace transform}%
  \index{Abelian theorems}%
  \index{Karamata's Tauberian theorem}%
  Abelian theorems relate the behaviour of $f(t)$ as $t\to\infty$ to that
  of $F(s)$ as $s\to 0^{+}$, and Tauberian theorems provide the converse
  under regularity conditions.  Karamata's Tauberian theorem is fundamental
  in analytic number theory and probability: if $F(s)\sim s^{-\rho}L(1/s)$
  with $L$ slowly varying, then
  $\int_{0}^{t}f(u)\,du\sim t^{\rho}L(t)/\Gamma(\rho+1)$.

\item \textbf{Uniqueness and the Lerch--Widder theorem.}%
  \index{Lerch's theorem!uniqueness}%
  \index{Widder's theorem}%
  \index{completely monotone functions}%
  Lerch's theorem guarantees that the Laplace transform is injective on
  functions continuous almost everywhere: if $F(s)=G(s)$ for all
  $\operatorname{Re}s>\sigma_{0}$, then $f=g$ a.e.  Widder's theorem
  characterises completely monotone functions as Laplace transforms of
  non-negative measures, connecting to Bernstein functions and
  L\'{e}vy processes.

\item \textbf{Laplace transform and the resolvent of semigroups.}%
  \index{operator semigroup!Laplace transform}%
  \index{resolvent operator}%
  \index{Hille--Yosida theorem}%
  For a strongly continuous semigroup $\{T(t)\}_{t\geq 0}$ on a Banach
  space, the resolvent $(sI-A)^{-1}=\int_{0}^{\infty}e^{-st}T(t)\,dt$
  is the Laplace transform of the semigroup.  The Hille--Yosida theorem
  characterises the generators of such semigroups through growth conditions
  on the resolvent, forming the mathematical backbone of evolution equations
  in PDEs and stochastic processes.
\end{enumerate}

%% -------------------------------------------------------------------
\subsubsection{17.12\quad Basic properties of the Laplace transform}

The operational properties of the Laplace transform---linearity, shifting,
scaling, differentiation, and integration rules---convert differential and
integral equations into algebraic ones.  The key properties are:
differentiation becomes multiplication ($\mathcal{L}\{f'\}=sF(s)-f(0)$),
convolution becomes multiplication
($\mathcal{L}\{f*g\}=F(s)G(s)$), and time delay becomes
exponential multiplication
($\mathcal{L}\{f(t-a)u(t-a)\}=e^{-as}F(s)$).  These properties, listed
in G\&R~17.12, are the foundation of every engineering application of
the Laplace transform.

\paragraph{Physics applications.}
\begin{enumerate}
\item \textbf{Initial and final value theorems in control systems.}%
  \index{initial value theorem}%
  \index{final value theorem}%
  \index{steady-state response}%
  \index{step response!final value}%
  The initial value theorem $f(0^{+})=\lim_{s\to\infty}sF(s)$ and
  final value theorem $\lim_{t\to\infty}f(t)=\lim_{s\to 0}sF(s)$ (when
  the limit exists) allow direct extraction of transient and steady-state
  behaviour from the $s$-domain representation without inverting the
  transform.  These are used routinely to check step-response settling
  values and initial jumps in control engineering.

\item \textbf{Convolution and linear system response.}%
  \index{convolution!Laplace transform}%
  \index{impulse response!convolution}%
  \index{Green's function!Laplace transform}%
  The output of a linear time-invariant system is
  $y(t)=(h*u)(t)=\int_{0}^{t}h(t-\tau)u(\tau)\,d\tau$, where $h$ is the
  impulse response.  The convolution theorem transforms this to
  $Y(s)=H(s)U(s)$, reducing the computation of system response to
  multiplication.  The impulse response $h(t)$ is itself the Green's function
  of the differential operator.

\item \textbf{Differentiation rule and the $s$-domain ODE.}%
  \index{differentiation!Laplace transform}%
  \index{ODE!Laplace transform solution}%
  \index{initial conditions!Laplace transform}%
  The rule $\mathcal{L}\{f^{(n)}\}=s^{n}F(s)-\sum_{k=0}^{n-1}s^{n-1-k}f^{(k)}(0)$
  automatically incorporates initial conditions into the algebraic
  equation.  For a second-order system
  $my''+cy'+ky=f(t)$, the transform gives
  $(ms^{2}+cs+k)Y(s)=F(s)+(\text{initial conditions})$,
  solved by partial fractions.

\item \textbf{$s$-shifting and damped oscillations.}%
  \index{$s$-shifting!Laplace transform}%
  \index{damped oscillations}%
  \index{frequency shifting}%
  \index{modulation theorem}%
  The $s$-shifting property $\mathcal{L}\{e^{at}f(t)\}=F(s-a)$ shifts
  poles in the $s$-plane.  A damped sinusoid
  $e^{-\alpha t}\sin(\omega t)$ has transform
  $\omega/((s+\alpha)^{2}+\omega^{2})$, with poles at
  $s=-\alpha\pm i\omega$ encoding both the damping rate and natural
  frequency directly in the pole locations.

\item \textbf{Integration rule and cumulative response.}%
  \index{integration!Laplace transform}%
  \index{integral equations!Laplace transform}%
  \index{Abel integral equation}%
  The rule $\mathcal{L}\{\int_{0}^{t}f(\tau)\,d\tau\}=F(s)/s$ converts
  Volterra integral equations of convolution type into algebraic equations.
  The Abel integral equation $g(x)=\int_{0}^{x}f(t)(x-t)^{-1/2}\,dt$
  is solved by Laplace transform using
  $G(s)=F(s)\cdot\Gamma(1/2)/s^{1/2}$, yielding
  $f(t)=(1/\pi)\,d/dt\int_{0}^{t}g(\tau)(t-\tau)^{-1/2}\,d\tau$.
\end{enumerate}

\paragraph{Mathematics applications.}
\begin{enumerate}
\item \textbf{Convolution algebras and Banach algebras.}%
  \index{convolution algebra}%
  \index{Banach algebra!$L^{1}$}%
  \index{Titchmarsh convolution theorem}%
  The space $L^{1}(\mathbb{R}^{+})$ with convolution as multiplication
  forms a commutative Banach algebra without identity.  The Laplace
  transform is a homomorphism from this algebra to an algebra of analytic
  functions.  The Titchmarsh convolution theorem states that if $f*g=0$
  on $[0,T]$, then $f=0$ on $[0,a]$ and $g=0$ on $[0,b]$ for some
  $a+b\geq T$.

\item \textbf{Generating functions and combinatorics.}%
  \index{generating functions!Laplace connection}%
  \index{exponential generating function}%
  \index{Borel summation}%
  The exponential generating function
  $\hat{f}(z)=\sum a_{n}z^{n}/n!$ is the Borel sum associated with
  the formal power series $\sum a_{n}z^{n}$.  The Borel summation method
  uses the Laplace transform to assign values to divergent series:
  $\sum a_{n}z^{n}\to\int_{0}^{\infty}e^{-t}\hat{f}(tz)\,dt$, connecting
  asymptotic analysis to the theory of integral transforms.

\item \textbf{Stieltjes transform and moment problems.}%
  \index{Stieltjes transform}%
  \index{moment problem!Stieltjes}%
  \index{continued fractions!Stieltjes transform}%
  The Stieltjes transform $S(s)=\int_{0}^{\infty}f(t)/(s+t)\,dt$ is the
  iterated Laplace transform: $S(s)=\mathcal{L}\{\mathcal{L}\{f\}\}(s)$.
  It arises in the Stieltjes moment problem---determining a measure from
  its moments $\mu_{n}=\int t^{n}d\mu(t)$---and is connected to
  continued fraction expansions of analytic functions.
\end{enumerate}

%% -------------------------------------------------------------------
\subsubsection{17.13\quad Table of Laplace transform pairs}

The table of Laplace transform pairs in G\&R~17.13 collects the standard
correspondences between time-domain functions and their $s$-domain
representations.  The most fundamental pairs include the exponential
$e^{at}\leftrightarrow 1/(s-a)$, the power function
$t^{n}\leftrightarrow n!/s^{n+1}$, and the damped sinusoids
$e^{at}\sin(bt)\leftrightarrow b/((s-a)^{2}+b^{2})$.
The table also contains transforms of special functions: the Bessel function
$J_{0}(at)\leftrightarrow 1/\sqrt{s^{2}+a^{2}}$, the error function
$\operatorname{erf}(a/\sqrt{t})\leftrightarrow e^{-a\sqrt{s}}/s$, and
the Heaviside step function $u(t-a)\leftrightarrow e^{-as}/s$.

\paragraph{Physics applications.}
\begin{enumerate}
\item \textbf{Inverse square root and diffusion.}%
  \index{Laplace transform pairs!diffusion}%
  \index{heat kernel!Laplace transform}%
  \index{diffusion equation!Green's function}%
  The pair $1/\sqrt{\pi t}\leftrightarrow 1/\sqrt{s}$ is the fundamental
  solution of the diffusion equation.  More generally,
  $t^{\alpha-1}/\Gamma(\alpha)\leftrightarrow s^{-\alpha}$ (for
  $\alpha>0$) underlies fractional calculus and anomalous diffusion
  processes where the mean square displacement grows as $t^{\alpha}$
  rather than linearly.

\item \textbf{Bessel function pairs and wave propagation.}%
  \index{Bessel function!Laplace transform}%
  \index{wave propagation!cylindrical}%
  \index{Laplace transform pairs!Bessel}%
  The transforms of Bessel functions ($J_{\nu}$, $I_{\nu}$, $K_{\nu}$)
  appear in cylindrical wave propagation, heat conduction in cylinders,
  and the sommerfeld integral for antenna radiation.  The pair
  $J_{0}(at)\leftrightarrow(s^{2}+a^{2})^{-1/2}$ is the starting point
  for the Hankel transform via the Fourier--Bessel connection.

\item \textbf{Rational function pairs and electrical engineering.}%
  \index{Laplace transform pairs!rational functions}%
  \index{partial fractions!circuit analysis}%
  \index{poles and zeros}%
  The rational pairs $1/(s-a)^{n}\leftrightarrow t^{n-1}e^{at}/(n-1)!$
  are the backbone of circuit analysis.  Every rational transfer function
  decomposes into partial fractions of this form, and the inverse transform
  is read off the table.  Complex pole pairs give damped oscillations;
  repeated poles give polynomial-times-exponential transients.
\end{enumerate}

\paragraph{Mathematics applications.}
\begin{enumerate}
\item \textbf{Mittag-Leffler function and fractional calculus.}%
  \index{Mittag-Leffler function!Laplace transform}%
  \index{fractional calculus!Laplace transform}%
  \index{Laplace transform pairs!Mittag-Leffler}%
  The Mittag-Leffler function
  $E_{\alpha,\beta}(z)=\sum_{k=0}^{\infty}z^{k}/\Gamma(\alpha k+\beta)$
  has Laplace transform
  $\mathcal{L}\{t^{\beta-1}E_{\alpha,\beta}(\lambda t^{\alpha})\}
  =s^{\alpha-\beta}/(s^{\alpha}-\lambda)$.  This generalises the
  exponential pair $e^{\lambda t}\leftrightarrow 1/(s-\lambda)$ and is
  the key to solving fractional differential equations.

\item \textbf{Bernstein's theorem and completely monotone functions.}%
  \index{Bernstein's theorem!Laplace transform}%
  \index{completely monotone functions!characterisation}%
  A function $f$ on $(0,\infty)$ is completely monotone
  ($(-1)^{n}f^{(n)}\geq 0$ for all $n$) if and only if it is the Laplace
  transform of a non-negative measure: $f(s)=\int_{0}^{\infty}e^{-st}\,d\mu(t)$.
  This characterisation is used in probability (infinitely divisible
  distributions) and in harmonic analysis on semigroups.
\end{enumerate}

%% -------------------------------------------------------------------
\subsubsection{17.21\quad Fourier transform}

The Fourier transform decomposes a function into its frequency components:
$\hat{f}(\omega)=\mathcal{F}\{f\}(\omega)
=\int_{-\infty}^{\infty}f(t)\,e^{-i\omega t}\,dt$,
with inverse $f(t)=\frac{1}{2\pi}\int_{-\infty}^{\infty}\hat{f}(\omega)\,e^{i\omega t}\,d\omega$.
The transform exists for $f\in L^{1}(\mathbb{R})$ and extends by density
to $L^{2}(\mathbb{R})$ as a unitary operator (Plancherel theorem).
Different normalisation conventions exist in the literature---G\&R uses the
asymmetric convention with the $1/(2\pi)$ factor on the inverse.  The
Fourier transform is arguably the single most important tool in mathematical
physics, signal processing, and harmonic analysis.

\paragraph{Physics applications.}
\begin{enumerate}
\item \textbf{CT reconstruction and the Fourier slice theorem.}%
  \index{Fourier transform!CT reconstruction}%
  \index{Fourier slice theorem}%
  \index{Radon transform!Fourier slice}%
  \index{computed tomography}%
  \index{filtered back-projection}%
  The Fourier slice theorem (central slice theorem) states that the
  one-dimensional Fourier transform of a parallel-beam projection of a
  two-dimensional object at angle $\theta$ equals a slice through the
  two-dimensional Fourier transform at the same angle:
  $\hat{P}_{\theta}(\omega)=\hat{f}(\omega\cos\theta,\omega\sin\theta)$.
  This is the mathematical foundation of computed tomography (CT):
  filtered back-projection reconstructs $f(x,y)$ by collecting projections
  at many angles and applying the inverse Fourier transform, enabling
  medical imaging that won Cormack and Hounsfield the 1979 Nobel Prize.

\item \textbf{X-ray crystallography and structure determination.}%
  \index{X-ray crystallography}%
  \index{Bragg diffraction}%
  \index{structure factor}%
  \index{phase problem!crystallography}%
  The diffraction pattern of a crystal is the squared modulus of the
  Fourier transform of the electron density:
  $I(\mathbf{k})\propto|\hat{\rho}(\mathbf{k})|^{2}$.
  The structure factor $F(\mathbf{h})=\sum_{j}f_{j}e^{2\pi i\mathbf{h}\cdot\mathbf{r}_{j}}$
  is a discrete Fourier transform over the unit cell.  The phase
  problem---recovering $\hat{\rho}(\mathbf{k})$ from $|\hat{\rho}(\mathbf{k})|$
  alone---is the central challenge, solved by Patterson methods, direct
  methods, and molecular replacement.

\item \textbf{Quantum mechanics and the momentum representation.}%
  \index{Fourier transform!quantum mechanics}%
  \index{momentum representation}%
  \index{uncertainty principle!Fourier}%
  \index{wave packet}%
  The position and momentum representations of a quantum state are
  related by the Fourier transform:
  $\tilde{\psi}(p)=\frac{1}{\sqrt{2\pi\hbar}}\int\psi(x)e^{-ipx/\hbar}\,dx$.
  The Heisenberg uncertainty principle $\Delta x\,\Delta p\geq\hbar/2$
  is a direct consequence of the Fourier uncertainty relation: a function
  and its Fourier transform cannot both be sharply localised.

\item \textbf{Signal processing and the sampling theorem.}%
  \index{sampling theorem!Nyquist--Shannon}%
  \index{Fourier transform!signal processing}%
  \index{aliasing}%
  \index{bandlimited signals}%
  A bandlimited signal with $\hat{f}(\omega)=0$ for $|\omega|>\Omega$
  is completely determined by samples at rate $2\Omega$ (Nyquist--Shannon
  sampling theorem).  Aliasing occurs when the sampling rate is
  insufficient, folding high-frequency components into lower frequencies.
  The entire theory of digital signal processing---filtering, spectral
  analysis, windowing---rests on the Fourier transform.

\item \textbf{Optics and Fraunhofer diffraction.}%
  \index{Fraunhofer diffraction}%
  \index{optical Fourier transform}%
  \index{diffraction pattern}%
  The far-field (Fraunhofer) diffraction pattern of an aperture is the
  Fourier transform of the aperture function: for an aperture $a(x,y)$,
  the field at the screen is
  $U(\xi,\eta)\propto\iint a(x,y)e^{-i(k_{x}x+k_{y}y)}\,dx\,dy$.
  A lens performs an optical Fourier transform in its focal plane, a
  principle exploited in spatial filtering and holography.
\end{enumerate}

\paragraph{Mathematics applications.}
\begin{enumerate}
\item \textbf{Harmonic analysis on locally compact abelian groups.}%
  \index{harmonic analysis!locally compact groups}%
  \index{Pontryagin duality}%
  \index{character group}%
  The Fourier transform on $\mathbb{R}$ is a special case of the
  Pontryagin duality theory: for any locally compact abelian group $G$,
  $\hat{f}(\chi)=\int_{G}f(g)\overline{\chi(g)}\,dg$ transforms
  functions on $G$ to functions on the dual group $\hat{G}$.
  For $G=\mathbb{R}$, $\hat{G}\cong\mathbb{R}$; for $G=\mathbb{Z}$,
  $\hat{G}\cong\mathbb{T}$ (Fourier series); for $G=\mathbb{Z}/N\mathbb{Z}$,
  $\hat{G}\cong\mathbb{Z}/N\mathbb{Z}$ (DFT).

\item \textbf{Schwartz space and tempered distributions.}%
  \index{Schwartz space}%
  \index{tempered distributions}%
  \index{Fourier transform!distributions}%
  The Fourier transform is an automorphism of the Schwartz space
  $\mathcal{S}(\mathbb{R})$ of rapidly decreasing functions, and extends
  by duality to tempered distributions $\mathcal{S}'(\mathbb{R})$.
  This gives rigorous meaning to $\mathcal{F}\{\delta\}=1$,
  $\mathcal{F}\{1\}=2\pi\delta$, and the transforms of polynomials,
  essential in PDE theory and quantum field theory.

\item \textbf{Paley--Wiener theorem and analyticity.}%
  \index{Paley--Wiener theorem}%
  \index{entire functions of exponential type}%
  \index{support!Fourier transform}%
  The Paley--Wiener theorem characterises the Fourier transforms of
  compactly supported distributions as entire functions of exponential
  type: $f$ is supported in $[-a,a]$ if and only if $\hat{f}$ extends
  to an entire function with $|\hat{f}(z)|\leq Ce^{a|\operatorname{Im}z|}$.
  This connects the support of a function to the growth rate of its
  analytic continuation.

\item \textbf{Fourier analysis and PDEs.}%
  \index{Fourier transform!PDE solution}%
  \index{heat equation!Fourier solution}%
  \index{dispersion relation}%
  The Fourier transform converts constant-coefficient PDEs to algebraic
  (or ODE) problems in the frequency variable.  The heat equation
  $u_{t}=\alpha u_{xx}$ transforms to $\hat{u}_{t}=-\alpha\omega^{2}\hat{u}$,
  giving $\hat{u}(\omega,t)=\hat{u}_{0}(\omega)e^{-\alpha\omega^{2}t}$
  and the Gaussian heat kernel by inverse transform.  The dispersion
  relation $\omega(k)$ for a wave equation determines the group and phase
  velocities directly in Fourier space.
\end{enumerate}

%% -------------------------------------------------------------------
\subsubsection{17.22\quad Basic properties of the Fourier transform}

The basic properties of the Fourier transform include linearity,
the shift theorem $\mathcal{F}\{f(t-a)\}=e^{-ia\omega}\hat{f}(\omega)$,
the modulation theorem
$\mathcal{F}\{e^{i\omega_{0}t}f(t)\}=\hat{f}(\omega-\omega_{0})$,
the scaling property $\mathcal{F}\{f(at)\}=|\!a\!|^{-1}\hat{f}(\omega/a)$,
the convolution theorem
$\mathcal{F}\{f*g\}=\hat{f}\cdot\hat{g}$, and Parseval's relation
$\int|f|^{2}\,dt=(2\pi)^{-1}\int|\hat{f}|^{2}\,d\omega$.
The differentiation property $\mathcal{F}\{f'\}=i\omega\hat{f}$ converts
differential operators to polynomial multiplication, the central mechanism
for solving PDEs via Fourier methods.

\paragraph{Physics applications.}
\begin{enumerate}
\item \textbf{Convolution theorem and linear filtering.}%
  \index{convolution theorem!Fourier}%
  \index{linear filter!frequency domain}%
  \index{low-pass filter}%
  \index{transfer function!Fourier domain}%
  The convolution theorem $\mathcal{F}\{f*g\}=\hat{f}\hat{g}$ is the
  foundation of linear filtering: a filter with impulse response $h(t)$
  multiplies the input spectrum by the frequency response $\hat{h}(\omega)$.
  Low-pass, high-pass, and band-pass filters are designed by specifying
  $\hat{h}(\omega)$, and the output is computed by inverse Fourier
  transform.  The Fast Fourier Transform (FFT) makes this convolution
  computationally efficient at $O(N\log N)$ cost.

\item \textbf{Parseval's theorem and energy spectral density.}%
  \index{Parseval's theorem!energy}%
  \index{energy spectral density}%
  \index{power spectrum}%
  \index{Wiener--Khinchin theorem}%
  Parseval's relation $\int|f(t)|^{2}\,dt=\frac{1}{2\pi}\int|\hat{f}(\omega)|^{2}\,d\omega$
  states that the total energy is the same in time and frequency domains.
  The energy spectral density $|\hat{f}(\omega)|^{2}$ describes how
  energy is distributed across frequencies.  For stationary random
  processes, the Wiener--Khinchin theorem identifies the power spectral
  density as the Fourier transform of the autocorrelation function.

\item \textbf{Time-frequency duality and the uncertainty principle.}%
  \index{uncertainty principle!time-frequency}%
  \index{Gabor limit}%
  \index{time-frequency analysis}%
  \index{short-time Fourier transform}%
  The scaling property shows that compressing a signal in time expands it
  in frequency and vice versa: $\Delta t\,\Delta\omega\geq 1/2$ (Gabor
  limit).  This fundamental trade-off governs radar pulse design, musical
  note resolution, and spectroscopic line widths.  The short-time Fourier
  transform $\text{STFT}(t,\omega)=\int f(\tau)w(\tau-t)e^{-i\omega\tau}\,d\tau$
  provides a compromise by windowing.

\item \textbf{Differentiation property and spectral methods for PDEs.}%
  \index{spectral methods!PDE}%
  \index{pseudospectral method}%
  \index{Fourier differentiation}%
  Since $\mathcal{F}\{f^{(n)}\}=(i\omega)^{n}\hat{f}$, derivatives in
  physical space become multiplications in Fourier space.  Pseudospectral
  methods compute spatial derivatives via FFT, multiply by $(i\omega)^{n}$,
  and transform back, achieving exponential convergence for smooth
  solutions.  This is the standard approach in direct numerical simulation
  of turbulence and weather prediction.
\end{enumerate}

\paragraph{Mathematics applications.}
\begin{enumerate}
\item \textbf{Plancherel theorem and $L^{2}$ isometry.}%
  \index{Plancherel theorem}%
  \index{$L^{2}$ isometry!Fourier}%
  \index{Fourier transform!$L^{2}$ extension}%
  The Fourier transform extends from $L^{1}\cap L^{2}$ to a unitary
  isomorphism $\mathcal{F}:L^{2}(\mathbb{R})\to L^{2}(\mathbb{R})$
  (Plancherel theorem).  This is the rigorous statement of Parseval's
  relation and is the cornerstone of $L^{2}$ harmonic analysis.

\item \textbf{Young's inequality and convolution estimates.}%
  \index{Young's inequality!convolution}%
  \index{Hausdorff--Young inequality}%
  \index{$L^{p}$ spaces!convolution}%
  Young's inequality $\|f*g\|_{r}\leq\|f\|_{p}\|g\|_{q}$ (with
  $1/p+1/q=1+1/r$) controls the $L^{r}$ norm of a convolution.  The
  Hausdorff--Young inequality $\|\hat{f}\|_{p'}\leq\|f\|_{p}$ for
  $1\leq p\leq 2$ (with $1/p+1/p'=1$) gives the sharp mapping
  properties of the Fourier transform between $L^{p}$ spaces, fundamental
  in PDE regularity theory.

\item \textbf{Poisson summation formula.}%
  \index{Poisson summation formula}%
  \index{sampling!Poisson summation}%
  \index{theta function!Poisson summation}%
  The Poisson summation formula
  $\sum_{n\in\mathbb{Z}}f(n)=\sum_{n\in\mathbb{Z}}\hat{f}(2\pi n)$
  connects sampling in the time domain to periodisation in the frequency
  domain.  It is the tool behind the functional equation of the Jacobi
  theta function $\theta(t)=\sum e^{-\pi n^{2}t}$, which in turn yields
  the functional equation of the Riemann zeta function.
\end{enumerate}

%% -------------------------------------------------------------------
\subsubsection{17.23\quad Table of Fourier transform pairs}

The table of Fourier transform pairs in G\&R~17.23 provides the standard
dictionary for converting between time/space domain functions and their
frequency representations.  The most fundamental entries are the Gaussian
$e^{-at^{2}}\leftrightarrow\sqrt{\pi/a}\,e^{-\omega^{2}/(4a)}$ (the
Gaussian is its own Fourier transform up to scaling), the rectangular
pulse $\operatorname{rect}(t)\leftrightarrow\operatorname{sinc}(\omega/2)$,
the exponential decay $e^{-a|t|}\leftrightarrow 2a/(a^{2}+\omega^{2})$
(Lorentzian), and the delta function
$\delta(t)\leftrightarrow 1$.

\paragraph{Physics applications.}
\begin{enumerate}
\item \textbf{Gaussian wave packets and minimum uncertainty.}%
  \index{Gaussian!Fourier transform pair}%
  \index{minimum uncertainty state}%
  \index{coherent state!Gaussian}%
  The Gaussian pair $e^{-t^{2}/2\sigma^{2}}\leftrightarrow
  \sigma\sqrt{2\pi}\,e^{-\sigma^{2}\omega^{2}/2}$ saturates the
  uncertainty inequality $\Delta t\,\Delta\omega=1/2$.  In quantum
  mechanics, Gaussian wave packets are the coherent states of the
  harmonic oscillator, and in optics, Gaussian beams are the fundamental
  modes of laser cavities.

\item \textbf{Lorentzian line shape and resonance.}%
  \index{Lorentzian!Fourier transform}%
  \index{spectral line shape}%
  \index{Breit--Wigner distribution}%
  The pair $e^{-\gamma|t|}\leftrightarrow 2\gamma/(\gamma^{2}+\omega^{2})$
  gives the Lorentzian spectral line shape, characteristic of damped
  harmonic oscillators and resonance phenomena.  In nuclear and particle
  physics, the Breit--Wigner distribution $|1/(E-E_{0}+i\Gamma/2)|^{2}$
  describes unstable particle resonances.

\item \textbf{Sinc function and ideal filters.}%
  \index{sinc function!Fourier pair}%
  \index{ideal filter!sinc reconstruction}%
  \index{Gibbs phenomenon}%
  The pair $\operatorname{sinc}(\pi t)\leftrightarrow\operatorname{rect}(\omega/2\pi)$
  shows that an ideal low-pass filter has a sinc impulse response.  The
  non-causal and slowly decaying nature of the sinc function means that
  ideal filters are unrealisable; the Gibbs phenomenon (9\% overshoot at
  discontinuities) is the manifestation in partial Fourier sums.
\end{enumerate}

\paragraph{Mathematics applications.}
\begin{enumerate}
\item \textbf{Schwartz functions as Fourier eigenfunctions.}%
  \index{Fourier eigenfunctions}%
  \index{Hermite functions!Fourier eigenfunctions}%
  \index{Schwartz space!Fourier stability}%
  The Hermite functions $h_{n}(t)=H_{n}(t)e^{-t^{2}/2}$ are
  eigenfunctions of the Fourier transform with eigenvalues $(-i)^{n}$.
  The Gaussian $h_{0}(t)=e^{-t^{2}/2}$ is the unique
  $L^{2}$-normalised eigenfunction with eigenvalue~1.  The Hermite
  expansion provides the spectral decomposition of the Fourier transform
  as an operator on $L^{2}$.

\item \textbf{Characteristic functions and probability.}%
  \index{characteristic function!probability}%
  \index{L\'evy continuity theorem}%
  \index{central limit theorem!Fourier proof}%
  The characteristic function $\varphi_{X}(\omega)=\mathbb{E}[e^{i\omega X}]
  =\hat{f}(-\omega)$ is the Fourier transform of the probability density.
  The L\'{e}vy continuity theorem states that convergence of
  characteristic functions implies convergence in distribution, providing
  the standard proof of the central limit theorem: the characteristic
  function of a normalised sum converges to $e^{-\omega^{2}/2}$, the
  transform of the Gaussian.
\end{enumerate}

%% -------------------------------------------------------------------
\subsubsection{17.24\quad Table of Fourier transform pairs for spherically symmetric functions}

For spherically symmetric (radial) functions $f(\mathbf{x})=f(r)$ in
$\mathbb{R}^{n}$, the $n$-dimensional Fourier transform reduces to a
one-dimensional integral involving Bessel functions.  In three dimensions,
$\hat{f}(k)=\frac{4\pi}{k}\int_{0}^{\infty}r\sin(kr)\,f(r)\,dr$,
which is essentially a Fourier sine transform of $rf(r)$ divided by~$k$.
The general formula for $\mathbb{R}^{n}$ involves the Hankel transform of
order $\nu=n/2-1$:
$\hat{f}(k)=(2\pi)^{n/2}k^{-\nu}\int_{0}^{\infty}r^{\nu+1}J_{\nu}(kr)\,f(r)\,dr$.
The table in G\&R~17.24 lists the most important radial pairs, which
appear throughout scattering theory, potential theory, and statistical
mechanics.

\paragraph{Physics applications.}
\begin{enumerate}
\item \textbf{Coulomb potential and scattering form factors.}%
  \index{Coulomb potential!Fourier transform}%
  \index{form factor!scattering}%
  \index{Yukawa potential!Fourier transform}%
  The fundamental pair $1/r\leftrightarrow 4\pi/k^{2}$ in three
  dimensions is the Fourier transform of the Coulomb potential,
  essential in electrostatics and quantum scattering.  The Yukawa
  potential $e^{-\mu r}/r\leftrightarrow 4\pi/(k^{2}+\mu^{2})$
  describes screened interactions.  Nuclear and particle form factors
  $F(k)=\hat{\rho}(k)/\hat{\rho}(0)$ are the spherical Fourier transforms
  of charge or matter distributions.

\item \textbf{Born approximation and scattering cross sections.}%
  \index{Born approximation!Fourier transform}%
  \index{scattering amplitude!radial}%
  \index{Rutherford scattering}%
  In the first Born approximation, the scattering amplitude is
  proportional to the Fourier transform of the potential:
  $f(\theta)\propto\hat{V}(|\mathbf{k}-\mathbf{k}'|)$.  For the
  Coulomb potential, this recovers the Rutherford scattering formula.
  The radial Fourier transform pairs in the table provide the scattering
  amplitudes for standard model potentials.

\item \textbf{Pair correlation functions in statistical mechanics.}%
  \index{pair correlation function}%
  \index{structure factor!radial}%
  \index{Ornstein--Zernike equation}%
  The static structure factor $S(k)=1+n\hat{h}(k)$ of a fluid is related
  to the pair correlation function $h(r)=g(r)-1$ through the spherical
  Fourier transform.  The Ornstein--Zernike equation
  $h(r)=c(r)+n\int c(|\mathbf{r}-\mathbf{r}'|)h(r')\,d\mathbf{r}'$
  becomes algebraic in Fourier space: $\hat{h}(k)=\hat{c}(k)/(1-n\hat{c}(k))$.
\end{enumerate}

\paragraph{Mathematics applications.}
\begin{enumerate}
\item \textbf{Hecke--Bochner theorem and radial Fourier analysis.}%
  \index{Hecke--Bochner theorem}%
  \index{spherical harmonics!Fourier decomposition}%
  \index{Hankel transform!radial Fourier}%
  The Hecke--Bochner theorem states that if
  $f(\mathbf{x})=f_{0}(r)Y_{\ell}^{m}(\hat{\mathbf{x}})$, then
  $\hat{f}(\mathbf{k})=(-i)^{\ell}\hat{f}_{0}^{(\ell)}(k)Y_{\ell}^{m}(\hat{\mathbf{k}})$,
  where $\hat{f}_{0}^{(\ell)}$ is a Hankel transform.  This separates
  the angular and radial parts of the Fourier transform, reducing
  multidimensional analysis to one-dimensional Hankel transforms.

\item \textbf{Positive definiteness and Schoenberg's theorem.}%
  \index{Schoenberg's theorem}%
  \index{positive definite functions!radial}%
  \index{radial basis functions}%
  Schoenberg's theorem characterises continuous radial positive definite
  functions in $\mathbb{R}^{n}$: $f(r)$ is positive definite if and only
  if its Hankel transform is a non-negative measure.  This is the
  foundation for radial basis function interpolation and Gaussian process
  regression with isotropic covariance kernels.
\end{enumerate}

%% -------------------------------------------------------------------
\subsubsection{17.31\quad Fourier sine and cosine transforms}

The Fourier sine and cosine transforms are the natural half-line analogues
of the Fourier transform, defined for $t\geq 0$:
\[
  \mathcal{F}_{s}\{f\}(\omega)=\int_{0}^{\infty}f(t)\sin(\omega t)\,dt,\qquad
  \mathcal{F}_{c}\{f\}(\omega)=\int_{0}^{\infty}f(t)\cos(\omega t)\,dt.
\]
Both are self-reciprocal: $\mathcal{F}_{s}^{-1}=(2/\pi)\mathcal{F}_{s}$
and $\mathcal{F}_{c}^{-1}=(2/\pi)\mathcal{F}_{c}$.  The sine transform
arises naturally from odd extensions of functions, and the cosine
transform from even extensions.  Their principal domain of application
is boundary value problems on the half-line and in semi-infinite
geometries, where the choice between sine and cosine is dictated by the
boundary condition at the origin: Dirichlet conditions select the sine
transform, Neumann conditions select the cosine transform.

\paragraph{Physics applications.}
\begin{enumerate}
\item \textbf{Heat conduction on a semi-infinite rod.}%
  \index{Fourier sine transform!heat equation}%
  \index{heat equation!semi-infinite domain}%
  \index{Dirichlet boundary condition!sine transform}%
  \index{Neumann boundary condition!cosine transform}%
  The heat equation $u_{t}=\alpha u_{xx}$ on $x\geq 0$ with
  $u(0,t)=0$ (Dirichlet) is solved by sine transform:
  $\hat{u}_{s}(\omega,t)=\hat{u}_{s}(\omega,0)\,e^{-\alpha\omega^{2}t}$.
  With $u_{x}(0,t)=0$ (Neumann, insulated end), the cosine transform
  applies instead.  The choice of transform automatically encodes the
  boundary condition.

\item \textbf{Elastic half-space and contact mechanics.}%
  \index{elastic half-space}%
  \index{contact mechanics!Fourier transform}%
  \index{Boussinesq problem}%
  \index{stress distribution!half-space}%
  The Boussinesq problem---a point load on an elastic half-space---is
  solved by Fourier (specifically Hankel) transforms that reduce to
  sine and cosine transforms for axially symmetric problems.  The
  surface displacement and stress distributions are expressed as inverse
  sine/cosine transforms of the applied load's transform, fundamental to
  contact mechanics and geotechnical engineering.

\item \textbf{Electromagnetic wave propagation in half-space.}%
  \index{electromagnetic!half-space}%
  \index{Sommerfeld integral}%
  \index{antenna!ground plane}%
  The radiation from an antenna above a conducting half-plane involves
  Fourier sine and cosine transforms (Sommerfeld integrals) to satisfy
  boundary conditions at the interface.  The vertical electric field
  component uses the cosine transform (vanishing tangential $E$ at the
  conductor), while the horizontal component uses the sine transform.

\item \textbf{Potential flow around semi-infinite bodies.}%
  \index{potential flow!semi-infinite}%
  \index{Laplace equation!half-plane}%
  \index{stream function!sine transform}%
  The velocity potential and stream function for irrotational flow past
  semi-infinite plates or wedges are computed via sine and cosine
  transforms of the Laplace equation on half-plane domains.  The
  Wiener--Hopf technique for mixed boundary value problems frequently
  decomposes into paired sine and cosine transform equations.
\end{enumerate}

\paragraph{Mathematics applications.}
\begin{enumerate}
\item \textbf{Hankel transforms and the connection to Bessel functions.}%
  \index{Hankel transform}%
  \index{Bessel function!integral representation}%
  \index{Fourier--Bessel transform}%
  The Hankel transform of order $\nu$,
  $\mathcal{H}_{\nu}\{f\}(k)=\int_{0}^{\infty}f(r)J_{\nu}(kr)\,r\,dr$,
  generalises the sine and cosine transforms: $\mathcal{H}_{-1/2}$
  reduces to the cosine transform and $\mathcal{H}_{1/2}$ to the sine
  transform (up to normalisation).  The Hankel transform is self-reciprocal
  and diagonalises the radial part of the Laplacian in cylindrical
  coordinates.

\item \textbf{Sturm--Liouville theory on the half-line.}%
  \index{Sturm--Liouville!half-line}%
  \index{eigenfunction expansion!half-line}%
  \index{Weyl--Titchmarsh theory}%
  The Fourier sine and cosine transforms are the eigenfunction expansions
  for the operator $-d^{2}/dx^{2}$ on $[0,\infty)$ with Dirichlet and
  Neumann boundary conditions, respectively.  The general
  Weyl--Titchmarsh theory extends this to arbitrary Sturm--Liouville
  operators, producing spectral measures and generalised eigenfunction
  transforms.

\item \textbf{Hardy space decomposition.}%
  \index{Hardy space!half-line}%
  \index{analytic signal}%
  \index{Hilbert transform!sine-cosine}%
  An $L^{2}$ function on the real line decomposes into analytic
  ($H^{2}_{+}$) and anti-analytic ($H^{2}_{-}$) parts.  The sine and
  cosine transforms of a causal function $f(t)u(t)$ are related by the
  Hilbert transform: $\mathcal{F}_{c}\{f\}$ and $\mathcal{F}_{s}\{f\}$
  form a Hilbert transform pair, encoding the Kramers--Kronig relations
  of linear response theory.

\item \textbf{Dual integral equations and mixed boundary problems.}%
  \index{dual integral equations}%
  \index{mixed boundary value problems}%
  \index{Sneddon's method}%
  Mixed boundary value problems (e.g., a crack in an elastic medium,
  or an electrified disc) lead to dual integral equations involving
  simultaneous sine or cosine transform relations on complementary
  intervals.  Sneddon's method reduces these to Abel integral equations,
  solvable in closed form using the properties of the sine and cosine
  transforms.
\end{enumerate}

%% -------------------------------------------------------------------
\subsubsection{17.32\quad Basic properties of the Fourier sine and cosine transforms}

The operational properties of the Fourier sine and cosine transforms
parallel those of the full Fourier transform, with important modifications
due to the half-line domain.  The differentiation formulas involve
boundary values:
$\mathcal{F}_{s}\{f''\}=-\omega^{2}\mathcal{F}_{s}\{f\}+\omega f(0)$ and
$\mathcal{F}_{c}\{f''\}=-\omega^{2}\mathcal{F}_{c}\{f\}-f'(0)$.
The convolution structure is more subtle than for the full Fourier
transform: neither the sine nor cosine transform has a simple
multiplicative convolution theorem, but specific half-range convolution
formulas exist.

\paragraph{Physics applications.}
\begin{enumerate}
\item \textbf{Differentiation rules and boundary value problems.}%
  \index{differentiation!sine transform}%
  \index{differentiation!cosine transform}%
  \index{boundary value problem!half-line}%
  \index{heat equation!boundary conditions}%
  The differentiation formulas
  $\mathcal{F}_{s}\{f''\}=-\omega^{2}\hat{f}_{s}+\omega f(0)$ and
  $\mathcal{F}_{c}\{f''\}=-\omega^{2}\hat{f}_{c}-f'(0)$ show precisely
  how boundary data enter the transformed equation.  For the heat equation
  $u_{t}=\alpha u_{xx}$ on $[0,\infty)$ with $u(0,t)=g(t)$, the sine
  transform gives $\hat{u}_{s,t}=-\alpha\omega^{2}\hat{u}_{s}
  +\alpha\omega g(t)$, a first-order ODE in~$t$ with a forcing term
  from the boundary.

\item \textbf{Parseval relations and energy on the half-line.}%
  \index{Parseval's theorem!sine/cosine}%
  \index{energy!half-line}%
  \index{spectral energy density!half-line}%
  The Parseval relations
  $\int_{0}^{\infty}|f(t)|^{2}\,dt=\frac{2}{\pi}\int_{0}^{\infty}|\hat{f}_{s}(\omega)|^{2}\,d\omega
  =\frac{2}{\pi}\int_{0}^{\infty}|\hat{f}_{c}(\omega)|^{2}\,d\omega$
  express conservation of energy on the half-line.  These are used in
  bounding solutions of half-space problems and in stability analysis of
  boundary layers in fluid dynamics.

\item \textbf{Scaling and self-similar solutions.}%
  \index{scaling!sine/cosine transforms}%
  \index{self-similar solutions}%
  \index{Barenblatt solution}%
  The scaling property $\mathcal{F}_{s}\{f(at)\}=a^{-1}\hat{f}_{s}(\omega/a)$
  is central to self-similar solutions of diffusion equations.  The
  Boltzmann similarity variable $\eta=x/\sqrt{4\alpha t}$ reduces the
  heat equation to an ODE whose solution is the error function---the
  sine transform of a Gaussian.

\item \textbf{Kramers--Kronig relations.}%
  \index{Kramers--Kronig relations}%
  \index{causality!sine-cosine connection}%
  \index{dielectric function!dispersion}%
  \index{optical constants}%
  The Kramers--Kronig relations connect the real and imaginary parts of
  a causal response function:
  $\chi'(\omega)=\frac{2}{\pi}\text{P}\!\int_{0}^{\infty}\frac{\omega'\chi''(\omega')}{\omega'^{2}-\omega^{2}}\,d\omega'$
  and its companion.  These are consequences of the fact that for a causal
  function, the Fourier cosine and sine transforms (the real and imaginary
  parts of the Fourier transform of a causal function) are Hilbert
  transform pairs, enforcing analyticity in the upper half-plane.
\end{enumerate}

\paragraph{Mathematics applications.}
\begin{enumerate}
\item \textbf{Integral representations of special functions.}%
  \index{integral representations!special functions}%
  \index{gamma function!sine transform}%
  \index{Riemann--Liouville integral}%
  Many special function identities are expressed through sine and cosine
  transforms.  For example,
  $\mathcal{F}_{s}\{t^{\alpha-1}\}=\Gamma(\alpha)\sin(\pi\alpha/2)/\omega^{\alpha}$
  for $0<\alpha<1$ gives integral representations of the gamma function,
  and the Riemann--Liouville fractional integral
  $I^{\alpha}f(x)=\frac{1}{\Gamma(\alpha)}\int_{0}^{x}(x-t)^{\alpha-1}f(t)\,dt$
  is diagonalised by the Fourier sine transform.

\item \textbf{Watson's lemma and asymptotic expansions.}%
  \index{Watson's lemma!sine/cosine}%
  \index{asymptotic expansion!oscillatory integrals}%
  \index{Riemann--Lebesgue lemma}%
  The asymptotic behaviour of Fourier sine and cosine integrals as
  $\omega\to\infty$ is governed by Watson's lemma: if
  $f(t)\sim\sum a_{n}t^{n+\lambda-1}$ as $t\to 0^{+}$, then
  $\mathcal{F}_{s}\{f\}\sim\sum a_{n}\Gamma(n+\lambda)\sin(\pi(n+\lambda)/2)/\omega^{n+\lambda}$.
  The Riemann--Lebesgue lemma guarantees that $\hat{f}_{s},\hat{f}_{c}\to 0$
  as $\omega\to\infty$ for $f\in L^{1}$.

\item \textbf{Completeness and the Fourier--Bessel expansion.}%
  \index{completeness!sine/cosine}%
  \index{Fourier--Bessel series}%
  \index{$L^{2}[0,\infty)$}%
  The systems $\{\sin(\omega t)\}_{\omega>0}$ and
  $\{\cos(\omega t)\}_{\omega>0}$ are complete in $L^{2}[0,\infty)$:
  every square-integrable function on the half-line has both a Fourier
  sine and a Fourier cosine representation.  The discrete analogues on
  $[0,a]$ give the Fourier sine and cosine series, and the radial
  generalisation gives Fourier--Bessel (Dini) series on $[0,a]$ using
  zeros of Bessel functions.
\end{enumerate}

%% -------------------------------------------------------------------
\subsubsection{17.33\quad Table of Fourier sine transforms}

The table of Fourier sine transforms in G\&R~17.33 lists the standard pairs
$f(t)\leftrightarrow\mathcal{F}_{s}\{f\}(\omega)$ for common functions on
the half-line.  Key entries include
$e^{-at}\leftrightarrow\omega/(a^{2}+\omega^{2})$,
$t^{-1}e^{-at}\leftrightarrow\arctan(\omega/a)$,
$1/t\leftrightarrow\pi/2$ (for all $\omega>0$), and
$t^{\alpha-1}\leftrightarrow\Gamma(\alpha)\sin(\pi\alpha/2)/\omega^{\alpha}$
for $0<\alpha<1$.  The sine transforms of Bessel functions and other special
functions are also tabulated, connecting to the Hankel transform theory.

\paragraph{Physics applications.}
\begin{enumerate}
\item \textbf{Odd-symmetry boundary problems in electrostatics.}%
  \index{Fourier sine transform!electrostatics}%
  \index{Laplace equation!half-plane}%
  \index{Dirichlet problem!half-plane}%
  The Laplace equation on a half-plane with Dirichlet data $u(0,y)=f(y)$
  for $y>0$ is solved by sine transform in~$y$:
  $\hat{u}_{s}(x,\omega)=\hat{f}_{s}(\omega)e^{-\omega x}$.  The pair
  $e^{-at}\leftrightarrow\omega/(a^{2}+\omega^{2})$ gives the potential
  due to a surface charge that decays exponentially along the boundary.

\item \textbf{Torsion of prismatic bars.}%
  \index{torsion problem!sine transform}%
  \index{Saint-Venant torsion}%
  \index{warping function}%
  The Saint-Venant torsion problem for a prismatic bar of rectangular
  cross section involves solving $\nabla^{2}\phi=-2$ with $\phi=0$ on the
  boundary.  Fourier sine series in one variable reduce this to an ODE
  in the other, and the table entries provide the explicit coefficients
  of the resulting hyperbolic sine/cosine solution.

\item \textbf{Pair distribution functions and neutron scattering.}%
  \index{neutron scattering!sine transform}%
  \index{pair distribution function!sine transform}%
  \index{radial distribution function}%
  The radial distribution function $g(r)$ of a liquid is related to the
  measured structure factor $S(k)$ by a Fourier sine transform:
  $r[g(r)-1]=\frac{1}{2\pi^{2}n}\int_{0}^{\infty}k[S(k)-1]\sin(kr)\,dk$.
  This is the fundamental data analysis tool in neutron and X-ray
  scattering experiments on liquids and amorphous materials.
\end{enumerate}

\paragraph{Mathematics applications.}
\begin{enumerate}
\item \textbf{The sine transform as an odd Fourier transform.}%
  \index{odd extension!sine transform}%
  \index{Fourier transform!odd functions}%
  If $f$ is defined on $(0,\infty)$ and $f_{\text{odd}}$ is its odd
  extension to $\mathbb{R}$, then
  $\mathcal{F}\{f_{\text{odd}}\}(\omega)=-2i\mathcal{F}_{s}\{f\}(\omega)$.
  This allows the sine transform table to be used for evaluating full
  Fourier transforms of odd functions, and conversely, symmetry
  arguments reduce certain full Fourier integrals to table look-ups
  in the sine transform table.

\item \textbf{Sine transform of power functions and Mellin connection.}%
  \index{Mellin transform!sine connection}%
  \index{power function!sine transform}%
  \index{beta function!integral representation}%
  The pair $t^{\alpha-1}\leftrightarrow\Gamma(\alpha)\sin(\pi\alpha/2)/\omega^{\alpha}$
  connects the sine transform to the Mellin transform: evaluating the
  Mellin transform of $\sin(\omega t)$ at $s=\alpha$ gives
  $\Gamma(\alpha)\sin(\pi\alpha/2)/\omega^{\alpha}$, which is also
  the analytic continuation of the integral
  $\int_{0}^{\infty}t^{s-1}\sin t\,dt$.  These identities are
  fundamental in the theory of Dirichlet series and the Riemann zeta
  function.
\end{enumerate}

%% -------------------------------------------------------------------
\subsubsection{17.34\quad Table of Fourier cosine transforms}

The table of Fourier cosine transforms in G\&R~17.34 provides the standard
pairs $f(t)\leftrightarrow\mathcal{F}_{c}\{f\}(\omega)$.  Key entries
include $e^{-at}\leftrightarrow a/(a^{2}+\omega^{2})$,
$e^{-at^{2}}\leftrightarrow\sqrt{\pi/(4a)}\,e^{-\omega^{2}/(4a)}$
(Gaussian), $\operatorname{sech}(\pi t/2)\leftrightarrow\operatorname{sech}(\omega)$
(the hyperbolic secant is a fixed point of the cosine transform up to
normalisation), and
$t^{\alpha-1}\leftrightarrow\Gamma(\alpha)\cos(\pi\alpha/2)/\omega^{\alpha}$
for $0<\alpha<1$.  The cosine transform table complements the sine
transform table and shares the same applications in half-line boundary
value problems.

\paragraph{Physics applications.}
\begin{enumerate}
\item \textbf{Even-symmetry problems and Neumann conditions.}%
  \index{Fourier cosine transform!Neumann condition}%
  \index{Neumann problem!half-plane}%
  \index{insulated boundary}%
  For the heat equation on $[0,\infty)$ with Neumann condition
  $u_{x}(0,t)=0$ (insulated end), the cosine transform gives
  $\hat{u}_{c,t}=-\alpha\omega^{2}\hat{u}_{c}$, with no boundary forcing
  term.  The pair $e^{-at^{2}}\leftrightarrow\sqrt{\pi/4a}\,e^{-\omega^{2}/4a}$
  gives the fundamental solution directly through the cosine transform
  table.

\item \textbf{Autocorrelation and power spectral density.}%
  \index{autocorrelation!cosine transform}%
  \index{power spectral density!cosine transform}%
  \index{Wiener--Khinchin!cosine form}%
  For a real stationary process, the autocorrelation $R(\tau)$ is an even
  function, and the Wiener--Khinchin theorem takes the form
  $S(\omega)=2\int_{0}^{\infty}R(\tau)\cos(\omega\tau)\,d\tau
  =2\mathcal{F}_{c}\{R\}(\omega)$.
  The cosine transform table thus directly provides the power spectra
  for standard autocorrelation models (exponential, Gaussian, etc.).

\item \textbf{Debye model and phonon specific heat.}%
  \index{Debye model!cosine transform}%
  \index{phonon density of states}%
  \index{specific heat!Debye}%
  The phonon density of states in the Debye model involves cosine
  transforms of lattice displacement correlation functions.  The pair
  $e^{-at}\leftrightarrow a/(a^{2}+\omega^{2})$ gives the spectral
  density for an exponentially decaying correlation, and the Debye
  function $D_{n}(x)$ can be expressed through integrals closely related
  to cosine transform pairs.
\end{enumerate}

\paragraph{Mathematics applications.}
\begin{enumerate}
\item \textbf{Even extension and the full Fourier transform.}%
  \index{even extension!cosine transform}%
  \index{Fourier transform!even functions}%
  If $f_{\text{even}}$ is the even extension of $f$ to $\mathbb{R}$,
  then $\mathcal{F}\{f_{\text{even}}\}=2\mathcal{F}_{c}\{f\}$.  The
  cosine transform table provides efficient evaluation of full Fourier
  transforms of even functions, and conversely, symmetry reduction halves
  the computational effort in numerical Fourier analysis of even data.

\item \textbf{Self-reciprocal functions.}%
  \index{self-reciprocal function!cosine}%
  \index{fixed points!cosine transform}%
  \index{sech!self-reciprocal}%
  A function $f$ satisfying $\mathcal{F}_{c}\{f\}=cf$ (up to a constant)
  is self-reciprocal under the cosine transform.  The Gaussian
  $e^{-t^{2}/2}$ and the function $1/\cosh(\pi t/2)$ are classical
  examples.  Self-reciprocal functions play a role in the theory of
  theta functions and modular forms, where the functional equation
  $\theta(1/t)=\sqrt{t}\,\theta(t)$ is a self-reciprocality statement
  for the Jacobi theta function under the Mellin transform.
\end{enumerate}

%% -------------------------------------------------------------------
\subsubsection{17.35\quad Relationships between transforms}

The Laplace, Fourier, sine, cosine, and Mellin transforms are all members
of a single family of integral transforms with exponential or power-law
kernels, and there are systematic relationships connecting them.  The
Fourier transform evaluated at imaginary argument recovers the Laplace
transform: for $f$ supported on $[0,\infty)$,
$F(s)=\hat{f}(-is)$.  The sine and cosine transforms are the imaginary
and real parts of the half-line Fourier transform.  The Mellin transform
$\mathcal{M}\{f\}(s)=\int_{0}^{\infty}t^{s-1}f(t)\,dt$ is related to
the Laplace transform by the substitution $t=e^{-u}$:
$\mathcal{M}\{f\}(s)=\mathcal{L}\{f(e^{-u})e^{-su}\}$ evaluated
appropriately.  These interconnections allow transform pairs from one table
to be translated into pairs for another.

\paragraph{Physics applications.}
\begin{enumerate}
\item \textbf{From Laplace to Fourier: steady-state frequency response.}%
  \index{Laplace to Fourier!$s=i\omega$}%
  \index{frequency response!from transfer function}%
  \index{Bode plot!Laplace-Fourier}%
  \index{steady-state response!frequency domain}%
  Setting $s=i\omega$ in the transfer function $H(s)$ gives the
  frequency response $H(i\omega)=|H(i\omega)|e^{i\phi(\omega)}$, provided
  the system is stable (all poles in the left half-plane).  This bridge
  between the Laplace and Fourier domains is the basis of Bode plots and
  all frequency-domain design methods in control engineering and
  electronic filter design.

\item \textbf{Wick rotation and Euclidean quantum field theory.}%
  \index{Wick rotation}%
  \index{Euclidean quantum field theory}%
  \index{imaginary time!thermal field theory}%
  \index{Matsubara frequencies}%
  The substitution $t\to-i\tau$ (Wick rotation) converts Minkowski
  spacetime path integrals to Euclidean ones, essentially rotating the
  Fourier transform variable from real to imaginary values.  In thermal
  field theory, the Euclidean time becomes periodic with period
  $\beta=1/(k_{B}T)$, and the continuous Fourier transform is replaced
  by a discrete sum over Matsubara frequencies $\omega_{n}=2\pi n/\beta$.

\item \textbf{Hilbert transform and causality.}%
  \index{Hilbert transform}%
  \index{causality!Hilbert transform}%
  \index{analytic signal!engineering}%
  \index{single-sideband modulation}%
  The Hilbert transform $\mathcal{H}\{f\}(t)=\frac{1}{\pi}\text{P}\!\int_{-\infty}^{\infty}\frac{f(\tau)}{t-\tau}\,d\tau$
  connects the cosine and sine transform components of a causal signal.
  The analytic signal $z(t)=f(t)+i\mathcal{H}\{f\}(t)$ has a one-sided
  Fourier transform (supported on $\omega>0$), the basis of single-sideband
  modulation in communications and envelope detection in signal processing.

\item \textbf{Two-sided Laplace transform and bilateral systems.}%
  \index{two-sided Laplace transform}%
  \index{bilateral systems}%
  \index{region of convergence}%
  The two-sided Laplace transform
  $\int_{-\infty}^{\infty}f(t)e^{-st}\,dt$ is the Fourier transform of
  $f(t)e^{-\sigma t}$ evaluated at $\omega$ (where $s=\sigma+i\omega$).
  The region of convergence in the $s$-plane determines whether the
  system is causal, anti-causal, or neither, and the intersection of
  the ROC with the imaginary axis determines the existence of the Fourier
  transform.
\end{enumerate}

\paragraph{Mathematics applications.}
\begin{enumerate}
\item \textbf{Mellin--Fourier connection and multiplicative harmonics.}%
  \index{Mellin--Fourier connection}%
  \index{multiplicative convolution}%
  \index{Haar measure!multiplicative group}%
  The substitution $t=e^{u}$ converts the Mellin transform to the
  Fourier transform: $\mathcal{M}\{f\}(\sigma+i\omega)
  =\mathcal{F}\{f(e^{u})e^{\sigma u}\}(\omega)$.  The Mellin convolution
  $(f\circledast g)(x)=\int_{0}^{\infty}f(x/t)g(t)\,dt/t$ becomes
  ordinary convolution under this substitution.  This reflects the Haar
  measure $dt/t$ on the multiplicative group $(\mathbb{R}^{+},\times)$
  and places the Mellin transform in the framework of harmonic analysis
  on groups.

\item \textbf{Laplace--Stieltjes transform and distribution theory.}%
  \index{Laplace--Stieltjes transform}%
  \index{distribution theory!transforms}%
  \index{Fourier--Laplace transform}%
  The Laplace--Stieltjes transform
  $\int_{0}^{\infty}e^{-st}\,d\mu(t)$ generalises the Laplace transform
  to measures and distributions.  The Fourier--Laplace transform
  $\hat{u}(\zeta)=\langle u,e^{-i\zeta\cdot x}\rangle$ for
  distributions $u\in\mathcal{E}'(\mathbb{R}^{n})$ yields entire
  functions of exponential type, unifying the Paley--Wiener and
  Laplace inversion theories.

\item \textbf{Ramanujan's master theorem.}%
  \index{Ramanujan's master theorem}%
  \index{Mellin transform!series expansion}%
  \index{analytic continuation!transform pairs}%
  Ramanujan's master theorem states that if
  $f(x)=\sum_{k=0}^{\infty}\frac{(-1)^{k}\varphi(k)}{k!}x^{k}$, then
  $\int_{0}^{\infty}x^{s-1}f(x)\,dx=\Gamma(s)\varphi(-s)$.
  This provides a powerful method for evaluating Mellin transforms (and
  hence Fourier and Laplace transforms via the inter-transform
  relationships) from the Taylor series expansion of the integrand,
  connecting power series coefficients to transform values by analytic
  continuation.
\end{enumerate}

%% -------------------------------------------------------------------
\subsubsection{17.41\quad Mellin transform}

The Mellin transform is defined by
$\mathcal{M}\{f\}(s)=\int_{0}^{\infty}t^{s-1}f(t)\,dt$
for $s$ in a vertical strip $a<\operatorname{Re}s<b$ (the fundamental
strip), with inverse
$f(t)=\frac{1}{2\pi i}\int_{c-i\infty}^{c+i\infty}t^{-s}\mathcal{M}\{f\}(s)\,ds$
for $a<c<b$.  The Mellin transform is the natural tool for problems with
multiplicative structure: it diagonalises the operator $t\,d/dt$ (the
generator of dilations), just as the Fourier transform diagonalises
$d/dt$ (the generator of translations).  The Mellin transform connects
analytic number theory, asymptotic analysis, and the special functions
of mathematical physics.

\paragraph{Physics applications.}
\begin{enumerate}
\item \textbf{Radiative transfer and the Milne problem.}%
  \index{Mellin transform!radiative transfer}%
  \index{Milne problem}%
  \index{radiative transfer equation}%
  \index{Chandrasekhar $H$-function}%
  The integral equation of radiative transfer in a semi-infinite
  atmosphere (the Milne problem) has a kernel with multiplicative
  structure that is diagonalised by the Mellin transform.
  Chandrasekhar's $H$-function, fundamental to astrophysical radiative
  transfer, satisfies a nonlinear integral equation whose analysis
  relies on Mellin transform techniques to establish existence,
  uniqueness, and asymptotic properties.

\item \textbf{Parton distribution functions in QCD.}%
  \index{parton distribution functions}%
  \index{DGLAP equations}%
  \index{Mellin transform!QCD}%
  \index{deep inelastic scattering}%
  The DGLAP evolution equations for parton distribution functions in
  quantum chromodynamics involve convolution integrals in the momentum
  fraction variable~$x$.  The Mellin transform converts these to ordinary
  differential equations in the Mellin variable~$N$:
  $d\tilde{f}(N,Q^{2})/d\ln Q^{2}=\tilde{P}(N)\tilde{f}(N,Q^{2})$,
  where $\tilde{P}(N)$ is the Mellin transform of the splitting function.
  Inverse Mellin transforms then give the evolved parton distributions.

\item \textbf{Gravitational lensing and the magnification distribution.}%
  \index{gravitational lensing!Mellin transform}%
  \index{magnification distribution}%
  \index{strong lensing!statistics}%
  The probability distribution of gravitational lensing magnifications
  has a power-law tail $P(\mu)\propto\mu^{-3}$ whose moments are
  naturally computed by the Mellin transform.  The Mellin convolution
  structure arises because successive lensing events multiply
  magnifications, and the total magnification distribution is the
  Mellin convolution of individual lens distributions.

\item \textbf{Dimensional regularisation in quantum field theory.}%
  \index{dimensional regularisation!Mellin transform}%
  \index{Feynman integrals!Mellin--Barnes}%
  \index{Mellin--Barnes representation}%
  Feynman loop integrals in dimensional regularisation are evaluated
  using the Mellin--Barnes representation: propagator products
  $1/(A_{1}^{a_{1}}\cdots A_{n}^{a_{n}})$ are written as Mellin--Barnes
  integrals (inverse Mellin transforms of products of gamma functions),
  reducing multi-loop integrals to contour integrals that can be
  evaluated by residues.  This is one of the principal techniques in
  modern perturbative quantum field theory.
\end{enumerate}

\paragraph{Mathematics applications.}
\begin{enumerate}
\item \textbf{The Riemann zeta function as a Mellin transform.}%
  \index{Riemann zeta function!Mellin transform}%
  \index{theta function!Mellin transform}%
  \index{functional equation!zeta function}%
  \index{Jacobi theta function}%
  The completed zeta function satisfies
  $\pi^{-s/2}\Gamma(s/2)\zeta(s)
  =\frac{1}{2}\int_{0}^{\infty}t^{s/2-1}[\theta(t)-1]\,dt$,
  where $\theta(t)=\sum_{n=-\infty}^{\infty}e^{-\pi n^{2}t}$ is the
  Jacobi theta function.  This is a Mellin transform, and the functional
  equation $\theta(1/t)=\sqrt{t}\,\theta(t)$ translates via the Mellin
  transform into the functional equation
  $\xi(s)=\xi(1-s)$ for the Riemann zeta function.

\item \textbf{Asymptotic analysis and the Mellin--Perron formula.}%
  \index{Mellin--Perron formula}%
  \index{asymptotic analysis!Mellin transform}%
  \index{algorithm analysis!Mellin}%
  \index{harmonic sums}%
  The Mellin--Perron formula
  $\sum_{n\leq x}a_{n}=\frac{1}{2\pi i}\int_{c-i\infty}^{c+i\infty}
  \left(\sum_{n=1}^{\infty}\frac{a_{n}}{n^{s}}\right)\frac{x^{s}}{s}\,ds$
  expresses partial sums of Dirichlet series as inverse Mellin transforms.
  In the analysis of algorithms, the average cost of divide-and-conquer
  algorithms involves harmonic sums
  $\sum_{k}f(x/b^{k})$ whose Mellin transforms are products of the form
  $\mathcal{M}\{f\}(s)\cdot\sum_{k}b^{-ks}=\mathcal{M}\{f\}(s)/(1-b^{-s})$,
  with poles determining the asymptotic growth rate.

\item \textbf{Dirichlet series and multiplicative number theory.}%
  \index{Dirichlet series!Mellin transform}%
  \index{multiplicative number theory}%
  \index{Euler product}%
  A Dirichlet series $\sum a_{n}n^{-s}$ is the Mellin transform of
  the sum $\sum a_{n}\delta(\log t-\log n)$.  The multiplicative
  convolution (Dirichlet convolution) $\sum_{d|n}f(d)g(n/d)$
  becomes pointwise multiplication of Dirichlet series.  Euler products
  $\prod_{p}(1-a_{p}p^{-s})^{-1}$ express the multiplicative structure
  of arithmetic functions through the Mellin transform framework.

\item \textbf{Gamma function identities and special function theory.}%
  \index{gamma function!Mellin transform pairs}%
  \index{beta function!Mellin convolution}%
  \index{Barnes integral}%
  The Mellin transforms of elementary functions involve the gamma function:
  $\mathcal{M}\{e^{-t}\}(s)=\Gamma(s)$,
  $\mathcal{M}\{(1+t)^{-a}\}(s)=B(s,a-s)=\Gamma(s)\Gamma(a-s)/\Gamma(a)$.
  The Barnes integral representations of hypergeometric functions
  are Mellin--Barnes integrals---inverse Mellin transforms of products of
  gamma functions---and provide the analytic continuation and asymptotic
  expansion of ${_{p}F_{q}}$ functions.
\end{enumerate}

%% -------------------------------------------------------------------
\subsubsection{17.42\quad Basic properties of the Mellin transform}

The basic properties of the Mellin transform include linearity,
the scaling property $\mathcal{M}\{f(at)\}(s)=a^{-s}\mathcal{M}\{f\}(s)$,
the multiplication property $\mathcal{M}\{t^{a}f(t)\}(s)=\mathcal{M}\{f\}(s+a)$,
the differentiation rule
$\mathcal{M}\{tf'(t)\}(s)=-s\mathcal{M}\{f\}(s)$ (assuming boundary terms
vanish), and the Mellin convolution theorem
$\mathcal{M}\{f\circledast g\}(s)=\mathcal{M}\{f\}(s)\cdot\mathcal{M}\{g\}(s)$,
where $(f\circledast g)(x)=\int_{0}^{\infty}f(x/t)g(t)\,dt/t$.
The Parseval formula is
$\int_{0}^{\infty}f(t)\overline{g(t)}\,dt
=\frac{1}{2\pi i}\int_{c-i\infty}^{c+i\infty}
\mathcal{M}\{f\}(s)\overline{\mathcal{M}\{g\}(\bar{s})}\,ds$.

\paragraph{Physics applications.}
\begin{enumerate}
\item \textbf{Scale invariance and power-law behaviour.}%
  \index{scale invariance!Mellin transform}%
  \index{power-law distributions}%
  \index{renormalisation group!scale invariance}%
  \index{critical phenomena!scaling}%
  The scaling property $\mathcal{M}\{f(at)\}=a^{-s}\mathcal{M}\{f\}$ shows
  that the Mellin transform diagonalises dilations: a function that is
  homogeneous of degree $-\alpha$ (i.e., $f(\lambda t)=\lambda^{-\alpha}f(t)$)
  has a Mellin transform proportional to $\delta(s-\alpha)$.  This makes
  the Mellin transform the natural tool for analysing power-law behaviour,
  critical exponents in phase transitions, and renormalisation group flows.

\item \textbf{Mellin convolution and cascade processes.}%
  \index{Mellin convolution!cascade}%
  \index{multiplicative processes}%
  \index{log-normal distribution}%
  \index{turbulent cascade}%
  In multiplicative cascade processes---turbulent energy cascade,
  fragmentation, and multiplicative noise---the output is a product of
  random factors.  The distribution of the product is the Mellin
  convolution of the individual factor distributions.  The Mellin
  convolution theorem converts this to multiplication in Mellin space,
  and the central limit theorem for products (yielding log-normal
  distributions) follows from the standard CLT applied to the Mellin
  (i.e., Fourier in the logarithmic variable) domain.

\item \textbf{Differentiation rule and moment equations.}%
  \index{Mellin differentiation!moment equations}%
  \index{Smoluchowski equation!Mellin transform}%
  \index{population balance equation}%
  The rule $\mathcal{M}\{t^{k}f^{(k)}\}(s)=(-1)^{k}s(s+1)\cdots(s+k-1)\mathcal{M}\{f\}(s)$
  converts Euler-type differential equations (with $t^{k}d^{k}/dt^{k}$
  terms) to algebraic equations.  The Smoluchowski coagulation equation
  and population balance equations have multiplicative kernel versions
  that are diagonalised by the Mellin transform, yielding evolution
  equations for the moments $M_{s}=\int_{0}^{\infty}t^{s}f(t)\,dt$.

\item \textbf{Parseval formula and spectral energy in log-frequency.}%
  \index{Parseval's theorem!Mellin}%
  \index{log-frequency!energy distribution}%
  \index{wavelet!Mellin transform connection}%
  The Mellin--Parseval formula distributes the $L^{2}$ norm of a function
  over the Mellin strip: signals with power-law spectra have their
  energy uniformly distributed on a logarithmic frequency scale.  This
  is closely related to the continuous wavelet transform, which is
  essentially a Mellin correlation with the analysing wavelet, and
  explains why wavelet analysis is natural for self-similar and
  fractal signals.
\end{enumerate}

\paragraph{Mathematics applications.}
\begin{enumerate}
\item \textbf{Euler differential equations and the Mellin transform.}%
  \index{Euler differential equation!Mellin}%
  \index{equidimensional equation}%
  \index{Mellin transform!differential equations}%
  The Euler (equidimensional) equation
  $\sum_{k=0}^{n}a_{k}t^{k}y^{(k)}(t)=g(t)$ has constant coefficients
  in the operator $\theta=t\,d/dt$: it becomes
  $\sum a_{k}\theta(\theta-1)\cdots(\theta-k+1)y=g$.  The Mellin transform
  converts this to the algebraic equation
  $p(s)\mathcal{M}\{y\}(s)=\mathcal{M}\{g\}(s)$ where $p(s)$
  is a polynomial, and solutions are obtained by inverse Mellin transform
  (contour integration picking up residues at the roots of $p$).

\item \textbf{Converse mapping theorem and singularity analysis.}%
  \index{converse mapping theorem!Mellin}%
  \index{singularity analysis!Mellin}%
  \index{asymptotic expansion!Mellin}%
  The asymptotic expansion of $f(t)$ as $t\to 0^{+}$ or $t\to\infty$
  is encoded in the poles of $\mathcal{M}\{f\}(s)$: a pole at $s=s_{0}$
  of order $m$ contributes a term $t^{-s_{0}}(\log t)^{m-1}$ to the
  asymptotic expansion.  This ``converse mapping theorem'' is the
  Mellin-transform analogue of the residue theorem for Laplace inversion
  and is the principal tool in the asymptotic analysis of harmonic sums
  and divide-and-conquer recurrences.

\item \textbf{Multiplicative number theory and Ramanujan's integral.}%
  \index{Ramanujan's integral}%
  \index{multiplicative functions!Mellin}%
  \index{Perron's formula}%
  Perron's formula $\sum_{n\leq x}a_{n}=\frac{1}{2\pi i}\int_{c-i\infty}^{c+i\infty}F(s)\frac{x^{s}}{s}\,ds$
  (where $F(s)=\sum a_{n}n^{-s}$) is the Mellin inversion formula
  applied to the partial sum of a Dirichlet series.  The residues of
  $F(s)x^{s}/s$ at the poles of $F$ give the main terms in the
  asymptotic expansion of $\sum_{n\leq x}a_{n}$, the basic method of
  analytic number theory.
\end{enumerate}

%% -------------------------------------------------------------------
\subsubsection{17.43\quad Table of Mellin transforms}

The table of Mellin transforms in G\&R~17.43 collects the fundamental pairs
$f(t)\leftrightarrow\mathcal{M}\{f\}(s)$.  The most important entries are
$e^{-t}\leftrightarrow\Gamma(s)$ (the defining property of the gamma
function), $(1+t)^{-a}\leftrightarrow\Gamma(s)\Gamma(a-s)/\Gamma(a)$
(the beta function), $e^{-t^{2}}\leftrightarrow\Gamma(s/2)/2$, and
$\sin(t)\leftrightarrow\Gamma(s)\sin(\pi s/2)$ for $-1<\operatorname{Re}s<1$.
The Mellin transforms of Bessel functions, hypergeometric functions, and
theta functions are also listed, providing the gateway to the Mellin--Barnes
integral representations used throughout special function theory and
mathematical physics.

\paragraph{Physics applications.}
\begin{enumerate}
\item \textbf{Gamma function and statistical mechanics.}%
  \index{gamma function!statistical mechanics}%
  \index{partition function!Mellin transform}%
  \index{ideal gas!partition function}%
  The pair $e^{-t}\leftrightarrow\Gamma(s)$ appears throughout statistical
  mechanics: the single-particle partition function of an ideal gas
  involves $\int_{0}^{\infty}\varepsilon^{s-1}e^{-\beta\varepsilon}\,d\varepsilon
  =\beta^{-s}\Gamma(s)$, and the thermodynamic functions (energy, entropy,
  specific heat) are obtained by differentiation with respect to~$s$.
  The Bose--Einstein and Fermi--Dirac integrals
  $f_{\nu}(z)=\frac{1}{\Gamma(\nu)}\int_{0}^{\infty}\frac{t^{\nu-1}}{z^{-1}e^{t}\mp 1}\,dt$
  are Mellin transforms that specialise to polylogarithms.

\item \textbf{Mellin--Barnes integrals for Feynman diagrams.}%
  \index{Mellin--Barnes integral!Feynman diagrams}%
  \index{hypergeometric function!Feynman integral}%
  \index{loop integral!Mellin--Barnes}%
  Multi-loop Feynman integrals are systematically evaluated using
  Mellin--Barnes representations.  The table entry
  $(1+t)^{-a}\leftrightarrow B(s,a-s)$ is the starting point: each
  propagator denominator is split using
  $\frac{1}{(A+B)^{a}}=\frac{1}{\Gamma(a)}\frac{1}{2\pi i}\int
  \Gamma(s)\Gamma(a-s)\frac{A^{-s}B^{-(a-s)}}{1}\,ds$,
  and the resulting multi-dimensional Mellin--Barnes integrals are
  evaluated by closing contours and summing residues.

\item \textbf{Bessel function Mellin transforms and diffraction.}%
  \index{Bessel function!Mellin transform}%
  \index{diffraction!Mellin transform}%
  \index{Airy pattern}%
  The Mellin transforms of Bessel functions, such as
  $\mathcal{M}\{J_{\nu}\}(s)
  =2^{s-1}\Gamma((s+\nu)/2)/\Gamma((\nu-s)/2+1)$, are used in
  computing diffraction patterns from circular apertures (the Airy
  pattern) and in evaluating radial integrals in atomic physics.
\end{enumerate}

\paragraph{Mathematics applications.}
\begin{enumerate}
\item \textbf{Theta function Mellin transform and $L$-functions.}%
  \index{theta function!Mellin transform table}%
  \index{$L$-function!Mellin transform}%
  \index{modular form!Mellin transform}%
  The Mellin transform of modular forms yields $L$-functions: if
  $f(\tau)=\sum a_{n}q^{n}$ (with $q=e^{2\pi i\tau}$) is a modular form,
  then $L(s,f)=\sum a_{n}n^{-s}=(2\pi)^{s}\Gamma(s)^{-1}\int_{0}^{\infty}f(it)t^{s-1}\,dt$.
  The modularity of $f$ translates via the Mellin transform into the
  functional equation of $L(s,f)$, a central theme in modern number theory.

\item \textbf{Hypergeometric function representations.}%
  \index{hypergeometric function!Mellin--Barnes integral}%
  \index{Barnes integral!hypergeometric}%
  \index{analytic continuation!hypergeometric}%
  The Barnes integral representation
  ${_{2}F_{1}}(a,b;c;z)=\frac{\Gamma(c)}{\Gamma(a)\Gamma(b)}
  \frac{1}{2\pi i}\int\frac{\Gamma(a+s)\Gamma(b+s)\Gamma(-s)}{\Gamma(c+s)}(-z)^{s}\,ds$
  is an inverse Mellin transform of a ratio of gamma functions.  This
  representation provides analytic continuation to $|z|>1$ and the
  connection formulas between different solutions of the hypergeometric
  equation, and is the prototype for Mellin--Barnes representations
  of all ${_{p}F_{q}}$ functions.
\end{enumerate}
