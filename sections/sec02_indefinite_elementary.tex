%% ============================================================
%% 2  Indefinite Integrals of Elementary Functions
%% ============================================================
\section{2\quad Indefinite Integrals of Elementary Functions}

\subsection{2.0\quad Introduction}

%% -------------------------------------------------------------------
\subsubsection{2.00\quad General remarks}
\subsubsection{2.01\quad The basic integrals}
\subsubsection{2.02\quad General formulas}

\paragraph{Physics applications.}
\begin{enumerate}
\item \textbf{Equations of motion from Newton's second law.}%
  \index{Newton's second law!integration}%
  \index{equations of motion!indefinite integral}%
  \index{velocity from acceleration}%
  The most elementary use of indefinite integrals in physics is
  $v(t)=\int a(t)\,dt$ and $x(t)=\int v(t)\,dt$.  Every kinematic
  formula in introductory mechanics---constant-acceleration results such
  as $x=x_{0}+v_{0}t+\tfrac{1}{2}at^{2}$---is an instance of the basic
  power-rule integral $\int t^{n}\,dt=t^{n+1}/(n+1)+C$.  The arbitrary
  constant~$C$ encodes initial conditions, the physicist's standard
  boundary data.

\item \textbf{Linearity and superposition.}%
  \index{superposition principle!integration}%
  \index{linearity!of integration}%
  \index{Fourier synthesis}%
  The linearity rule $\int[\alpha f+\beta g]\,dx
  =\alpha\int f\,dx+\beta\int g\,dx$ (G\&R~2.02) is the integral
  counterpart of the superposition principle.  In circuit theory, the
  response to a sum of inputs is the sum of individual responses, each
  computed by a separate antiderivative.  Fourier synthesis builds
  arbitrary waveforms from sinusoidal antiderivatives
  $\int\sin(n\omega t)\,dt=-\cos(n\omega t)/(n\omega)$.

\item \textbf{Integration by parts and the interaction picture.}%
  \index{integration by parts!physical applications}%
  \index{interaction picture}%
  \index{Dyson series}%
  Integration by parts $\int u\,dv=uv-\int v\,du$ (G\&R~2.02)
  transfers derivatives between factors.  In quantum field theory, the
  analogous operation moves derivatives off fields to obtain equations
  of motion from action principles; the boundary terms determine
  surface contributions that vanish for fields decaying at infinity.

\item \textbf{Substitution and coordinate changes.}%
  \index{substitution rule!coordinate change}%
  \index{change of variables!indefinite integral}%
  \index{canonical transformation}%
  The substitution rule $\int f(g(x))g'(x)\,dx=\int f(u)\,du$
  (G\&R~2.02) is the one-dimensional version of coordinate
  transformation.  In Hamiltonian mechanics, canonical transformations
  $(q,p)\to(Q,P)$ exploit substitutions that simplify the Hamiltonian,
  reducing integrals to standard forms catalogued in G\&R~2.01.
\end{enumerate}

\paragraph{Mathematics applications.}
\begin{enumerate}
\item \textbf{The fundamental theorem of calculus.}%
  \index{fundamental theorem of calculus}%
  \index{antiderivative!existence}%
  \index{Riemann integral!antiderivative}%
  Every continuous function on a closed interval possesses an
  antiderivative, given by $F(x)=\int_{a}^{x}f(t)\,dt$.  The
  fundamental theorem connects the two faces of calculus: the
  antiderivative (G\&R~2.01) and the definite integral (G\&R~3--4).
  The table of basic integrals is, in effect, a table of inverse
  derivatives.

\item \textbf{Liouville's theorem on elementary antiderivatives.}%
  \index{Liouville's theorem!elementary integrals}%
  \index{elementary functions!integration}%
  \index{differential algebra}%
  Not every elementary function has an elementary antiderivative---the
  classic examples $\int e^{-x^{2}}\,dx$, $\int\sin(x)/x\,dx$, and
  $\int dx/\ln x$ require special functions (see G\&R~8--9).
  Liouville's theorem (1835) and its modern extension by Risch (1969)
  give a decision procedure for when an elementary antiderivative
  exists, founding the field of differential algebra.

\item \textbf{Reduction formulas and recursion.}%
  \index{reduction formulas}%
  \index{recursion!integration}%
  \index{Wallis-type integrals}%
  The general formulas of G\&R~2.02 include reduction formulas such as
  $\int x^{n}e^{ax}\,dx=\frac{x^{n}e^{ax}}{a}
  -\frac{n}{a}\int x^{n-1}e^{ax}\,dx$.
  These are discrete recursions in the exponent~$n$, connecting
  indefinite integration to difference equations and combinatorial
  identities.
\end{enumerate}

%% ===================================================================
\subsection{2.1\quad Rational Functions}

%% -------------------------------------------------------------------
\subsubsection{2.10\quad General integration rules}
\subsubsection{2.11--2.13\quad Forms containing the binomial $a + bx^k$}

\paragraph{Physics applications.}
\begin{enumerate}
\item \textbf{Partial fractions in circuit analysis.}%
  \index{partial fractions!circuit analysis}%
  \index{transfer function!poles}%
  \index{Laplace transform!inversion by partial fractions}%
  Inverse Laplace transforms in linear circuit theory require
  decomposing a rational transfer function $H(s)=P(s)/Q(s)$ into
  partial fractions.  Each simple pole $1/(s-p_{k})$ inverts to an
  exponential $e^{p_{k}t}$, while repeated poles produce terms
  $t^{n}e^{pt}$.  The partial-fraction rules of G\&R~2.10 are the
  workhorse of this procedure.

\item \textbf{Gravitational and Coulomb potentials in one dimension.}%
  \index{Coulomb potential!one-dimensional integral}%
  \index{gravitational potential!antiderivative}%
  \index{inverse-square law!integration}%
  Integrating the inverse-square force $F=k/x^{2}$ gives the
  potential energy $U=-k/x+C$, an instance of $\int x^{-n}\,dx$
  from G\&R~2.11.  More generally, power-law forces $F\propto x^{-n}$
  and their potentials catalogue the basic binomial integrals.

\item \textbf{Logistic and Verhulst population models.}%
  \index{logistic equation!partial fractions}%
  \index{Verhulst model}%
  \index{population dynamics}%
  The logistic equation $dN/dt=rN(1-N/K)$ separates to
  $\int\frac{dN}{N(1-N/K)}=rt$, resolved by partial fractions into
  $\ln|N|-\ln|1-N/K|=rt+C$.  This is a direct application of the
  rational-function techniques of G\&R~2.10--2.13 and yields the
  sigmoid growth curve ubiquitous in ecology, epidemiology, and
  machine learning.
\end{enumerate}

\paragraph{Mathematics applications.}
\begin{enumerate}
\item \textbf{Partial fraction decomposition and the residue theorem.}%
  \index{partial fraction decomposition!algebraic}%
  \index{residue theorem!partial fractions}%
  \index{rational functions!integration}%
  Every rational function $P(x)/Q(x)$ with $\deg P<\deg Q$ decomposes
  into partial fractions, each integrable in closed form (logarithms
  and arctangents).  Over $\mathbb{C}$, the coefficients are residues
  of the complex function $P(z)/Q(z)$, linking the algebraic
  decomposition of G\&R~2.10 to the Cauchy residue theorem.

\item \textbf{Ostrogradsky--Hermite method.}%
  \index{Ostrogradsky--Hermite method}%
  \index{rational part!of integral}%
  \index{squarefree factorisation}%
  The Ostrogradsky--Hermite method separates
  $\int P/Q\,dx$ into a rational part plus a logarithmic part without
  fully factoring~$Q(x)$, using only the squarefree decomposition.
  This is more efficient than full partial fractions and is the basis
  of modern computer algebra algorithms for rational integration.

\item \textbf{Chebyshev's theorem on binomial integrals.}%
  \index{Chebyshev's theorem!binomial integrals}%
  \index{binomial integral!elementary conditions}%
  Chebyshev (1853) proved that $\int x^{p}(a+bx^{r})^{q}\,dx$ is
  elementary only when $(p+1)/r$, $q$, or $(p+1)/r+q$ is an integer,
  providing a complete classification for the binomial integrals of
  G\&R~2.11--2.13.
\end{enumerate}

%% -------------------------------------------------------------------
\subsubsection{2.14\quad Forms containing the binomial $1\pm x^n$}
\subsubsection{2.15\quad Forms containing pairs of binomials: $a + bx$ and $\alpha +\beta x$}

\paragraph{Physics applications.}
\begin{enumerate}
\item \textbf{Scattering cross sections and angular integrals.}%
  \index{scattering cross section!angular integral}%
  \index{Rutherford scattering}%
  \index{angular integration!rational}%
  Rutherford scattering involves integrals of the form
  $\int d(\cos\theta)/(1-\cos\theta)^{2}$, an instance of
  $\int dx/(1\pm x)^{n}$ from G\&R~2.14.  More complex angular
  distributions produce paired-binomial integrands when the
  differential cross section involves two angular scales.

\item \textbf{Voltage divider and impedance networks.}%
  \index{voltage divider!integration}%
  \index{impedance!paired binomials}%
  \index{RC circuit!transfer function}%
  Transfer functions for cascaded RC networks involve rational
  expressions in two linear factors $a+bs$ and $\alpha+\beta s$.
  Inverse Laplace transforms of such expressions use exactly the
  paired-binomial decompositions catalogued in G\&R~2.15.

\item \textbf{Chemical kinetics with competing reactions.}%
  \index{chemical kinetics!competing reactions}%
  \index{rate equations!partial fractions}%
  \index{consecutive reactions}%
  Consecutive first-order reactions $A\to B\to C$ with different rate
  constants $k_{1}\neq k_{2}$ lead to integrals
  $\int dt/[(k_{1}-k_{2})e^{-k_{1}t}+\cdots]$ whose rational
  pre-images after substitution $u=e^{-t}$ involve paired binomials.
\end{enumerate}

\paragraph{Mathematics applications.}
\begin{enumerate}
\item \textbf{Cyclotomic polynomials and roots of unity.}%
  \index{cyclotomic polynomials}%
  \index{roots of unity!factorisation}%
  \index{partial fractions!over cyclotomic fields}%
  The factorisation $1-x^{n}=\prod_{d|n}\Phi_{d}(x)$ into cyclotomic
  polynomials refines the integrands of G\&R~2.14 into irreducible
  factors over~$\mathbb{Q}$.  The resulting partial fractions involve
  logarithms and arctangents evaluated at roots of unity, connecting
  indefinite integration to algebraic number theory.

\item \textbf{Heaviside cover-up method.}%
  \index{Heaviside cover-up method}%
  \index{partial fractions!distinct linear factors}%
  For distinct linear factors $(a+bx)(\alpha+\beta x)\cdots$, the
  Heaviside cover-up method evaluates each partial-fraction coefficient
  by substituting the root of the corresponding factor into the
  remaining expression.  This is the practical algorithm behind the
  formulas of G\&R~2.15.
\end{enumerate}

%% -------------------------------------------------------------------
\subsubsection{2.16\quad Forms containing the trinomial $a + bx^k + cx^{2k}$}
\subsubsection{2.17\quad Forms containing the quadratic trinomial $a + bx + cx^2$ and powers of $x$}
\subsubsection{2.18\quad Forms containing the quadratic trinomial $a + bx + cx^2$ and the binomial $\alpha +\beta x$}

\paragraph{Physics applications.}
\begin{enumerate}
\item \textbf{Resonance curves and the damped harmonic oscillator.}%
  \index{damped harmonic oscillator!integral}%
  \index{resonance!quadratic denominator}%
  \index{Lorentzian!integration}%
  The steady-state response of a damped oscillator driven at frequency
  $\omega$ involves integrals with denominator
  $(\omega_{0}^{2}-\omega^{2})^{2}+4\gamma^{2}\omega^{2}$, a
  quadratic trinomial in~$\omega^{2}$.  Completing the square and
  applying the arctangent integral (G\&R~2.17) gives the Lorentzian
  lineshape $\sim\arctan[(\omega^{2}-\omega_{0}^{2})/(2\gamma\omega)]$.

\item \textbf{Relativistic kinematics.}%
  \index{relativistic kinematics!quadratic forms}%
  \index{rapidity!integration}%
  \index{Lorentz transformation}%
  Phase-space integrals in relativistic kinematics involve
  $\int dp/\sqrt{p^{2}+m^{2}c^{2}}$, which after the substitution
  $p=mc\sinh\phi$ reduces to the rapidity variable.  More general
  two-body phase-space integrals produce quadratic trinomials in
  the momentum transfer variable.

\item \textbf{RLC circuit transient response.}%
  \index{RLC circuit!transient response}%
  \index{quadratic trinomial!circuit analysis}%
  \index{overdamped and underdamped response}%
  The characteristic equation of an RLC circuit
  $Ls^{2}+Rs+1/C=0$ has roots that determine the transient
  response.  Inverse Laplace transforms of $1/(Ls^{2}+Rs+1/C)$
  use exactly the completing-the-square technique of G\&R~2.17,
  yielding exponentially decaying sinusoids (underdamped) or pure
  exponentials (overdamped).
\end{enumerate}

\paragraph{Mathematics applications.}
\begin{enumerate}
\item \textbf{Completing the square and the Euler substitutions.}%
  \index{completing the square}%
  \index{Euler substitutions}%
  \index{quadratic form!canonical form}%
  Completing the square reduces $a+bx+cx^{2}$ to
  $c(x+b/2c)^{2}+(a-b^{2}/4c)$, unifying the integrals of G\&R~2.17
  into the two standard forms $\int du/(u^{2}+k^{2})=\frac{1}{k}\arctan(u/k)$
  and $\int du/(u^{2}-k^{2})=\frac{1}{2k}\ln|(u-k)/(u+k)|$.
  This canonical reduction is the prototype of diagonalising a
  quadratic form.

\item \textbf{Discriminant and the nature of antiderivatives.}%
  \index{discriminant!quadratic trinomial}%
  \index{arctangent!from negative discriminant}%
  \index{logarithm!from positive discriminant}%
  The sign of the discriminant $\Delta=b^{2}-4ac$ determines whether
  the antiderivative involves logarithms ($\Delta>0$, real roots),
  arctangents ($\Delta<0$, complex conjugate roots), or degenerates to
  a power-law integral ($\Delta=0$, repeated root).  This trichotomy
  is the real-variable shadow of the factorisation over~$\mathbb{C}$.

\item \textbf{Algebraic curves and genus.}%
  \index{algebraic curves!rational parametrisation}%
  \index{genus zero curves}%
  \index{rational parametrisation}%
  Every integral $\int R(x,\sqrt{ax^{2}+bx+c})\,dx$ with $R$ rational
  can be evaluated in terms of elementary functions because the curve
  $y^{2}=ax^{2}+bx+c$ is a conic (genus~0) admitting a rational
  parametrisation.  The Euler substitutions of G\&R~2.25 implement
  this parametrisation explicitly.
\end{enumerate}

%% ===================================================================
\subsection{2.2\quad Algebraic Functions}

%% -------------------------------------------------------------------
\subsubsection{2.20\quad Introduction}
\subsubsection{2.21\quad Forms containing the binomial $a + bx^k$ and $\sqrt{x}$}
\subsubsection{2.22--2.23\quad Forms containing $\sqrt[n]{(a + bx)^k}$}

\paragraph{Physics applications.}
\begin{enumerate}
\item \textbf{Kepler's equation and orbital mechanics.}%
  \index{Kepler's equation}%
  \index{orbital mechanics!algebraic integrals}%
  \index{eccentric anomaly}%
  The radial equation of Keplerian orbits
  $\int\frac{dr}{\sqrt{2(E-V(r))-\ell^{2}/r^{2}}}=t+C$ involves
  square roots of quadratic and higher-degree polynomials in~$r$.
  For inverse-square potentials $V=-k/r$, the substitution
  $r=a(1-e\cos u)$ (eccentric anomaly) reduces the integral to
  algebraic forms catalogued in G\&R~2.21--2.23.

\item \textbf{Brachistochrone and variational problems.}%
  \index{brachistochrone}%
  \index{variational problems!algebraic integrands}%
  \index{cycloid}%
  The brachistochrone problem minimises $\int_{0}^{x_{1}}\sqrt{(1+y'^{2})/(2gy)}\,dx$,
  whose Euler--Lagrange equation leads to an integral involving
  $\sqrt{y/(a-y)}$.  This is a binomial-with-square-root form from
  G\&R~2.21, and its evaluation yields the parametric cycloid solution.

\item \textbf{Thomas--Fermi screening.}%
  \index{Thomas--Fermi model}%
  \index{electron screening}%
  \index{power-law potentials!integration}%
  The Thomas--Fermi equation for atomic screening involves integrals
  $\int x^{p}(a+bx^{k})^{q}\,dx$ where the exponents arise from the
  electron density expressed as a power of the electrostatic potential.
  These are precisely the binomial integrals tabulated in G\&R~2.11--2.23.
\end{enumerate}

\paragraph{Mathematics applications.}
\begin{enumerate}
\item \textbf{Abel's theorem on algebraic integrability.}%
  \index{Abel's theorem!algebraic integrals}%
  \index{algebraic functions!integration}%
  \index{Abelian integrals}%
  Abel (1826) proved that $\int R(x,y)\,dx$ with $y$ algebraic over
  $\mathbb{C}(x)$ can always be expressed as a sum of algebraic terms,
  logarithms, and Abelian integrals (integrals on algebraic curves of
  genus~$\geq 1$).  The formulas of G\&R~2.20--2.23 enumerate the
  genus-0 cases where the result is fully elementary.

\item \textbf{Rationalising substitutions.}%
  \index{rationalising substitution}%
  \index{substitution!$t=\sqrt[n]{a+bx}$}%
  \index{rational function!after substitution}%
  The substitution $t=\sqrt[n]{a+bx}$ converts integrals involving
  $n$th roots into rational functions of~$t$, reducible by partial
  fractions.  This is the standard technique behind every formula in
  G\&R~2.22--2.23 and illustrates the principle that algebraic
  integrals of genus-0 curves are always elementary.
\end{enumerate}

%% -------------------------------------------------------------------
\subsubsection{2.24\quad Forms containing $\sqrt{a + bx}$ and the binomial $\alpha +\beta x$}
\subsubsection{2.25\quad Forms containing $\sqrt{a + bx + cx^2}$}
\subsubsection{2.26\quad Forms containing $\sqrt{a + bx + cx^2}$ and integral powers of $x$}
\subsubsection{2.27\quad Forms containing $\sqrt{a + cx^2}$ and integral powers of $x$}
\subsubsection{2.28\quad Forms containing $\sqrt{a + bx + cx^2}$ and first- and second-degree polynomials}

\paragraph{Physics applications.}
\begin{enumerate}
\item \textbf{Arc length and proper time.}%
  \index{arc length!quadratic radical}%
  \index{proper time!integration}%
  \index{geodesic!arc length}%
  The arc length $\int\sqrt{1+y'^{2}}\,dx$ and the relativistic
  proper time $\int\sqrt{1-v^{2}/c^{2}}\,dt$ are prototypical
  integrals involving $\sqrt{a+cx^{2}}$.  In general relativity, the
  geodesic equation in Schwarzschild spacetime produces integrals
  $\int dr/\sqrt{E^{2}-(1-r_{s}/r)(1+\ell^{2}/r^{2})}$ involving
  square roots of polynomials in~$r$.

\item \textbf{Central-force orbits.}%
  \index{central-force problem!orbit integral}%
  \index{orbit equation!quadratic radical}%
  \index{Binet equation}%
  The orbit equation $\theta=\int\frac{\ell\,dr}{r^{2}\sqrt{2m(E-V)-\ell^{2}/r^{2}}}$
  for a central-force potential with $V(r)$ polynomial in $1/r$
  yields square roots of quadratic trinomials upon substituting
  $u=1/r$.  The conic-section orbits of the Kepler problem emerge from
  the $\sqrt{a+bu+cu^{2}}$ integrals of G\&R~2.25.

\item \textbf{Catenary and elastica.}%
  \index{catenary}%
  \index{elastica}%
  \index{hanging chain}%
  The shape of a hanging chain satisfies
  $\int dy/\sqrt{1+(dy/dx)^{2}}=x/a$, a $\sqrt{a+cx^{2}}$ form
  (G\&R~2.27).  The elastica (thin elastic rod under load) leads
  to integrals $\int d\theta/\sqrt{a+b\cos\theta}$ that, after
  half-angle substitution, become algebraic integrals of the type
  in G\&R~2.25--2.28.

\item \textbf{Charged particle in combined electric and magnetic fields.}%
  \index{charged particle!combined fields}%
  \index{drift velocity!integration}%
  \index{$\mathbf{E}\times\mathbf{B}$ drift}%
  The trajectory of a charged particle in crossed electric and
  magnetic fields involves integrals
  $\int dt/\sqrt{a+bt+ct^{2}}$ arising from the energy conservation
  equation.  The quadratic-trinomial-under-radical forms of
  G\&R~2.25--2.28 catalogue these antiderivatives.
\end{enumerate}

\paragraph{Mathematics applications.}
\begin{enumerate}
\item \textbf{Euler substitutions.}%
  \index{Euler substitutions!three types}%
  \index{quadratic radical!rationalisation}%
  \index{conic!rational parametrisation}%
  Every integral $\int R(x,\sqrt{ax^{2}+bx+c})\,dx$ is reducible to a
  rational integral by one of Euler's three substitutions:
  $\sqrt{ax^{2}+bx+c}=t\pm x\sqrt{a}$ (if $a>0$),
  $=t\pm\sqrt{c}$ (if $c>0$), or
  $=(x-\alpha)t$ (if $\alpha$ is a real root).  These implement the
  rational parametrisation of the conic $y^{2}=ax^{2}+bx+c$ and
  underlie every formula in G\&R~2.24--2.28.

\item \textbf{Weierstrass substitution as a special case.}%
  \index{Weierstrass substitution!special case}%
  \index{half-angle substitution}%
  \index{trigonometric substitution!and radicals}%
  The trigonometric substitutions $x=a\sin\theta$, $x=a\tan\theta$,
  $x=a\sec\theta$ that eliminate $\sqrt{a^{2}-x^{2}}$,
  $\sqrt{a^{2}+x^{2}}$, $\sqrt{x^{2}-a^{2}}$ are special cases of
  the Euler substitutions when $b=0$.  They reduce the integrals of
  G\&R~2.27 to trigonometric antiderivatives (G\&R~2.5--2.6).

\item \textbf{Differential Galois theory.}%
  \index{differential Galois theory!algebraic integrals}%
  \index{Picard--Vessiot extension}%
  The fact that all integrals in G\&R~2.24--2.28 are elementary
  (no special functions needed) can be proved systematically via
  differential Galois theory: the differential Galois group of
  $y'=R(x,\sqrt{ax^{2}+bx+c})$ is solvable, guaranteeing a
  Liouvillian antiderivative.
\end{enumerate}

%% -------------------------------------------------------------------
\subsubsection{2.29\quad Integrals that can be reduced to elliptic or pseudo-elliptic integrals}

\paragraph{Physics applications.}
\begin{enumerate}
\item \textbf{Pendulum period and elliptic integrals.}%
  \index{pendulum!elliptic integral}%
  \index{elliptic integral!physical origin}%
  \index{large-amplitude oscillations}%
  The exact period of a simple pendulum
  $T=4\sqrt{\ell/g}\int_{0}^{\pi/2}d\theta/\sqrt{1-k^{2}\sin^{2}\theta}
  =4\sqrt{\ell/g}\,K(k)$ is the complete elliptic integral of the first
  kind with $k=\sin(\theta_{0}/2)$.  The corresponding indefinite
  integral is the incomplete elliptic integral $F(\phi,k)$, the
  prototype entry in G\&R~2.29.

\item \textbf{Geodesics on an ellipsoid.}%
  \index{geodesics!ellipsoid}%
  \index{ellipsoid!geodesic}%
  \index{Jacobi's theorem!geodesics}%
  The geodesic equations on the surface of an ellipsoid reduce to
  integrals involving $\sqrt{P(u)}$ where $P$ is a cubic or quartic
  polynomial---elliptic integrals.  This is the classical result of
  Jacobi (1839) and lies behind precise geodetic calculations on the
  Earth's surface.

\item \textbf{Nonlinear oscillators and Duffing's equation.}%
  \index{Duffing equation}%
  \index{nonlinear oscillator!elliptic integral}%
  \index{anharmonic oscillator}%
  The Duffing oscillator $\ddot{x}+\alpha x+\beta x^{3}=0$ conserves
  energy $E=\tfrac{1}{2}\dot{x}^{2}+\tfrac{1}{2}\alpha x^{2}
  +\tfrac{1}{4}\beta x^{4}$, so the period involves
  $\int dx/\sqrt{E-\tfrac{1}{2}\alpha x^{2}-\tfrac{1}{4}\beta x^{4}}$,
  an elliptic integral from G\&R~2.29.
\end{enumerate}

\paragraph{Mathematics applications.}
\begin{enumerate}
\item \textbf{Elliptic curves and genus-1 integrals.}%
  \index{elliptic curves!genus one}%
  \index{genus!elliptic integrals}%
  \index{Weierstrass normal form}%
  An integral $\int R(x,\sqrt{P(x)})\,dx$ with $P$ of degree~3 or~4
  generically defines an elliptic curve of genus~1.  Such integrals
  cannot be expressed in elementary functions (Liouville--Abel), and
  their inversion leads to elliptic functions (G\&R~8.1).

\item \textbf{Pseudo-elliptic integrals.}%
  \index{pseudo-elliptic integrals}%
  \index{algebraic miracle!pseudo-elliptic}%
  \index{Abel's addition theorem!pseudo-elliptic}%
  A pseudo-elliptic integral looks elliptic (involves $\sqrt{P(x)}$
  with $\deg P\geq 3$) but is actually elementary due to hidden
  algebraic relations.  Detecting these requires Abel's addition
  theorem or Risch-type algorithms.  G\&R~2.29 includes both genuinely
  elliptic and pseudo-elliptic cases, distinguished by the algebraic
  structure of the integrand.

\item \textbf{Hyperelliptic integrals and Abelian varieties.}%
  \index{hyperelliptic integrals}%
  \index{Abelian varieties}%
  \index{Jacobian variety}%
  When $\deg P\geq 5$, the integral $\int dx/\sqrt{P(x)}$ defines a
  hyperelliptic curve of genus $g=\lfloor(\deg P-1)/2\rfloor$.
  Inversion leads to Abelian functions on a $g$-dimensional torus
  (the Jacobian variety), a vast generalisation of elliptic functions.
\end{enumerate}

%% ===================================================================
\subsection{2.3\quad The Exponential Function}

%% -------------------------------------------------------------------
\subsubsection{2.31\quad Forms containing $e^{ax}$}
\subsubsection{2.32\quad The exponential combined with rational functions of $x$}

\paragraph{Physics applications.}
\begin{enumerate}
\item \textbf{Radioactive decay and reaction kinetics.}%
  \index{radioactive decay!integration}%
  \index{first-order kinetics}%
  \index{exponential decay!antiderivative}%
  First-order kinetics $dN/dt=-\lambda N$ gives $N(t)=N_{0}e^{-\lambda t}$
  after separation and integration of $\int dN/N=-\lambda\int dt$.
  The cumulative number of decays $\int_{0}^{t}|\dot{N}|\,dt'
  =N_{0}(1-e^{-\lambda t})$ is the prototype of G\&R~2.31.  Bateman's
  equations for decay chains involve sums of exponentials whose
  coefficients require the rational-times-exponential integrals
  of G\&R~2.32.

\item \textbf{Quantum tunnelling amplitudes.}%
  \index{quantum tunnelling!WKB integral}%
  \index{WKB approximation}%
  \index{transmission coefficient}%
  The WKB tunnelling amplitude
  $T\sim\exp\!\bigl(-\frac{2}{\hbar}\int_{x_{1}}^{x_{2}}
  \sqrt{2m(V-E)}\,dx\bigr)$ involves an exponential of an integral.
  When the barrier is approximated by polynomials, the inner integral
  reduces to algebraic antiderivatives (G\&R~2.2), and the overall
  expression involves exponentials combined with powers.

\item \textbf{Damped oscillations and signal processing.}%
  \index{damped oscillation!exponential integral}%
  \index{signal processing!windowed integrals}%
  \index{Gaussian window}%
  Products $x^{n}e^{ax}$ arise when computing moments of exponentially
  decaying signals.  The reduction formula
  $\int x^{n}e^{ax}\,dx=\frac{x^{n}e^{ax}}{a}
  -\frac{n}{a}\int x^{n-1}e^{ax}\,dx$ (G\&R~2.32) is used
  repeatedly in signal processing to evaluate windowed integrals and
  to derive the moments of the gamma distribution.

\item \textbf{Partition functions in statistical mechanics.}%
  \index{partition function!integration}%
  \index{statistical mechanics!exponential integrals}%
  \index{Boltzmann factor}%
  The canonical partition function
  $Z=\int e^{-\beta H(q,p)}\,dq\,dp$ involves exponentials of the
  Hamiltonian.  For harmonic or polynomial potentials, the multidimensional
  integral factorises into products of one-dimensional integrals of the
  type $\int x^{n}e^{-ax^{2}}\,dx$, linking G\&R~2.31--2.32 to
  the Gaussian integrals of G\&R~3.32.
\end{enumerate}

\paragraph{Mathematics applications.}
\begin{enumerate}
\item \textbf{The exponential function and Lie groups.}%
  \index{exponential map!Lie group}%
  \index{Lie group!exponential}%
  \index{matrix exponential}%
  The matrix exponential $e^{tA}=\sum_{n=0}^{\infty}(tA)^{n}/n!$ is
  the solution to $\dot{X}=AX$, obtained by integrating the constant-coefficient
  ODE.  The scalar integrals $\int e^{ax}\,dx=e^{ax}/a$ are the
  one-dimensional case.  The Baker--Campbell--Hausdorff formula
  $\ln(e^{A}e^{B})=A+B+\tfrac{1}{2}[A,B]+\cdots$ generalises the
  addition rule $e^{a}e^{b}=e^{a+b}$ to non-commuting generators.

\item \textbf{Laplace and Fourier transforms.}%
  \index{Laplace transform!as indefinite integral}%
  \index{Fourier transform!exponential kernel}%
  \index{integral transform!exponential}%
  The Laplace transform $\hat{f}(s)=\int_{0}^{\infty}f(t)e^{-st}\,dt$
  converts convolutions to products and differential equations to
  algebraic ones.  Its building blocks are the indefinite integrals
  $\int t^{n}e^{-st}\,dt$ from G\&R~2.32, and its inversion (the
  Bromwich integral) closes the circle back to exponential
  antiderivatives.

\item \textbf{Asymptotic expansions and Watson's lemma.}%
  \index{Watson's lemma}%
  \index{asymptotic expansion!Laplace integrals}%
  \index{Laplace method}%
  Watson's lemma gives the asymptotic expansion of
  $\int_{0}^{\infty}t^{\alpha}e^{-\lambda t}f(t)\,dt$ as
  $\lambda\to\infty$ by expanding $f$ and integrating term by term
  using $\int t^{n+\alpha}e^{-\lambda t}\,dt=\Gamma(n+\alpha+1)/\lambda^{n+\alpha+1}$.
  The individual antiderivatives are instances of G\&R~2.32.
\end{enumerate}

%% ===================================================================
\subsection{2.4\quad Hyperbolic Functions}

%% -------------------------------------------------------------------
\subsubsection{2.41--2.43\quad Powers of $\sinh x$, $\cosh x$, $\tanh x$, and $\coth x$}
\subsubsection{2.44--2.45\quad Rational functions of hyperbolic functions}

\paragraph{Physics applications.}
\begin{enumerate}
\item \textbf{Relativistic velocity addition and rapidity.}%
  \index{rapidity!hyperbolic functions}%
  \index{relativistic velocity addition}%
  \index{Lorentz boost!rapidity}%
  The rapidity $\phi=\operatorname{arctanh}(v/c)$ linearises Lorentz
  boosts: rapidities add, $\phi_{12}=\phi_{1}+\phi_{2}$.  The
  energy--momentum relations $E=mc^{2}\cosh\phi$,
  $p=mc\sinh\phi$ make $\int\cosh\phi\,d\phi=\sinh\phi$ and
  $\int\sinh\phi\,d\phi=\cosh\phi$ (G\&R~2.41) the basic kinematic
  integrals of special relativity.

\item \textbf{Catenary and suspension bridges.}%
  \index{catenary!hyperbolic cosine}%
  \index{suspension bridge!cable shape}%
  \index{surface of revolution!catenoid}%
  The catenary $y=a\cosh(x/a)$ is the shape of an ideal hanging chain
  under uniform gravitational load.  Its arc length
  $\int\sqrt{1+\sinh^{2}(x/a)}\,dx=a\sinh(x/a)$ and area of the
  catenoid (minimal surface of revolution) both reduce to the
  hyperbolic antiderivatives of G\&R~2.41.

\item \textbf{Solitons and the $\operatorname{sech}^{2}$ potential.}%
  \index{soliton!$\operatorname{sech}^2$ potential}%
  \index{KdV equation}%
  \index{reflectionless potential}%
  The one-soliton solution of the Korteweg--de Vries equation is
  $u(x,t)=-2\kappa^{2}\operatorname{sech}^{2}(\kappa(x-4\kappa^{2}t))$.
  Energy and momentum integrals of this solution involve
  $\int\operatorname{sech}^{2n}(x)\,dx$ and
  $\int\operatorname{sech}^{2n}(x)\tanh^{m}(x)\,dx$, all catalogued
  in G\&R~2.41--2.43.

\item \textbf{Fermi--Dirac and Bose--Einstein integrals.}%
  \index{Fermi--Dirac distribution!integration}%
  \index{Bose--Einstein distribution}%
  \index{polylogarithm!from hyperbolic integrals}%
  Thermal occupation numbers $(e^{(\varepsilon-\mu)/kT}\pm1)^{-1}$
  can be rewritten in terms of $\coth$ and $\tanh$.  Integrals of
  rational functions of $\sinh$ and $\cosh$ from G\&R~2.44--2.45
  appear in the thermodynamics of quantum gases, ultimately connecting
  to polylogarithms and the Riemann zeta function.
\end{enumerate}

\paragraph{Mathematics applications.}
\begin{enumerate}
\item \textbf{Hyperbolic--exponential duality.}%
  \index{hyperbolic functions!exponential representation}%
  \index{$\sinh$ and $\cosh$!as exponentials}%
  \index{Osborn's rule}%
  Since $\sinh x=(e^{x}-e^{-x})/2$ and $\cosh x=(e^{x}+e^{-x})/2$,
  every hyperbolic antiderivative in G\&R~2.41--2.45 can be rewritten
  as an exponential integral (G\&R~2.31--2.32) and vice versa.
  Osborn's rule translates trigonometric identities to hyperbolic ones
  by replacing $\sin\to i\sinh$, $\cos\to\cosh$, converting the
  formulas of G\&R~2.5 to those of G\&R~2.4.

\item \textbf{Weierstrass-type substitution for hyperbolics.}%
  \index{Weierstrass substitution!hyperbolic}%
  \index{$t=\tanh(x/2)$ substitution}%
  \index{rational parametrisation!hyperbola}%
  The substitution $t=\tanh(x/2)$ gives $\sinh x=2t/(1-t^{2})$,
  $\cosh x=(1+t^{2})/(1-t^{2})$, $dx=2\,dt/(1-t^{2})$, reducing
  rational functions of $\sinh$ and $\cosh$ to rational functions of~$t$.
  This is the hyperbolic analogue of the Weierstrass substitution
  $t=\tan(x/2)$ for trigonometric integrals (G\&R~2.55).

\item \textbf{Reduction formulas and recursion relations.}%
  \index{reduction formulas!hyperbolic powers}%
  \index{recursion!hyperbolic integrals}%
  The reduction formula $\int\sinh^{n}x\,dx
  =\frac{\sinh^{n-1}x\cosh x}{n}-\frac{n-1}{n}\int\sinh^{n-2}x\,dx$
  is a two-term recursion in~$n$, solved by the same techniques as
  difference equations.  The closed forms involve binomial coefficients
  and connect to the beta function $B(p,q)$ through the substitution
  $u=\cosh x$.
\end{enumerate}

%% -------------------------------------------------------------------
\subsubsection{2.46\quad Algebraic functions of hyperbolic functions}
\subsubsection{2.47\quad Combinations of hyperbolic functions and powers}
\subsubsection{2.48\quad Combinations of hyperbolic functions, exponentials, and powers}

\paragraph{Physics applications.}
\begin{enumerate}
\item \textbf{Magnetic susceptibility and the Langevin function.}%
  \index{Langevin function}%
  \index{paramagnetism!Langevin model}%
  \index{magnetic susceptibility}%
  The Langevin function $\mathcal{L}(x)=\coth x-1/x$ describes
  classical paramagnetism.  Thermodynamic quantities such as the
  susceptibility involve integrals $\int x^{n}\coth(x)\,dx$ and
  $\int x^{n}/\sinh(x)\,dx$, combinations of hyperbolic functions
  and powers from G\&R~2.47.

\item \textbf{Black-body radiation: Planck spectrum moments.}%
  \index{Planck spectrum!integration}%
  \index{black-body radiation}%
  \index{Stefan--Boltzmann law!derivation}%
  The Planck spectral density involves $x^{3}/(e^{x}-1)$,
  expressible via $\coth(x/2)-1$.  Moments
  $\int x^{n}/(e^{x}-1)\,dx$ are combinations of exponentials,
  powers, and hyperbolic functions (G\&R~2.48) whose definite-integral
  counterparts yield the Stefan--Boltzmann law and Wien's displacement
  law.

\item \textbf{Transmission-line theory.}%
  \index{transmission line!hyperbolic functions}%
  \index{characteristic impedance}%
  \index{standing wave ratio}%
  Voltage and current on a lossy transmission line are expressed in
  terms of $\cosh(\gamma z)$ and $\sinh(\gamma z)$.  Power integrals
  along the line involve products $\sinh(\gamma z)\cosh(\gamma z)$
  and $x^{n}\sinh(\gamma x)$, catalogued in G\&R~2.46--2.48.
\end{enumerate}

\paragraph{Mathematics applications.}
\begin{enumerate}
\item \textbf{Bernoulli numbers from generating functions.}%
  \index{Bernoulli numbers!generating function}%
  \index{$x/\sinh x$ expansion}%
  \index{generating function!Bernoulli numbers}%
  The function $x/\sinh x=\sum_{n=0}^{\infty}(2-2^{2n})B_{2n}x^{2n}/(2n)!$
  generates the Bernoulli numbers.  Term-by-term integration of this
  expansion yields power series for the integrals
  $\int x^{m}/\sinh^{n}x\,dx$, connecting G\&R~2.46--2.47 to
  the number-theoretic properties of Bernoulli numbers and
  values of the Riemann zeta function at even integers.

\item \textbf{Elliptic-function degeneration.}%
  \index{elliptic functions!degeneration to hyperbolic}%
  \index{modular parameter!$k\to 1$}%
  \index{Jacobi elliptic functions!limits}%
  As the elliptic modulus $k\to 1$, the Jacobi elliptic functions
  degenerate: $\operatorname{sn}(u,k)\to\tanh u$,
  $\operatorname{cn}(u,k)\to\operatorname{sech}u$,
  $\operatorname{dn}(u,k)\to\operatorname{sech}u$.
  Consequently, the elliptic integrals of G\&R~2.58--2.62 reduce to
  the hyperbolic integrals of G\&R~2.41--2.48 in this limit.
\end{enumerate}

%% ===================================================================
\subsection{2.5--2.6\quad Trigonometric Functions}

%% -------------------------------------------------------------------
\subsubsection{2.50\quad Introduction}
\subsubsection{2.51--2.52\quad Powers of trigonometric functions}

\paragraph{Physics applications.}
\begin{enumerate}
\item \textbf{Intensity patterns in optics.}%
  \index{diffraction!intensity integral}%
  \index{interference pattern}%
  \index{Malus's law}%
  Fraunhofer diffraction from a single slit gives an intensity
  $I\propto\operatorname{sinc}^{2}(\beta)$.  Averaging over angles and
  computing total power through apertures involves integrals
  $\int\sin^{2n}\theta\,d\theta$ and $\int\cos^{2n}\theta\,d\theta$
  (G\&R~2.51), evaluated by the standard reduction formulas.  Malus's
  law $I=I_{0}\cos^{2}\theta$ for polarised light is the simplest case.

\item \textbf{Action-angle variables in celestial mechanics.}%
  \index{action-angle variables}%
  \index{celestial mechanics!trigonometric integrals}%
  \index{secular perturbation theory}%
  In secular perturbation theory for planetary orbits, the disturbing
  function is expanded in powers of $\sin(i)$ and $\cos(i)$ (orbital
  inclination), and averaging over the fast angle yields integrals
  $\int\sin^{m}\theta\cos^{n}\theta\,d\theta$ from G\&R~2.51--2.52.

\item \textbf{Angular distribution in scattering.}%
  \index{angular distribution!scattering}%
  \index{partial-wave expansion}%
  \index{Legendre polynomials!trigonometric powers}%
  Scattering cross sections in partial-wave analysis involve integrals
  of $P_{\ell}(\cos\theta)P_{\ell'}(\cos\theta)\sin\theta$ over the
  solid angle.  Since Legendre polynomials are polynomials in
  $\cos\theta$, these reduce to the power-of-cosine integrals of
  G\&R~2.51--2.52.
\end{enumerate}

\paragraph{Mathematics applications.}
\begin{enumerate}
\item \textbf{Wallis's product and the beta function.}%
  \index{Wallis product}%
  \index{beta function!Wallis integral}%
  \index{reduction formula!trigonometric powers}%
  The definite integral $\int_{0}^{\pi/2}\sin^{n}\theta\,d\theta$
  satisfies a two-term recursion whose ratio of consecutive values
  yields Wallis's product $\pi/2=\prod_{n=1}^{\infty}4n^{2}/(4n^{2}-1)$.
  The indefinite antiderivatives in G\&R~2.51 are the building blocks;
  the connection to the beta function
  $B(p,q)=2\int_{0}^{\pi/2}\sin^{2p-1}\theta\cos^{2q-1}\theta\,d\theta$
  unifies these formulas.

\item \textbf{Chebyshev polynomials and trigonometric identities.}%
  \index{Chebyshev polynomials!from trigonometric powers}%
  \index{trigonometric identities!power reduction}%
  \index{linearisation formulas}%
  The power-reduction formulas
  $\cos^{n}\theta=\sum_{k}a_{k}\cos(k\theta)$ and their sine
  analogues express powers as linear combinations of multiple-angle
  functions.  These are equivalent to the expansion in Chebyshev
  polynomials $T_{n}(\cos\theta)=\cos(n\theta)$, and they reduce
  the integrals of G\&R~2.51--2.52 to those of G\&R~2.53--2.54.
\end{enumerate}

%% -------------------------------------------------------------------
\subsubsection{2.53--2.54\quad Sines and cosines of multiple angles and of linear and more complicated functions of the argument}
\subsubsection{2.55--2.56\quad Rational functions of the sine and cosine}

\paragraph{Physics applications.}
\begin{enumerate}
\item \textbf{Fourier analysis of periodic signals.}%
  \index{Fourier series!coefficient integrals}%
  \index{orthogonality!trigonometric system}%
  \index{signal decomposition}%
  The Fourier coefficients $a_{n}=\frac{1}{\pi}\int f(x)\cos(nx)\,dx$,
  $b_{n}=\frac{1}{\pi}\int f(x)\sin(nx)\,dx$ are indefinite integrals
  of products of sines and cosines at different frequencies (G\&R~2.53).
  The orthogonality relation
  $\int\sin(mx)\cos(nx)\,dx$ underlies the entire edifice of spectral
  analysis, from acoustics to quantum mechanics.

\item \textbf{Phase-sensitive detection (lock-in amplifiers).}%
  \index{lock-in amplifier}%
  \index{phase-sensitive detection}%
  \index{product-to-sum formulas}%
  A lock-in amplifier multiplies a signal by a reference sine wave
  and integrates: $\int V(t)\sin(\omega_{r}t+\phi)\,dt$.  The
  product-to-sum formula
  $\sin(A)\sin(B)=\tfrac{1}{2}[\cos(A-B)-\cos(A+B)]$ (the identity
  behind G\&R~2.53) isolates the component at the reference frequency
  from broadband noise.

\item \textbf{Geometric optics and ray tracing.}%
  \index{Snell's law!integration}%
  \index{ray tracing!trigonometric integrals}%
  \index{graded-index fibre}%
  Ray paths in graded-index optical fibres satisfy
  $\int d\theta/(n^{2}(\theta)-\text{const})$, where $n(\theta)$ is a
  trigonometric function of the ray angle.  Rational functions of
  $\sin\theta$ and $\cos\theta$ arise naturally, and the Weierstrass
  substitution $t=\tan(\theta/2)$ of G\&R~2.55 converts these to
  rational integrals (G\&R~2.1).
\end{enumerate}

\paragraph{Mathematics applications.}
\begin{enumerate}
\item \textbf{The Weierstrass substitution.}%
  \index{Weierstrass substitution!$t=\tan(x/2)$}%
  \index{universal trigonometric substitution}%
  \index{rational parametrisation!unit circle}%
  The substitution $t=\tan(x/2)$ gives $\sin x=2t/(1+t^{2})$,
  $\cos x=(1-t^{2})/(1+t^{2})$, $dx=2\,dt/(1+t^{2})$, converting
  every rational function of $\sin x$ and $\cos x$ into a rational
  function of~$t$.  This is the rational parametrisation of the unit
  circle and the universal method behind G\&R~2.55--2.56.

\item \textbf{Dirichlet kernel and summability.}%
  \index{Dirichlet kernel}%
  \index{Fej\'er kernel}%
  \index{summability!trigonometric series}%
  The Dirichlet kernel
  $D_{N}(x)=\sum_{n=-N}^{N}e^{inx}=\sin((N+\tfrac{1}{2})x)/\sin(x/2)$
  is a rational function of $\sin$ and $\cos$.  Its integral
  $\int_{0}^{x}D_{N}(t)\,dt$ connects to the partial sums of Fourier
  series and to the convergence theory of trigonometric series.

\item \textbf{Eisenstein series and modular forms.}%
  \index{Eisenstein series}%
  \index{modular forms!trigonometric sums}%
  \index{cotangent sums}%
  Partial-fraction expansions of $\cot(\pi z)$ and
  $\csc^{2}(\pi z)$ produce series that, when integrated, yield
  the logarithmic integrals $\int\ln\sin x\,dx$ and related
  expressions.  These connect to the Eisenstein series $G_{2k}(\tau)$
  of modular form theory via the $q$-expansion.
\end{enumerate}

%% -------------------------------------------------------------------
\subsubsection{2.57\quad Integrals containing $\sqrt{a\pm b\sin x}$ or $\sqrt{a\pm b\cos x}$}
\subsubsection{2.58--2.62\quad Integrals reducible to elliptic and pseudo-elliptic integrals}

\paragraph{Physics applications.}
\begin{enumerate}
\item \textbf{Pendulum beyond small angles.}%
  \index{pendulum!exact solution}%
  \index{elliptic integral!from trigonometric radical}%
  \index{Jacobi amplitude}%
  The pendulum integral $\int d\theta/\sqrt{a-b\cos\theta}$ (G\&R~2.57)
  is, after the substitution $\sin(\theta/2)=k\sin\phi$, exactly the
  incomplete elliptic integral of the first kind $F(\phi,k)$.  The
  inversion $\theta(t)=2\operatorname{am}(\omega t,k)$ gives the exact
  angular motion in terms of the Jacobi amplitude function.

\item \textbf{Magnetic field of a circular loop.}%
  \index{magnetic field!circular loop}%
  \index{Biot--Savart law!elliptic integral}%
  \index{mutual inductance!elliptic integral}%
  The Biot--Savart integral for the magnetic field of a circular
  current loop involves $\int d\phi/\sqrt{a+b\cos\phi}$, an elliptic
  integral from G\&R~2.57.  Mutual inductance between coaxial loops
  (Neumann's formula) similarly reduces to complete elliptic integrals
  $K(k)$ and $E(k)$.

\item \textbf{Geodesics on surfaces of revolution.}%
  \index{geodesics!surface of revolution}%
  \index{Clairaut's relation}%
  \index{torus!geodesics}%
  Clairaut's relation $r\cos\alpha=\text{const}$ for geodesics on a
  surface of revolution leads to integrals
  $\int d\theta/\sqrt{f(\theta)-c^{2}}$ where $f$ involves
  trigonometric functions of the latitude angle, producing
  trigonometric-radical forms from G\&R~2.57 that are generically
  elliptic.
\end{enumerate}

\paragraph{Mathematics applications.}
\begin{enumerate}
\item \textbf{Reduction to Legendre normal form.}%
  \index{Legendre normal form!elliptic integral}%
  \index{elliptic integrals!three kinds}%
  \index{reduction to standard form}%
  The integrals of G\&R~2.58--2.62 reduce, by linear-fractional or
  trigonometric substitutions, to the three standard Legendre forms:
  $F(\phi,k)$, $E(\phi,k)$, and $\Pi(\phi,n,k)$---elliptic integrals
  of the first, second, and third kinds.  This reduction is the
  classical programme of Legendre (1825) and Jacobi (1829).

\item \textbf{Arithmetic-geometric mean and fast computation.}%
  \index{arithmetic-geometric mean}%
  \index{Gauss AGM}%
  \index{elliptic integral!fast computation}%
  The complete elliptic integral $K(k)=\pi/(2\,\mathrm{AGM}(1,k'))$
  is computed to arbitrary precision by the arithmetic-geometric mean
  iteration, converging quadratically.  This makes the evaluation of
  the indefinite elliptic integrals in G\&R~2.58--2.62 practical for
  numerical work.

\item \textbf{Uniformisation of elliptic curves.}%
  \index{uniformisation!elliptic curve}%
  \index{Weierstrass $\wp$-function!uniformisation}%
  \index{Abel--Jacobi map}%
  Inverting the elliptic integral $u=\int_{z_{0}}^{z}R(t,\sqrt{P(t)})\,dt$
  yields the Weierstrass $\wp$-function $z=\wp(u)$, which uniformises
  the elliptic curve $y^{2}=P(x)$.  The Abel--Jacobi map
  $z\mapsto u$ identifies the curve with the complex torus
  $\mathbb{C}/\Lambda$.
\end{enumerate}

%% -------------------------------------------------------------------
\subsubsection{2.63--2.65\quad Products of trigonometric functions and powers}
\subsubsection{2.66\quad Combinations of trigonometric functions and exponentials}
\subsubsection{2.67\quad Combinations of trigonometric and hyperbolic functions}

\paragraph{Physics applications.}
\begin{enumerate}
\item \textbf{Multipole moments and radiation patterns.}%
  \index{multipole expansion!angular integrals}%
  \index{radiation pattern!trigonometric-power integrals}%
  \index{antenna theory}%
  The radiation power pattern of a multipole of order~$\ell$ is
  $\int|Y_{\ell}^{m}(\theta,\phi)|^{2}\sin\theta\,d\theta$, which
  reduces to $\int\sin^{2\ell+1}\theta\,P_{\ell}^{m}(\cos\theta)^{2}
  \,d\theta$---a product of trigonometric functions and powers
  (G\&R~2.63--2.65).  Antenna directivity and radar cross sections
  involve the same type of integrals.

\item \textbf{Damped oscillations: $e^{ax}\sin(bx)$ and $e^{ax}\cos(bx)$.}%
  \index{damped oscillation!trig-exponential integral}%
  \index{phasor method}%
  \index{resonance!transient response}%
  The transient response of any underdamped linear system is a sum of
  terms $e^{-\gamma t}\sin(\omega t+\phi)$.  Integrals
  $\int x^{n}e^{ax}\sin(bx)\,dx$ and $\int x^{n}e^{ax}\cos(bx)\,dx$
  from G\&R~2.66 give the impulse response, step response, and energy
  dissipated in such systems.  The phasor method---replacing
  $\sin(bx)$ by $\operatorname{Im}(e^{ibx})$---unifies these into
  complex-exponential integrals.

\item \textbf{Waveguide mode coupling.}%
  \index{waveguide!mode coupling}%
  \index{coupled-mode theory}%
  \index{trigonometric-hyperbolic products}%
  In tapered or lossy waveguides, coupling between propagating and
  evanescent modes involves overlap integrals of the form
  $\int\sin(n\pi x/a)\sinh(\kappa x)\,dx$---products of trigonometric
  and hyperbolic functions (G\&R~2.67).  These arise wherever
  oscillatory and exponentially growing/decaying modes coexist, as in
  optical couplers and tunnelling junctions.

\item \textbf{AC circuit power with harmonics.}%
  \index{AC circuits!power integral}%
  \index{harmonic distortion!power}%
  \index{$\sin\cdot\cos$ products}%
  The average power in an AC circuit with harmonic distortion is
  $P=\frac{1}{T}\int_{0}^{T}v(t)\,i(t)\,dt$, where $v$ and $i$
  are sums of sinusoids at different harmonics.  The cross terms
  involve products $\sin(m\omega t)\cos(n\omega t)$---the integrals
  of G\&R~2.63 vanish for $m\neq n$ (orthogonality) and give the
  per-harmonic power for $m=n$.
\end{enumerate}

\paragraph{Mathematics applications.}
\begin{enumerate}
\item \textbf{Integration by parts and exponential-trigonometric integrals.}%
  \index{integration by parts!exponential-trigonometric}%
  \index{complex exponential method}%
  \index{Euler's formula!integration}%
  The integral $\int e^{ax}\sin(bx)\,dx$ is most elegantly evaluated
  by writing $\sin(bx)=\operatorname{Im}(e^{ibx})$ and integrating
  $\int e^{(a+ib)x}\,dx=e^{(a+ib)x}/(a+ib)$.  Separating real and
  imaginary parts simultaneously gives both sine and cosine integrals.
  This complex-exponential method extends to all formulas in
  G\&R~2.66.

\item \textbf{Orthogonality and Hilbert space structure.}%
  \index{orthogonality!trigonometric functions}%
  \index{$L^2$ inner product}%
  \index{Hilbert space!trigonometric basis}%
  The system $\{1,\cos(nx),\sin(nx)\}_{n\geq 1}$ is a complete
  orthogonal basis for $L^{2}([0,2\pi])$, and the orthogonality
  relations are verified by the product integrals of G\&R~2.53
  and~2.63.  Completeness (Parseval's theorem) asserts that every
  square-integrable function is determined by its Fourier coefficients.

\item \textbf{Laplace transform of trigonometric functions.}%
  \index{Laplace transform!of $\sin$ and $\cos$}%
  \index{transfer function!trigonometric}%
  The Laplace transforms $\mathcal{L}\{e^{at}\sin(bt)\}=b/((s-a)^{2}+b^{2})$
  and $\mathcal{L}\{e^{at}\cos(bt)\}=(s-a)/((s-a)^{2}+b^{2})$ are
  obtained by integrating the exponential-trigonometric products of
  G\&R~2.66 from $0$ to $\infty$.  These are the transfer-function
  building blocks for all second-order linear systems.
\end{enumerate}

%% ===================================================================
\subsection{2.7\quad Logarithms and Inverse-Hyperbolic Functions}

%% -------------------------------------------------------------------
\subsubsection{2.71\quad The logarithm}
\subsubsection{2.72--2.73\quad Combinations of logarithms and algebraic functions}
\subsubsection{2.74\quad Inverse hyperbolic functions}

\paragraph{Physics applications.}
\begin{enumerate}
\item \textbf{Entropy and information theory.}%
  \index{entropy!logarithmic integral}%
  \index{Shannon entropy}%
  \index{information theory!integration}%
  The Shannon entropy $H=-\int p(x)\ln p(x)\,dx$ is the continuous
  analogue of $-\sum p_{i}\ln p_{i}$.  Computing $H$ for standard
  distributions (exponential, Gaussian, beta) requires the
  antiderivatives $\int x^{n}\ln x\,dx$ and
  $\int\ln(a+bx)\,dx$ from G\&R~2.71--2.72.  In statistical mechanics,
  the Boltzmann entropy $S=k_{B}\ln\Omega$ connects the logarithm to
  the second law of thermodynamics.

\item \textbf{Rocket equation and logarithmic mass ratio.}%
  \index{Tsiolkovsky rocket equation}%
  \index{rocket equation!logarithmic}%
  \index{mass ratio}%
  The Tsiolkovsky rocket equation $\Delta v=v_{e}\ln(m_{0}/m_{f})$
  follows from $\int dv=v_{e}\int dm/m$, the simplest logarithmic
  integral.  Optimal staging problems involve $\int\ln(a+bx)\,dx$
  and products of logarithms with polynomials (G\&R~2.72).

\item \textbf{Electrostatic potential of line charges.}%
  \index{electrostatic potential!line charge}%
  \index{line charge!logarithmic potential}%
  \index{capacitance!per unit length}%
  The potential of an infinite line charge is
  $\Phi=-(\lambda/2\pi\varepsilon_{0})\ln r$, and the capacitance per
  unit length of coaxial conductors involves $\int dr/(r)=\ln r$.
  More complex geometries produce integrals
  $\int\ln(a+bx+cx^{2})\,dx$ from G\&R~2.72--2.73.

\item \textbf{Relativistic Doppler effect and rapidity.}%
  \index{Doppler effect!relativistic}%
  \index{inverse hyperbolic functions!rapidity}%
  \index{rapidity!inverse hyperbolic}%
  The rapidity $\phi=\operatorname{arctanh}(v/c)
  =\tfrac{1}{2}\ln[(1+v/c)/(1-v/c)]$ connects the inverse hyperbolic
  functions of G\&R~2.74 to the logarithms of G\&R~2.71.  The
  relativistic Doppler factor $\sqrt{(1+\beta)/(1-\beta)}=e^{\phi}$
  shows that rapidity is the natural logarithmic measure of relativistic
  velocity.
\end{enumerate}

\paragraph{Mathematics applications.}
\begin{enumerate}
\item \textbf{The logarithmic integral and prime number theorem.}%
  \index{logarithmic integral $\operatorname{li}(x)$}%
  \index{prime number theorem}%
  \index{prime counting function}%
  The logarithmic integral
  $\operatorname{li}(x)=\int_{0}^{x}dt/\ln t$ approximates the prime
  counting function $\pi(x)$, the central result of analytic number
  theory.  The antiderivative $\int dx/\ln x$ is not elementary
  (Liouville), illustrating the boundary between G\&R sections~2
  (elementary antiderivatives) and~5 (special-function antiderivatives).

\item \textbf{Polylogarithms and iterated integrals.}%
  \index{polylogarithm!iterated integral}%
  \index{iterated integrals}%
  \index{dilogarithm}%
  The dilogarithm $\operatorname{Li}_{2}(x)=-\int_{0}^{x}\ln(1-t)/t\,dt$
  is an iterated integral of two logarithmic forms.  Higher
  polylogarithms $\operatorname{Li}_{n}(x)$ arise from deeper iterations.
  The basic logarithmic antiderivatives of G\&R~2.71--2.72 are the
  building blocks of this hierarchy, which appears throughout
  perturbative quantum field theory.

\item \textbf{Inverse hyperbolic functions as logarithms.}%
  \index{inverse hyperbolic functions!logarithmic form}%
  \index{$\operatorname{arcsinh}$!$=\ln(x+\sqrt{x^2+1})$}%
  \index{$\operatorname{arctanh}$!$=\frac{1}{2}\ln\frac{1+x}{1-x}$}%
  The identities $\operatorname{arcsinh}(x)=\ln(x+\sqrt{x^{2}+1})$,
  $\operatorname{arccosh}(x)=\ln(x+\sqrt{x^{2}-1})$, and
  $\operatorname{arctanh}(x)=\tfrac{1}{2}\ln((1+x)/(1-x))$ show that
  G\&R~2.74 is a notational variant of G\&R~2.71--2.73 combined with
  algebraic functions.  The inverse hyperbolic notation is more natural
  when the argument arises from a hyperbolic substitution.
\end{enumerate}

%% ===================================================================
\subsection{2.8\quad Inverse Trigonometric Functions}

%% -------------------------------------------------------------------
\subsubsection{2.81\quad Arcsines and arccosines}
\subsubsection{2.82\quad The arcsecant, the arccosecant, the arctangent, and the arccotangent}

\paragraph{Physics applications.}
\begin{enumerate}
\item \textbf{Phase shifts in scattering theory.}%
  \index{phase shift!scattering}%
  \index{scattering theory!arctangent}%
  \index{Born approximation}%
  The $s$-wave scattering phase shift is
  $\delta_{0}(k)=\arctan(-ka)$ (scattering length~$a$), and higher
  partial waves contribute $\delta_{\ell}(k)=\arctan(f_{\ell}(k))$.
  Energy integrals of phase shifts, e.g.,
  $\int\delta_{\ell}(k)\,dk$ and $\int k\,\arctan(k/k_{0})\,dk$,
  involve the inverse-trigonometric integrals of G\&R~2.81--2.82.

\item \textbf{Geometric optics: angles of refraction and reflection.}%
  \index{Snell's law!arcsine}%
  \index{refraction!arcsine integral}%
  \index{critical angle}%
  Snell's law $n_{1}\sin\theta_{1}=n_{2}\sin\theta_{2}$ gives
  $\theta_{2}=\arcsin(n_{1}\sin\theta_{1}/n_{2})$.  Ray-tracing
  through a graded-index medium integrates angle changes:
  $\int\arcsin(n(x)/n_{0})\,dx$, an arcsine-with-algebraic-function
  integral from G\&R~2.81.

\item \textbf{Control theory: phase margin.}%
  \index{phase margin!arctangent}%
  \index{Bode plot!phase}%
  \index{control theory!stability}%
  The phase of a transfer function $G(i\omega)$ involves
  $\arg(G)=\sum_{k}\arctan(\omega/\omega_{k})$.  The gain--phase
  relation $\int_{0}^{\infty}\frac{d(\ln|G|)}{d\omega}\ln\omega\,d\omega
  =\frac{\pi}{2}\arg(G)$ (Bode's integral) connects arctangent
  functions to logarithmic integrals, intertwining G\&R~2.82 with
  G\&R~2.71.
\end{enumerate}

\paragraph{Mathematics applications.}
\begin{enumerate}
\item \textbf{Inverse functions and integration by parts.}%
  \index{inverse function!integration by parts}%
  \index{integration by parts!inverse trigonometric}%
  The standard technique $\int\arcsin x\,dx=x\arcsin x+\sqrt{1-x^{2}}+C$
  uses integration by parts with $u=\arcsin x$, $dv=dx$.  This
  illustrates the general principle: integrals of inverse functions
  reduce via $\int f^{-1}(x)\,dx=xf^{-1}(x)-\int x\,d(f^{-1}(x))$
  to integrals of the forward function.

\item \textbf{Arctangent and the Gregory--Leibniz series.}%
  \index{Gregory--Leibniz series}%
  \index{arctangent!power series}%
  \index{$\pi$!computation}%
  The Maclaurin series $\arctan x=\sum_{n=0}^{\infty}(-1)^{n}x^{2n+1}/(2n+1)$
  gives, at $x=1$, the Gregory--Leibniz series $\pi/4=1-1/3+1/5-\cdots$.
  The integral representation $\arctan x=\int_{0}^{x}dt/(1+t^{2})$
  connects G\&R~2.82 to the rational-function integrals of G\&R~2.17
  and to Machin-type formulas for computing~$\pi$.
\end{enumerate}

%% -------------------------------------------------------------------
\subsubsection{2.83\quad Combinations of arcsine or arccosine and algebraic functions}
\subsubsection{2.84\quad Combinations of the arcsecant and arccosecant with powers of $x$}
\subsubsection{2.85\quad Combinations of the arctangent and arccotangent with algebraic functions}

\paragraph{Physics applications.}
\begin{enumerate}
\item \textbf{Probability distributions and the arcsine law.}%
  \index{arcsine distribution}%
  \index{random walk!arcsine law}%
  \index{probability!inverse trigonometric}%
  The arcsine distribution with density
  $p(x)=1/(\pi\sqrt{x(1-x)})$ on $(0,1)$ has CDF
  $F(x)=(2/\pi)\arcsin(\sqrt{x})$.  Its moments
  $\int x^{n}\arcsin(\sqrt{x})\,dx$ are combinations of arcsine and
  algebraic functions from G\&R~2.83, arising in the theory of random
  walks and Brownian motion (L\'evy's arcsine law).

\item \textbf{Antenna radiation resistance.}%
  \index{antenna!radiation resistance}%
  \index{radiation resistance!integral}%
  \index{thin-wire antenna}%
  The radiation resistance of a thin-wire antenna involves integrals
  $\int_{0}^{L}\arctan(f(x))\cdot g(x)\,dx$ where $f$ and $g$ are
  algebraic functions of the position along the wire.  These are
  arctangent-with-algebraic-function integrals from G\&R~2.85.

\item \textbf{Fluid flow past a wedge.}%
  \index{wedge flow!inverse trigonometric}%
  \index{potential flow!arctangent}%
  \index{conformal mapping!wedge}%
  The potential flow around a wedge of half-angle $\alpha$ involves
  the complex potential $w=Az^{\pi/\alpha}$, whose streamlines are
  curves $\psi=\text{const}$ given by arctangent expressions in the
  Cartesian coordinates.  Integrated quantities such as pressure force
  involve $\int x^{n}\arctan(y/x)\,dx$, forms from G\&R~2.85.
\end{enumerate}

\paragraph{Mathematics applications.}
\begin{enumerate}
\item \textbf{Clausen's integral and related functions.}%
  \index{Clausen integral}%
  \index{$\operatorname{Cl}_2(\theta)$}%
  \index{dilogarithm!imaginary part}%
  Clausen's integral
  $\operatorname{Cl}_{2}(\theta)=-\int_{0}^{\theta}\ln|2\sin(t/2)|\,dt
  =\sum_{n=1}^{\infty}\sin(n\theta)/n^{2}$ arises naturally when
  integrating products of inverse trigonometric and algebraic functions,
  as the boundary between elementary and non-elementary antiderivatives.
  It is the imaginary part of the dilogarithm on the unit circle.

\item \textbf{Moments of inverse trigonometric functions.}%
  \index{moments!inverse trigonometric}%
  \index{beta function!arctangent moments}%
  \index{integration by parts!iterated}%
  The integrals $\int x^{n}\arctan(x)\,dx$ (G\&R~2.85) evaluate to
  polynomial-times-arctangent plus a rational correction, obtained by
  iterated integration by parts.  These moments connect to the beta
  function: $\int_{0}^{1}x^{n}\arctan(x)\,dx$ can be expressed in
  terms of the digamma function $\psi(n)$ and Catalan's constant
  $G=\sum(-1)^{k}/(2k+1)^{2}$.
\end{enumerate}
